\documentclass{article}
\usepackage[utf8]{inputenc}
\usepackage{algorithm}
\usepackage{algorithmic}
\usepackage{amsfonts}
\usepackage{amsmath}
\usepackage{amssymb}
\usepackage{amsthm}
\usepackage{bm}
\usepackage{bbm}
\usepackage{booktabs}
\usepackage{dsfont}
\usepackage{enumitem}
\usepackage{extarrows}
\usepackage{float} 
\usepackage{graphicx}
\usepackage{hyperref}
\usepackage{inconsolata}
\usepackage{listings}
\usepackage{makecell}
\usepackage{mathrsfs}
\usepackage{multicol}
\usepackage{multirow}
\usepackage{setspace}
\usepackage{subfigure} 
\usepackage{threeparttable}
\usepackage{ulem}
\usepackage{tikz}
\usetikzlibrary{positioning, arrows.meta}
\setitemize[1]{itemsep=0.8pt,partopsep=0.8pt,parsep=\parskip,topsep=0.8pt}
\DeclareMathAlphabet{\mathpzc}{OT1}{pzc}{m}{it}
%% Number of equations.
\numberwithin{equation}{section}
%% New symbols.
\newcommand{\rmb}{\mathrm{b}}
\newcommand{\e}{\mathrm{e}}
\newcommand{\E}{\mathbb{E}}
\newcommand{\ind}{\perp\!\!\!\perp}
\newcommand{\bfw}{\mathbf{w}}
\newcommand{\bbC}{\mathbb{C}}
\newcommand{\bbN}{\mathbb{N}}
\newcommand{\bbP}{\mathbb{P}}
\newcommand{\bbQ}{\mathbb{Q}}
\newcommand{\bbR}{\mathbb{R}}
\newcommand{\bbT}{\mathbb{T}}
\newcommand{\bbZ}{\mathbb{Z}}
\newcommand{\scr}{\mathscr}
\renewcommand{\cal}{\mathcal}
\newcommand{\loc}{\mathrm{loc}}
\newcommand{\ol}{\overline}
\newcommand{\wh}{\widehat}
\newcommand{\wt}{\widetilde}
\DeclareFontFamily{U}{mathx}{}
\DeclareFontShape{U}{mathx}{m}{n}{<-> mathx10}{}
\DeclareSymbolFont{mathx}{U}{mathx}{m}{n}
\DeclareMathAccent{\widecheck}{0}{mathx}{"71}
\DeclareMathOperator{\id}{Id}
\DeclareMathOperator{\gr}{Gr}
\DeclareMathOperator{\tr}{tr}
\DeclareMathOperator{\Le}{Le}
\DeclareMathOperator{\cov}{Cov}
\DeclareMathOperator{\var}{Var}
\DeclareMathOperator{\conv}{Conv}
\DeclareMathOperator{\supp}{supp}
\DeclareMathOperator{\diam}{diam}
\DeclareMathOperator{\esssup}{ess\,sup}
\DeclareMathOperator{\argmax}{argmax}
\DeclareMathOperator{\argmin}{argmin}
\renewcommand{\d}{\mathrm{d}}
\renewcommand{\Re}{\mathrm{Re}}
\renewcommand{\Im}{\mathrm{Im}}
\renewcommand{\i}{\mathrm{i}}
\renewcommand{\proofname}{\textit{Proof}}
\renewcommand*{\thesubfigure}{(\arabic{subfigure})}
\renewcommand{\baselinestretch}{1.20}

\theoremstyle{plain}
\newtheorem{theorem}{Theorem}[section]
\newtheorem{lemma}[theorem]{Lemma}
\newtheorem{proposition}[theorem]{Proposition}
\newtheorem{corollary}[theorem]{Corollary}
\newtheorem{example}[theorem]{Example}
\theoremstyle{definition}
\newtheorem{definition}[theorem]{Definition}
\newtheorem*{remark}{Remark}
\title{\bf Notes on Introductory Point-Set Topology}
\usepackage{geometry}
\geometry{a4paper, scale=0.80}
\author{\textsc{Jyunyi Liao}}
\date{}
\begin{document}
\maketitle
\tableofcontents

\newpage
\setcounter{section}{-1}
\section{Notations}
\begin{description}
	\item[$x\in A$ or $A\ni x$]: $x$ belongs to set $A$
	\item[$A\cup B$]: The union of $A$ and $B$, i.e. $\{x:x\in A\ \text{or}\ x\in B\}$
	\item[$A\cap B$]: The intersection of $A$ and $B$, i.e. $\{x:x\in A\ \text{and}\ x\in B\}$
	\item[$A \subset B$ or $B \supset A$]: $A$ is a subset of $B$, i.e. $\forall x\in A,\ x\in B$
	\item[$A \subsetneq B$ or $B \supsetneq A$]: $A$ is a proper subsetof $B$, i.e. $A\subset B$ and $\exists b\in B$ such that $b\notin A$
	\item[$B\backslash A$]: The difference of $B$ from $A$, i.e. $\{b:b\in B,b\notin A\}$ 
	\item[$A\times B$]: The cartesian product of $A$ and $B$, i.e. the set of tuples $\{(a,b):a\in A,b\in B\}$
	\item[$\bigcup_{\alpha\in J}A_\alpha$]: The union of collection $\{A_\alpha,\alpha\in J\}$, i.e. $\{a:\exists\alpha\in J,\ a\in A_\alpha\}$
	\item[$\bigcap_{\alpha\in J}A_\alpha$]: The intersection of collection $\{A_\alpha,\alpha\in J\}$, i.e. $\{a:\forall\alpha\in J,\ a\in A_\alpha\}$
	\item[$\prod_{\alpha\in J}A_\alpha$]: The cartesian product of collection $\{A_\alpha,\alpha\in J\}$, i.e. $\{(x_\alpha)_{\alpha\in J}: x_\alpha\in A_\alpha\ \text{for each}\ \alpha\in J\}$
	\item[$\pi_\beta$]: The projection map that carries a tuple $\mathbf{x}=(x_\alpha)_{\alpha\in J}$ to its $\beta$-th component $x_\beta$
	\item[$\emptyset$]: The empty set
	\item[$\overline{A}$]: The closure of subspace $A$
	\item[$\mathring{A}$]: The interior of subspace $A$
	\item[$\partial A$]: The frontier of subspace $A$
	\item[$O(x_0,\epsilon)$]: The open ball centered at $x_0$ of radius $\epsilon$ in a metric space $(X,d)$, i.e. $\{x\in X:d(x,x_0)<\epsilon\}$
	\item[$d(x,A)$] The distance from a point $x$ to a set $A$ in a metric space $(X,d)$, i.e. $d(x,A)=\inf_{a\in A}d(x,a)$
	\item[$x_n\to x$]: The point sequence $x_n$ converges to $x$
	\item[$Y^X$]: The set of all functions from $X$ to $Y$, namely, the Cartesian product $\prod_{x\in X}Y$
	\item[$f|_A$]: The restriction of function $f:X\to Y$ on subspace $A$, i.e. $f|_A:A\to Y, a\mapsto f(a)$
	\item[$\chi_A$] The indicator function of a subset $A\subset X$, i.e. $\chi_A:X\to\{0,1\}$ with $\chi_A(A)=\{1\}$ and $\chi_A(X\backslash A)=\{0\}$
	\item[$f_n\rightrightarrows f$]: The function sequence $f_n$ converges uniformly to $f$
	\item[$\mathbb{N}$]: The set of all positive integers, i.e. $\{1,2,\cdots,n,n+1,\cdots\}$
	\item[$\mathbb{N}_0$]: $\mathbb{N}\cup\{0\}$
	\item[$\mathbb{Z}$]: The set of all integers, i.e. $\mathbb{N}\cup\{0\}\cup(\mathbb{-\mathbb{N}})$
	\item[$\mathbb{Q}$]: The set of all rational numbers, i.e. $\{n/m:n\in\mathbb{Z},m\in\mathbb{Z}\}$
	\item[$\mathbb{R}$]: The set of all real numbers, the completion of $(\mathbb{Q},|\cdot - \cdot|)$
	\item[$\mathbb{R}_+$]: The set of nonnegative real numbers, i.e. $\{x:x\in\mathbb{R},x\geq 0\}$
	\item[$\mathbb{C}$]: The set of complex numbers, i.e. $\{a+b\mathrm{i}:a\in\mathbb{R},b\in\mathbb{R}\}$, where $\mathrm{i}^2=-1$
\end{description}
\newpage

\section{Open and Closed Sets}
\subsection{Open and Closed Sets in Topological Spaces}
\begin{definition}[Topology and open sets]\label{def:1.1}
Let $X$ be a nonempty open set. A \textit{topology} on $X$ is a collection $\mathscr{T}$ of nonempty subsets of $X$, called \textit{open sets}, such that
\begin{itemize}
	\item[(i)] any union of open sets is open,
	\item[(ii)] any finite intersection of open sets is open, and
	\item[(iii)] both $X$ and $\emptyset$ are open.
\end{itemize}
A set $X$ together with a topology $\mathscr{T}$ on it is called a \textit{a topological space}, denoted by $(X,\mathscr{T})$. Without ambiguity, we drop $\mathscr{T}$ and say $X$ is a topological space.
\end{definition}

\paragraph{Examples of topological spaces.}
\begin{itemize}
	\item (Euclidean Space). Let $X = \mathbb{R}^n.$ A subset $U$ of $X$ is open if for every $\mathbf{x}\in U,$ there exists $\delta > 0$ such that the ball $O(\mathbf{x},\delta) = \left\{\mathbf{y}\in\mathbb{R}^n:\Vert\mathbf{y}-\mathbf{x}\Vert_2 < \delta\right\}$ lies entirely in $U$.
	\item (Discrete topology). Let $X$ be a non-empty set. Every subset of $X$ is an open set.
	\item (Subspace topology/induced topology). Let $Y$ be a non-empty subset of $X$. A subset $U$ of $Y$ is open if there exists an open set $O$ in $X$ such that $U=O\cap Y$.
\end{itemize}

\begin{definition}[Neighborhood]\label{def:1.2}
Let $X$ be a topological space. Given a point $x\in X$, a subset $N$ of $X$ is called a \textit{neighborhood} of $x$, if we can find an open set $O$ in $X$ such that $x\in O \subset N$.
\end{definition} 
By definition, an open set $O\subset X$ is a neighborhood of each of its points. Conversely, if $O$ is a neighborhood of each of its points, we can find an open $N_x$ for each $x\in O$ such that $x\in N_x\subset O$. Then $O=\bigcup_{x\in O}N_x$ as a union of open sets is itself open.

\begin{proposition}[Properties of neighborhoods]\label{prop:1.3}
 Let $X$ be a topological space, and $x$ is a point in $X$. Then the following statements hold:
\begin{itemize}
	\item[(i)] $x$ lies in each of its neighborhood.
	\item[(ii)] The intersection of two neighborhoods of $x$ is itself a neighborhood of $x$.
	\item[(iii)] If $N$ is a neighborhood of $x$ and $N\subset M \subset X$, then $M$ is a neighborhood of $x$.
	\item[(iv)] If $N$ is a neighborhood of $x$, then $\mathring{N}=\{y\in N:N \ \text{is a neighborhood of}\ y\}$ is also a neighborhood of $x$.
\end{itemize}
\end{proposition}
\begin{proof}
The first three statements are trivial. We prove the fourth statement. Let $N$ be a neighborhood of $x$, then there exists an open set $O$ in $X$ such that $x\in O \subset N$. Since $O$, as an open set, is a neighborhood of each of its points, we have $O\subset\mathring{N}$, which concludes the proof.
\end{proof}

\paragraph{Remark.} The four properties in Proposition 1.3 form an alternative construction of a topological space. More specifically, let $X$ be a non-empty sets, for each point $x\in X$ we define the collection of its neighborhoods as satisfying (i)-(iv). A subset $O$ of $X$ is called an open set if it is a neighborhood of each of its point. Then we can verify that the collection of open sets in $X$ satisfies \hyperref[def:1.1]{Definition 1.1}.

\begin{definition}[Closed sets]\label{def:1.4}
A subset $A$ of a topological space $X$ is said to be a \textit{closed} set in $X$, if its complement $X\backslash A$ is open.
\end{definition}

\paragraph{Remark.} Combining \hyperref[def:1.1]{Definition 1.1} and \hyperref[def:1.4]{Definition 1.4}, it is clear that any intersection of closed sets, any finite union of closed sets, the entire space $X$ and the empty set $\emptyset$ are closed. To characterize closed sets in a topological space, we introduce the following definition.

\begin{definition}[Limit points/accumulation points]\label{def:1.5} 
Let $A$ be a subset of a topological space $X$. A point $p\in X$ is called a \textit{limit point} of $A$ if every neighborhood of $p$ contains at least one point of $A\backslash\{p\}$.
\end{definition}

\newpage
\begin{theorem}[Characterization of closed sets]\label{thm:1.6}
A set is closed if and only if it contains all its limit points.
\end{theorem}
\begin{proof}
Let $A$ be a closed set in a topological space $X$. Then its complement $X\backslash A$, being an open set, is a neighborhood of each of its points. Then any $x\in X\backslash A$ is not a limit point of $A$, and $A$ contains all its limit points. Conversely, if $A$ contains all of its limit points, then for each $x\in X\backslash A$, there exists a neighborhood of $x$ lying in $X\backslash A$. Therefore $X\backslash A$ is a neighborhood of each of its points.
\end{proof}

\begin{definition}[Closure]\label{def:1.7}
Let $A$ be a subset of a topological space $X$. The union of $A$ and all its limit points, denoted by $\overline{A}$, is called the \textit{closure} of $A$.
\end{definition}

\begin{theorem}\label{thm:1.8}
Let $A$ be a subset of a topological space $X$. Then $\overline{A}$ is the intersection of all closed sets in $X$ that contains $A$. In other words, $\overline{A}$ is the smallest closed set that contains $A$.
\end{theorem}
\begin{proof}
We first prove that $\overline{A}$ is closed. For every $x\in X\backslash\overline{A}$, we can find an open neighborhood $O$ of $x$ such that $O$ does not intersect with $A$. If $O$ contains a limit point of $A$, denoted by $p$, then $O$ as a neighborhood of $p$ contains a point of $A$, a contradiction! Hence $O\subset X\backslash\overline{A}$, showing $\overline{A}$ is closed. Now let $B\supset A$ be a closed set in $X$. It suffices to show that any limit point $p$ of $A$ is contained in $B$. To see this, suppose $p\notin B$. Since $X\backslash B$ is open, it is a neighborhood of $p$. Then $X\backslash B$ contains at least one point in $A$, again a contradiction!
\end{proof}

The following conclusion immediately follows from \hyperref[thm:1.8]{Theorem 1.8}.
\begin{corollary}\label{cor:1.9}
A set is closed if and only if it is equal to its closure.
\end{corollary}

\begin{definition}[Interior]\label{def:1.10}
Let $A$ be a subset of a topological space $X$. The \textit{interior} of $A$, denoted by $\mathring{A}$, is the union of all subsets of $A$ that are open in $X$. A point that is in $\mathring{A}$ is an \textit{interior point} of $A$.
\end{definition}

It is clear that a set is open if and only if it is equal to its interior. We can also check that a point $x$ lies in $\mathring{A}$ if and only if $A$ is a neighborhood of $x$, which is consistent with the notation we use in \hyperref[prop:1.3]{Proposition 1.3}.

\begin{definition}[Frontier]\label{def:1.11} Let $A$ be a subset of a topological space $X$. We define the \textit{frontier} of $A$ as the intersection of its closure and the closure of its complements, $\partial A:= \overline{A}\cap\overline{(X\backslash A)}$.
\end{definition}

\begin{proposition}\label{prop:1.12}
Let $A$ be a subset of a topological space $X$. Then $\mathring{A}\cap\partial A = \emptyset$, and $\mathring{A}\cup\partial A = \overline{A}$.
\end{proposition}
\begin{proof}
Let $\{O_\lambda:\lambda\in\Lambda\}$ be the collection of all open subsets of $A$. Then $\{X\backslash O_\lambda:\lambda\in\Lambda\}$ is the collection of all closed sets in $X$ that contains $X\backslash A$. By \hyperref[def:1.10]{Definition 1.10} and \hyperref[thm:1.8]{Theorem 1.8}, we have $\mathring{A}=\bigcup_{\lambda\in\Lambda}O_\lambda$, and $\overline{(X\backslash A)}=\bigcap_{\lambda\in\Lambda}(X\backslash O_\lambda) = X\backslash\mathring{A}$. Hence $\mathring{A}\cap\partial A=\emptyset$, and $\partial A = \overline{A}\cap(X\backslash\mathring{A}) = \overline{A}\backslash\mathring{A}$.
\end{proof}

\paragraph{Remark.} In the above proof, we obtain an alternative definition of the interior: $\mathring{A} = X\backslash\overline{(X\backslash A)}$

\begin{proposition}\label{prop:1.13} Let $A$ and $B$ be two subsets of a topological space $X$. The following statements hold:\\
(i) $(A\cup B)^{\circ} \supset \mathring{A}\cup\mathring{B}$; (ii) $(A\cap B)^{\circ} = \mathring{A}\cap\mathring{B}$; (iii) $(\mathring{A})^\circ = \mathring{A}.$\\
(iv) $\overline{A\cup B} = \overline{A}\cup\overline{B};$ (v)  $\overline{A\cap B} \subset \overline{A}\cap\overline{B}$; (vi)  $\overline{\overline{A}} = \overline{A}$;
\end{proposition}
\begin{proof}
Applying (i) and (ii) to $X\backslash A$ and $X\backslash B$ yields (v) and (iv), respectively. The result (iii) holds because $\mathring{A}$ is open, and (vi) holds because $\overline{A}$ is closed. Hence it remains to show (i) and (ii).\\
(i):
$x\in\mathring{A}\cup\mathring{B}\ \Leftrightarrow\ \text{either}\ A\ \text{or}\ B\ \text{is a neighborhood of}\ x \Rightarrow\ A\cup B\ \text{is a neighborhood of}\ x\ \Leftrightarrow\ x\in(A\cup B)^\circ.$\\
(ii):
$x\in(A\cap B)^{\circ}\ \Leftrightarrow\ A\cap B\ \text{is a neighborhood of}\ x\ \Leftrightarrow\ \text{both}\ A, B\ \text{are neighborhoods of}\ x\ \Leftrightarrow\ x\in\mathring{A}\cup\mathring{B}.$
\end{proof}

\paragraph{Remark.} The equality does not necessarily holds in (i) and (v). As a counterexample of (i), consider $A=[-1,0]$ and $B=[0,1]$ in $X=\mathbb{R}$.

\subsection{Density and Separability}
\begin{definition}[Dense sets]\label{def:1.14} Let $A$ be a subset of a topological space $X$. A subset $D$ of $X$ is said to be \textit{dense} in $A$ if $A\subset\overline{D}$. Specifically, $D$ is a \textit{dense set} if $\overline{D}=X$. 
\end{definition}

\paragraph{Remark.} $D$ is also called an \textit{everywhere dense set} when $\overline{D} = X$.

\begin{proposition}\label{prop:1.15} Let $A$ be a subset of a topological space $X$. Then $A$ is dense if and only if it intersects with every nonempty open set in $X$.
\end{proposition}
\begin{proof}
Suppose that $A$ intersects with every nonempty open set in $X$. It suffices to show that for any $x\in X\backslash A$, $x$ is a limit point of $A$. This is clear because every neighborhood of $x$, containing a nonempty open set in $X$, intersects with $A$. Conversely, let $O$ be a nonempty open set in $X$, and let $A$ be dense in $O$. Choose $x\in O$. The conclusion is clear if $x\in A$, so it remains to prove the case $x\notin A$. Since $A$ is dense in $O$, $x$ is a limit point of $A$. Hence $O$ as a neighborhood of $x$ contains at least one point of $A$, which concludes the proof.
\end{proof}

\begin{proposition}\label{prop:1.16} Let $A$ be a dense set in a topological space $X$. Then for every nonempty open set $O\subset X$, $A\cap O$ is dense in $O$.
\end{proposition}
\begin{proof}
Choose $x\in O$, we want to show that $x\in\overline{A\cap O}$. It suffices to prove the case $x\notin A$. Let $N$ be an arbitrary neighborhood of $x$. By \hyperref[prop:1.3]{Proposition 1.3} (ii), $N\cap O$ is a neighborhood of $x$, and contains at least one point of $A$. Hence $N \cap (O\cap A)\neq\emptyset$, and $x$ is a limit point of $A\cap O$.
\end{proof}

\begin{definition}[Basis]\label{def:1.17} Let $X$ be a topological space.
\begin{itemize}
\item[(i)] A \textit{basis} for the topological space $X$ is a family $\mathscr{B}$ of open sets such that every open set in $X$ is a union of members of $\mathscr{B}$. Elements of $\mathscr{B}$ are called \textit{base sets}.
\item[(ii)] A \textit{neighborhood basis} for $X$ at $x$ is a family $\mathscr{B}_x$ of open sets containing $x$ such that every neighborhood of $X$ contains at least one member of $\mathscr{B}_x$.
\end{itemize}
\end{definition}

\paragraph{Remark.} A family of sets $\mathscr{B}$ is a basis for $X$ if and only if $\mathscr{B}$ contains a neighborhood basis at each $x\in X$.

\begin{theorem}\label{thm:1.18}
Let $\mathscr{B}$ be a nonempty collection of subsets of a set $X$. If the intersection of any finite number of members of $\mathscr{B}$ is always in $\mathscr{B}$, and if $\bigcup_{B\in\mathscr{B}}B = X$, then $\mathscr{B}$ is a basis for a topology on $X$. 
\end{theorem} 
\begin{proof}
Let $\mathscr{T}$ be the collection of all unions of members of $\mathscr{B}$. Then $\mathscr{T}$ forms a topology on $X$. (To see this, just check the three conditions in \hyperref[def:1.1]{Definition 1.1}.)
\end{proof}

\paragraph{Remark.} Given any family $\mathscr{E}$ of sets in $X$, we can generate a topology on $X$ which consists of all unions of finite intersections of members of $\mathscr{E}$.

\begin{definition}[Second countable space]\label{def:1.19} A topological space is said to be a \textit{second countable} space if it has a countable basis.
\end{definition}

\begin{definition}[Separable space]\label{def:1.20} A topological space is \textit{separable} if it has a countable dense subset.
\end{definition}
\begin{theorem}\label{thm:1.21}
A second countable topological space is separable.
\end{theorem} 
\begin{proof}
Let $X$ be a second countable topological space with a basis $\mathscr{B}=\{B_n:n\in\mathbb{N}\}$, and without loss of generality let each $B_n$ be nonempty since empty sets can be discarded. Choose $x_n\in B_n$ for each $n$ and let $A=\{x_n,n\in\mathbb{N}\}$, then $X$ is separable if we can show that $A$ is dense in $X$. 

By \hyperref[prop:1.15]{Proposition 1.15}, it suffices to show that $A$ intersects with every nonempty open set in $X$. Let $O$ be an arbitrary nonempty open set in $X$. Then there exists $B_n$ such that $B_n\subset O$. Consequently, $O$ and $A$ meet at the point $x_n$.
\end{proof}

\subsection{The Subspace Topology}
\begin{definition}[Subspace topology]\label{def:1.22}
Let $Y$ be a non-empty subset of a topological space $(X,\mathscr{T}_X)$. Let $\mathscr{T}_Y = \{O\cap Y: O\in\mathscr{T}_X\}$ be the collection of open sets in $Y$, then $\mathscr{T}_Y$ defines a topology on $Y$, which is called the \textit{subspace topology}. The topological space $(Y,\mathscr{T}_Y)$ is also called a \textit{subspace}. Without ambiguity we can drop $\mathscr{T}_Y$ and say $Y$ is a subspace of $X$.
\end{definition}

\begin{proposition}\label{prop:1.23} If $Y$ is a subspace of $X$, and $Z$ is a subspace of $Y$, then $Z$ is a subspace of $X$.
\end{proposition}
\begin{proof}
By definition, we have $Z\subset Y\subset X$, and $\mathscr{T}_Y = \{O\cap Y:O\in\mathscr{T}_X\}$, $\mathscr{T}_Z = \{O^\prime\cup Z:O^\prime\in\mathscr{T}_Y\}.$\\
$\forall O^\prime\cap Z\in\mathscr{T}_Z,$ $\exists O\in\mathscr{T}_X$ such that $O^\prime\cap Z = (O\cap Y) \cap Z = O\cap Z.$\\
And $\forall O\in\mathscr{T}_X$, we have $O\cap Z = O\cap (Y\cap Z) = O^\prime\cap Z\in\mathscr{T}_Z$, where $O^\prime:=O\cap Y\in\mathscr{T}_Y$.\\
Hence $\mathscr{T}_Z = \{O\cup Z:O\in\mathscr{T}_X\}$, and $Z$ is a subspace of $X$.
\end{proof}

\begin{proposition}[Closed sets in a subspace]\label{prop:1.24} Let $Y$ be a subspace of $X$. Then a subset of $Y$ is closed if and only if it is the intersection of $Y$ with a closed set in $X$.
\end{proposition}
\begin{proof}
For the ``if'' statement, let $K=B\cap Y,$ where $B$ is closed in $X$. Then $Y\backslash K = Y\backslash B = Y\cap (X\backslash B)$ is open in $Y$, because $X\backslash B$ is open in $X$. Therefore $K$ is closed in $Y$.

For the ``only if'' statement, suppose $K$ is closed in $Y$. Then we know that $Y\backslash K$ is open in $Y$, and $\exists$ open $O$ in $X$ such that $Y\backslash K = O\cap Y$. Hence $(X\backslash O)\cap Y= Y\backslash (O\cap Y) = K,$ which concludes the proof.
\end{proof}

\begin{lemma}[Open and closed subspaces]\label{lemma:1.25}
Let $Y$ be a subspace of $X$ such that $Y$ is open (closed) in $X$, and $A$ be a subset of $Y$. Then $A$ is open (closed) in $Y$ if and only if $A$ is open (closed) in $X$.
\end{lemma}
\begin{proof}
By \hyperref[def:1.22]{Definition 1.22} and \hyperref[prop:1.24]{Proposition 1.24}.
\end{proof}

\begin{proposition}[Closures and interiors in a subspace]\label{prop:1.26}
Let $Y$ be a subspace of $X$, and $A$ be a subset of $Y$. Denoted by $\overline{A}_X$ and $\overline{A}_Y$ the closure of $A$ in $X$ and in $Y$, respectively, and similarly $\mathring{A}_X$ and $\mathring{A}_Y$ the interiors. Then: (i) $\overline{A}_Y = \overline{A}_X\cap Y$, (ii) $\mathring{A}_Y\supset\mathring{A}_X$.
\end{proposition}
\begin{proof}
(i) Let $\{B_\lambda:\lambda\in\Lambda\}$ be the collection of all closed subsets of $X$ that contains $A$. By \hyperref[prop:1.24]{Proposition 1.24}, $\{B_\lambda\cap Y:\lambda\in\Lambda\}$ is the collection of all closed subsets of $Y$ that contains $A$, and (i) follows from \hyperref[thm:1.8]{Theorem 1.8}.\\
(ii) We only show the case $\mathring{A}_X\neq\emptyset$. If $a\in\mathring{A}_X$, then $\exists$ an open set $O$ in $X$ such that $a\in O\subset\mathring{A}_X$. Since $O\cap Y$ is open in $Y$, and we have $a\in O\cap Y\subset A,$ $A$ is a neighborhood of $a$ in subspace $Y$. Hence $a\in\mathring{A}_Y$.
\end{proof}

\paragraph{Remark.} In \hyperref[prop:1.26]{Proposition 1.26}, the equality in (ii) does not necessarily holds. As a counterexample, consider Euclidean spaces $X=\mathbb{R}^2$ and $Y=\mathbb{R}$, $A = (0,1)\subset Y$. Then $\mathring{A}_X = \emptyset$, $\mathring{A}_Y=A$.

\newpage
\section{Continuity}
\subsection{Continuous Functions}
\paragraph{Definition 2.1\label{def:2.1}} (Continuous functions). Let $X$ and $Y$ be two topological spaces. A function $f:X\to Y$ is said to be \textit{continuous} if the inverse image of each open set in $Y$ is open in $X$, i.e., for each open set $O\subset Y$, the inverse image $f^{-1}(O):=\{x\in X:f(x)\in O\}$ is open in $X$.

\paragraph{Proposition 2.2\label{prop:2.2}} (Neighborhood characterization of continuity). Let $X$ and $Y$ be two topological spaces. A function $f:X\to Y$ is continuous if and only if for each point $x\in X$ and each neighborhood $N$ of $f(x)$ in $Y$, the inverse image $f^{-1}(N)$ is a neighborhood of $x$ in $X$.
\begin{proof}
\textit{``If'' part:} Let $O$ be an open set in $Y$. Then for every $x\in f^{-1}(O)$, $O$ is a neighborhood of $f(x)$ in $Y$. By our assumption, $f^{-1}(O)$ is a neighborhood of $x$ in $X$.\\
\textit{``Only if'' part:} Let $x\in X$ and $N$ be a neighborhood of $f(x)$ in $Y$. Then $\exists$ an open set $O$ such that $f(x)\in O\subset N$. Since $f^{-1}(O)$ is open, $f^{-1}(N)\supset f^{-1}(O)\ni x$ is a neighborhood of $x$ in $X$.
\end{proof}

\paragraph{Remark.} In some literature, \hyperref[prop:2.2]{Proposition 2.2} is also used as the definition of continuous functions.

\paragraph{Theorem 2.3.\label{thm:2.3}} The composition of two continuous functions is continuous.
\begin{proof}
Let $X,Y,Z$ be topological spaces, and $f:X\to Y,\ g:Y\to Z$ be continuous functions. Let $O$ be an open set in $Z$, then $g^{-1}(O)$ is open in $Y$, and $f^{-1}g^{-1}(O)$ is open in $X$. Since $(g\circ f)^{-1}(O) = f^{-1}g^{-1}(O)$, we conclude that $g\circ f:X\to Z$ is continuous.
\end{proof}

\paragraph{Theorem 2.4. \label{thm:2.4}} Let $X$ and $Y$ be two topological spaces, and $f:X\to Y$ be a continuous function. Let $A\subset X$ have the subspace topology. Then the restriction $f|_A:A\to Y$ is continuous.
\begin{proof}
Let $O$ be an open set in $Y$, then $f^{-1}(O)$ is open in $X$, and $(f|_A)^{-1}(O) = A\cap f^{-1}(O)$ is open in the subspace topology on $A$. Then $f|_A$ is continuous.
\end{proof}

\paragraph{Remark.} The function from $X$ to $X$ which sends each point $x\in X$ to itself is called the \textit{identity map}, denoted by $I_X$. If we restrict $I_X$ to a subspace $A$ of $X$, we obtain the \textit{inclusion map}, denoted by $\iota:A\to X$.

\paragraph{Theorem 2.5. \label{thm:2.5}} Let $X$ and $Y$ be two topological spaces. The following statements are equivalent:
\begin{itemize}
	\item[(i)] $f:X\to Y$ is continuous.
	\item[(ii)] If $\mathscr{B}$ is a basis for the topology of $Y$, then the inverse image of every member of $\mathscr{B}$ is open in $X$.
	\item[(iii)] $f(\overline{A}) \subset \overline{f(A)}$ for any subset $A$ of $X$.
	\item[(iv)] $\overline{f^{-1}(B)}\subset f^{-1}(\overline{B})$ for any subset $B$ of $Y$.
	\item[(v)] The inverse image of each closed set in $Y$ is closed in $X$.
\end{itemize}
\begin{proof}
(i) $\Rightarrow$ (ii): By \hyperref[def:2.1]{Definition 2.1}.\\
(ii) $\Rightarrow$ (iii): Let $A$ be a subset of $X$. Since $f(A)\subset\overline{f(A)}$, it suffices to show $f(x)$ is a limit point of $f(A)$ for $x\in \overline{A}\backslash A$ such that $f(x)\notin f(A)$. If $N$ is a neighborhood of $f(x)$ in $Y$, then $\exists B\in\mathscr{B}$ such that $f(x)\in B\subset N$, and $f^{-1}(B)$ is an open neighborhood of $x$. Since $x$ is a limit point of $A$, $f^{-1}(B)$ contains at least one point in $A$. As a result $B$ and $N$ both contain at least on point in $f(A)$, which concludes the proof.\\
(iii) $\Rightarrow$ (iv): Let $A = f^{-1}(B)$ in (iii).\\
(iv) $\Rightarrow$ (v): Let $B$ be a closed set in $Y$. Then $B=\overline{B}$, and by (iv) $f^{-1}(B) \subset\overline{f^{-1}(B)}\subset f^{-1}(\overline{B}) = f^{-1}(B)$. \\
(v) $\Rightarrow$ (i): Let $O$ be an open set in $Y$. Then by (iv) $f^{-1}(Y\backslash O) = X\backslash f^{-1}(O)$ is closed, and $f^{-1}(O)$ is open.
\end{proof}

\paragraph{Definition 2.6\label{def:2.6}} (Homeomorphism). Let $X$ and $Y$ be two topological spaces. A \textit{homeomorphism} is a function $h:X\to Y$ that is continuous, bijective and that has continuous inverse.

\paragraph{Remark.} In \hyperref[def:2.6]{Definition 2.6}, the condition of continuous inverse is required. Consider function $f:[0,1)\to\{z\in\mathbb{C}:\vert z\vert = 1\},\ x\mapsto\mathrm{e}^{\mathrm{i}2\pi x}$, which is continuous and bijective. The inverse $f^{-1}:\{z\in\mathbb{C}:\vert z\vert = 1\}\to[0,1),z\mapsto\frac{1}{2\pi}\arg z$ is not continuous. For example, it maps $\left\{\mathrm{e}^{\mathrm{i}\theta}:\theta\in[0,\pi)\right\}$ to $[0,1/2)$, the inverse image of an open set is not open!

\subsection{Metric Spaces and Tietze Extension Theorem\label{subsec:2.2}}

\paragraph{Definition 2.7\label{def:2.7}} (Metric, metric spaces and metric topology). Let $X$ be a nonempty set. A \textit{metric} on $X$ is a function $d:X\times X\to\mathbb{R}$ such that for all $x,y,z\in X$ the following conditions are satisfied:\\
(i) $d(x,y)\geq 0,$ and $d(x,y)=0$ holds if and only if $x=y;$\\
(ii) $d(x,y) = d(y,x);$\quad (iii) $d(x,z) + d(z,y)\geq d(x,y)$.

A set $X$ together with a metric $d$ is called a \textit{metric space}, denoted by $(X,d)$. Without ambiguity, we drop $d$ and say $X$ is a metric space. 

Given a metric $d$ on a set $X$, we let $O(x,\epsilon):=\{y:d(x,y)<\epsilon\}$ the \textit{open ball} centered at $x$ of radius $\epsilon > 0$. Then a topology can be induced as follows: a subset $U$ of $X$ is open, if for each $x\in U$, there exists $\epsilon > 0$ such that $O(x,\epsilon)$ is contained in $U$. This topology satisfies the axioms in \hyperref[def:1.1]{Definition 1.1}. It is also referred to as the \textit{metric topology}.

\paragraph{Remark.} We can check that \hyperref[def:2.1]{Definition 2.1} is consistent with the definition of continuity in metric spaces, which is characterized by $\epsilon$-$\delta$ the condition: given any $x\in X$ and any $\epsilon > 0$, there exists $\delta > 0$ such that $d_X(x,x^\prime)<\delta$ implies $d_Y(f(x),f(x^\prime))<\epsilon$. 

Let $(X,d_X)$ and $(Y,d_Y)$ be two metric spaces, and $f:X\to Y$ be a function. Suppose the $\epsilon$-$\delta$ condition holds, and let $U$ be an open set in $Y$. Then for every $x\in f^{-1}(U)$, we can find some $\epsilon > 0$ such that the open ball $O_Y(f(x),\epsilon)$ lies in $U$. Moreover, there exists $\delta > 0$ such that for all $x^\prime\in O_X(x,\delta)$ we have $f(x^\prime)\in O_Y(f(x),\epsilon)\subset U$. Hence $O(x,\delta)\subset f^{-1}(U)$. Hence $f^{-1}(U)$ is open.

Conversely, suppose $f$ is continuous. Then for any $\epsilon > 0$ and $x\in X$, $f^{-1}\left(O_Y(f(x),\epsilon)\right)$ must be an open neighborhood of $x$ in $X$. As a result, there exists $\delta > 0$ such that $O_X(x,\delta)\subset\{x^\prime\in X:d_Y(f(x),f(x^\prime))<\epsilon\}$.

\paragraph{Lemma 2.8.\label{lemma:2.8}} Let $A$ be a subset in a metric space $(X,d)$. For a point $x\in X$, define its distance to set $A$ as $d(x,A) = \inf_{y\in A}d(x,y)$. Then the function $x\mapsto d(x,A)$ is continuous on $X$.
\begin{proof}
Let $x\in X$ and let $N$ be a neighborhood of $d(x,A)$ on the real line. Choose a small $\epsilon > 0$ such that $(d(x,A)-\epsilon,d(x,A)+\epsilon)\subset N$, and $a\in A$ such that $d(x,a)<d(x,A) + \epsilon/2$. For $z\in O(x,\epsilon/2)$, we have
\begin{align*}
	d(z,A) \leq d(z,a) \leq d(z,x) + d(x,a) < d(x,A) + \epsilon.
\end{align*}
Similarly, we have $d(x,A) < d(z,A) + \epsilon$. Hence $O(x,\epsilon/2)$ is mapped inside $(d(x,A)-\epsilon,d(x,A)+\epsilon)\subset N$, and the inverse image of $N$ is a neighborhood of $N$. Following \hyperref[prop:2.2]{Proposition 2.2} completes the proof.
\end{proof}

\paragraph{Lemma 2.9.\label{lemma:2.9}} Following \hyperref[lemma:2.8]{Lemma 2.8}, $d(x,A)=0$ if and only if $x\in\overline{A}$.
\begin{proof}
For the ``if'' statement, it suffices to show the case $x\in\overline{A}\backslash A$, which implies that $x$ is a limit point of $A$, and $O(x,\epsilon)\cap A\neq\emptyset$ for all $\epsilon > 0$. Hence $0\leq d(x,A) = \inf_{y\in A}d(x,y) <\epsilon$ for all $\epsilon > 0$, and $d(x,A) = 0$.

For the ``only if'' statement, suppose $x$ is neither a point nor a limit point of $A$. Then there exists $\epsilon > 0$ such that $O(x,\epsilon)\cap A = \emptyset$. As a result, $d(x,A) \geq \epsilon > 0$. Hence $d(x,A)=0$ only if $x\in\overline{A}$. 
\end{proof}

\paragraph{Corollary 2.10.\label{cor:2.10}} Following \hyperref[lemma:2.8]{Lemma 2.8}, $d(x,A)=d(x,\overline{A})$ for all $x\in X$.
\begin{proof}
Fix $x\in X$. Since $A\subset\overline{A}$, it suffices to show $d(x,A)\leq d(x,\overline{A})$. For all $z\in\overline{A},$ by \hyperref[lemma:2.9]{Lemma 2.9}, we have $d(x,A)\leq d(x,z) + d(z,A) = d(x,z)$, which completes the proof.
\end{proof}

\paragraph{Lemma 2.11\label{lemma:2.11}} Let $A$ and $B$ be two disjoint closed subsets in a metric space $(X,d)$. Then there exists a continuous $\mathbb{R}$-valued function on $X$ such that $f(A) = \{1\}, f(B)=\{-1\}$ and $f(X\backslash(A\cup B))=(-1,1)$.
\begin{proof}
Since $A$ and $B$ are closed and disjoint, \hyperref[lemma:2.9]{Lemma 2.9} implies that $d(x,A) + d(x,B) > 0$ for all $x\in X$. Hence we can define
\begin{align*}
	f(x) = \frac{d(x,A) - d(x,B)}{d(x,A) + d(x,B)},\ x\in X
\end{align*}
which takes on the required values. Moreover, the continuity of $f$ follows from \hyperref[lemma:2.8]{Lemma 2.8}.
\end{proof}

\paragraph{} Let $A$ be a subspace of topological space $X$. Given a continuous function $f:A\to\mathbb{R}$, we are interested if we are able to extend $f$ to the whole space $X$ without damage its continuity. More explicitly, we want to find a continuous $\mathbb{R}$-valued function on $X$ such that its restriction on $A$ is $f$.

\paragraph{Theorem 2.12\label{thm:2.12}} (Tietze extension theorem). Any real-valued continuous function defined on a closed subset of a metric space can be extended over the whole space.

\paragraph{} We left the proof of \hyperref[thm:2.12]{Theorem 2.12} to \hyperref[thm:5.16]{Theorem 5.16}. To prove this conclusion in the metric space case, we can apply \hyperref[lemma:2.11]{Lemma 2.11} instead of the Urysohn lemma. The proof also uses the Weierstrass M-test, which is introduced in \hyperref[thm:2.20]{Theorem 2.20}.

\subsection{Convergence and Uniform Convergence}
\paragraph{Definition 2.13\label{def:2.13}} (Metrizable spaces). Let $X$ be a topological space. Then $X$ is said to be \textit{metrizable} if there exists a metric $d$ on $X$ that induces the topology of $X$. Under this definition, a metric space $(X,d)$ is a metrizable space $X$ together with a specific metric $d$ that gives the topology of $X$.

\paragraph{Remark.} Without ambiguity, we share the terms ``metric space'' and ``metrizable space'' in later sections.

\paragraph{Definition 2.14\label{def:2.14}} (Boundedness). Let $(X,d)$ be a metric space. A subset $A$ of $X$ is said to be bounded if there exists $M>0$ such that $d(x,x^\prime)>0$ for every pair $x,x^\prime$ of points of $A$. If $A$ is bounded and nonempty, then the \textit{diameter} of $A$ is defined as $D_A=\sup_{x,x^\prime\in A}d(x,x^\prime)$.

\paragraph{Remark.} Boundedness of a set is not a topological property, for it depends on the metric $d$ defined on $X$. In fact, we can find  for every metric space $(X,d)$ a metric $\overline{d}$ which induces the same topology on $X$ and is bounded.

\paragraph{Theorem 2.15\label{thm:2.15}} (Standard bounded metric). Let $(X,d)$ be a metric space. Define $\overline{d}:X\times X\to\mathbb{R}$ by the equation
\begin{align*}
	\overline{d}(x,x^\prime) = \min\{d(x,x^\prime),1\},\ \forall x,x^\prime\in X.
\end{align*}
Then $\overline{d}$ is a metric that induces the same topology as $d$.
\begin{proof}
It is easy to verify that $\overline{d}$ is a valid metric which satisfies the three conditions in \hyperref[def:2.7]{Definition 2.7}. To check the third condition, note that
\begin{align*}
	\overline{d}(x,y) + \overline{d}(y,z) = \min\{d(x,y),1\} + \min\{d(y,z),1\} &= \min\{d(x,y)+d(y,z),1+d(x,y),1+d(y,z),2\}\\
	&\geq\min\{d(x,y)+d(y,z),1\}\geq\overline{d}(x,z).
\end{align*}
For both $d$ and $\overline{d}$, note that the collection of open balls $\{O(x,\epsilon),x\in X,\epsilon<1\}$ forms a basis for any metric topology. Then the topologies induced by $d$ and $\overline{d}$ share the same basis.
\end{proof}

Now we introduce the convergence of point sequences in a topological space.

\paragraph{Definition 2.16\label{thm:2.16}} (Convergent sequences). Let $X$ be a topological space, and let $(x_n)$ be a sequence of points of $X$. We say that the sequence $(x_n)$ \textit{converges} to the point $x_0\in X$, if for any neighborhood $U$ of $x_0$, there exists $N$ such that $x_n\in U$ for all $n \geq N$.

Let $A\subset X$. If there is a sequence of points of $A$ that converges to $x\in X$, then by definition $x\in\overline{A}$. Moreover, the converse holds for any metrizable $X$: for any $x\in \overline{A}$, we can construct a sequence $(x_n)$ of points of $A$ by choosing $x_n\in O(x,n^{-1})\cap A$.

\paragraph{Theorem 2.17.\label{thm:2.17}} Let $X$ and $Y$ be topological spaces; let $f:X\to Y$.
\begin{itemize}
	\item[(i)] If $f$ is continuous, then for every convergent sequence $x_n\to x$ in $X$, $\{f(x_n)\}$ converges to $f(x)$.
	\item[(ii)] If $X$ is metrizable and $f(x_n)\to f(x)$ for every convergent sequence $x_n\to x$ in $X$, then $f$ is continuous.
\end{itemize}
\begin{proof}
(i) For any neighborhood $U$ of $f(x)$ in $Y$, $f^{-1}(U)$ is a neighborhood of $x$ in $X$ by continuity of $f$. Then $\exists N$ such that $x_n\in f^{-1}(U)$ for all $n\geq N$, consequently $f(x_n)\in U$.

(ii) Let $A\subset X$, where $X$ is metrizable. Then for any $x\in\overline{A}$, there exists a convergent sequence $x_n\to x$. By our assumption, $f(x_n)\to f(x)\in\overline{f(A)}$. Hence $f(\overline{A})\subset \overline{f(A)}$, and $f$ is continuous by \hyperref[thm:2.5]{Theorem 2.5 (iii)}.
\end{proof}

\paragraph{Definition 2.18\label{def:2.18}} (Uniform convergence). Let $f_n:X\to Y$ be a sequence of functions from a topological space $X$ to a metric space $Y$. We say that the sequence $\{f_n\}$ \textit{converges uniformly} to the function $f:X\to Y$ if for any $\epsilon > 0$, there exists $N$ such that $d_Y(f_n(x),f(x))<\epsilon$ for all $n\geq N$ and all $x\in X$.

\paragraph{Remark.} The condition for uniform convergence can be rewritten as
\begin{align*}
	\sup_{x\in X} d_Y\left(f_n(x),f(x)\right) < \epsilon,\ \forall n\geq N.
\end{align*}
We use the notation $f_n\rightrightarrows f$ to stand for uniform convergence.

\paragraph{Theorem 2.19\label{thm:2.19}} (Uniform limit theorem). Let $f_n:X\to Y$ be a sequence of continuous functions from a topological space $X$ to a metric space $Y$. If $\{f_n\}$ converges uniformly to $f$, then $f$ is continuous.
\begin{proof}
Let $V$ be open in $Y$, and let $x_0\in f^{-1}(V)$. We want to find a neighborhood $U$ of $x_0$ in $X$ such that $U\subset f^{-1}(V)$. We choose $\epsilon > 0$ such that the open ball $O(f(x_0),\epsilon)$ lies in $V$. Then we can use the uniform convergence to choose $N$ such that $d_Y(f_n(x),f(x)) < \epsilon/3$ for all $n\geq N$ and all $x\in X$.

Now we fix $n\geq N$. Since $f_n$ is continuous, we can choose $U=f_n^{-1}(O(f_n(x_0),\epsilon/3))$, which is a neighborhood of $x_0$ in $X$. Then for any $x\in U$, we have
\begin{align*}
	d_Y(f(x),f(x_0)) &\leq d_Y(f(x),f_n(x)) + d_Y(f_n(x),f_n(x_0)) + d_Y(f_n(x_0),f(x_0)) < \epsilon.
\end{align*}
Hence $U\subset O(f(x_0),\epsilon)\subset V$, completing the proof.
\end{proof}

\paragraph{Theorem 2.20\label{thm:2.20}} (Weierstrass M-test). Let $X$ be a topological space and $f_n:X\to\mathbb{R}$ be a sequence of functions. Define
\begin{align*}
	S_n(x) = \sum_{j=1}^n f_j(x).
\end{align*}
\textit{Weierstrass M-test} for uniform convergence: If $\vert f_j(x)\vert \leq M_j$ for all $x\in X$ and all $j\in\mathbb{N}$, and if the series $\sum_{n=1}^\infty M_n$ converges, then the sequence $\{S_n\}$ converges uniformly to a function $S$.
\begin{proof}
Let $r_n=\sum_{j=n+1}^\infty M_j$, which converges to $0$ as $n\to\infty$. Fix $x\in X$. For $n>m$, we have
\begin{align*}
	\left\vert S_n(x) - S_m(x)\right\vert \leq \sum_{j=m+1}^n \left\vert f_j(x)\right\vert \leq \sum_{j=m+1}^n M_j \leq r_m.
\end{align*}
Hence $\{S_n(x)\}$ is a Cauchy sequence, which must converge to some $S(x)\in\mathbb{R}$. Thus we obtain a function $S:X\to\mathbb{R}$ of pointwise convergence. It remains to show the uniform convergence. To show this, let $n\to\infty$ in the equation above:
\begin{align*}
	\left\vert S(x)-S_m(x)\right\vert \leq r_m,\ \forall x\in X.
\end{align*}
Then we conclude the proof.
\end{proof}

\newpage
\section{Connectedness}
\subsection{Connected Spaces}
\paragraph{Definition 3.1\label{def:3.1}} (Connected space). A topological space $X$ is \text{connected} if it cannot be decomposed as the union of two disjoint nonempty open sets. 

By saying a subset $Y$ of $X$ is connected, we mean that $Y$ is \text{connected} in its subspace topology.

\paragraph{Proposition 3.2\label{prop:3.2}} (Alternative definition of connectedness). The following statements are equivalent:
\begin{itemize}
	\item[(i)] $X$ is a connected space.
	\item[(ii)] Any decomposition $X=A\cup B$ of nonempty subsets of $X$ satisfies $\overline{A}\cap B\neq\emptyset$ or $A\cap\overline{B}\neq\emptyset$. 
	\item[(iii)] $X$ cannot be decomposed as the union of two disjoint nonempty closed sets.
	\item[(iv)] The only sets that are both open and closed in $X$ are $\emptyset$ and $X$ itself.
	\item [(v)] There exists no onto continuous function from $X$ to a discrete space that contains more than one points.
\end{itemize}
\begin{proof}
(i) $\Rightarrow$ (ii): Assume there exist nonempty subsets $A$ and $B$ of $X$ such that $X=A\cup B$, $\overline{A}\cap B = \emptyset$ and $A\cap\overline{B}=\emptyset$. Then $X=A\cup B\subset \overline{A}\cup B = X$, hence $B=X\backslash\overline{A}$ is open in $X$. Similarly $A$ is open in $X$. Hence $X$ is the union of two disjoint nonempty open sets $A$ and $B$, a contradiction!

(ii) $\Rightarrow$ (iii): Assume there exist two nonempty closed sets $A$ and $B$ such that $A\cap B = \emptyset,\ A\cup B=X$, then $A=\overline{A}$ and $B=\overline{B}$, which contradicts (ii).


(iii) $\Rightarrow$ (iv):  If there exists a both open and closed subset $A$ in $X$ such that $A\neq X$ and $A\neq\emptyset$, then $X=A\cup(X\backslash A)$ is a decomposition of two disjoint nonempty closed sets.

(iv) $\Rightarrow$ (i): If $X=A\cup B$ is a decomposition of two disjoint nonempty open sets, then $A$ must be a both open and closed in $X$ such that $A\neq\emptyset$ and $A\neq X$.

(i) $\Rightarrow$ (v): Let $Y$ be a discrete space with more than one 
point and $f:X\to Y$ an onto continuous function. Break up $Y$ as a union $U\cup V$ of two disjoint nonempty open sets. Then $X=(f^{-1}U)\cup (f^{-1}V)$.

(v) $\Rightarrow$ (ii): Assume there exist two nonempty sets $A$ and $B$ such that $A\cap B = \emptyset,\ A\cup B=X$, then both $A=X\backslash\overline{B}$ and $B=X\backslash\overline{A}$ are open. Define $f = \mathds{1}_A - \mathds{1}_B:X\to\{-1,1\}$, then $f$ is continuous and onto.
\end{proof}

\paragraph{Theorem 3.3\label{thm:3.3}} (Connectedness of the real line). The real line $\mathbb{R}$ is a connected space.
\begin{proof}
We argue that $\mathbb{R}$ by checking the condition (ii) in \hyperref[prop:3.2]{Proposition 3.2}.
Let $\mathbb{R}=A\cup B$ be a partition of $\mathbb{R}$, i.e. $A$ and $B$ are nonempty, and $A\cap B=\emptyset$. Choose $a\in A, b\in B$, and without loss of generality suppose $a < b$. Then $\{x\in A:x<b\}$ is nonempty. Let $s = \sup\{x\in A: x< b\}$. By the very definition of supremum, we have $s\in\overline{A}$. If $s\notin B$, then $s\in \mathbb{R}\backslash B = A$, and $s<b$. Moreover, $(s,b]$ lies in $B$, then $s$ is a limit point of $B$, showing $s\in\overline{B}$. Therefore $s$ lies either in $\overline{A}\cup B$ or in $A\cup\overline{B}$.
\end{proof}

\paragraph{Theorem 3.4.\label{thm:3.4}} (Connected subsets of the real line). A nonempty subset of $\mathbb{R}$ is connected if and only if it is an interval. (Note that any single point $a\in\mathbb{R}$ is also an interval $[a,a]$.)
\begin{proof}
Akin to the proof of \hyperref[thm:3.3]{Theorem 3.3}, we can show that any interval is connected. If a nonempty set $A$ is not an interval, then we can find $a<p<b$ such that $p\notin A$ and $a,b\in A$. Let $B=\{x\in A:x<p\},$ then both $B$ and $A\backslash B$ are nonempty. Since $p\notin A$, we have $B\in(-\infty,p)$ and $A\backslash B\in (p,\infty)$. Hence $\overline{B}\cup (A\backslash B) = B\cup\overline{(A\backslash B)} = \emptyset$, showing $A$ is not connected.
\end{proof}

\paragraph{Theorem 3.5\label{thm:3.5}} (Connected dense set). Let $X$ be a topological space and let $Y$ be a subspace of $X$. If $Y$ is connected, and if $Y$ is dense in $X$, then $X$ is connected. 
\begin{proof}
Let $A$ be a nonempty subset of $X$ which is both open and closed. Since 
$Y$ is dense in $X$, $Y$ intersects every nonempty open subset of 
$X$, $A\cap Y$ is nonempty. Note that $A\cap Y$ is both open and closed in $Y$, and $Y$ is connected, we have $A\cap Y = Y$, i.e., $Y\subset A$. Therefore $X = \overline{Y}\subset\overline{A} = A\subset X$, meaning $A=X$.
\end{proof}

\paragraph{Corollary 3.6.\label{cor:3.6}} Let $X$ be a topological space and let $Y$ be a connected subspace of $X$. If $Y\subset Z\subset\overline{Y}$, then $Z$ is connected. Particularly, the closure $\overline{Y}$ of a connected subspace $Y$ is connected.
\begin{proof}
By \hyperref[prop:1.26]{Proposition 1.26 (i)}, $Y$ is dense in $Z$. Applying \hyperref[thm:3.5]{Theorem 3.5} yields the wanted result.
\end{proof}

\paragraph{Lemma 3.7.\label{lemma:3.7}} If topological space $X=A\cup B$, where $A$ and $B$ are disjoint open sets, and if $Y$ is a connected subspace of $X$, then $Y$ lies entirely within either $A$ or $B$.
\begin{proof}
We observe that both $A\cap Y$ and $B\cap Y$ are open sets in $Y$, and they forms a partition of $Y$. Since $Y$ is connected, at least one of them should be empty.
\end{proof}

\paragraph{Theorem 3.8\label{thm:3.8}} (Union of connected subspaces). Let $\mathscr{X}=\{X_\alpha,\alpha\in J\}$ be a collection of connected subspaces of $X$ such that $\bigcap_{\alpha\in J}X_\alpha\neq\emptyset$. Then $\bigcup_{\alpha\in J} X_\alpha$ is connected.
\begin{proof}
Let $p\in\bigcap_{\alpha\in J}X_\alpha$, and $Y:=\bigcup_{\alpha\in J}X_\alpha = A\cup B$, where $A$ and $B$ are disjoint open sets in $Y$. Then $p$ is in one of $A$ and $B$. Without loss of generality, let $p\in A$. For each $\alpha\in J$, $X_\alpha\ni p\in A$. Since $X_\alpha$ is connected, by \hyperref[lemma:3.7]{Lemma 3.7}, $X_\alpha\subset A$. Hence $A=Y$ and $B=\emptyset$.
\end{proof}

Now we introduce the concepts of set product and box topology. Let $\mathscr{X}=\{X_\alpha,\alpha\in J\}$ be a collection of indexed sets. The \textit{cartesian product} of this indexed collection, denoted by $\prod_{\alpha\in J}X_\alpha$, is defined to be the set of all $J$-tuples $(x_\alpha)_{\alpha\in J}$ such that $x_\alpha\in X_\alpha$ for each $\alpha\in J$. Equivalently, it is the set of all functions $\mathbf{x}:J\to\bigcup_{a\in J}X_\alpha$ such that $\mathbf{x}(\alpha)\in X_\alpha$ for each $\alpha\in J$. 

\paragraph{Definition 3.9\label{def:3.9}} (Box topology on a product). We take as a basis for a topology on the product $\prod_{\alpha\in J}X_\alpha$ the collection of the sets of the form $\prod_{\alpha\in J}O_\alpha$, with $O_\alpha$ open in $X_\alpha$ for each $\alpha\in J$.  The topology generated by this basis is called the \textit{box topology}.

\paragraph{Remark.} To check the basis we choose is valid, we use \hyperref[thm:1.18]{Theorem 1.18}. The first condition is satisfied because 
$\prod_{\alpha\in J}X_\alpha$ is itself a basis element. The second condition is satisfied because the intersection of any two basis elements is another basis element:
\begin{align*}
	\biggl(\prod_{\alpha\in J}U_\alpha\biggr)\cap\biggl(\prod_{\alpha\in J}Y_\alpha\biggl) = \prod_{\alpha\in J}(U_\alpha\cap V_\alpha).
\end{align*}

\paragraph{Theorem 3.10.\label{thm:3.10}} The cartesian product of finitely many connected spaces is connected.
\begin{proof}
It suffices to show that the cartesian product of two connected spaces is connected. Let $X$ and $Y$ be two connected topological space. For each $x\in\mathcal{X}$, $\{x\}\times Y$, being homeomorphic to $Y$, is connected (we will interpret this in \hyperref[cor:3.15]{Corollary 3.15}). Similarly, $X\times\{y\}$ is connected for each $y\in Y$. By \hyperref[thm:3.8]{Theorem 3.8}, the cross-shaped set $C_{x,y}:=(\{x\}\times Y)\cup(X\times\{y\})$ is connected. Fix $(x_0,y_0)\in X\times Y$. Then $X\times Y = \bigcup_{y\in Y}C_{x_0,y}$, and $(x_0,y_0)\in\bigcap_{y\in Y}C_{x_0,y}$. Again by \hyperref[thm:3.8]{Theorem 3.8}, the product space $X\times Y$ is connected.
\end{proof}

\subsection{Path-connected Spaces}
\paragraph{Definition 3.11\label{def:3.11}} (Path-connected space). A topological space $X$ is \textit{path-connected} if for each $a,b\in X$, there exists a \textit{path} in $X$ from $a$ to $b$, that is, a continuous function $f:[0,1]\to X$ with $f(0)=a$ and $f(1)=b$.

\paragraph{Lemma 3.12.\label{lemma:3.12}} A path-connected space is connected.
\begin{proof}
Let $X$ be a path-connected space. If $X$ is not connected, there exists a partition $X=A\cup B$ such that $A$ and $B$ are disjoint nonempty open sets in $X$. Choose $a\in A,b\in B$, then there exists a continuous function $f:[0,1]\mapsto X$ such that $f(0)=a$ and $f(1)=b$. Then $f^{-1}(A)$ and $f^{-1}(B)$ are nonempty disjoint open sets in $[0,1]$ whose unions are $[0,1]$, contradicting the connectedness of $[0,1]$.
\end{proof}

\paragraph{Remark.} A connected space is not necessarily path-connected. We will give a counterexample afterwards. \hyperref[thm:3.13]{Theorem 3.13} establishes a relation between connected sets and path-connected sets in Euclidean spaces.

\paragraph{Theorem 3.13.\label{thm:3.13}} Any connected open set in a euclidean space is path-connected.
\begin{proof}
Consider euclidean space $\mathbb{R}^n$. Let $X$ be a connected open set in $\mathbb{R}^n$ and fix $x\in X$. Let $U(x)$ be the set of all points in $X$ that can be joined to $x$ by a path in $X$. By construction, $U(x)$ is path-connected. It suffices to show $U(x)=X$. Let $y\in U(x)$ and choose an open ball $O(y,\epsilon)$ that lies entirely in $X$. Then we can join $z$ to $x$ whenever $z\in O(y,\epsilon)$. Hence $O(y,\epsilon)\subset U(x)$, and $U(x)$ is open in $X$. Also, $X\backslash U(x) = \bigcup_{y\in X\backslash U(x)} U(y)$ as the union of a collection of open sets is open, then $U(x)$ is closed. Recall that $X$ is connected, $U(x)=X$.
\end{proof}

\paragraph{Theorem 3.14\label{thm:3.14}} (The continuous image of connected/path-connected sets). Let function $f:X\to Y$ be continuous and onto. (i) If $X$ is connected, so is $Y$; (ii) If $X$ is path-connected, so is $Y$.
\begin{proof}
(i) Let $Y=A\cup B$, where $A$ and $B$ are disjoint open set in $Y$. Then $X=f^{-1}(A)\cup f^{-1}(B)$, with $f^{-1}(A)$ and $f^{-1}(B)$ being disjoint and open in $X$. Since $X$ is connected, one of $f^{-1}(A)$ and $f^{-1}(B)$ is empty. $f$ is onto, hence one of $A$ and $B$ is empty.

(ii) For each $a,b\in Y$, choose $u\in f^{-1}(\{a\})$ and $v\in f^{-1}(\{b\})$, whose nonemptiness is ensured by the surjectivity of $f$. Since $X$ is path-connected, we can find a path $g$ in $X$ from $u$ to $v$. Since the composition preserves continuity, $f\circ g$ is a path in $Y$ from $a$ to $b$.
\end{proof}
\hyperref[thm:3.14]{Theorem 3.14} immediately implies the following conclusion.
\paragraph{Corollary 3.15.\label{cor:3.15}} If $h:X\to Y$ is a homeomorphism, then $X$ is connected (path-connected) if and only if $Y$ is connected (path-connected). In other words, connectedness (path-connectedness) is a topological property.

\subsection{Local Connectedness and Local Path-connectedness}
\paragraph{Definition 3.16\label{def:3.16}} (Locally connected sets and locally path-connected sets). A topological space $X$ is said to be \textit{locally connected at} $x$ if for every neighborhood $U$ of $x$, there is a connected neighborhood $V$ of $x$ contained in $U$. If $X$ is locally connected at each of its points, it is said simply to be \textit{locally connected}. 

Similarly, a topological space $X$ is said to be \textit{locally path-connected at} $x$ if for every neighborhood $U$ of $x$, there is a path-connected neighborhood $V$ of $x$ contained in $U$. If $X$ is locally path-connected at each of its points, then it is said to be \textit{locally path-connected}.

\paragraph{} To characterize local connectedness and local path-connectedness, we need to introduce the concepts of components and path components.

\paragraph{Definition 3.17\label{def:3.17}} (Components and path components). Let $X$ be a topological space, and define an equivalence relation on $X$ by letting $x\sim y$ if there is a connected subspace of $X$ containing both $x$ and $y$. The equivalence classes are called the \textit{(connected) components} of $X$.

Define another equivalence relation on $X$ by letting $x\sim y$ if there is a path in $X$ joining $x$ and $y$. The equivalence classes are called the \textit{path components} of $X$.

\paragraph{Remark.} We need to verify the validity of the equivalence relations we define. For the first statement, the symmetry and reflexivity is clear, and the transitivity follows from \hyperref[thm:3.8]{Theorem 3.8}. 

For the second statement, the symmetry holds because when $f:[0,1]\to X$ is a path from $x$ to $y$ then $g:t\mapsto f(1-t)$ is a path from $y$ to $x$, and the reflexivity follows from the continuity of constant functions. For the transitivity, suppose $f:[0,1]\to X$ is a path from $x$ to $y$, and $g:[0,1]\to X$ a path from $y$ to $z$. Then we can construct a path from $x$ to $z$ by $h:t\mapsto f(2t)\chi_{[0,1/2]}(t) + g(2t - 1)\chi_{(1/2, 1]}(t)$.

\paragraph{Theorem 3.18\label{thm:3.18}} (Component decomposition). The components of $X$ are disjoint connected subspaces of $X$ whose union is $X$, and each nonempty connected subspace of $X$ lies in one of them.  

The path-components of $X$ are disjoint path-connected subspaces of $X$ whose union is $X$, and each nonempty path-connected subspace of $X$ lies in one of them.
\begin{proof}
We first prove the first statement. Being equivalence classes, the components of $X$ are disjoint and their union is $X$. For each connected subspace $A$ of $X$, if there exists $p_1,p_2\in A$ such that $p_1\in C_1$ and $p_2\in C_2$, where both $C_1$ and $C_2$ are components of $X$, then $C_1=C_2$ because $p_1\sim p_2$. Hence $A$ intersects with only one component of $X$, and it must lie entirely in that component.

It remains to show that each component $C$ is connected. To argue this, choose $x_0\in C$, for each $x\in C$, $x\sim x_0$, and there exists a connected subspace containing $x_0$ and $x$. By the result just proved, $A_x\subset C$, and $C=\bigcup_{x\in C}A_x$ is connected by \hyperref[thm:3.8]{Theorem 3.8}.

For the second statements, we make a slight modification on \hyperref[thm:3.8]{Theorem 3.8}: for a collection of path-connected subspaces $\{X_\alpha,\alpha\in J\}$ in $X$, if $\exists p\in\bigcap_{\alpha\in J} X_\alpha$, then we can construct a path between any two points in $\bigcup_{\alpha\in J} X_\alpha$ that meets $p$.
\end{proof}

\paragraph{Remark.} By \hyperref[thm:3.18]{Theorem 3.18}, we can set that the components (path-components) are the collection of maximal connected (path-connected) subspaces of a topological space.

\paragraph{Theorem 3.19.\label{thm:3.19}} A topological space $X$ is locally connected if and only if for every open set $U$ of $X$, each component of $U$ is open in $X$. Similarly,  $X$ is locally path-connected if and only if for every open set $U$ of $X$, each path component of $U$ is open in $X$.

\begin{proof}
We only prove the first statement, since the proof of the second is parallel. Suppose $X$ is locally connected and $U$ is an open set in $X$. If $C$ is a component of $U$ and $x\in C$, then we can choose some $V\subset U$ such that $V$ is a connected neighborhood of $x$. By \hyperref[thm:3.18]{Theorem 3.18}, $V\subset C$, and $C$ is open in $X$.

Conversely, given $x\in X$ and a neighborhood $U$ of $x$ (without loss of generality suppose it is open), let $C$ be the component of $U$ that contains $x$. Then $C$ is a connected neighborhood of $x$ if $C$ is open. Since $C\subset U$, $X$ is locally connected at $x$.
\end{proof}

\paragraph{Theorem 3.20.\label{thm:3.20}} Let $X$ be a topological space. Then every path component of $X$ lies in a component of $X$. If $X$ is locally path-connected, then the components and the path components of $X$ are the same.
\begin{proof}
Let $C$ be a component of $X$ and $x\in C$. Suppose $P$ is the path component containing $x$. Since $P$ is connected, $P\subset C$. It remains to show $P=C$ if $X$ is locally path-connected.

Assume $P\subsetneq C$. Denote by $Q$ the union of all the path components of $X$ that is different from $P$ and meets $C$. Then $C=P\cup Q$. By \hyperref[thm:3.19]{Theorem 3.19}, each path component of $X$ is open in $X$. Then $P$ and $Q$ are disjoint nonempty open sets whose union is $C$, contradicting the connectedness of $C$.
\end{proof}

The following statements immediately follows from \hyperref[thm:3.20]{Theorem 3.20}.
\paragraph{Corollary 3.21.\label{cor:3.21}} If a topological space $X$ is connected and locally path-connected, then $X$ is path-connected.

\paragraph{Remark.} As an end of this section, let's see an example of connected space that is not path-connected. Consider the following closed set in euclidean space $\mathbb{R}^2$:
\begin{equation*}
	A=\overline{\left\{(x,y)\in\mathbb{R}^2:y=\sin\frac{1}{x}, x>0\right\}}
\end{equation*} 
The line segment $L=\{(0,y):-1\leq y\leq 1\}$ lies in $A$.

Consider the function $f:\mathbb{R}\to\mathbb{R}^2, x\mapsto(x,\sin \frac{1}{x})$, which is continuous. Then $f((0,\infty))$ as the image of a connected set $(0,\infty)$ is connected, and $A=\overline{f((0,\infty))}$ as the closure is also connected.

Let $f:[0,1]\to X$ be a path starting at a point in $L$. Then $f^{-1}(L)$ is closed since $f$ is continuous. If we can show $f^{-1}(L)$ is open, then $f^{-1}(L)=[0,1]$ because $[0,1]$ is connected and $f^{-1}(L)$ is nonempty. Hence $f([0,1])\in L$, and there is no path joining a point in $A$ to a point in $B$.

Fix $t\in f^{-1}(L)$, and choose an open ball $U=O(f(t),\epsilon)$ in $\mathbb{R}^2$. Then $U\cap A$ has infinitely many path components including $U\cap L$. Since $f$ is continuous, $f^{-1}(U)\ni t$ is an open set in $[0,1]$. Then there exists an interval $I\subset f^{-1}(U)$ such that $I$ is open in $[0,1]$ and $I\ni t$. Note that $I$ is path-connected, then $f(I)$, being path-connected, lies in $U\cap L$ by \hyperref[thm:3.18]{Theorem 3.18}. Therefore $I\subset f^{-1}(L)$, and $t$ is an interior point of $f^{-1}(L)$. Since $t$ is arbitrarily chosen, $f^{-1}(L)$ is an open set in $[0,1]$, which concludes our proof.

\begin{figure}[H]
\centering
\includegraphics[width=\linewidth]{sinq.jpg}
\end{figure}

\newpage
\section{Compactness}
\subsection{Compact Sets}
\paragraph{Definition 4.1\label{def:4.1}} (Cover). A collection $\mathscr{A} = \{A_\alpha,\alpha\in J\}$ of subsets of a topological space $X$ is said to be a \textit{cover of} $X$ (or briefly, a \textit{cover}), if $X=\bigcup_{\alpha\in J}A_\alpha$. It is called an \textit{open cover} of $X$ if its elements are open sets in $X$. A subcollection of a cover $\mathscr{A}$ whose union is equal to $X$ is called a \textit{subcover}.

\paragraph{Definition 4.2\label{def:4.2}} (Compact sets). Let $X$ be a topological space. $X$ is said to be \textit{compact}, if every open cover of $X$ has a finite subcover.

\paragraph{Remark.} If $Y$ is a subspace of $X$, a collection $\mathscr{A}=\{A_\alpha,\alpha\in J\}$ of (open) subsets of $X$ is said to be a cover (an open cover) of $Y$ if the union of its elements contains $Y$. By definition, we can verify that a subspace $Y$ of $X$ is compact if and only if every open cover of $Y$ contains a finite subcover of $Y$.

\paragraph{Lemma 4.3.\label{lemma:4.3}} Every closed subspace of a compact space is compact.
\begin{proof}
Let $X$ be a compact space and let $K$ be a closed subspace of $X$. Then for every open cover $\mathscr{A}$ of $K$, then $\mathscr{B}=\mathscr{A}\cup\{X\backslash K\}$ forms an open cover of $X$. By the very definition of compactness, $\mathscr{B}$ contains a finite subcover $\mathscr{B}^\prime$ of $X$. Since $X\backslash K$ does not intersect $K$, we can remove $X\backslash K$ from $\mathscr{B}^\prime$ if required, resulting in a finite subcover $\mathscr{A}^\prime\subset\mathscr{A}$ of $K$.
\end{proof}

\paragraph{Theorem 4.4\label{thm:4.4}} (Continuity and compactness). The continuous image of a compact space is compact.
\begin{proof}
	Let $X$ be a compact space and let $f:X\to Y$ be a continuous function. Let $\mathscr{A}$ be an open cover of $f(X)$ in $Y$. By continuity, $\{f^{-1}(A):A\in\mathscr{A}\}$ is an open cover of $X$, from which we can find finite many $f^{-1}(A_1),\cdots,f^{-1}(A_n)$ that cover $X$. Then $A_1,\cdots,A_n\in\mathscr{A}$ form a finite subcover of $f(X)$.
\end{proof}

Now we investigate the compactness of products of compact sets.

\paragraph{Lemma 4.5\label{lemma:4.5}} (Tube lemma). Let $X$ and $Y$ be two topological spaces, and let $Y$ be compact. For every $x\in X$, if an open set $O$ in $X\times Y$ contains $\{x\}\times Y$, then there exists a neighborhood $U_x$ of $x$ such that $U_x\times Y\subset O$.
\begin{proof}
Fix $x\in X$, and let $O$ be an open set in $X\times Y$ containing slice $\{x\}\times Y$. By the property of box topology, we can cover $\{x\}\times Y$ by an collection of basis elements in form of $U\times V$ lying in $O$. Note that the space $\{x\}\times Y$ is compact (because $Y$ is compact, and $\{x\}$ itself is open in the subspace topology), we can cover it with finitely many basis sets $U_1\times V_1,\cdots,U_n\times V_n$. 

Let $U_x=\bigcap_{j=1}^n U_j$, then $U_x$ is an open neighborhood of $x$ in $X$. Then for each $(x^\prime,y^\prime)\in U_x\times Y$, $y^\prime$ must lie in some $V_j$, and $x^\prime$ lies in $U_x\subset U_j$. Hence $(x^\prime,y^\prime)\in O$, as desired.
\end{proof}

\paragraph{Theorem 4.6\label{thm:4.6}} (Product of compact spaces). The product of finitely many compact spaces is compact.
\begin{proof}
It suffices to show the product of two compact spaces $X$ and $Y$ is compact. Let $\mathscr{A}$ be an open cover of $X\times Y$. Then for each $x\in X$, the slice $\{x\}\times Y$ is compact, and there exist finitely many $A_1^x,\cdots,A_{n_x}^x\in\mathscr{A}$ with $\bigcup_{j=1}^{n_x}A_{n_x}^x\supset \{x\}\times Y$. By \hyperref[lemma:4.5]{Lemma 4.5}, there exists an open neighborhood $U_x$ of $x$ such that $U_x\times Y$ is covered by finitely many elements of $\mathscr{A}$. Noticing that $\{U_x,x\in X\}$ is an open cover of compact space $X$, there exists finitely many $U_x\times Y$ that covers $X\times Y$, with each $U_x\times Y$ covered by finitely many elements of $\mathscr{A}$. Then $X\times Y$ is covered by finitely many elements of $\mathscr{A}$.
\end{proof}

\paragraph{Lemma 4.7\label{lemma:4.7}} (Projection). For two topological spaces $X$ and $Y$, define $\pi_1:X\times Y\to X,(x,y)\mapsto x$.
\begin{itemize}
	\item[(i)] $\pi_1$ is an open map, that is, it carries open sets to open sets.
	\item[(ii)] If $Y$ is compact, then $\pi_1$ is a closed map, that is, it carries closed sets to closed sets.
\end{itemize}
\begin{proof}
(i) Let $O$ be an open set in $X\times Y$. Then for any $x\in \pi_1(O)$, we can find some $(x,y)\in O$. By the property of box topology, we can find a basis set $U\times V$ such that $(x,y)\in U\times V\subset O$, where $U$ and $V$ are open sets in $X$ and $Y$, respectively. As a result, $x\in U \subset \pi_1(O)$.

(ii) Let $C$ be a closed sets in $X\times Y$, where $Y$ is compact. We are about to show $X\backslash\pi_1(C)$ is open. Take $x\notin\pi_1(C)$. The slice $\{x\}\times Y$ is disjoint from $C$. Since $Y$ is compact, by \hyperref[lemma:4.5]{Lemma 4.5}, there exists a neighborhood $U_x\ni x$ such that $U_x\times Y\subset( X\times Y)\backslash C$. Therefore $U_x$ is a neighborhood of $x$ which is disjoint from $\pi_1(C)$, completing the proof.
\end{proof}

Now we investigate the compact sets in euclidean spaces.

\paragraph{Theorem 4.8.\label{thm:4.8}} A closed interval $[a,b]$ is compact.

\begin{proof}
The case $a=b$ is trivial, so we may assume $a<b$. 

\textit{Step I:} Let $\mathscr{A}$ be an open cover of $[a,b]$. We first prove that if $x\in[a,b]\backslash\{b\}$, then $\exists y>x$ of $[a,b]$ such that $[x,y]$ can be covered by at finitely many elements of $\mathscr{A}$. Choose $A\in\mathscr{A}$ such that $A\ni x$. Since $x\neq b$ and $A$ is open, $A$ contains an interval of the form $[a,c)$ for some $c\in[a,b]$. Choose $y\in(x,c)$, then $[x,y]$ is covered by a single element of $\mathscr{A}$.

\textit{Step II:} Let $C$ be the set of all points $y>a$ of $[a,b]$ such that $[a,y]$ can be covered by finitely many elements of $\mathscr{A}$. By our conclusion in Step I, $C$ is nonempty. Let $c$ be the \textbf{least upper bound} of $C$, then $a<c\leq b$. We show that $c\in C$. Choose $B\in\mathscr{A}$ such that $B\ni c$. $B$ is open, so it contains an interval of the form $(d,c]$ for some $d\in[a,b]$. If $c\notin C$, then there exists $z\in C$ lying in $(d,c)$, otherwise $d$ would be an upper bound of $C$ smaller than $c$. Since $z\in C$, $[a,z]$ is able to be covered by finitely many elements of $\mathscr{A}$, so is $[a,z]\cup[z,c]$, contradicting $c\notin C$!

\textit{Step III:} It remains to show $c=b$, which completes our proof. Assume $c<b$, then applying Step I can we find some $y>c$ in $[a,b]$ such that $[c,y]$ is covered by finitely many elements of $\mathscr{A}$. So is $[a,y]=[a,c]\cup[c,y]$. However this means $C\ni y > c$, another contradiction!
\end{proof}

\paragraph{Theorem 4.9\label{thm:4.9}} (Heine-Borel). A subspace of an euclidean space $\mathbb{R}^n$ is compact if and only if it is closed and bounded.
\begin{proof}
\textit{``If'' part:} Provided $K$ is bounded in $\mathbb{R}^n$, we can find a cell $[-b,b]^n\supset K$, which is compact by Theorems \hyperref[thm:4.6]{4.6} and \hyperref[thm:4.8]{4.8}. By \hyperref[lemma:4.3]{Lemma 4.3}, if $K$ is closed, then it is compact.

\textit{``Only if'' part:} Suppose $K$ is compact. The collection of centered open balls $\{O(0,n),n=1,2,\cdots\}$ is a cover of $\mathbb{R}^n$, so there exist finitely many balls that cover $K$. Then $K$ must be contained in an open ball of finite radius, and it suffices to show that $K$ is closed. 

Argue by contraction. Assume $x$ is a limit point of $K$ that is not contained in $K$. Then we can construct an open cover $\bigl\{\mathbb{R}\backslash\overline{O(x,n^{-1})},n=1,2,\cdots\bigr\}$ of $K$. Since $x$ is a limit point, any $O(x,n^{-1})$ contains at least one point in $K$, and we cannot find a finite subcover of $K$ from our construction.
\end{proof}

\paragraph{Theorem 4.10\label{thm:4.10}} (Extreme value theorem). Let $X\to\mathbb{R}$ be a continuous function. If $X$ is compact, then there exists points $a,b\in X$ such that $f(a)\leq f(x)\leq f(b)$ for every $x\in X$.
\begin{proof}
By the continuity of $f$, the image $f(X)$ is compact in $\mathbb{R}$. Then it suffices to show that $f(X)$ has a largest element $M$ and a smallest element $m$. Argue by contradiction. Suppose $f(X)$ has no largest element. Then $\{(-\infty,a):a\in f(X)\}$ is an open cover of $f(X)$ because for every $x\in X$ there exists $a>x$ in $X$. However, every finite subcollection $\{(-\infty,a_j):j=1,\cdots,n\}$ does not cover $X$ because there exists $b > \max_{j=1,\cdots,n} a_j$ in $X$, contradicting the compactness of $f(X)$. Similarly we can prove that $f(X)$ has a smallest element.
\end{proof}

Now we introduce the concept of uniform continuity.
\paragraph{Definition 4.11\label{def:4.11}} (Uniform continuity). Let $(X,d_X)$ and $(Y,d_Y)$ be two metric spaces. A function $f:X\to Y$ is said to be \textit{uniformly continuous}, if given any $\epsilon > 0$, $\exists\delta > 0$ such that for every $x,x^\prime\in X$, $d_X(x,x^\prime)<\delta$ implies $d_Y(f(x),f(x^\prime)) < \epsilon$.

\paragraph{} By definition, a uniformly continuous function must be continuous, but the converse is not true: in the definition of uniformly continuity, the choice of $\delta$ only depends on $\epsilon$ but not on location $x$. The following \hyperref[thm:4.13]{Theorem 4.13} tells us in what case does a continuous function become uniformly continuous. We first introduce a technical lemma. Recall that for a bounded subset $B$ of a metric space $(X,d)$, we denote by $D_B=\sup_{x,x^\prime\in B} d(x,x^\prime)$ the diameter of $B$.

\paragraph{Lemma 4.12\label{lemma:4.12}} (Lebesgue number lemma). Let $\mathscr{A}$ be an open cover of a compact metric space $(X,d)$. Then there exists a $\delta > 0$ such that for each subset of $X$ having diameter less than $\delta$, there exists an element of $\mathscr{A}$ that contains it. The number $\delta$ is called a \textit{Lebesgue number} for the cover $\mathscr{A}$.
\begin{proof}
If $X\in\mathscr{A}$, then any positive number is a Lebesgue number of $\mathscr{A}$. So we assume $X\notin\mathscr{A}$. By compactness of $X$, there exist finite many $A_1,\cdots,A_n\in\mathscr{A}$ whose union contains $X$, and we set $C_j = X\backslash A_j$ for $j=1,\cdots,n$. Define $f:X\to\mathbb{R},x\mapsto\sum_{j=1}^n d(x,C_j)$, which is a continuous function by \hyperref[lemma:2.8]{Lemma 2.8}. By \hyperref[thm:4.10]{Theorem 4.10}, there exists $x_0\in X$ such that $f(x)\geq f(x_0):=\delta$ for all $x\in X$. Since $A_1,\cdots,A_n$ is an open cover of $X$, there exists $\epsilon >0$ such that the open ball $O(x_0,\epsilon)$ lies in some $A_j$. Then $d(x_0,C_j)\geq\epsilon$, and $\delta=f(x)\geq \epsilon/n > 0$.

Now we prove $\delta$ is a Lebesgue number of $\mathscr{A}$. Let $B$ be a subset of $X$ of diameter less than $\delta$. Choose any $b\in B$, then $O(b,\delta)\supset B$. Let $m\in\mathrm{argmax}_{j\in\{1,\cdots,n\}}d(b,C_j)$, then $\delta\leq f(b)\leq d(b,C_m)$. Consequently, $B\subset O(b,\delta)\subset X\backslash C_m = A_m$, completing the proof.
\end{proof}

\paragraph{Theorem 4.13\label{thm:4.13}} (Uniform continuity theorem). Let $f:X\to Y$ be a continuous function on a compact metric space $(X,d_X)$ to a metric space $(Y,d_Y)$. Then $f$ is uniformly continuous.
\begin{proof}
By the continuity of $f$, the image $f(X)$ is compact in $Y$. Fix $\epsilon > 0$, then the collection of open balls $\{O_Y(y,\epsilon/2):y\in f(X)\}$ covers $f(X)$, and there exist finite many open balls $O_Y(y_1,\epsilon/2),\cdots,O_Y(y_n,\epsilon/2)$ that cover $f(X)$. Take $\delta$ to be a Lebesgue number of $\{U_j:=f^{-1}O_Y(y_j,\epsilon/2),j=1,\cdots,n\}$, which is an open cover of $X$. Then for any $d_X(x,x^\prime) < \delta$, $\{x,x^\prime\}$ as a point set of diameter less than $\delta$ must lie in some $U_j$, and $d_Y(f(x),f(x^\prime))\leq d_Y(f(x),y_j) + d_Y(y_j,f(x^\prime)) < \epsilon$, completing the proof.
\end{proof}

\paragraph{Theorem 4.14\label{thm:4.14}} (Closed set criterion for compactness). Let $X$ be a topological space. Then $X$ is compact if and only if for every collection $\mathscr{C}$ of closed sets in $X$ having the \textit{finite intersection property}, that is, for every finite subcollection $\{C_1,\cdots,C_n\}$ of $\mathscr{C}$, their intersection $\bigcap_{i=1}^n C_i$ is nonempty, the intersection $\bigcap_{C\in\mathscr{C}}C$ of all elements of $\mathscr{C}$ is nonempty.

\begin{proof}
Given a collection $\mathscr{A}$ of subsets of $X$, let $\mathscr{C}=\{X\backslash A:A\in\mathscr{A}\}$ be the collection of their complements. Then the following statements hold:
\begin{itemize}
	\item[(i)] $\mathscr{A}$ is a collection of open sets in $X$ if and only if $\mathscr{C}$ is a collection of closed sets.
	\item[(ii)] $\mathscr{A}$ covers $X$ if and only if $\bigcap_{C\in\mathscr{C}} C$ is empty.
	\item[(iii)] A finite subcollection $\{A_1,\cdots,A_n\}\subset\mathscr{A}$ covers $X$ if and only if the intersection of the finite subcollection $\{C_j=X\backslash A_j:j=1,\cdots,n\}\subset\mathscr{C}$ is empty.
\end{itemize}
Then we can derive three equivalent characterizations of compactness:
\begin{itemize}
	\item Given any collection $\mathscr{A}$ of open sets in $X$, if $\mathscr{A}$ covers $X$, then there exists a finite subcollection of $\mathscr{A}$ that covers $X$.
	\item Given any collection $\mathscr{A}$ of open sets in $X$, if no finite subcollection of $\mathscr{A}$ covers $X$, then $\mathscr{A}$ does not cover $X$.
	\item Given any collection $\mathscr{C}$ of closed sets in $X$, if every finite subcollection of $\mathscr{C}$ has nonempty intersection, then $\bigcap_{C\in\mathscr{C}}C$ is nonempty.
\end{itemize}
The last statement is the condition of our theorem.
\end{proof}

The following corollary immediately follows from \hyperref[thm:4.14]{Theorem 4.14}.
\paragraph{Corollary 4.15\label{cor:4.15}} For a \textit{nested sequence} $C_1\supset C_2 \supset \cdots \supset C_n \supset C_{n+1} \supset\cdots$ of nonempty closed sets in a compact space $X$, the intersection $\bigcap_{n=1}^\infty C_n$ is nonempty.

\subsection{Hausdorff Spaces}
\paragraph{Motivation.} One’s experience with open and closed sets and limit points in euclidean spaces can be misleading when considering general topological space. For example, in an euclidean space, every single point set $\{x_0\}$ is closed because for every $x\neq x_0$ we can find one of its neighborhood $O(x,\epsilon)$ not containing $x_0$ when $\epsilon$ is sufficiently small. However, this property does not hold for arbitrary topological spaces.

We can also consider the properties of convergent sequences. In a topological space $X$, a sequence $\{x_n\}_{n=1}^\infty$ of points is said to \textit{converge} to a point $x_0\in X$ if for every neighborhood $N$ of $x_0$, there exists a positive integer $N$ such that $x_n\in N$ for all $n\geq N$. It is clear that $x_0$ is a limit point of any set that contains $\{x_n\}_{n=1}^\infty$. In euclidean spaces a convergent sequence never converges to more than one point.

On a three-point set $\{a,b,c\}$, consider the topology $\tau = \{\emptyset,\{b\},\{a,b\},\{b,c\},\{a,b,c\}\}$. The one-point set $\{b\}$ is not closed, because its complement $\{a,c\}$ is not open. Also, the sequence defined by $\{x_n=b,n=1,2,\cdots\}$ converges not only to $b$, but also to points $a$ and $c$ since their neighborhoods always contain $b$.

In this section we consider a special class of topological spaces, which enjoys some nice properties.

\paragraph{Definition 4.16\label{def:4.16}} (Hausdorff spaces/$T_2$ spaces). A topological space $X$ is called a \textit{Hausdorff} space if for each pair of distinct points $x,y\in X$, there exists a neighborhood $U$ of $x$ and a neighborhood $V$ of $y$ such that $U$ and $V$ are disjoint.

\paragraph{Proposition 4.17\label{prop:4.17}} (Properties of Hausdorff spaces). Suppose $X$ is a Hausdorff space.
\begin{itemize}
	\item[(i)] Every finite point set in $X$ is closed;
	\item[(ii)] A sequence of points of $X$ converges to at most one point of $X$;
	\item[(iii)] The product of Hausdorff spaces $\{X_\alpha\}_{\alpha\in J}$ is Hausdorff;
	\item[(iv)] Any subspace of $X$ is a Hausdorff space.
\end{itemize}
\begin{proof}
(i) Fix $x_0\in X$. For any $x\neq x_0$ in $X$, we can find two disjoint neighborhoods $U$ and $V$ of $x_0$ and $x$, respectively. Since $x\notin U$, $x\notin\overline{\{x_0\}}$. Consequently, $\{x_0\}=\overline{\{x_0\}}$.

(ii) Let $(x_n)_{n=1}^\infty$ be a sequence of points of $X$ that converges to $x\in X$. Then for any $x^\prime\in X$ distinct from $x$, let $U\ni x$ and $V\ni x^\prime$ be their disjoint neighborhoods. Then there exists infinite many elements of $\{x_n\}_{n\in\bbN}$ that do not lies in $V$.

(iii) For any pair of distinct points $\mathbf{x}=(x_\alpha)_{\alpha\in J}$ and $\mathbf{y}=(y_\alpha)_{\alpha\in J}$ in $\prod_{\alpha\in J}X_\alpha$, There exists $\beta\in J$ with $x_\beta\neq y_\beta$. Then there exist disjoint neighborhoods $U\ni x_\beta$ and $V\ni y_\beta$ in $X_\beta$. As a result, $\pi_\beta^{-1}(U)$ and $\pi_\beta^{-1}(V)$ are disjoint neighborhoods of $\mathbf{x}$ and $\mathbf{y}$, respectively, in $\prod_{\alpha\in J}X_\alpha$.

(iv) Let $A$ be a subspace of $X$. For each pair of distinct points $x_1,x_2\in A\subset X$, there exist distinct neighborhoods $U\ni x_1$ and $V\ni x_2$ in $X$. Then $U\cap A$ and $V\cap A$ are disjoint neighborhoods of $x_1$ and $x_2$, respectively, in the subspace topology.
\end{proof}

Now let's investigate the compact sets in Hausdorff spaces.

\paragraph{Lemma 4.18.\label{lemma:4.18}} If $K$ is a compact subspace of a Hausdorff space $X$, and $x_0\in X$ is not in $K$. Then there exists disjoint open sets $U$ and $V$ in $X$ such that $U\ni x_0$ and $V\supset K$.
\begin{proof}
For each point $y\in K$, we are able to choose two disjoint open neighborhoods $U_y\ni x_0$ and $V_y\ni y$. The collection $\{V_y:y\in K\}$ is an open cover of $K$, then there exist finitely many $y_1,\cdots,y_n\in K$ such that the $V:=\bigcup_{j=1}^n V_{y_j}\supset K$. As a result, $U:=\bigcap_{j=1}^nU_{y_j}$ is an open neighborhood of $x$ that does not intersect $K$.
\end{proof}

\paragraph{Theorem 4.19\label{thm:4.19}} (Compact sets in Hausdorff spaces). Every compact subspace of a Hausdorff space is closed.
\begin{proof}
Let $K$ be a compact subspace of a Hausdorff space $X$. \hyperref[lemma:4.18]{Lemma 4.18} tells us $X\backslash K$ is an open set, because every $x_0\in X\backslash K$ lies in the interior of $X\backslash K$. Thus we complete the proof.
\end{proof}

\paragraph{} One important use of \hyperref[thm:4.19]{Theorem 4.19} is as a tool for verifying that a function is a homeomorphism.

\paragraph{Theorem 4.20.\label{thm:4.20}} Let $f:X\to Y$ be a bijective continuous function. If $X$ is compact and $Y$ is Hausdorff, then $f$ is a homeomorphism.
\begin{proof}
We show that images of closed sets of $X$ under $f$ are closed in $Y$, which implies the continuity of $f^{-1}$. This is clear: $K$ is closed in $X$ $\Rightarrow$ $K$ is compact $\Rightarrow$ $f(K)$ is compact $\Rightarrow$ $f(K)$ is closed in $Y$.
\end{proof}

\paragraph{Theorem 4.21\label{thm:4.21}} (Closed graph). Let $f:X\to Y$, and define the \textit{graph} of $f$ as $G_f=\{(x,f(x)):x\in X\}$.
\begin{itemize}
	\item[(i)] If $G_f$ is closed and $Y$ is compact, then $f$ is continuous.
	\item[(ii)] If $f$ is continuous and $Y$ is Hausdorff, then $G_f$ is closed.
\end{itemize}

\begin{proof}
(i) Let $O$ be an open set in $Y$, we need to show $f^{-1}(O)$ is open in $X$. The intersection $G_f\cap X\times(Y\backslash O)$ is closed in $X\times Y$. By \hyperref[lemma:4.7]{Lemma 4.7} (ii), the compactness of $Y$ implies that $\pi_1(G_f\cap X\times(Y\backslash O)) = f^{-1}(Y\backslash O)$ is closed in $X$. Then $f^{-1}(O)$ is open in $X$.

(ii) For any $(x,y)\in(X\times Y)\backslash G_f$, we have $y\neq f(x)$. Since $Y$ is Hausdorff, we can find two disjoint open sets $U\ni y$ and $V\ni f(x)$. Then $(x,y)\in f^{-1}(V)\times U$. Moreover, for any point $(z,f(z))\in G_f$, if it lies in $f^{-1}(V)\times U$, then $z\in f^{-1}(V)$, however $f(z)$ lies in $V$ which is disjoint from $U$, a contradiction! Hence $f^{-1}(V)\times U$ is a neighborhood of $(x,y)$ which is disjoint from $G_f$.
\end{proof}

Now we introduce the definition of isolated points in topological spaces.
\paragraph{Definition 4.22\label{def:4.22}} (Isolated points). Let $X$ be a topological space. An \textit{isolated point} of $X$ is a point $x$ of $X$ such that the one-point set $\{x\}$ is open in $X$.

\paragraph{Theorem 4.23.\label{thm:4.23}} Let $X$ be a nonempty compact Hausdorff space. If $X$ has no isolated point, then $X$ is uncountable.
\begin{proof}
\textit{Step I:} We first show that for any nonempty open set $U$ in $X$ and any $x\in X$, we can find a nonempty open set $V$ contained in $U$ such that $x\notin \overline{V}$. By our assumption that $X$ has no isolated points, we can always find some $y\in U$ such that $y\neq x$. Since $X$ is Hausdorff, we can find two disjoint open sets $W_y\ni y$ and $W_x\ni x$. Letting $V=W_y\cap U$ yields the desired result.

\textit{Step II:} It suffices to show that any function $f:\mathbb{N}\to X$ is not surjective. Let $x_n=f(n),n=1,2,\cdots.$ For $x_1\in X$, we can find a nonempty open set $V_1$ such that $x_1\notin\overline{V_1}$. Then we can iteratively find $V_{n+1}\subset V_n$ such that $x_{n+1}\notin\overline{V_{n+1}}$ for each $n\in\mathbb{N}$. Then we obtain a nested sequence $\overline{V_1}\supset\overline{V_2}\supset\cdots$ of nonempty closed sets in $X$. By \hyperref[cor:4.15]{Corollary 4.15}, $\bigcap_{n=1}^\infty \overline{V_n}$ is nonempty, that is, there exists $x\in\bigcap_{n=1}^\infty \overline{V_n}\subset X$ such that $x\notin\{x_n\}_{n=1}^\infty$, which concludes the proof.
\end{proof}

The uncountability of real numbers immediately follows from \hyperref[thm:4.23]{Theorem 4.23}.
\paragraph{Corollary 4.24.\label{cor:4.24}} Every closed interval in $\mathbb{R}$ is uncountable.

\paragraph{} As supplementary, let's discuss another class of spaces called $T_1$ spaces. They are weaker than Hausdorff spaces and less commonly used. The proof of \hyperref[lemma:4.26]{Lemma 4.26} can be adapted from \hyperref[prop:4.17]{Proposition 4.17 (i)}.
\paragraph{Definition 4.25\label{def:4.25}} ($T_1$ spaces). A topological space $X$ is called a $T_1$ \textit{space} if for each pair of distinct points $x,y\in X$ there exists a neighborhood $U$ of $x$ such that $y\notin U$, and a neighborhood $V$ of $y$ such that $x\notin V$.
\paragraph{Lemma 4.26.\label{lemma:4.26}} Let $X$ be a $T_1$ space. Then every finite point set in $X$ is closed.
\paragraph{Theorem 4.27.\label{thm:4.27}} Let $X$ be a $T_1$ space; let $A$ be a subset of $X$. Then a point $x\in X$ is a limit point of $A$ if and only if every neighborhood of $x$ contains infinitely many points of $A$.
\begin{proof}
The sufficiency is clear, so we need to prove the necessity. We let $x$ be a limit point of $A$, and choose an arbitrary neighborhood $N$ of $x$ in $X$. If $N$ contains only finitely many points $a_1,\cdots,a_n$ of $A\backslash\{x\}$, then $U:=N\cap (X\backslash\{a_1,\cdots,a_n\})$ is also a neighborhood of $x$, since $\{a_1,\cdots,a_n\}$ is closed by \hyperref[lemma:4.26]{Lemma 4.26}. However, $U$ as a neighborhood of the limit point $x$ should contains at least one point of $A\backslash\{x\}$, a contradiction!
\end{proof}

\paragraph{Remark.} By definition, a Hausdorff space must be a $T_1$ space. but not conversely. As a counterexample, consider the finite complement topology on $\mathbb{N}$: $\mathscr{T}=\{U:U\subset\mathbb{N}\ \text{and}\ \mathbb{N}\backslash U\ \text{is finite}\}\cup\{\emptyset\}$. This is a $T_1$ space, because for any distinct $m,n\in\mathbb{N}$, we can choose neighborhoods $N\backslash\{m\}$ and $N\backslash\{n\}$ that separates $m$ and $n$. However, it is not Hausdorff because any two nonempty open sets are not disjoint!

\subsection{Limit Point Compact, Countably Compact and Sequentially Compact Sets}
In this section we introduce other formulations of compactness that are commonly used.

\paragraph{Definition 4.28\label{def:4.28}} (Limit point compactness/Fréchet compactness/Bolzano-Weierstrass property). Let $X$ be a topological space. Then $X$ is said to be \textit{limit point compact} if every infinite subset of $X$ has a limit point.

\paragraph{Theorem 4.29\label{thm:4.29}} (Compactness implies limit point compactness). A compact space $X$ is limit point compact.
\begin{proof}
Let $X$ be a compact space. We prove the contrapositive: if a subset $A$ of $X$ has no limit point, then $A$ is finite. Suppose $A$ has no limit point. Then $A$ is closed because it contains all its limit points (which is the empty set). Since $X$ is compact, $A$ is also compact. Furthermore, for each $a\in A$ we can choose an open neighborhood $U_a$ of $a$ such that $U_a\cap A=\{a\}$. Then space $A$ is covered by open sets $\{U_a,a\in A\}$, and we can find a finite subcollection $\{U_{a_1},\cdots,U_{a_n}\}$ that contains $A$. Hence $A=\{a_1,\cdots,a_n\}$.
\end{proof}

\paragraph{Remark.} Conversely, limit point compactness does not necessarily imply compactness. Consider $A=\{a_1,a_2\}$ with a topology $\{A,\emptyset\}$. Given $\mathbb{N}$ the discrete topology, the space $X=\mathbb{N}\times A$ is limit point compact, because every nonempty subset of $X$ has a limit point. To see this, suppose $(a_1,n)$ lies in a set $U\subset X$, then $(a_2,n)$ must be a limit point of $U$ because any neighborhood of $(a_2,n)$ contains $(a_1,n)$. However $X$ is not compact, since the open cover $\{\{n\}\times A,n\in\mathbb{N}\}$ has no finite subcover.

\paragraph{Definition 4.30\label{def:4.30}} (Countable compactness). A space $X$ is said to be \textit{countably compact} if every countable open cover of $X$ contains a finite subcover of $X$.

\paragraph{Theorem 4.31.\label{thm:4.31}} Let $X$ be a topological space. (i) If $X$ is countably compact, then it is limit point compact;
(ii) If $X$ is a limit point compact $T_1$ space, then it is countably compact.
\begin{proof}
(i) Let $A$ be an infinite subset of $X$ that has no limit point. We can assume $A$ to be countable because if $A$ has no limit point, so does its countable subsets. Since $A$ has no limit points, $A$ is closed. Moreover, for each $a\in A$ we can find an open set $U_a$ in $X$ with $U_a\cap A=\{a\}$. Then $X\backslash A$ and $\{U_a: a\in A\}$ form a countable open cover of $X$ that has no finite subcover.

(ii) Argue by contradiction. Let $\{A_n\}_{n=1}^\infty$ be an countable open cover of $X$. If there exists no finite subcover, then we choose $x_n\in X\backslash\bigcup_{j=1}^n A_n$ for each $n$. Since $X$ is limit point compact, $B:=\{x_n:n\in\mathbb{N}\}$ has a limit point $x$. Moreover, there exists at least one element $A_m$ that contains $x$, and $A_m\cap B\subset\{x_1,\cdots,x_m\}$. However, $A_m$ as a neighborhood of $x$ contains only finite points of $B$, contradicting with \hyperref[thm:4.27]{Theorem 4.27}!
\end{proof}

\paragraph{Definition 4.32\label{def:4.32}} (Sequential compactness). A topological space $X$ is said to be \textit{sequentially compact} if every sequence of points of $X$ has a convergent subsequence.

\paragraph{Lemma 4.33\label{lemma:4.33}} (Lebesgue number lemma for sequentially compact metric space). Let $\mathscr{A}$ be an open cover of a sequentially compact metric space $(X,d)$. Then there exists a $\delta > 0$ such that for each subset of $X$ having diameter less than $\delta$, there exists an element of $\mathscr{A}$ that contains it.
\begin{proof}
Argue by contradiction. Let $\mathscr{A}$ be an open cover of $X$, we assume that there exists no $\delta >0$ such that each set of diameter less than $\delta$ has an element of $\mathscr{A}$ containing it. Then for each $n\in\mathbb{N}$, there exists a set $C_n$ of diameter less than $1/n$ that is not contained by any element of $\mathscr{A}$. Choose a point $x_n\in C_n$ for each $n$. Since $X$ is sequentially compact, there exists a subsequence $\{x_{n_k}\}_{k\in\mathbb{N}}$ that converges to some $x_\infty\in X$. Since $x_\infty\in A$ for some element $A$ of $\mathscr{A}$, we can choose some $\epsilon > 0$ such that $O(x_\infty,\epsilon)\subset A$. Then for a large enough $k$ such that $1/n_k < \epsilon / 2$ and $d(x_{n_k},x_\infty) < \epsilon/2$, $C_{n_k}\subset O(x_{n_k},\epsilon/2)\subset O(x_\infty, \epsilon)\subset A$, a contradiction!
\end{proof}

\paragraph{Definition 4.34\label{def:4.34}} (Totally bounded sets). A metric space $(X,d)$ is said to be \textit{totally bounded}, if for every $\epsilon >0$, there exists a finite cover of $X$ by open $\epsilon$-balls.

\paragraph{Remark.} It is clear that a compact metric space is totally bounded, since we can find a finite subcover of the open cover $\{O(x,\epsilon):x\in X\}$ of $X$ for any $\epsilon > 0$. Conversely, a totally bounded space is not always compact. As a simple counterexample, consider the open interval $(0,1)$.

\paragraph{Lemma 4.35.\label{lemma:4.35}} A metrizable space $X$ is totally bounded if it is sequentially compact.
\begin{proof}
Argue by contradiction. Suppose that $\exists\epsilon>0$ such that $X$ cannot be covered by finitely many open $\epsilon$-balls. We construct a sequence $\{x_n\}$ as follows. Choose any $x_1\in X$, then $O(x_1,\epsilon)\subsetneq X$. For any $n>1$, choose $x_n\in X\backslash\bigcup_{j=1}^{n-1}O(x_j,\epsilon)$. By construction, $d(x_{n},x_j)\geq\epsilon$ for $j=1,\cdots,n-1$. Then $\{x_n\}$ does not converge in $X$, contradicting the sequential compactness of $X$.
\end{proof}

Now we are prepared to introduce a profound conclusion of metric spaces.

\paragraph{Theorem 4.36\label{thm:4.36}} (Equivalence of four kinds of compactness in metric spaces). Let $X$ be a metrizable space. The following are equivalent: (i) $X$ is compact; (ii) $X$ is limit point compact; (iii) $X$ is countably compact; (iv) $X$ is sequentially compact.
\begin{proof}
(i) $\Rightarrow$ (ii): By \hyperref[thm:4.29]{Theorem 4.29}.\quad (ii) $\Leftrightarrow$ (iii): A metric space is $T_1$. Apply \hyperref[thm:4.31]{Theorem 4.31}.

(ii) $\Rightarrow$ (iv): Assume $X$ is limit point compact, and let $\{x_n\}_{n=1}^\infty$ be a sequence of points of $X$. Consider the set $A=\{x_n:n\in\mathbb{N}\}$. If $A$ is finite, then we can choose infinitely many $x_n$ that coincides with some $x\in A$, which form a convergent subsequence. On the other hand, if $A$ is infinite, then $A$ has a limit point $x\in X$. We can construct a convergent subsequence as follows. We first choose $n_1$ such that $x_{n_1}$ lies in open ball $O(x,1)$. For every $k\geq 2$, we can also find $N_k$ such that $x_n\in O(x,1/k)$ for all $n\geq N_k$. If $x_{n_{k-1}}$ is given, we can choose $n_k\geq\max\{n_{k-1},N_k\}$ so that $x_{n_k}\in O(x,1/k)$. Then $\{x_{n_k}\}_{k=1}^\infty$ converges to $x\in X$.

(iv) $\Rightarrow$ (i): Let $\mathscr{A}$ be an open cover of a sequentially compact metric space $(X,d)$. By \hyperref[lemma:4.33]{Lemma 4.33}, $\mathscr{A}$ has a Lebesgue number $\delta >0$. By \hyperref[lemma:4.35]{Lemma 4.35}, we can cover $X$ by finitely many open $\delta/3$-balls. Each of these balls has diameter no greater than $2\delta/3$, hence lies in some elements of $\mathscr{A}$. By choosing this elements of $\mathscr{A}$ we immediately obtain a finite subcover of $X$.
\end{proof}

\subsection{Local Compactness and Compactification}
\paragraph{Definition 4.37\label{def:4.37}} (Locally compact spaces). A topological space $X$ is said to be \textit{locally compact at} $x$ if there is some compact subspace $C$ of $X$ that contains a neighborhood of $x$. If $X$ is locally compact at each of its points, $X$ is said to be \textit{locally compact}.

\paragraph{Example.} (i) The euclidean space $\mathbb{R}^n$ is locally compact, because for any $\mathbf{x}\in\mathbb{R}^n$, we can always find a compact closed cell $[-b,b]^n$ containing a neighborhood of $\mathbf{x}$, where $b>\Vert\mathbf{x}\Vert_\infty$.

(ii) The rational numbers $\mathbb{Q}$ as a subspace of $\mathbb{R}$ is not locally compact. For any $q\in Q$, choose an open neighborhood $N_q:=\mathbb{Q}\cap O(q,\epsilon)$ of $q$. Since $N_q$ is countable, denote by $\{q_1,q_2,\cdots\}$ its elements. Then $\mathscr{A} = \{O(q_n,2^{-n}\epsilon),n=1,2,\cdots\}$ is an open cover of $N_q$. However, any finite subcollection of $\mathscr{A}$, with total length less than $2\epsilon$, does not cover $N_q$.

\paragraph{Theorem 4.38.\label{thm:4.38}} Let $X$ be a topological space. Then $X$ is locally compact Hausdorff if and only if there exists a space $Y$ satisfying the following conditions:
\begin{itemize}
	\item[(i)] $X$ is a subspace of $Y$;
	\item[(ii)] $Y\backslash X$ consists of a single point;
	\item[(iii)] $Y$ is a compact Hausdorff space.
\end{itemize}
If $Y$ and $Y^\prime$ are two spaces satisfying these conditions, then there is a homeomorphism of $Y$ with $Y^\prime$ that equals the identity map on $X$.

\begin{proof}
\textit{Step I:} We first verify the uniqueness. Let $Y$ and $Y^\prime$ be two spaces satisfying (i)-(iii). Define $h:Y\to Y^\prime$ by letting $h$ map the single point $p$ of $Y\backslash X$ to the point $q$ of $Y^\prime\backslash X$, and let $h$ equal the identity on $X$. We show that if $U$ is open in $Y$, then $h(U)$ is open in $Y^\prime$. Symmetry then implies that $h$ is a homeomorphism.

First, consider the case $p\notin U$. Since $U$ is open in $Y$ and $U\subset X$, it is open in $X$. Noticing $X=Y^\prime\backslash\{q\}$ and $Y^\prime$ is Hausdorff, $X$ is open in $T^\prime$. Then $h(U)=U$ is open in $Y^\prime$.

Second, let $p\in U$. Since $C=Y\backslash U$ is closed in $Y^\prime$ and $Y^\prime$ is compact, $C$ is compact as a subspace of $Y$. Since $C\subset X$, it is a compact subspace of $X$. Because $X$ is a subspace of $Y^\prime$, $C$ is also compact in $Y^\prime$. Since $Y^\prime$ is Hausdorff, $C$ is closed in $Y$, and $Y^\prime\backslash C = h(U)$ is open in $Y^\prime$.
\vspace{0.2cm}

\textit{Step II:} Now we suppose $X$ is a locally compact Hausdorff space $X$ and construct $Y$. Let us take some object that is not a point
of $X$, denoted by the symbol $\infty$ for convenience, and adjoin it to $X$, forming the set $Y=X\cup\{\infty\}$. Inspired by Step I, we topologize $Y$ by defining the collection of open sets of $Y$ to consist of 
\begin{itemize}
	\item[(i)] all sets $U$ that are open in $X$, and
	\item[(ii)] all sets of the form $Y\backslash C$, where $C$ is a compact subspace of $X$.
\end{itemize}

We first check that such collection is indeed a topology on $Y$. 
\begin{itemize}
	\item Clearly, $\emptyset$ and $Y$ are open sets of type (i) and (ii), respectively.
	\item For the intersection condition, let $U_1$ and $U_2$ be open sets of $X$, and let $C_1$ and $C_2$ be compact sets in $X$. Then $U_1\cap U_2$ is of type (i), $(Y\backslash C_1)\cap (Y\backslash C_2) = Y\backslash(C_1\cup C_2)$ is of type (ii), and $U\cap (Y\backslash C) = U\cap(X\backslash C)$ is of type (i) because $X$ is Hausdorff.
	\item For the union condition, let $\{U_\alpha\}$ be a collection of open sets of $X$, and let $\{C_\beta\}$ be a collection of compact sets in $X$. Then $\bigcup_{\alpha}U_\alpha = U$ is of type (i), $\bigcup_{\beta}(Y\backslash C_\beta) = Y\backslash\bigcap_\beta C_\beta = Y\backslash C$ is of type (ii), and $U \cup (Y\backslash C) = Y\backslash (C\backslash U)$ is of type (ii) because $C\backslash U$ is a closed subset of compact set $C$.
\end{itemize} 

Then we need to verify that $X$ is a subspace of $Y$:
\begin{itemize}
	\item Given any open set in $Y$, its intersection with $X$ is open in $X$. If the open set is of type (i), it is clearly open in $X$. If it is of type (ii), then $(Y\backslash C)\cap X = X\backslash C$ is open in Hausdorff space $X$.
	\item Conversely, given any open set in $X$, it is a type (i) open set in $Y$.
\end{itemize}

Now we show $Y$ is compact. Let $\mathscr{A}$ be an open cover of $Y$, Then it must contain at least one open set of type (ii), denoted by $Y\backslash C$, to contain $\infty$. Taking all members in $\mathscr{A}$ but $Y\backslash C$ and intersect them with $X$, we obtain a cover of $X$. Since $C$ is a compact subspace of $X$, finitely many of them cover $C$. Then the corresponding finite collection of elements of $\mathscr{A}$ along with $Y\backslash C$ form a cover of $Y$.

It remains to show $Y$ is Hausdorff. Let $x$ and $y$ be two elements of $Y$:
\begin{itemize}
	\item Both $x$ and $y$ lies in $X$, which is a clear case since $X$ is Hausdorff.
	\item Assume $y=\infty$. By the local compactness of $X$, we can choose a compact set $C$ in $X$ that contains a neighborhood $U$ of $x$, then $U$ and $Y\backslash C$ are disjoint neighborhoods of $x$ and $\infty$, respectively, in $Y$.
\end{itemize}

\vspace{0.2cm}
\textit{Step III:} Finally, we prove the converse. Suppose a space $Y$ satisfying conditions (i)–(iii) exists. Then $X$ is Hausdorff, because it is a subspace of Hausdorff space $Y$. Now fix $x\in X$. Choose disjoint open sets $U\ni x$ and $V\supset\{Y\backslash X\}$ in $Y$. Then $C=Y\backslash V$ is closed in $Y$, and is compact. Since $C$ is contained in $X$, it is also compact in $X$. Furthermore, it contains a neighborhood $U$ of $x$.
\end{proof}

\paragraph{Definition 4.39\label{def:4.39}} (Compactification). If $Y$ is a compact Hausdorff space, and $X$ is a proper dense subspace of $Y$, then $Y$ is said to be a \textit{compactification} of $X$. If $Y\backslash X$ equals a single point, then $Y$ is called the \textit{one-point compactification} of $X$.

\paragraph{Remark.} By \hyperref[thm:4.38]{Theorem 4.38}, $X$ has a one-point compactification $Y$ if and only if $X$ is a locally compact Hausdorff space that is not itself compact. Moreover, it is uniquely determined up to a homeomorphism.

\paragraph{Example.} (i) The one-point compactification of real line $\mathbb{R}$ is homeomorphic to the circle $S^1$. To see this, we define $f:\mathbb{R}\cup\{\infty\}\to S_1$ as follows:
\begin{align*}
	f(t):=\begin{cases}
		\left(\frac{1-t^2}{1+t^2},\frac{2t}{1+t^2}\right),\ & t\in\mathbb{R},\\
		(-1,0),\ & t=\infty,
	\end{cases},\quad
    f^{-1}(x,y):=\begin{cases}
    	\frac{y}{1+x},\ &(x,y)\neq(-1,0),\\
    	\infty,\ &(x,y)=(-1,0).
    \end{cases}
\end{align*}
By construction, $f|_\mathbb{R}$ is a homeomorphism between $\mathbb{R}$ and $S_1\backslash\{(-1,0)\}$. Furthermore, the compactification of $S_1\backslash\{(-1,0)\}$ is its closure $S_1$.

(ii) The one-point compactification of $\mathbb{R}^2$ is homeomorphic to the sphere $S^2$. If $\mathbb{R}^2$ is looked at as the space $\mathbb{C}$ of complex numbers, then $\mathbb{C}\cup\{\infty\}$ is called the \textit{Riemann sphere}, or the \textit{extended complex plane}.

\paragraph{} Here we also give another formulation of local compactness which aligns with \hyperref[def:3.16]{Definition 3.16}. The two formulations are equivalent in Hausdorff spaces.

\paragraph{Theorem 4.40\label{thm:4.40}} (Another equivalent characterization of local compactness in Hausdorff spaces). Let $X$ be a Hausdorff space. Then $X$ is locally compact if and only if given any $x\in X$ and any neighborhood $U$ of $x$, there exists a neighborhood $V$ of $x$ such that $\overline{V}$ is compact and $\overline{V}\subset U$.
\begin{proof}
Clearly this characterization implies local compactness of $X$, and this direction does not require $X$ to be Hausdorff. We prove the converse.

Suppose $X$ is locally compact and Hausdorff, then we can take the one-point compactification $Y$ of $X$. For any $x\in X$ and any neighborhood $U$ of $x$, let $C=Y\backslash U$. Then $C$ is compact since $Y$ is compact Hausdorff and $U$ is open. Applying \hyperref[lemma:4.18]{Lemma 4.18}, we can find disjoint open sets $V$ and $W$ in $Y$ such that $V\ni x$ and $W\supset C$. Then $\overline{V}$ is compact because it is a closed subset of a compact space $Y$, and $\overline{V}\subset U$ because $V$ is disjoint from a open set $W$ containing $C$.
\end{proof}

\paragraph{Corollary 4.41.\label{cor:4.41}} (i) A closed subspace of a locally compact space is locally compact; (ii) An open subspace of a locally compact Hausdorff space is locally compact. 
\begin{proof}
(i) Suppose that $A$ is a closed subspace of a locally compact space $X$. For any $x\in A$, let $X$ be a compact subspace of $X$ that contains a neighborhood $U$ of $x$. Then $C\cap A$ as a closed subset of $C$ is compact. Furthermore, it contains the neighborhood $U\cap A$ of $x$ in $A$.

(ii) Now suppose that $A$ is an open subspace of a locally compact Hausdorff space $X$. Then for any $x\in A$ and any neighborhood $U$ of $x$ in $A$, $U$ is also a neighborhood of $x$ in $X$. Hence we find can a neighborhood $V$ of $x$ in $X$ such that $\overline{V}$ is compact and $\overline{V}\subset U$.
\end{proof}

Combining \hyperref[thm:4.40]{Theorem 4.38} and \hyperref[cor:4.41]{Corollary 4.41} immediately yields the following result.
\paragraph{Corollary 4.42.\label{cor:4.42}} A subspace of $X$ is homeomorphic to an open subspace of a compact Hausdorff subspace if and only if $X$ is locally compact Hausdorff.
\begin{proof}
\textit{``If'' part:} If $Y$ is an open subspace of a locally compact Hausdorff space $X$, then $Y$ is locally compact by \hyperref[cor:4.41]{Corollary 4.41}.

\textit{``Only if'' part:} If $Y$ is locally compact, then $Y$ is a subspace of its one-point compactification $X$, which is compact and Hausdorff by \hyperref[thm:4.38]{Theorem 4.38}.
\end{proof}

\newpage
\section{Countability and Trennungsaxiom}
\subsection{First Countability, Second Countability, Separability and Lindelöf Spaces}
\paragraph{Definition 5.1\label{def:5.1}} (First countable spaces). Let $X$ be a topological spaces. Given a point $x\in X$, the space $X$ is said to be \textit{first countable at} $x$, if there exists a countable collection $\mathscr{B}$ of neighborhoods of $x$ such that each neighborhood of $x$ contains at least one of the elements of $\mathscr{B}$. In other words, there exists a countable neighborhood basis for $\mathscr{T}$ at $x$.

\paragraph{} By definition, a metrizable space is always first-countable: for any $x$ of a metrizable space $X$, the collection $\mathscr{B}=\{O(x,1/n),n\in\mathbb{N}\}$ of open balls satisfies \hyperref[def:5.1]{Definition 5.1}. But the converse is not true.

We give a generalization of \hyperref[thm:2.17]{Theorem 2.17} as below. It states that in a first-countable space, convergent sequences are adequate to detect limit points of sets and to
check continuity of functions.
\paragraph{Theorem 5.2.\label{thm:5.2}} Let $X$ be a first countable space.
\begin{itemize}
	\item[(i)] Let $A\subset X$. If $x\in\overline{A}$, then we can find a sequence $\{x_n\}$ of points of $A$ that converges to $x$.
	\item[(ii)] Let $f:X\to Y$. If for every convergent sequence $x_n\to x$ in $X$, the sequence $f(x_n)$ converges to $f(x)$, then $f$ is continuous.
\end{itemize}
\begin{proof}
(i) Fix $x\in\overline{A}$. Then there exists a countable neighborhood basis $\mathscr{B}=\{B_n,n\in\mathbb{N}\}$ at $x$, and each neighborhood of $x$ contains at least one of the elements of $\mathscr{B}$. Choose $x_n\in A\cap\bigl(\bigcap_{j=1}^n B_n\bigr).$ Then for any neighborhood $U$ of $x$, it contains some $B_N\in\mathscr{B}$. Hence $x_n\in U$ for all $n\geq N$.

(ii) By (i), $f(\overline{A})\subset\overline{f(A)}$ for all $A\subset X$. Then $f$ is continuous.
\end{proof}

Recalling Definition \hyperref[def:1.19]{1.19}, a topological space is said to be \textit{second countable} if it has a countable basis. A second countable space must be first countable, because we can always find a countable collection of neighborhoods that satisfying \hyperref[def:5.1]{Definition 5.1} by choosing the base sets containing $x$.

Conversely, a first-countable space is not always second-countable. As a simplest example, consider the metric space $\mathbb{R}$ (or any uncountable set) with metric $d(x,y):=\mathds{1}_{\{x\neq y\}}$. Then $(X,d)$ is first-countable, because for any $x\in\mathbb{R}$, $\{x\}$ is a neighborhood of $x$, and any countable collection of neighborhoods of $x$ containing $\{x\}$ satisfies \hyperref[def:5.1]{Definition 5.1}. However, a basis for this topology must contains all singletons, hence is uncountable.

\paragraph{Theorem 5.3.\label{thm:5.3}} A subspace of a first-countable (second-countable) space is first-countable (second-countable). A finite product of first-countable (second-countable) spaces is first countable (second-countable).
\begin{proof}
Suppose $X$ is first-countable, and $Y$ is a subspace of $X$. Fix $y\in Y$. If $\mathscr{B}$ is a collection of neighborhoods of $y$ in $X$ such that each neighborhood of $y$ in $X$ contains at least one element of $\mathscr{B}$, then the same condition holds for $\{B\cap Y:B\in\mathscr{B}\}$ in $Y$. 

Suppose $\{X_j,j=1,\cdots,n\}$ are first-countable spaces. Then for each $x=(x_1,\cdots,x_n)\in\prod_{j=1}^n X_j$, there exists a countable collection $\mathscr{B}_j$ of neighborhoods of $x_j$ that satisfies \hyperref[def:5.1]{Definition 5.1} for each $j=1,\cdots,n$. Consequently, any neighborhood of $x$ in $\prod_{n=1}^\infty X_n$ contains at least one element of the countable collection
\begin{align*}
	\left\{\prod_{j=1}^n B_j:B_j\in\mathscr{B}_j,\ \forall j\in\mathbb{N}\right\}.
\end{align*}
The statement for second countability is similar.
\end{proof}

Recall \hyperref[def:1.20]{Definition 1.20}, a topological space is said to be separable if it has a dense subset. We also show in \hyperref[thm:1.21]{Theorem 1.21} that a second-countable space must be separable. The following theorem reveals the relationship among compactness, separability and second countability.

\paragraph{Theorem 5.4.\label{thm:5.4}} (i) A compact metrizable space is separable. (ii) A separable metrizable space is second-countable.
\begin{proof}
(i) Let $X$ be a compact metric space. Let $\mathscr{B}_1$ be a finite cover of $1$-balls of $X$. Similarly, let $\mathscr{B}_n$ be a finite cover of $1/n$-balls of $X$, $n=2,3,\cdots$. Let $A$ be the centers of these balls, then $A$ being the union of countable finite sets is countable. Clearly, $A$ is dense in $X$.

(ii) Let $X$ be a separable metrizable space. Then there exists a dense subset $A=\{x_n,n\in\mathbb{N}\}$ of $X$. Let $\mathscr{B}=\{O(x,k^{-1}):x\in A,k\in\mathbb{N}\}$ be a collection of open balls in $X$. We show that $\mathscr{B}$ is a basis for $X$.

Let $U$ be a nonempty open set in $X$. For each $y\in U$, we can find some $\epsilon > 0$ such that $O(y,\epsilon)\subset U$. Since $A$ is dense in $X$, there exists $x\in A$ and $2/\epsilon < k\in\mathbb{N}$ such that $y\in O(x,k^{-1})\subset O(y,\epsilon)$ by the triangle inequality. Denote by $B_y$ the open ball in $\mathscr{B}$ with $y\in B_y\subset U$. Then $U = \bigcup_{y\in U}B_y$.
\end{proof}

Now we introduce another countability axiom.

\paragraph{Definition 5.5\label{def:5.5}} (Lindelöf spaces). Let $X$ be a topological space. $X$ is said to be a \textit{Lindelöf space}, if every open cover of $X$ contains a countable subcover of $X$.

\paragraph{Theorem 5.6.\label{thm:5.6}} (i) A second-countable space is Lindelöf; (ii) A metrizable Lindelöf space is second-countable.
\begin{proof}
(i) Suppose $X$ is a second-countable space with countable basis $\mathscr{B}=\{B_n,n\in\mathbb{N}\}$. Let $\mathscr{A}$ be an open cover of $X$. Then each element of $\mathscr{A}$ contains some element of $\mathscr{B}$. Furthermore, for any $x\in X$, it is contained in some $A\in\mathscr{A}$, hence is contained in some $B_n\in\mathscr{B}$ such that $x\in B_n\subset A$.

Let $\mathscr{B}^\prime = \{B_n\in\mathscr{B}:\exists A\in\mathscr{A}\ \text{such that}\ B_n\subset A\}$. For each $B\in\mathscr{B}^\prime$, we choose one $A\in\mathscr{A}$ with $A\supset B$. Then we obtain a finite subcover of $X$ from $\mathscr{A}$.

(ii) Let $X$ be a Lindelöf metric space. Let $\mathscr{B}_1$ be a countable cover of $1$-balls of $X$. Similarly, let $\mathscr{B}_n$ be a countable cover of $1/n$-balls of $X$, $n=2,3,\cdots$. We show that $\mathscr{B}=\bigcup_{n=1}^\infty\mathscr{B}_n$ is a countable basis for $X$.

Let $U$ be a nonempty open set in $X$. For each $y\in U$, $\exists\epsilon > 0$ such that $O(y,\epsilon)\subset U$. Let $n > 2/\epsilon$, then we can choose some $B_y:=O(x,n^{-1})\in\mathscr{B}_n$ such that $y\in O(x,n^{-1})\subset O(y,\epsilon)$. Then $U=\bigcup_{y\in U}B_y$.
\end{proof}

\paragraph{Remark.} Combining \hyperref[thm:5.4]{Theorem 5.4} and \hyperref[thm:5.6]{Theorem 5.6}, it is clear that the second countability axiom, the separability axiom, and the Lindelöf axiom are equivalent for metrizable spaces.

\paragraph{Theorem 5.7.\label{thm:5.7}} Let $f:X\to Y$ be continuous.
(i) If $X$ is first-countable, so is $f(X)$; (ii) If $X$ is second-countable, so is $f(X)$; (iii) If $X$ is separable, so is $f(X)$; (iv) If $X$ is Lindelöf, so is $f(X)$.
\begin{proof}
(i) Fix $x\in X$. Let $\mathscr{B}$ be a countable collection of neighborhoods of $x$ such that any neighborhood of $x$ contains a member of $\mathscr{B}$. Then for any neighborhood $V$ of $f(x)$ in $f(X)$, $f^{-1}(V)$, being a neighborhood of $x$ in $X$, must contain some $B\in\mathscr{B}$. Hence $V\supset f(B)$.

(ii) Similar to (i), we can verify that if $\mathscr{B}$ is a basis for $X$, then $\{f(B):B\in\mathscr{B}\}$ is a basis for $f(X)$.

(iii) Let $A$ be a countable dense subset of $X$, then $f(X)=f(\overline{A})\subset\overline{f(A)}$.

(iv) Let $\mathscr{A}$ be an open cover of $f(X)$. By continuity $\{f^{-1}(A):A\in\mathscr{A}\}$ is an open cover of $X$, from which we can choose countably many $\{f^{-1}(A_n)\}$ that cover $X$. Then $\{A_n\}$ covers $f(X)$.
\end{proof}

\paragraph{Example: Lower limit topology $\mathbb{R}_\ell$.} Let $\mathscr{B}=\{[a,b):a<b\}$ be the collection of all half-open intervals on $\mathbb{R}$ of the form $[a,b)$. Then the topology generated by $\mathscr{B}$ is called the \textit{lower limit topology} on $\mathbb{R}$, and we denote this topological space by $\mathbb{R}_\ell$. The space $\mathbb{R}_\ell$ is first-countable, separable and Lindelöf. However, it is not second-countable.

We first show that $\mathbb{R}_\ell$ is not second countable. Let $\mathscr{B}^\prime$ be a basis for $\mathbb{R}_\ell$. $\forall x\in\mathbb{R}$, $\exists B_x\in\mathscr{B}^\prime$ such that $x\in B_x\subset[x,x+1)$. Then $x\neq y$ implies $B_x\neq B_y$ since $x=\inf B_x\neq \inf B_y=y$. Hence $\mathscr{B}^\prime$ is uncountable.

Given $x\in\mathbb{R}_\ell$, the set $\{[x,x+1/n),n\in\mathbb{N}\}$ is a countable basis at $x$, which implies first-countability. Following this, every neighborhood of $x$ includes a rational number, hence $x\in\overline{\mathbb{Q}^{\mathbb{R}_\ell}}$. Since $x$ is arbitrarily chosen, the set of all rational numbers $\mathbb{Q}$ is dense in $\mathbb{R}_\ell$, which implies separability.

Now it remains to show that $\mathbb{R}_\ell$ is Lindelöf. Since every open cover of $\mathbb{R}_\ell$ can be rewritten as a union of basis elements, we show that every open cover of $\mathbb{R}_\ell$ by basis elements contains a countable subcover of $\mathbb{R}_\ell$. Let $\mathscr{A}=\{[a_\alpha,b_\alpha)\}_{\alpha\in J}$ be such a cover, we wish to find a countable subcover of $\mathscr{A}$.

\textit{Step I:} Let $C=\bigcup_{\alpha\in J}(a_\alpha,b_\alpha)$, we show that $\mathbb{R}_\ell\backslash C$ is countable. Let $x\in\mathbb{R}_\ell\backslash C$. Then $x$ lies in no open interval $(a_\alpha,b_\alpha)$, and there exists $\beta\in J$ such that $x=a_\beta$. We choose a rational $q_x$ such that $q_x\in(a_\beta,b_\beta)$. Similarly, we can choose such $q_y$ for $y\in\mathbb{R}_\ell\backslash C$ with $y > x$. If $q_y\leq q_x$, then $y\in (x,q_y)\subset (x,q_x) \subset (a_\beta,b_\beta)$, contradicting $y\notin C$! Therefore $q_x < q_y$, and the map $x\mapsto q_x,\mathbb{R}_\ell\backslash C\to\mathbb{Q}$ is injective.

\textit{Step II:} We show that some countable subcollection of $\mathscr{A}$ covers $\mathbb{R}_\ell$. By Step I, we find a countable subcollection $\mathscr{A}^\prime=\{[a_\beta,b_\beta):\beta\in J\ \text{and}\ a_\beta\in\mathbb{R}_\ell\backslash C\}$ that covers $\mathbb{R}_\ell\backslash C$. To cover $C$, note that $C$ is an open set in the standard topology on $\mathbb{R}$, which is second-countable by \hyperref[thm:5.4]{Theorem 5.4 (ii)}. Since $C$ is covered by $\{(a_\alpha,b_\alpha)\}_{\alpha\in J}$, there exists a countable subcollection $\{(a_{\alpha_n},b_{\alpha_n})\}_{n\in\mathbb{N}}$ that covers $C$. Let $\mathscr{A}^{\prime\prime}=\{[a_{\alpha_n},b_{\alpha_n})\}_{n\in\mathbb{N}}$, then $\mathscr{A}^{\prime}\cup\mathscr{A}^{\prime\prime}$ is a countable subcollection of $\mathscr{A}$ that covers $\mathbb{R}_\ell$.

\paragraph{Example: Sorgenfrey plane $\mathbb{R}_\ell^2$.} The Sorgenfrey plane, as the product of two $\mathbb{R}_\ell$ space, has as basis all sets of the form $[a,b)\times[c,d)$. We claim that $\mathbb{R}_\ell^2$ is not Lindelöf.

Consider the subspace $L=\{(x,-x):x\in\mathbb{R}_\ell\}$ of $\mathbb{R}_\ell^2$. For any $(y,z)\notin L$, we can find a basis element containing $(y,z)$ and not intersecting $L$. Hence $L$ is closed in $\mathbb{R}_\ell^2$.

Then we construct an open cover of $\mathbb{R}_\ell^2$ by $\mathbb{R}_\ell^2\backslash L$ and by all basis elements of the form $[x,b)\times[-x,d)$. Clearly, $L$ is uncountable, and each of these open sets intersects $L$ in at most one point. Therefore we require uncountably many open sets to cover $L$. As a result, $\mathbb{R}_\ell^2$ is not Lindelöf.

\subsection{Regular Spaces and Normal Spaces}
\paragraph{Definition 5.8} (Regular spaces/$T_3$ spaces). Let $X$ be a topological space where one-point sets are closed. Then $X$ is said to be \textit{regular} if for each pair consisting of a point $x$ of $X$ and a closed set $B$ disjoint from $x$, there exist disjoint open sets containing $x$ and $B$, respectively.

\paragraph{Definition 5.9\label{def:5.9}} (Normal spaces/$T_4$ spaces). Let $X$ be a topological space where one-point sets are closed. Then $X$ is said to be \textit{normal} if for each pair of disjoint closed sets $A$ and $B$, there exist disjoint open sets containing $A$ and $B$, respectively.

\paragraph{Example.} The left limit topology $\mathbb{R}_\ell$ is normal. It is clear that one-point sets in $\mathbb{R}_\ell$ are closed. Let $A$ and $B$ be two disjoint closed sets in $\mathbb{R}_\ell$. For each point $a$ of $A$, we can choose a basis element $[a,x_a)$ not intersecting $B$ since $B$ is closed. Similarly, we choose a basis element $[b,x_b)$ not intersecting $A$ for each $b\in B$. Then
\begin{align*}
	U=\bigcup_{a\in A}[a,x_a)\ \ \text{and}\ \ V=\bigcup_{b\in B}[b,x_b)
\end{align*}
are two disjoint open sets containing $A$ and $B$, respectively. Hence $\mathbb{R}_\ell$ is normal.

\paragraph{Lemma 5.10.\label{lemma:5.10}} Let $X$ be a topological space where one-point sets are closed.
\begin{itemize}
	\item[(i)] $X$ is regular if and only if given any point $x$ of $X$ and any neighborhood $U$ of $x$, there exists an open neighborhood $V$ of $x$ such that $\overline{V}\subset U$.
	\item[(ii)] $X$ is normal if and only if given any closed set $A$ in $X$ and any open set $U\supset A$, there is an open set $V$ containing $A$ such that $\overline{V}\subset U$.
\end{itemize}
\begin{proof}
(i) For the ``if'' part, let $x$ be a point of $X$ and $B$ be a closed set disjoint from $x$. Then $X\backslash B$ is a neighborhood of $x$, and we can find a neighborhood $V$ of $x$ in $X$ such that $\overline{V}\subset X\backslash B$.  Let $O\subset U$ be an open set containing $x$. Then $O$ and $X\backslash\overline{V}$ are disjoint open sets that contain $x$ and $B$, respectively.

Conversely, suppose $X$ is regular. For any $x\in X$ and any neighborhood $U$ of $x$, let $O\subset U$ be an open set containing $x$. Then $X\backslash O$ and $x$ can be separated with two open sets $F\supset X\backslash O$ and $G\ni x$. For any $y\notin O$, we have $y\in F$. Since $F$ is open and disjoint from $G$, $y\notin \overline{G}$. Hence $\overline{G}\subset O\subset U$, the proof is completed.

(ii) is similar to (i) by replace $x$ with $A$.
\end{proof}

We also have the following properties of regular spaces alike \hyperref[prop:4.17]{Proposition 4.17}.

\paragraph{Lemma 5.11.\label{lemma:5.11}} (i) A product of regular spaces is regular; (ii) A subspace of regular space is regular.
\begin{proof}
(i) Let $\{X_\alpha\}_{\alpha\in J}$ be a collection of regular spaces, and let $X=\prod_{\alpha\in J}X_\alpha$ be the product. By \hyperref[prop:4.17]{Proposition 4.17 (iii)}, $X$ is Hausdorff, then one-point sets in $X$ are closed. Let $\mathbf{x}=(x_\alpha)_{\alpha\in J}$ be a point of $X$ and $U$ be a neighborhood of $\mathbf{x}$ in $X$. Choose a basis element $\prod_{\alpha\in J}U_\alpha\subset U$ containing $\mathbf{x}$. By \hyperref[lemma:5.11]{Lemma 5.11}, there exists a neighborhood $V_\alpha$ of $x_\alpha$ in $X_\alpha$ such that $\overline{V}_\alpha\subset U_\alpha$ for every $\alpha\in J$. Moreover, $V=\prod{V}_\alpha$ is a neighborhood of $\mathbf{x}$ in $X$ whose closure is contained in $U$. Applying \hyperref[lemma:5.11]{Lemma 5.11} again concludes the proof.

(ii) Let $Y$ be a subspace of a regular space $X$. Let $x$ be a point of $Y$ and $B$ be a closed set in $Y$ disjoint from $x$. Let $\overline{B^X}$ be the closure of $B$ in $X$, then $B=\overline{B^X}\cap Y$ by \hyperref[prop:1.26]{Proposition 1.26 (i)}. Moreover, $x\notin \overline{B^X}$. The regularity of $X$ allows us to find disjoint open sets $U\ni x$ and $V\supset\overline{B^X}$ in $X$, and the conclusion follows from taking their intersection with $Y$.
\end{proof}

\paragraph{Remark.} There is no analogous conclusion for normal spaces. As a counterexample, the left limit topology on real line $\mathbb{R}_\ell$ is normal, but the Sorgenfrey plane $\mathbb{R}_\ell^2$ is not normal.

Now we are going to discuss some common conclusions about normal spaces.

\paragraph{Theorem 5.12.\label{thm:5.12}} A second-countable regular space is normal.
\begin{proof}
Let $X$ be a second-countable regular space with a countable basis $\mathscr{B}$. Let $A$ and $B$ be two disjoint closed sets in $X$. Using the regularity, each $a\in A$ has a neighborhood $U$ not intersecting $B$, and we can choose a neighborhood $V$ of $a$ whose closure lies in $U$ by \hyperref[lemma:5.10]{Lemma 5.10}. Furthermore, we can choose a basis element from $\mathscr{B}$ containing $a$ and contained by $V$. By choosing such a basis element for each $a\in A$, we obtain a countable open cover of $A$ whose closure does not intersect $B$. We denote  by $\{U_n\}$ the elements of this cover.

Similarly, we can construct a countable open cover $\{V_n\}$ of $B$ whose closure does not intersect $A$. Define
\begin{align*}
	U_n^\prime = U_n\backslash\left(\bigcup_{i=1}^n\overline{V}_i\right),\ V_n^\prime = V_n\backslash\left(\bigcup_{i=1}^n\overline{U}_i\right),\ n=1,2,\cdots.
\end{align*}
Then $U^\prime=\bigcup_n U_n^\prime$ is still an open cover of $A$, since $\bigcup_n\overline{V}_n$ does not intersect $A$. Also, $V^\prime=\bigcup_n V_n^\prime$ is an open cover of $B$. Moreover, $U^\prime$ and $V^\prime$ are disjoint, for if there exists $x\in U^\prime\cap V^\prime$, then there exists $i$ and $j$ such that $x\in U_i^\prime\cap V_j^\prime\subset U_i\cap V_j$. If $j\geq i$, then $x\in V_j^\prime$ and $x\notin \overline{U}_i$, a contradiction! A similar contradiction occurs when $i\geq j$. Then we conclude the proof.
\end{proof}

\paragraph{Theorem 5.13.\label{thm:5.13}} A compact Hausdorff space is normal.
\begin{proof}
Let $X$ be a compact Hausdorff space, and let $A$ and $B$ be two disjoint closed sets in $X$. 

We first prove the regularity of $X$. Fix $a\in A$, then we can find disjoint open sets $U_b\ni a$ and $V_b\ni b$ for every $b\in B$. Since $B$ is closed in $X$, $B$ is compact, and there exist finitely many $V_{b_1},\cdots,V_{b_n}$ that cover $B$. As a result, $\bigcap_{i=1}^n U_{b_i}$ and $\bigcup_{i=1}^n V_{b_i}$ are disjoint open sets that contains $a$ and $B$, respectively.

Now we prove the normality. Using the above conclusion, we can find disjoint open sets $U_a\ni a$ and $V_a\supset B$ for every $a\in A$. Since $A$ is closed in $X$, we can find finitely many $a_1,\cdots,a_m$ such that $U:=\bigcup_{i=1}^m U_{a_i}\supset A$. Meanwhile, $V:=\bigcap_{i=1}^m V_{a_i}\supset B$ is disjoint from $U$, which concludes the proof.
\end{proof}
\paragraph{Theorem 5.14.\label{thm:5.14}} A metrizable space is normal.
\begin{proof}
Let $(X,d)$ be a metric space. Let $A$ and $B$ be two disjoint closed sets in $X$. For each $a\in A$, $a\notin B=\overline{B}$, hence we can find $n_a$ such that the open ball $O(a,1/n_a)$ does not intersect $B$. Similarly, we can find $n_b$ for each $b\in B$ such that $O(b,1/n_b)$ does not intersect $A$. Define 
\begin{align*}
	U=\bigcup_{a\in A}O\left(a,\frac{1}{3n_a}\right),\ V=\bigcup_{b\in B}O\left(b,\frac{1}{3n_b}\right).
\end{align*}
Then $U$ and $V$ are open sets containing $A$ and $B$, respectively. It suffices to show $U$ and $V$ are disjoint. Suppose there exists $x\in U\cap V$. Then there exists $a\in A$ and $b\in B$ such that $x\in O(a,n_a^{-1}/3)\cap O(b,n_b^{-1}/3)$, which implies $d(a,b)\leq d(x,a) + d(x,b) < \frac{1}{3}(n_a^{-1}+n_b^{-1})$. However, by construction of $n_a$ and $n_b$, we have $d(a,b)\geq \max\{n_a^{-1},n_b^{-1}\}$, a contradiction!
\end{proof}

\paragraph{Review.} The separation axioms are listed in order of increasing strength.
\begin{itemize}
	\item $T_1$: For every pair of distinct points, each has a neighborhood not containing the other point.
	\item $T_2$ (Hausdorff): For every pair of distinct points, there exists disjoint neighborhoods of each.
	\item $T_3$ (Regular): One-point sets are closed; For each closed set and a point not contained in the closed set, there exists disjoint open sets containing each.
	\item $T_4$ (Normal): One-point sets are closed; For each pair of disjoint closed sets, there exists disjoint open sets containing each.
\end{itemize}

The letter ``T'' comes from the German ``Trennungsaxiom'', which means ``separation axiom''.

\subsection{Urysohn Lemma and Tietze Extension Theorem}
In normal spaces, we have similar conclusions as we have discussed in \hyperref[subsec:2.2]{Section 2.2}. We first introduce the Urysohn lemma, which is a generalization of \hyperref[lemma:2.11]{Lemma 2.11}.
\paragraph{Theorem 5.15} (Urysohn lemma). Let $X$ be a normal space. Let $A$ and $B$ be disjoint closed sets in $X$. Then there exists a continuous map $f:X\to [0,1]$ such that $f(A)=\{0\}$ and $f(B)=\{1\}$.
\begin{proof}
\textit{Step I:} Let $[\mathbb{Q}]_0^1=\mathbb{Q}\cap[0,1]$. We shall construct a collection $\{U_q\}_{q\in [\mathbb{Q}]_0^1}$ of open sets in $X$ indexed by the rational numbers in $[0,1]$, such that whenever $p < q$, we have $\overline{U}_p\subset U_q$. Since $[\mathbb{Q}]_0^1$ is countable, we can follow a sequence of elements in $[\mathbb{Q}]_0^1$ and define the sets $U_q$ by induction. Let the first two elements of the sequence be $1$ and $0$, and denote by $Q_n$ the first $n$ numbers in the sequence.

Let $U_1=X\backslash B$, then $U_1\supset A$. Since $A$ is closed in $X$ and $X$ is normal, we can find an open set $U_0$ such that $A\subset U_0\subset\overline{U}_0\subset U_1$ by \hyperref[lemma:5.10]{Lemma 5.10}. Then the first two sets are defined, and we wish to define $U_q$ for general $q\in[\mathbb{Q}]_0^1$ by induction. Suppose $U_{q}$ is defined for all $q\in Q_{n-1}$, where $n\geq 3$, and let $r$ be the next number in the sequence, i.e. $Q_n = Q_{n-1}\cup\{r\}$. 

Clearly, $0<r<1$. Since $Q_{n-1}$ is finite, we can always find the predecessor $p=\max\{x\in Q_{n-1}:x<r\}$ and successor $q=\min\{x\in Q_{n-1}:x>r\}$ of $r$ in $Q_n$. By the induction assumption, we have $\overline{U}_p\subset U_q$. Using \hyperref[lemma:5.10]{Lemma 5.10} again, we choose an open set $U_r$ such that $\overline{U}_p\subset U_r\subset\overline{U}_r\subset U_q$. Since $p$ is the predecessor of $r$ in $Q_n$ and $q$ is the successor, the inclusion order $p<q\Rightarrow\overline{U}_p\subset U_q$ is guaranteed for all $p,q\in Q_n$. By induction, $U_q$ is defined for all $q\in[\mathbb{Q}]_0^1$.
\vspace{0.12cm}

\textit{Step II:} Following Step I, we extend the definition of $U_q$ to all $q\in\mathbb{Q}$ by setting $U_q = \emptyset$ for $q < 0$ and $U_q = X$ for $q > 1$. The condition $p<q\Rightarrow \overline{U}_p\subset U_q$ still holds for any pair $p,q\in\mathbb{Q}$.
\vspace{0.12cm}

\textit{Step III:} For any point $x$ of $X$, define $\mathbb{Q}_x=\{q\in\mathbb{Q}:x\in U_q\}$, which is the set of all rational indices whose corresponding open sets contain $x$. By definition, $\mathbb{Q}_x$ contains no rational numbers less than $0$ and all rational numbers greater than $1$. Since $\mathbb{Q}_x$ is bounded below, we define
\begin{align*}
	f(x)=\inf\mathbb{Q}_x = \inf\{q\in\mathbb{Q}:x\in U_q\}, x\in X.
\end{align*}
Then $f$ is bounded by $[0,1]$. For any $x\in A$, $x\in U_0$, and $\mathbb{Q}_x=\mathbb{Q}\cap[0,\infty)$. For any $x\in B$, $x\notin U_1$, and $\mathbb{Q}_x=\mathbb{Q}\cap[1,\infty)$. Hence $f(A)=\{0\}$, and $f(B)=\{1\}$.
\vspace{0.12cm}

\textit{Step IV:} It remains to show that $f:X\to[0,1]$ is continuous: 
given a point $x_0$ of $X$ and an open interval $(a,b)$ that contains $f(x_0)$, we show that there exists a neighborhood $U$ of $x_0$ such that $f(U)\subset (a,b)$. 

Let $p$ and $q$ be two rational numbers with $a<p<f(x_0)<q<b$. Since $f(x_0)<q$, we have $q\in\mathbb{Q}_{x_0}$, which implies $x_0\in U_q$. Moreover, we can find some $p^\prime\in\mathbb{Q}$ such that $p<p^\prime<f(x_0)$, which implies $p^\prime\notin\mathbb{Q}_{x_0}$, and $x_0\notin U_{p^\prime}\supset \overline{U}_{p}$. Hence $x_0\in U_q\backslash\overline{U}_p$.

Now we prove that $U=U_q\backslash\overline{U}_p$ is the desired neighborhood of $x_0$. Fix $x\in U$. Since $x\in U_q$, $q\in\mathbb{Q}_x$, and $f(x)\leq q < b$. Moreover, $x\notin\overline{U}_p$ implies $p\notin\mathbb{Q}_x$, and $f(x)\geq p$. Hence $f(x)\in[p,q]\subset(a,b)$, as desired.
\end{proof}

The following Tietze extension theorem is a generalization of \hyperref[thm:2.12]{Theorem 2.12}. It is an immediate corollary of the Urysohn lemma.

\paragraph{Theorem 5.16\label{thm:5.16}} (Tietze extension theorem). Let $X$ be a normal space, and let $A$ be a closed subspace of $X$.
\begin{itemize}
	\item[(i)] Any continuous map of $A$ into the closed interval $[a,b]$ of $\mathbb{R}$ can be extended to a continuous map of all of $X$ into $[a,b]$ (with the same bound).
	\item[(ii)] Any continuous map of $A$ into $\mathbb{R}$ may be extended to a continuous map of all	of $X$ into $\mathbb{R}$.
\end{itemize}

\begin{proof}
\textit{Step I:} Let $A$ be a closed subset in a metric space $X$, and $f:A\to\mathbb{R}$ a continuous function. We first consider a bounded function $f:A\to\mathbb{R}$, with $\sup_{x\in A}\vert f(x)\vert\leq M < \infty$. We claim that there exists a continuous function $g:X\to\mathbb{R}$ such that
\begin{align*}
	\sup_{x\in X}\vert g(x)\vert \leq \frac{1}{3}M,\ \ \sup_{x\in A}\vert f(x)-g(x)\vert \leq \frac{2}{3}M.
\end{align*}
Let $U=\{x\in A:f(x)\geq M/3\}$ and $L=\{x\in A:f(x)\leq -M/3\}$. By \hyperref[lemma:1.25]{Lemma 1.25}, $U$ and $L$ are disjoint closed subsets of $A$ in $X$. Using the Urysohn lemma (\hyperref[thm:5.16]{Theorem 5.16}), we can find a continuous function $g:X\to[-M/3,M/3]$ such that $g(U)=\{M/3\}$ and $g(L)=\{-M/3\}$. Clearly, $g$ is the desired function.
\vspace{0.12cm}

\textit{Step II.} Now we prove the part (i). Without loss of generality, let $f:A\to [-1,1]$. By step (i), there exists a continuous function $g_1$ such that
\begin{align*}
	\sup_{x\in X}\vert g_1(x)\vert \leq \frac{1}{3},\ \ \sup_{x\in A}\vert f(x)-g_1(x)\vert \leq \frac{2}{3}.
\end{align*}
Now consider function $f-g_1$. Applying Step I again, we can find a continuous function $g_2:X\to\mathbb{R}$ such that
\begin{align*}
	\sup_{x\in X}\vert g_2(x)\vert \leq \frac{2}{9},\ \ \sup_{x\in A}\vert f(x) - g_1(x) - g_2(x)\vert \leq \frac{4}{9}.
\end{align*}
Repeat this procedure, we obtain a sequence of functions $g_n:X\to\mathbb{R}$ such that
\begin{align*}
	\sup_{x\in X}\vert g_n(x)\vert \leq \frac{1}{3}\left(\frac{2}{3}\right)^{n-1},\ \ \sup_{x\in A}\left\vert f(x) - g_1(x) - \cdots - g_n(x)\right\vert \leq \left(\frac{2}{3}\right)^{n}.
\end{align*}
By Weierstrass M-test (\hyperref[thm:2.20]{Theorem 2.20}), the series $\sum_{n=1}^\infty g_n$ converges uniformly on $X$, and we denote it by $g$. For any $x\in A$, we have
\begin{align*}
	\left\vert f(x) - \sum_{k=1}^n g_k(x)\right\vert \leq\left(\frac{2}{3}\right)^n \to 0\ \text{as}\ n\to\infty.
\end{align*}
Then $g$ and $f$ agree on $A$, and $g$ is uniformly bounded by $\sum_{n=1}^\infty 2^{n-1}/3^n = 1$. Furthermore, $g$ is continuous by \hyperref[thm:2.19]{Theorem 2.19}. Hence $g$ extends $f$ to the whole space $X$. 
\vspace{0.15cm}
	
\textit{Step III:} Now we prove the case for general continuous $f$, which is the part (ii). Choose a homeomorphism $h:\mathbb{R}\to(-1,1)$ and consider the composition $h\circ f$, which is bounded. Then we extend it to a continuous $\mathbb{R}$-valued function $g:X\to[-1,1]$ as in Step II. 

We need to deal with the endpoints. Let $E=g^{-1}(\{-1,1\})$, which is closed by continuity of $g$. Since $g$ and $h\circ f$ agrees on $A$, $g(A)\subset(-1,1)$. Then $A$ and $E$ are disjoint closed sets, and we can find a continuous function $\phi:X\to\mathbb[0,1]$ such that $\phi(E)=\{0\}$ and $\phi(A)=\{1\}$ by the Urysohn lemma. Then we have $g(x)\phi(x)\in(-1,1)$ for all $x\in X$, and $g(a)\phi(a)=g(a)$ for all $a\in A$. Now $h^{-1}\circ(g\cdot\phi)$ is well-defined and continuous, and by construction it extends $f$ over $X$. Thus we complete the proof.
\end{proof}

\newpage
\section{Product Spaces}
\subsection{The Product Topology}
In \hyperref[def:3.9]{Definition 3.9}, we have introduced the concepts of set product and the box topology. However, the box topology does not behave well in some cases. For example, in the box topology, an infinite product of compact spaces is not necessarily compact. In this section, we are going to introduce another topology on products of topological spaces, which enjoys some good properties.

\paragraph{Definition 6.1\label{def:6.1}} (Product topology). Let $X=\prod_{\alpha\in J}X_\alpha$ be the product of topological spaces $\{X_\alpha,\alpha\in J\}$. Let $\mathscr{S}_\beta$ denote the collection
\begin{align*}
	\mathscr{S}_\beta=\{\pi_\beta^{-1}(U_\beta):\ U_\beta\ \text{is open in}\ X_\beta\},
\end{align*}
and let $\mathscr{S}=\bigcup_{\beta\in J}\mathscr{S}_\beta$ be the union of these collections. Then the topology generated by the subbasis $\mathscr{S}$ is called the \textit{product topology}. In this topology $X=\prod_{\alpha\in J}X_\alpha$ is called a \textit{product space}.

\paragraph{Remark.} Let's consider the basis $\mathscr{B}$ that $\mathscr{S}$ generates. By \hyperref[thm:1.18]{Theorem 1.18}, $\mathscr{B}$ consists of all finite intersections of elements of $\mathscr{S}$. Since a finite intersection of open sets are still open, we do not produce anything new if we intersect the elements from the same collection $\mathscr{S}_\beta$. Hence we intersect elements from different collections $\mathscr{S}_\beta$: Let $\beta_1,\cdots,\beta_n$ be a finite set of distinct indices from $J$, and let $U_{\beta_i}$ be an open set in $X_{\beta_i}$ for $i=1,\cdots,n$. Then we generate a typical element of $\mathscr{B}$ as
\begin{align*}
	B=\bigcup_{i=1}^n\pi_{\beta_i}^{-1}(U_{\beta_i})
\end{align*}
If a point $\mathbf{x}=(x_\alpha)$ belongs to $B$, then $x_{\beta_i}\in U_{\beta_i}$ for $i=1,\cdots,n$, and there is no restriction for $x_\alpha$ if $\alpha\notin\{\beta_1,\cdots,\beta_n\}$. Therefore, we can write $B$ as the product $B=\prod_{\alpha\in J}U_\alpha$, where $U_\alpha=X_\alpha$ for $\alpha\notin\{\beta_1,\cdots,\beta_n\}$.

\paragraph{Comparison of the box and product topologies.} The box topology on $\prod_{\alpha\in J} X_\alpha$ has as basis all sets of the form $\prod_{\alpha\in J}U_\alpha$, where $U_\alpha$ is open in $X_\alpha$ for each $\alpha\in J$. The product topology on $\prod_{\alpha\in J} X_\alpha$ has as basis all sets of the form $\prod_{\alpha\in J}U_\alpha$, where $U_\alpha$ is open in $X_\alpha$ for each $\alpha\in J$, and $U_\alpha\neq X_\alpha$ for only finitely many values of $\alpha\in J$. Clearly, two topologies are equivalent when the index set $J$ is finite.

As a result, we can generate the bases for the two topologies from the bases $\{\mathscr{B}_\alpha\}$ of topological spaces $\{X_\alpha,\alpha\in J\}$ as follows. The collection of all sets of the form $\prod_{\alpha\in J}B_\alpha$, where $B_\alpha\in\mathscr{B}_\alpha$ for all $\alpha\in J$, serves as a basis for the box topology on $\prod_{\alpha\in J}X_\alpha$. The collection of all sets of the form $\prod_{\alpha\in J}B_\alpha$, where $B_\alpha\in\mathscr{B}_\alpha$ for finitely many indices $\alpha$ and $B_\alpha=X_\alpha$ for all the remaining indices, serves as a basis for the product topology on $\prod_{\alpha\in J}X_\alpha$.

\paragraph{Lemma 6.2\label{lemma:6.2}} (Subspace topology of product spaces). Let $Y_\alpha\subset X_\alpha$ for each $\alpha\in J$. Then $\prod_{\alpha\in J}Y_\alpha$ is a subspace of $\prod_{\alpha\in J}X_\alpha$ if both products are given the product topology, or if both are given the box topology.
\begin{proof}
Check that if $\mathscr{B}$ is a basis for $\prod_{\alpha\in J}X_\alpha$, then $\{B\cap Y,B\in\mathscr{B}\}$ is a basis for $Y=\prod_{\alpha\in J}Y_\alpha$.
\end{proof}

\paragraph{Lemma 6.3.\label{lemma:6.3}} Let $A_\alpha\subset X_\alpha$ for each $\alpha\in J$. If $\prod_{\alpha\in J} X_\alpha$ is given either the box topology or the product topology, then
\begin{align*}
	\prod_{\alpha\in J}\overline{A}_\alpha = \overline{\prod_{\alpha\in J}A_\alpha}.
\end{align*}
\begin{proof}
Let $\mathbf{x}=(x_\alpha)$ be a point of $\prod_{\alpha\in J}\overline{A}_\alpha$. Let $U=\prod_{\alpha\in J}U_\alpha$ be a basis element for either the box or product topology that contains $\mathbf{x}$. Since $x_\alpha\in\overline{A}_\alpha$, there exists $y_\alpha\in U_\alpha\cap A_\alpha$ for each $\alpha\in J$. Then $\mathbf{y}=(y_\alpha)\in U\cap\prod_{\alpha\in J}A_\alpha$. Since $U$ is arbitrary, $\mathbf{x}\in\overline{\prod_{\alpha\in J}A_\alpha}$.

Conversely, let $\mathbf{x}=(x_\alpha)\in\overline{\prod_{\alpha\in J}A_\alpha}$ in either topology. Fix $\beta\in J$. Let $V_\beta$ be an arbitrary open set in $X_\beta$ that contains $x_\beta$. Then $\pi_\beta^{-1}(V_\beta)$ is a neighborhood of $\mathbf{x}$ in $\prod_{\alpha\in J}X_\alpha$, and it intersects $\prod_{\alpha\in J}A_\alpha$. Hence $V_\beta$ intersects $A_\beta$, and $x_\beta\in\overline{A}_\beta$.
\end{proof}

\paragraph{Remark.} In \hyperref[prop:4.17]{Proposition 4.17 (iii)} and \hyperref[lemma:5.11]{Lemma 5.11 (i)}, we use the term ``product'' and do not distinguish box topology and product topology. In fact, our proofs hold for both. Therefore, a product of Hausdorff/regular spaces is still Hausdorff/regular, under both box and product topologies.

\paragraph{} Now we introduce two important conclusions of product topologies.

\paragraph{Lemma 6.4.\label{lemma:6.4}} Let $X=\prod_{\alpha\in J}X_\alpha$ be a product space. Let $(\mathbf{x}_n)$ be a sequence of points of $X$. Then $\mathbf{x}_n\to\mathbf{x}$ if and only if $\pi_\alpha(\mathbf{x}_n)\to\pi_\alpha(\mathbf{x})$ for all $\alpha\in J$.
\begin{proof}
Consider the projection mapping $\pi_\alpha:\prod_{\alpha\in J}X_\alpha\to X_\alpha$. If $U_\alpha$ is open in $X_\alpha$, then $\pi^{-1}(U_\alpha)$ is also open. Since the projection mapping $\pi_\alpha:X\to X_\alpha$ is continuous, it preserves convergent sequences. 

Conversely, if $\pi_\alpha(\mathbf{x}_n)\to\pi_\alpha(\mathbf{x})$ for every $\alpha\in J$, we choose a basis element $U=\prod_{\alpha\in J}U_\alpha$ that contains $\mathbf{x}$. For every $\alpha\in J$ with $U_\alpha\neq X_\alpha$, choose $N_\alpha$ such that $\pi_\alpha(\mathbf{x}_n)\in U_\alpha$ for all $n\geq N_\alpha$. Since $\{N_\alpha\}$ is finite, let $N=\max\{N_\alpha:\alpha\in J,\,U_\alpha\neq X_\alpha\}$. Then for all $n\geq N$, $\mathbf{x}_n\in U$. Since $U$ is arbitrary, $\mathbf{x}_n\to\mathbf{x}$.
\end{proof}

\paragraph{Remark.} \hyperref[lemma:6.4]{Lemma 6.4} states that the coordinate convergence implies the convergence under the product topology. However, there is no similar conclusion for box topologies when $J$ is infinite, since $\{N_\alpha\}$ can be unbounded and we are not able to find an appropriate $N$.

\paragraph{Theorem 6.5.\label{thm:6.5}} Let $f:X\to\prod_{\alpha\in J}Y_\alpha$ be given by the equation
\begin{align*}
	f(x) = \bigl(f_\alpha(x)\bigr)_{\alpha\in J},
\end{align*}
where $f_\alpha:X\to Y_\alpha$ for each $\alpha\in J$. Give $\prod_{\alpha\in J}Y_\alpha$ the product topology. Then $f$ is continuous if and only if each function $f_\alpha$ is continuous.
\begin{proof}
Suppose $f$ is continuous. For any $\alpha\in J$, since $\pi_\alpha$ is continuous, $f_\alpha = \pi_\alpha\circ f$ is also continuous.

Conversely, suppose $f_\alpha$ is continuous for all $\alpha\in J$. By \hyperref[thm:2.5]{Theorem 2.5 (ii)}, it suffices to show that the inverse image of each element of the subbasis $\mathscr{S}$ is open in $X$:
\begin{align*}
	f^{-1}(\pi_\beta^{-1}(U_\beta)) = f_\beta^{-1}(U_\beta).
\end{align*}
Since $f_\beta$ is continuous, the inverse image is open in $X$. Then we conclude the proof.
\end{proof}

\subsection{The Tychonoff Theorem}
In \hyperref[thm:4.6]{Theorem 4.6}, we have shown that the product of finitely many compact spaces is compact. Naturally, we wonder if a similar conclusion holds for the product of infinitely many compact spaces. The Tychonoff theorem gives a positive answer, under the product topology.

Recall the definition of the \textit{finite intersection property}, which is given in \hyperref[thm:4.14]{Theorem 4.14}: A collection $\mathscr{C}$ of sets is said to have the finite intersection property, if for every finite subcollection $\{C_1,\cdots,C_n\}$ of $\mathscr{C}$, their intersection $\bigcap_{i=1}^n C_n$ is nonempty.

\paragraph{Lemma 6.6\label{lemma:6.6}} (Maximal element having the finite intersection property). Let $X$ be a nonempty set. Let $\mathscr{A}$ be a collection of subsets of $X$ having the finite intersection property. Then there is a collection $\mathscr{D}$ of subsets of $X$ such that $\mathscr{D}\supset\mathscr{A}$, and $\mathscr{D}$ has the finite intersection property, and no collection of subsets of $X$ that properly contains $\mathscr{D}$ has this property.

\begin{proof}
We construct $\mathscr{D}$ by Zorn's lemma: Suppose a partially ordered set $P$ has the property that every simply ordered subset of $P$ has an upper bound in $P$, then $P$ has at least one maximal element.

We use $C$ to denote a subset of $X$, use $\mathscr{C}$ to denote a collection of subsets of $X$, and use $\mathbb{C}$ to denote a superset whose elements are collections of subsets of $X$. We shall apply Zorn's lemma on such a superset.

By assumption, we have a collection $\mathscr{A}$ of subsets of $X$ that has the finite intersection property. Let $\mathbb{A}$ be the superset consists of all collections $\mathscr{B}$ of subsets of $X$ such that $\mathscr{B}\supset\mathscr{A}$ and $\mathscr{B}$ has the finite intersection property. We use proper inclusion $\subsetneq$ as our strict partial order on $\mathbb{A}$. To prove the conclusion, we need to show that $\mathbb{A}$ has a maximal element $\mathscr{D}$.

We show that if $\mathbb{B}$ is a ``subsuperset'' of $\mathbb{A}$ that is simply ordered by proper inclusion, then $\mathbb{B}$ has an upper bound in $\mathbb{A}$. Then the conclusion follows from Zorn's Lemma. In fact, the collection $\mathscr{C}=\bigcup_{\mathscr{B}\in\mathbb{B}}\mathscr{B}$ is the desired upper bound. To show this, it suffices to show that $\mathscr{C}\in\mathbb{A}$.

Clearly, $\mathscr{C}\supset\mathscr{A}$. To show the finite intersection property, choose finitely many $C_1,\cdots,C_n\in\mathscr{C}$. Since $\mathscr{C}$ is a union of elements of $\mathbb{B}$, we can find some $\mathscr{B}_i\in\mathbb{B}$ containing $C_i$ for each $i\in[n]$. Then we obtain a finite superset $\{\mathscr{B}_i\}_{i\in[n]}$, which is contained in a simply ordered set $\mathbb{B}$ and has a maximal element $\mathscr{B}_k$ such that $\mathscr{B}_k\supset\mathscr{B}_i$ for each $i$. Hence $C_1,\cdots,C_n\in\mathscr{B}_k$, and their intersection is nonempty, as desired.
\end{proof}

\paragraph{Lemma 6.7.\label{lemma:6.7}} Let $X$ be a nonempty set, and let $\mathscr{D}$ be a collection of subsets of $X$ that is maximal with respect to the finite intersection property. Then: (i) Every finite intersection of elements of $\mathscr{D}$ is an element of $\mathscr{D}$;
(ii) If $A$ is a subset of $X$ that intersects every element of $\mathscr{D}$, then $A$ is an element of $\mathscr{D}$.
\begin{proof}
(i) Let $B$ be an intersection of finitely many elements of $\mathscr{D}$, and define $\mathscr{E}=\mathscr{D}\cup\{B\}$. If we can show that $\mathscr{E}$ has the finite intersection property, then $\mathscr{E}=\mathscr{D}$ by maximality of $\mathscr{D}$, which implies $B\in\mathscr{D}$.

Take $E_1,\cdots,E_n\in\mathscr{E}$. The case $B\notin\{E_1,\cdots,E_n\}$ is clear. Without loss of generality, assume $E_1=B$. Then $B\cap E_2\cap\cdots\cap E_n$ is also an intersection of finite many elements of $\mathscr{D}$, hence is nonempty. 

(ii) Let $\mathscr{F}=\mathscr{D}\cup\{A\}$, it suffices to show $\mathscr{F}$ has the finite intersection property. Take $D_1,\cdots,D_n\in\mathscr{D}$. Since $\bigcap_{i=1}^n D_i$ is nonempty, and $A$ intersects every element of $\mathscr{D}$, $\bigl(\bigcap_{i=1}^n D_i\bigr)\cap A$ is nonempty.
\end{proof}

\paragraph{Theorem 6.8\label{thm:6.8}} (Tychonoff theorem). The product of any collection of compact topological spaces is compact in the product topology.
\begin{proof}
Let $\{X_\alpha\}_{\alpha\in J}$ be a collection of compact topological spaces, and define $X=\prod_{\alpha\in J} X_\alpha$. Let $\mathscr{A}$ be a collection of subsets of $X$ having the finite intersection property. Following \hyperref[thm:4.14]{Theorem 4.14}, it suffices to show that the intersection $\bigcap_{A\in\mathscr{A}}\overline{A}$ is nonempty.

By \hyperref[lemma:6.6]{Lemma 6.6}, we choose a collection $\mathscr{D}$ of subsets of $X$ such that $\mathscr{D}\supset\mathscr{A}$ and $\mathscr{D}$ is maximal with respect to the finite intersection property. We show that the intersection $\bigcap_{D\in\mathscr{D}}\overline{D}$ is nonempty.

For each $\alpha\in J$, the collection $\{\pi_\alpha(D):D\in\mathscr{D}\}$
of subsets of $X_\alpha$ also has the finite intersection property as $\mathscr{D}$ does. By compactness of $X_\alpha$, there exists $x_\alpha\in\bigcap_{D\in\mathscr{D}}\overline{\pi_\alpha(D)}$. Therefore, we are able to find a point $\mathbf{x}=(x_\alpha)_{\alpha\in J}$ of $X$. We shall show that $\mathbf{x}\in\overline{D}$ for every $D\in\mathscr{D}$, which concludes the proof.

Fix $\beta\in J$, and let $\pi_\beta^{-1}(U_\beta)$ be a subbasis element containing $\mathbf{x}$. By definition, $x_\beta\in\overline{\pi_\beta(D)}$ for all $D\in\mathscr{D}$. Then $U_\beta$, being a neighborhood of $x_\beta$ in $X_\beta$, intersects every $\pi_\beta(D)$. By \hyperref[lemma:6.7]{Lemma 6.7 (ii)}, $\pi_\beta^{-1}(U_\beta)\in\mathscr{D}$, and by \hyperref[lemma:6.7]{Lemma 6.7 (i)}, every basis element of the product topology that contains $\mathbf{x}$ belongs to $\mathscr{D}$. Since $\mathscr{D}$ has the finite intersection property, every basis element containing $\mathbf{x}$ intersects every element of $\mathscr{D}$. Hence $\mathbf{x}\in\overline{D}$ for every $D\in\mathscr{D}$ as desired.
\end{proof}

\newpage
\section{Metric Spaces and Function Spaces}
\subsection{Completeness and Compactness}
\paragraph{Definition 7.1\label{def:7.1}} (Cauchy sequence and completeness). Let $(X,d)$ be a metric space. A sequence $(x_n)$ of points of $X$ is said to be a \textit{Cauchy sequence}, if $\forall\epsilon >0$, $\exists N\in\mathbb{N}$ such that $n,m\geq N$ implies $d(x_n,x_m)<\epsilon$. The metric space $(X,d)$ is said to be \textit{complete} if every Cauchy sequence in $X$ converges with respect to $d$.

\paragraph{Lemma 7.2.\label{lemma:7.2}} Let $(X,d)$ be a metric space.
\begin{itemize}
\item[(i)] If $(X,d)$ is complete, $A$ is a closed subspace of $X$, and $d_A$ is the restricted metric of $A$, i.e. $d_A(x,y)=d(x,y)\ \forall x,y\in A$, then $(A,d_A)$ is a complete metric space.
\item[(ii)] Let $\bar{d}$ be the standard bounded metric, i.e. $\bar{d}(x,y)=\min\{d(x,y),1\}\ \forall x,y\in X$, then $(X,\bar{d})$ is complete if and only if $(X,d)$ is complete.
\end{itemize}
\begin{proof}
(i) Let $(x_n)$ be a Cauchy sequence in $A$ under $d_A$. Then $(x_n)$ is also a Cauchy sequence in $X$ under $d$, and it converges to some $x\in X$. By definition, any neighborhood $U$ of $x$ contains infinitely many points of $(x_n)$. Hence $x$ is a limit point of $A$. Since $A$ is closed, $x\in A$, and $(x_n)$ converges with respect to $d_A$.

(ii) Let $(x_n)$ be a sequence of points of $X$. Then the conclusion is clear if we can prove:
\begin{itemize}
	\item $(x_n)$ is a Cauchy sequence relative to $\bar{d}$ if and only if it is a Cauchy sequence relative to $d$, and
	\item For $x\in X$, $\bar{d}(x_n,x)\to 0$ if and only if $d(x_n,x)\to 0$.
\end{itemize}

We only prove the first claim. Since $\bar{d}$ is never greater than $d$, the ``if'' direction is clear. Conversely, let $(x_n)$ be a Cauchy sequence relative to $\bar{d}$. Given $\epsilon > 0$, we set $\epsilon^\prime=\min\{\epsilon,1/2\}$. Then there exists $N$ such that $n,m\geq N$ implies $\bar{d}(x_n,x_m)<\epsilon^\prime \leq\epsilon$. In this case, $d(x_n.x_m)=\bar{d}(x_n,x_m) < 1/2$, hence $(x_n)$ is also a Cauchy sequence relative to $d$. The proof for the second claim is similar.
\end{proof}

Now we introduce a criterion for a metric space to be complete.

\paragraph{Lemma 7.3\label{lemma:7.3}} (Subsequence criterion).  A metric space $(X,d)$ is complete if every Cauchy sequence in $X$ has a convergent subsequence.
\begin{proof}
Let $(x_n)$ be a Cauchy sequence in $X$, and let $(x_{n_k})$ be a convergent subsequence of $(x_n)$. Fix $\epsilon > 0$. We first choose a positive integer $N$ such that $n,m\geq N$ implies $d(x_n,x_m)<\epsilon / 2$.

Suppose that $(x_{n_k})$ converges to $x\in X$. We choose a sufficiently large integer $K$ so that $n_K\geq N$ and $k\geq K$ implies $d(x_{n_k},x)<\epsilon/2$. Then for any $n\geq N$, we have
\begin{align*}
	d(x_n,x) \leq d(x_{n_k},x_n) + d(x_{n_k},x) < \frac{\epsilon}{2} + \frac{\epsilon}{2} = \epsilon.
\end{align*}
Since $\epsilon$ is arbitrarily chosen, $(x_n)$ converges to $x$.
\end{proof}

\paragraph{Example.} Let $k\in\mathbb{N}$. The Euclidean space $(\mathbb{R}^k,d)$ is complete, where $d(\mathbf{x},\mathbf{y}) = \left(\sum_{i=1}^k (x_i-y_i)^2\right)^{1/2}$.

To see this, let $(\mathbf{x}_n)$ be a Cauchy sequence in $\mathbb{R}^k$. Fix $\epsilon = 1$, then we can find $N$ such that $n,m\geq N$ implies $d(\mathbf{x}_n,\mathbf{x}_m)<1$. Let $B$ be the closed ball of radius $\max\{d(\mathbf{0},\mathbf{x}_1),\cdots,d(\mathbf{0},\mathbf{x}_{N-1}),d(\mathbf{0},\mathbf{x}_N)+1\}$ centered at $\mathbf{0}$. Then $B$ is a compact set that contains the sequence $(\mathbf{x}_n)$. By \hyperref[thm:4.36]{Theorem 4.36}, $B$ is sequentially compact, hence $(x_n)$ has a convergent subsequence. By \hyperref[lemma:7.3]{Lemma 7.3}, the space $\mathbb{R}^k$ is complete.

\paragraph{} Now we discuss the compactness in metric spaces. In \hyperref[thm:4.36]{Theorem 4.36}, we have established the equivalence of four kinds of compactness in metric spaces, which is a very important conclusion. Following this, we are going to introduce another two kinds of ``compactness''.

\paragraph{Definition 7.4\label{def:7.4}} (Relatively compactness and relative sequential compactness). Let $(X,d)$ be a metric space, and let $A$ be a subset of $X$. 
\begin{itemize}
	\item[(i)] $A$ is said to be \textit{relatively compact} (or \textit{precompact}) if its closure is compact;
	\item[(ii)] $A$ is said to be \textit{relatively sequentially compact}, if for every sequence $(x_n)\subset A$ there exists a subsequence that converges to some $x\in X$. (Clearly, $x\in\overline{A}$.)
\end{itemize}

The following theorem reveals the equivalence of these two definitions.
\paragraph{Theorem 7.5\label{thm:7.5}} (Equivalence of relative sequential compactness and relative compactness). Let $(X,d)$ be a metric space. Let $A\subset X$. Then $A$ is relatively sequentially compact if and only if $A$ is relatively compact.
\begin{proof}
Suppose that $\overline{A}$ is compact. Then $\overline{A}$ is also sequentially compact by \hyperref[thm:4.36]{Theorem 4.36}, and the relative sequential compactness of $A$ is clear.

Conversely, suppose that $A$ is relatively sequentially compact. We show that $\overline{A}$ is sequentially compact. Let $(x_n)$ be a sequence of points of $\overline{A}$. For every $n\in\mathbb{N}$, since $x_n\in\overline{A}$, we can choose $y_n\in A$ such that $d(x_n,y_n)<1/n$. By relative sequential compactness of $A$, there is a subsequence $(y_{n_k})$ with $y_{n_k}\to y\in\overline{A}$. 

Fix $\epsilon > 0$. We first choose $K_1\in\mathbb{N}$ such that $d(y_{n_k},y)<\epsilon/2$ for all $k\geq K_1$. Then we choose $K_2\in\mathbb{N}$ such that $n_k > 2/\epsilon$ for all $k\geq K_2$, which implies $d(x_{n_k},y_{n_k})<\epsilon/2$. Hence $d(x_{n_k},y) < \epsilon$ for all $k\geq\max\{K_1,K_2\}$, and the subsequence $(x_{n_k})$ converges to $y\in\overline{A}$ as $k\to\infty$. Therefore $\overline{A}$ is sequentially compact.
\end{proof}

Next, we are going to establish the equivalence between relatively compact sets and totally bounded sets in complete metric spaces. Recall \hyperref[def:4.34]{Definition 4.34}: a metric space is said to be totally bounded if it has a finite $\epsilon$-net (a cover consists of open $\epsilon$-balls) for any $\epsilon > 0$.

Let $(X,d)$ be a metric space. Then a subset $A$ of $X$ is said to be \textit{totally bounded} if $A$ can be covered by finitely many $\epsilon$-balls in $X$ for any $\epsilon > 0$. Clearly, if $A$ is totally bounded, so is $\overline{A}$. To see this, let $\epsilon > 0$ be given. Then $A$ can be covered by finitely many $\frac{\epsilon}{2}$-balls. For any $x\in\overline{A}$, there exists $y\in A$ with $d(x,y)<\epsilon / 2$. Hence we can cover $\overline{A}$ by expanding the radii of the balls to $\epsilon$.

\paragraph{Theorem 7.6\label{thm:7.6}} (Hausdorff). Let $X$ be a metric space. Let $A\subset X$.
\begin{itemize}
	\item[(i)] If $A$ is relatively compact, then $A$ is totally bounded.
	\item[(ii)] If $X$ is complete and $A$ is totally bounded, then $A$ is relatively compact.
\end{itemize}
\begin{proof}
(i) Consider a cover of $\overline{A}$ consists of open $\epsilon$-balls, the conclusion is clear by finding a finite subcover.

(ii) We shall prove that $\overline{A}$ is sequentially compact. Let $(x_n)$ be a sequence of points, it suffices to construct a subsequence of $(x_n)$ that is a Cauchy sequence, which converges by completeness of $\overline{A}$.

We first cover $\overline{A}$ by finitely many $1$-balls. At least one of these balls, denoted by $O_1$, contains infinitely many elements of $(x_n)$. We denote by $J_1=\{n\in\mathbb{N}:x_n\in O_1\}$ the index set of these elements.

Next, cover $\overline{A}$ by finitely may $1/2$-balls. Since $J_1$ is infinite, at least one of these balls, denoted by $O_2$, contains infinitely many elements of $\{x_n:n\in J_1\}$. Similarly, let $J_2=\{n\in J_1:x_n\in O_2\}$. By repeating this procedure, we obtain a finite cover of $\overline{A}$ by $1/k$-balls and an infinite index set $J_k=\{n\in J_{k-1}:x_n\in O_{k}\}$ for arbitrarily large $k$, with $J_k\subset J_{k-1}\subset\cdots\subset J_1$.

Choose $n_1\in J_1$. Given $n_{k-1}$, choose $n_k\in J_k$ with $n_k > n_{k-1}$, which is feasible because $J_k$ is infinite. For any $l,m\geq k$, $n_l,n_m\in J_k$, and $x_{n_l},x_{n_m}\in O_k$, implying $d(x_{n_l},x_{n_m})<2/k$. Hence the subsequence $(x_{n_k})$ is a Cauchy sequence, as desired.
\end{proof}

\paragraph{Corollary 7.7.\label{cor:7.7}} A metric space $X$ is compact if and only if it is complete and totally bounded.
\begin{proof}
By \hyperref[lemma:7.3]{Lemma 7.3} (``only if'' part) and \hyperref[thm:7.6]{Theorem 7.6} (``if'' part).
\end{proof}

Now we are going to introduce some criterions for a metric space to be complete.
\paragraph{Lemma 7.8.\label{lemma:7.8}} Every Cauchy sequence in a metric space is totally bounded.
\begin{proof}
Let $(X,d)$ be a metric space, and let $(x_n)$ be a Cauchy sequence in $X$. Given $\epsilon > 0$, there exists $N$ such that $d(x_n,x_m)<\epsilon$ for all $n,m\geq N$, which implies $O(x_N,\epsilon)\supset\{x_n,n\geq N\}$. Then the open balls $O(x_1,\epsilon),\cdots,O(x_N,\epsilon)$ form a cover of $X$. 
\end{proof}

\paragraph{Theorem 7.9.\label{thm:7.9}} A metric space $X$ is complete if every totally bounded set in $X$ is relatively compact.
\begin{proof}
Let $(x_n)$ be a Cauchy sequence in $X$. By \hyperref[lemma:7.8]{Lemma 7.8}, $(x_n)$ is totally bounded. Then $(x_n)$ is relatively sequentially compact, and the completeness follows from \hyperref[lemma:7.3]{Lemma 7.3}.
\end{proof}

\paragraph{Lemma 7.10.\label{lemma:7.10}} If $A$ is a dense subset of a metric space $(X,d)$, and every Cauchy sequence in $A$ converges in $X$, then $(X,d)$ is complete.
\begin{proof}
Let $(x_n)$ be a Cauchy sequence in $X$. Since $\overline{A}=X$, there exists find $a_n\in A$ such that $d(a_n,x_n)<1/n$ for each $n\in\mathbb{N}$. Fix $\epsilon > 0$. Then there exists $N$ such that $d(x_n,x_m)<\epsilon/3$ for all $n,m\geq N$. By setting $n,m\geq\max\{N,3\epsilon^{-1}\}$, we have
\begin{align*}
	d(a_n,a_m) \leq d(a_n,x_n) + d(x_n,x_m) + d(x_m,a_m) < \epsilon.
\end{align*}
Hence $(a_n)$ is a Cauchy sequence in $A$, and it converges to some $x\in X$. Since $d(x_n,a_n)\to 0$, and $d(a_n,x)\to 0$, we have $d(x_n,x)\to 0$, which concludes the proof.
\end{proof}

Finally, we present a useful lemma in analysis.
\paragraph{Lemma 7.11.\label{lemma:7.11}} Any relatively compact set is separable.
\begin{proof}
Let $A$ be a relatively compact subset of a metric space $X$. Since $A$ is totally bounded, we can cover it by finitely many $1$-balls. We denote the centers of these balls by $C_1$. Similarly, we can cover $A$ by finitely many $1/n$-balls for arbitrarily large $n\in\mathbb{N}$, and extract their centers $C_n$. Then $\bigcup_{n=1}^\infty C_n$, being the union of countably many finite sets, is a countable dense set in $X$.
\end{proof}

\subsection{Completion of Metric Spaces}
\paragraph{Definition 7.12\label{def:7.12}} (Completion). Let $(X,d)$ be a metric space. A complete metric space $(Y,\tilde{d})$ is said to be a \textit{completion} of $(X,d)$, if there exists an injective mapping $\iota:X\to Y$ such that (i) $\iota$ is isometric, i.e. $\tilde{d}(\iota(x),\iota(x^\prime))=d(x,x^\prime)$ for any pair $x,x^\prime\in X$, and (ii) $\overline{\iota(X)}=Y$. In this case, $\iota$ is called an \textit{imbedding}.

\paragraph{} The following theorem states that every incomplete metric space has at least one completion.

\paragraph{Theorem 7.13\label{thm:7.13}} (Existence of a completion). Let $(X,d)$ be a metric space. Then there exists a completion of $(X,d)$. Namely, there exists an isometric imbedding from $X$ to a complete metric space.
\begin{proof}
We prove this theorem by constructing a complete metric space which consists of equivalence classes of Cauchy sequences in $X$.

\textit{Step I:} Let $Y^\prime$ be the set of all Cauchy sequences $\mathbf{x}=(x_1,x_2,\cdots)$ in $X$. Let $d^\prime(\mathbf{x},\mathbf{y}):=\lim_{n\to\infty}d(x_n,y_n)$. Then $d^\prime$ is a pseudometric on $Y^\prime$, that is, $d^\prime:Y^\prime\times Y^\prime\to\mathbb{R}_+$ satisfies symmetry and triangle inequality.

\vspace{0.12cm}
\textit{Step II:} Define a relation $\sim$ on $Y^\prime$: for $\mathbf{x}=(x_n)$ and $\mathbf{y}=(y_n)$ in $Y^\prime$, 
\begin{align*}
	\mathbf{x}\sim\mathbf{y}\ \overset{\mathrm{def.}}{\Leftrightarrow}\ \lim_{n\to\infty}d(x_n,y_n)=0.
\end{align*}
It is clear that $\sim$ is an equivalence relation on $Y^\prime$, i.e., $\sim$ has reflexivity, symmetry and transitivity. Let $\widetilde{Y}=Y^\prime/\sim$ be the set of equivalence classes on $Y^\prime$, and define $\tilde{d}:\widetilde{Y}\times\widetilde{Y}\to\mathbb{R}_+$ as
\begin{align*}
	\tilde{d}([\mathbf{x}],[\mathbf{y}])=\lim_{n\to\infty} d(x_n,y_n).
\end{align*}
Note that $\tilde{d}([\mathbf{x}],[\mathbf{y}])=d^\prime(\mathbf{x},\mathbf{y})$. Following Step I, $\tilde{d}$ is a metric on $\widetilde{Y}$.

\vspace{0.12cm}
\textit{Step III:} Define $\iota:X\to\widetilde{Y},x\mapsto[(x,x,\cdots)]$, which maps a point of $X$ to an equivalence class of a constant sequence. Clearly, $\tilde{d}(\iota(x),\iota(y))=d(x,y)$, which implies that $\iota$ is an isometric imbedding.

Now we show that $\overline{\iota(X)}=\widetilde{Y}$. Given any Cauchy sequence $\mathbf{x}=(x_n)\in Y^\prime$, we have
\begin{align*}
	\lim_{n\to\infty}\tilde{d}(\iota(x_n),[\mathbf{x}])=\lim_{n,m\to\infty} d(x_n,x_m) = 0,
\end{align*}
which implies $[\mathbf{x}]\in\overline{\iota(X)}$. Since $\mathbf{x}$ is arbitrary, we have $\overline{\iota(X)}=\widetilde{Y}$.

\vspace{0.12cm}
\textit{Step IV:} It remains to show the completeness of $(\widetilde{Y},\tilde{d})$. By \hyperref[lemma:7.10]{Lemma 7.10}, it suffices to show that every Cauchy sequence in $\iota(X)$ converges in $\widetilde{Y}$.

Let $\{[\mathbf{x}^{(n)}]\}_{n\in\mathbb{N}}$ be a Cauchy sequence in $\iota(X)$, where $\mathbf{x}^{(n)}=(x_n,x_n,\cdots)$ for each $n\in\mathbb{N}$. By definition, $\tilde{d}([\mathbf{x}^{(n)}],[\mathbf{x}^{(m)}]) = d(x_n,x_m)$, which implies that $\mathbf{x}=(x_n)$ is a Cauchy sequence in $X$. Moreover,
\begin{align*}
	\lim_{n\to\infty}\tilde{d}\left([\mathbf{x}^{(n)}],[\mathbf{x}]\right) = \lim_{n\to\infty}\left[\lim_{k\to\infty} d(x_n,x_k)\right] = 0,
\end{align*}
which implies $[\mathbf{x}^{(n)}]\to[\mathbf{x}]\in\widetilde{Y}$. Therefore we obtain a completion of $X$.
\end{proof}

By construction, we showed that every metric space has at least one completion. Naturally, we wonder if the completion is unique. We have the following theorem.

\paragraph{Theorem 7.14\label{thm:7.14}} (Uniqueness of the completion). The completion of a metric space $(X,d)$ is uniquely determined up to an isometry. Namely, if $\iota_1:X\to Y_1=\overline{\iota_1(X)}$ and $\iota_2:X\to Y_2=\overline{\iota_2(X)}$ are two isometric imbeddings from $X$ to a complete metric space, then there exists an isometric bijection from $Y_1$ to $Y_2$.
\begin{proof}
\textit{Step I:} Define map $\phi_0:\iota_1(X)\to\iota_2(X)$, $\iota_1(x)\mapsto\iota_2(x)$, which is bijective and isometric from $\iota_1(X)$ to $\iota_2(X)$. We extend $\phi_0$ to $\phi:Y_1\to Y_2$ as follows: Given $y_1\in Y_1$, choose a sequence $(x_n)$ of points of $X$ such that $d_{Y_1}(\iota_1(x_n),y_1)\to 0$, which is feasible because $Y_1=\overline{\iota_1(X)}$, and define $$\phi(y_1)=\lim_{n\to\infty}\phi_0(\iota_1(x_n))=\lim_{n\to\infty}\iota_2(x_n).$$
\vspace{0.12cm}

\textit{Step II:} check that $\phi$ is well-defined. Since $(\iota_1(x_n))$ converges to $y_1\in Y_1$, it is a Cauchy sequence in $Y_1$. Note that $\iota_1$ and $\iota_2$ are isometric, $(x_n)$ is a Cauchy sequence in $X$, and $(\iota_2(x_n))$ is a Cauchy sequence in $Y_2$. By completeness of $Y_2$, $(\iota_2(x_n))$ converges to some point $y_2$ of $Y_2$.

Suppose $(x_n^\prime)$ is another sequence of points of $X$ such that $\iota_1(x_n^\prime)\to y_1$. Repeat the above procedure, there exists $y_2^\prime\in Y_2$ such that $\iota_2(x^\prime_n)\to y_2^\prime$. Moreover,
\begin{align*}
	d_{Y_2}(y_2,y_2^\prime) = \lim_{n,m\to\infty}d_{Y_2}(\iota_2(x_n),\iota_2(x_m^\prime)) = \lim_{n,m\to\infty} d(x_n,x^\prime_m) = \lim_{n,m\to\infty}d_{Y_1}(\iota_1(x_n),\iota_1(x_m^\prime)) = d_{Y_1}(y_1,y_1)=0.
\end{align*}
Hence $y_2=y^\prime_2$. Therefore, $\phi:Y_1\to Y_2$ is well-defined and agrees with $\iota_2\circ\iota_1^{-1}$ on $\iota_1(X)$.

\paragraph{}
\textit{Step III:} It remains to show that $\phi$ is isometric. Given $y,y^\prime\in Y_1$, we choose two sequences $(x_n)$ and $(x_n^\prime)$ from $X$ such that $\iota_1(x_n)\to y$ and $\iota_1(x_n^\prime)\to y^\prime$. Then we have
\begin{align*}
	d_{Y_2}(\phi(y),\phi(y^\prime))=\lim_{n,m\to\infty}d_{Y_2}(\iota_2(x_n),\iota_2(x_m^\prime)) = \lim_{n,m\to\infty} d(x_n,x^\prime_m) = \lim_{n,m\to\infty}d_{Y_1}(\iota_1(x_n),\iota_1(x_m^\prime)) = d_{Y_1}(y,y^\prime).
\end{align*}
Hence $\phi$ is an isometric bijection from $Y_1$ to $Y_2$.
\end{proof}

\paragraph{Remark.} Combining \hyperref[thm:7.13]{Theorem 7.13} and \hyperref[thm:7.14]{Theorem 7.14}, we conclude that for every metric space, there exists a unique completion up to an isometry.

\subsection{Function Spaces and Arzelà–Ascoli Theorem}
Let $Y$ be a topological space, and let $J$ be a set. Then each point $\mathbf{y}=(y_\alpha)_{\alpha\in J}$ of the cartesian product $Y^J$ can be viewed as a function $\mathbf{y}:J\to Y,\alpha\mapsto y_\alpha$. In this section, we shift from the classical tuple notation $\mathbf{y}=(y_\alpha)_{\alpha\in J}$ to the functional notation $f:J\to Y$.

\paragraph{Definition 7.15\label{def:7.15}} (Uniform metric). Let $(Y,d)$ be a metric space, and let $\bar{d}(x,y)=\min\{d(x,y),1\}$ be the standard bounded metric on $Y$ derived from $d$. If $f,g\in Y^J$ are functions from $J$ to $Y$, let
\begin{align*}
	\bar{\rho}(f,g)=\sup_{\alpha\in J}\bar{d}\left(f(\alpha),g(\alpha)\right).
\end{align*}
Then $\bar{\rho}$ is said to be the \textit{uniform metric} on $Y^J$ corresponding to the metric $d$ on $Y$.

\paragraph{Theorem 7.16.\label{thm:7.16}} Let $(Y,d)$ be a complete metric space, and let $\bar{\rho}$ be the uniform metric on $Y^J$ corresponding to $d$. Then $(Y^J,\bar{\rho})$ is complete.
\begin{proof}
By \hyperref[lemma:7.2]{Lemma 7.2 (ii)}, $(Y,\bar{d})$ is a complete metric space, where $\bar{d}$ is the standard bounded metric corresponding to $d$. Suppose $\{f_n\}_{n\in\mathbb{N}}$ is a Cauchy sequence in $(Y^J,\bar{\rho})$. Then for any $\alpha\in J$, we have
\begin{align*}
	\bar{d}\left(f(\alpha),f(\alpha)\right)\leq\bar{\rho}\left(f_n,f_m\right),\ n,m\in\mathbb{N}.
\end{align*}
Since $\{f_n\}_{n\in\mathbb{N}}$ is a Cauchy sequence in $(Y^J,\bar{\rho})$, $(f_n(\alpha))$ is a Cauchy sequence in $(Y,\bar{d})$, which converges to some $f(\alpha)\in Y$ by completeness of $(Y,\bar{d})$. As a result, we obtain $f\in Y^J$ such that $f_n(\alpha)\to f(\alpha)$ in metric $\bar{d}$ for each $\alpha\in J$. We claim that $f_n\to f$ in metric $\bar{\rho}$.

Given $\epsilon > 0$, we choose a sufficiently large $N$ such that $\bar{\rho}(f_n,f_m)<\epsilon/2$ for all $n,m\geq N$. Then $\bar{d}\left(f_n(\alpha),f_m(\alpha)\right)<\epsilon/2$ holds for all $n,m \geq N$ and all $\alpha\in J$. Setting $m\to\infty$, we have $\bar{d}\left(f_n(\alpha),f(\alpha)\right)\leq\epsilon/2$ for all $n \geq N$ and all $\alpha\in J$. Therefore
\begin{align*}
	\bar{\rho}\left(f_n,f\right)=\sup_{\alpha\in J}\bar{d}\left(f_n(\alpha),f(\alpha)\right) \leq \frac{\epsilon}{2} < \epsilon,
\end{align*}
for all $n\geq N$, as desired.
\end{proof}

Up to now, we have not imposed any assumption on set $J$ when discussing the function space $Y^J$. Let's go one step further: consider the set $Y^X$ of all functions $f:X\to Y$, where $X$ is a topological space and $(Y,d)$ is a metric space. We are going to investigate two special subsets of $Y^X$:
\begin{itemize}
	\item $C(X,Y)$ is the set of all \textit{continuous} functions $f:X\to Y$. A function $f:X\to Y$ is said to be \textit{continuous} if the inverse image of any open set in $Y$ is open in $X$.
	\item  $B(X,Y)$ is the set of all \textit{bounded} functions $f:X\to Y$. A function $f:X\to Y$ is said to be \textit{bounded} if its image $f(X)$ is a bounded subset of $(Y,d)$.
\end{itemize}

\paragraph{Theorem 7.17.\label{thm:7.17}} Let $X$ be a topological space and let $(Y,d)$ be a metric space. Then both the set $C(X,Y)$ of all continuous functions and the set $B(X,Y)$ of all bounded functions are closed in $Y^X$ under the uniform metric $\bar{\rho}$ corresponding to $d$. As a result, both $C(X,Y)$ and $B(X,Y)$ are complete in $\bar{\rho}$ if $Y$ is complete in $d$.
\begin{proof}
We first show that if a sequence of elements $\{f_n\}$ of $Y^X$ converges to $f\in Y^X$ relative to $\bar{\rho}$, then it converges uniformly to $f$ relative to the standard bounded metric $\bar{d}$. Given $\epsilon >0$, we can choose $N\in\mathbb{N}$ such that $\bar{\rho}\left(f_n,f\right) < \epsilon$ for all $n\geq N$. Then $\bar{d}\left(f_n(x),f(x)\right) \leq \bar{\rho}\left(f_n,f\right) < \epsilon$ for all $n\geq N$ and all $x\in X$.

Now we prove that $C(X,Y)$ is closed in $Y^X$ under the topology induced by $\bar{\rho}$. Let $f\in\overline{C(X,Y)}$, then there exists a sequence $\{f_n\}\subset C(X,Y)$ that converges to $f$ relative to $\bar{\rho}$. As a result, $\{f_n\}$ converges to $f$ uniformly relative to $\bar{d}$. By the uniform limit theorem (\hyperref[thm:2.19]{Theorem 2.19}), $f$ is continuous. Hence $f\in C(X,Y)$.

\paragraph{} Finally, we prove that $B(X,Y)$ is closed in $Y^X$ under the topology induced by $\bar{\rho}$. Let $f\in\overline{B(X,Y)}$, then there exists a sequence $\{f_n\}\subset B(X,Y)$ that converges to $f$ relative to $\bar{\rho}$. Set $\epsilon=1/2$, then we can choose $N\in\mathbb{N}$ such that $\bar{\rho}\left(f_n,f\right)<1/2$ for all $n\geq N$. Since  $d\left(f_N(x),f(x)\right)=\bar{d}\left(f_N(x),f(x)\right)\leq\bar{\rho}\left(f_n,f\right)<1/2$ for all $x\in X$, and $f_N$ is bounded, we can bound the diameter of $f(X)$ as follows:
\begin{align*}
\sup_{x,x^\prime\in X} d(f(x),f(x^\prime)) &\leq \sup_{x,x^\prime\in X}\bigl\{d(f(x),f_N(x)) + d(f_N(x),f_N(x^\prime)) + d(f_N(x^\prime),f(x^\prime))\bigr\} \\
&< \sup_{x,x^\prime\in X} d(f_N(x),f_N(x^\prime)) + 1 < \infty.
\end{align*}
Hence $f\in B(X,Y)$, and $B(X,Y)$ is closed.

Following \hyperref[lemma:7.2]{Lemma 7.2 (i)} and \hyperref[thm:7.16]{Theorem 7.16}, we conclude that $C(X,Y)$ and $B(X,Y)$ are complete in $\bar{\rho}$ when $(Y,d)$ is complete.
\end{proof}

\paragraph{Definition 7.18\label{def:7.18}} (Supremum metric). Let $X$ be a topological space and let $(Y,d)$ be a metric space. If $f,g\in B(X,Y)$, define
\begin{align*}
	\rho(f,g)=\sup_{x\in X} d\left(f(x),g(x)\right).
\end{align*}
Then $\rho$ is a well-defined metric on $B(X,Y)$, since the set $f(X)\cup g(X)$, being the union of two bounded sets, is bounded. The metric $\rho$ is called the \textit{supremum metric}.

\paragraph{Remark.} (i) Suppose $f,g\in B(X,Y)$. Then:
\begin{align*}
\begin{cases}
\rho\left(f,g\right)\leq 1\ \Rightarrow\ \forall x\in X,\ d\left(f(x),g(x)\right)\leq 1\ \Rightarrow\ \forall x\in X,\ \bar{d}\left(f(x),g(x)\right)=d\left(f(x),g(x)\right) \Rightarrow\ \bar{\rho}\left(f,g\right)=\rho\left(f,g\right);\\
\rho\left(f,g\right) > 1\ \Rightarrow\ \exists x_0\in X,\ d\left(f(x_0),g(x_0)\right)> 1\ \Rightarrow\ \bar{d}\left(f(x_0),g(x_0)\right)=1 \Rightarrow\ \bar{\rho}(f,g)=1;
\end{cases}
\end{align*}
Hence we obtain a simple relation between the supremum metric $\rho$ and the uniform metric $\bar{\rho}$:
\begin{align*}
	\rho\left(f,g\right)=\min\{\bar{\rho}\left(f,g\right),1\}.
\end{align*}
Applying \hyperref[thm:7.16]{Theorem 7.16} and \hyperref[lemma:7.2]{Lemma 7.2 (ii)}, we conclude that $(B(X,Y),\rho)$ is complete if $(Y,d)$ is complete.

\paragraph{} (ii) Suppose $X$ is a compact space. Then for any $f\in C(X,Y)$, $f(X)$ is compact. Hence it is always bounded in $Y$, and $f\in B(X,Y)$. In this case, $C(X,Y)$ is a closed subspace of $B(X,Y)$. Furthermore, if the metric space $(Y,d)$ is complete, then $(C(X,Y),\rho)$ is also complete.

\paragraph{} In the remaining part of this section, we investigate the compactness in function spaces.

\paragraph{Definition 7.19\label{def:7.19}} (Equicontinuity). Let $X$ be a topological space and let $(Y,d)$ be a metric space. Let $\mathcal{F}$ be a subset of the function space $C(X,Y)$. Given $x_0\in X$, the function class $\mathcal{F}$ is said to be \textit{equicontinuous at} $x_0$ if for any $\epsilon > 0$, there exists a neighborhood $U$ of $x_0$ such that $d(f(x),f(x_0)) < \epsilon$ for all $x\in U$ and all $f\in\mathcal{F}$. If $\mathcal{F}$ is equicontinuous at $x_0$ for all $x_0\in X$, then $\mathcal{F}$ is said simply to be \textit{equicontinuous}.

\paragraph{Remark.} By definition, any finite subset of $C(X,Y)$ is equicontinuous. Furthermore, equicontinuity depends on the specific metric $d$ rather than merely on the topology of $Y$.

\paragraph{Lemma 7.20.\label{lemma:7.20}} Let $X$ be a topological space and let $(Y,d)$ be a metric space. Let $\mathcal{F}$ be a subset of $C(X,Y)$. If $\mathcal{F}$ is totally bounded under the uniform metric corresponding to $d$, then $\mathcal{F}$ is equicontinuous under $d$.
\begin{proof}
Suppose $\mathcal{F}$ is totally bounded. Given $\epsilon\in (0,1)$ and $x_0\in X$, we wish to find a neighborhood $U$ of $x_0$ such that $d(f(x),f(x_0))<\epsilon$ for all $x\in U$ and all $f\in\mathcal{F}$.

Since $\mathcal{F}$ is totally bounded, we cover $\mathcal{F}$ by finitely many $\epsilon/3$-balls
\begin{align*}
	O\left(f_1,\frac{\epsilon}{3}\right),\cdots,O\left(f_n,\frac{\epsilon}{3}\right)
\end{align*}
in $C(X,Y)$. For each $j=1,\cdots,n$, $f_j$ is continuous, and we can choose a neighborhood $U_j$ of $x_0$ such that $d(f_j(x),f_j(x_0)) < \epsilon/3$ for all $x\in U_j$. We show that $U=\bigcap_{j=1}^n U_j$ is the desired neighborhood of $x_0$.

Let $f$ be any function in $\mathcal{F}$. Then there exists $1\leq k\leq n$ such that $\bar{\rho}\left(f,f_k\right)<\epsilon/3$, where $\bar{\rho}$ is the uniform metric corresponding to $d$. By definition, we have for all $x\in X$ that
\begin{align*}
	d\left(f(x),f_k(x)\right) = \overline{d}\left(f(x),f_k(x)\right) \leq \bar{\rho}\left(f,f_k\right) < \frac{\epsilon}{3}.
\end{align*}
For all $x\in U$,  $x\in U_k$. Therefore
\begin{align*}
	d\left(f(x),f(x_0)\right) \leq \underbrace{d\left(f(x),f_k(x)\right)}_{<\epsilon/3} + \underbrace{d\left(f_k(x),f_k(x_0)\right)}_{<\epsilon/3} + \underbrace{d\left(f_k(x_0),f(x_0)\right)}_{<\epsilon/3} < \epsilon,
\end{align*}
which concludes the proof.
\end{proof}

The converse of \hyperref[lemma:7.20]{Lemma 7.20} holds if we introduce some additional assumptions.

\paragraph{Lemma 7.21.\label{lemma:7.21}} Let $X$ be a compact space and let $(Y,d)$ be a compact metric space. Let $\mathcal{F}$ be a subset of $C(X,Y)$. If $\mathcal{F}$ is equicontinuous under $d$, then $\mathcal{F}$ is totally bounded under both the uniform and supremum metrics corresponding to $d$.

\begin{proof}
Since $X$ is compact, the supremum metric $\rho$ corresponding to $d$ is well-defined on $C(X,Y)$. Since the uniform metric $\bar{\rho}$ never exceeds the supremum metric $\rho$, and every $\epsilon$-ball under $\rho$ is also an $\epsilon$-ball under $\bar{\rho}$ whenever $\epsilon<1$, total boundedness does not vary between $\rho$ and $\bar{\rho}$. Hence we use $\rho$ throughout.

Assume $\mathcal{F}$ is equicontinuous. Given $\epsilon > 0$, we wish to find finitely many functions $f_1,\cdots,f_N \in C(X,Y)$ such that for every $f\in\mathcal{F}$, there exists $l\in\{1,\cdots,N\}$ such that $\rho(f,f_l)<\epsilon$.

For each $z\in X$, we can find an open set $U_z$ containing $z$ such that $d(f(x),f(z))<\epsilon/3$ for all $x\in U_z$ and $f\in\mathcal{F}$ by equicontinuity of $\mathcal{F}$. Since $X$ is compact, we can choose finitely many $z_1,\cdots,z_k\in X$ with $\bigcup_{i=1}^k U_{z_i}\supset X$. For every $x\in X$, there exists $z_i$ such that $d(f(x),f(z_i))<\epsilon/3$.

For the compact metric space $(Y,d)$, we choose finitely many $\epsilon/6$-balls $O_1,\cdots,O_m$ that cover the whole space. Now for every tuple $(i,j)\in\{1,\cdots,k\}\times\{1,\cdots,m\}$, if there exists $f\in \mathcal{F}$ such that $f(z_i)\in O_j$, then we set this function as $f_{ij}$. We prove that $\{f_{ij},1\leq i\leq k, 1\leq j\leq m\}\subset\mathcal{F}$ is the desired finite subset.

Given $x\in X$, we choose $i\in\{1,\cdots,k\}$ such that $d(f(x),f(z_i))<\epsilon/3$ for all $f\in\mathcal{F}$. Fix $f\in\mathcal{F}$, then $f(z_i)$ is contained in some $O_j$, and we have $f_{ij}$ such that $f_{ij}(z_i)\in O_j$. As a result, $d(f(z_i),f_{ij}(z_i)) < \epsilon/3$, and
\begin{align*}
	d\left(f(x),f_{ij}(x)\right) \leq d\left(f(x),f(z_i)\right) + d\left(f(z_i),f_{ij}(z_i)\right) + d\left(f_{ij}(z_i),f_{ij}(x)\right) < \epsilon.
\end{align*}
Since $x$ is arbitrary, we have $\rho(f,f_{ij})<\epsilon$.  Hence $\mathcal{F}$ is covered by the $\epsilon$-balls centered at $\{f_{ij}\}$, as desired.
\end{proof}

\paragraph{Definition 7.22\label{def:7.22}} (Pointwise boundedness). Let $X$ be a set and let $(Y,d)$ be a metric space. A subset $\mathcal{F}$ of $C(X,Y)$ is said to be \textit{pointwise bounded} if for each $x\in X$, the subset $\mathcal{F}(x)=\{f(x):f\in\mathcal{F}\}$ of $Y$ is bounded.

\paragraph{} The Arzelà–Ascoli theorem characterizes relatively compact subsets in space $C(X,Y)$, where $X$ is compact and $Y$ is a metric space in which all closed bounded subsets are compact.

\paragraph{Theorem 7.23\label{thm:7.23}} (Arzelà–Ascoli). Let $X$ be a compact space. Let $(Y,d)$ be a metric space in which all closed bounded subspaces are compact, and give $C(X,Y)$ the corresponding uniform topology. Then a subset $\mathcal{F}$ of $C(X,Y)$ is relatively compact if and only if $\mathcal{F}$ is equicontinuous and pointwisely bounded.
\begin{proof}
Since $X$ is compact, the uniform topology on $C(X,Y)$ is induced by the supremum metric $\rho$ corresponding to $d$, which is well-defined.
\vspace{0.12cm}

\textit{Step I:} We first show that $\mathcal{F}$ is equicontinuous and pointwise bounded when it is relatively compact, from which the ``only if'' direction follows.

By \hyperref[thm:7.6]{Theorem 7.6 (i)}, $\mathcal{F}$ is totally bounded under both $\rho$ and $\bar{\rho}$. Following \hyperref[lemma:7.20]{Lemma 7.20}, it is equicontinuous under $d$. To prove the pointwise boundedness, note that $d(f(x),g(x))\leq\rho(f,g)$ for all $f,g\in\mathcal{F}$ and all $x\in X$. Hence the diameter of $\mathcal{F}(x)$ never exceeds that of $\mathcal{F}$, which is bounded by compactness.

\vspace{0.12cm}
\textit{Step II:} We show that if $\mathcal{F}$ is equicontinuous and pointwise bounded, so is $\overline{\mathcal{F}}$.

Suppose $\mathcal{F}$ is equicontinuous, and let $g\in\overline{\mathcal{F}}$. For any $\epsilon > 0$, there exists $f_g\in\mathcal{F}$ such that $\rho(f_g,g)<\epsilon/3$. Now fix $x_0\in X$. By equicontinuity of $\mathcal{F}$, there exists a neighborhood $U$ of $x_0$ in $X$ such that $d(f(x),f(x_0))<\epsilon/3$ for all $x\in U$ and $f\in\mathcal{F}$. Then we have
\begin{align*}
	d\left(g(x),g(x_0)\right) \leq d\left(g(x),f_g(x)\right) + d\left(f_g(x),f_g(x_0)\right) + d\left(f_g(x_0),g(x_0)\right) < \epsilon,\ \forall x\in U.
\end{align*}
Since $g$ is arbitrary, $\overline{\mathcal{F}}$ is equicontinuous at $x_0$.

Now suppose $\mathcal{F}$ is pointwise bounded. For any $\epsilon>0$ and $g,h\in\overline{\mathcal{F}}$, there exists $f_g,f_h\in\mathcal{F}$ such that $\rho(f_g,g)<\epsilon/2$ and $\rho(f_h,h)<\epsilon/2$. Fix $x\in X$, then
\begin{align*}
	d\left(g(x),h(x)\right) &\leq d\left(g(x),f_g(x)\right) + d\left(f_g(x),f_h(x)\right) + d\left(f_h(x),h(x)\right)\\
	&\leq \rho\left(f_g,g\right) + d\left(f_g(x),f_h(x)\right) + \rho\left(f_h,h\right) < \mathrm{diam}(\mathcal{F}(x)) + \epsilon.
\end{align*}
Taking supremum with respect to $g$ and $h$ yields $\mathrm{diam}(\mathcal{F}(x))\leq \mathrm{diam}(\overline{\mathcal{F}}(x)) \leq \mathrm{diam}(\mathcal{F}(x)) + \epsilon$ for all $\epsilon > 0$. Hence $\mathrm{diam}(\overline{\mathcal{F}}(x)) = \mathrm{diam}(\mathcal{F}(x)) < \infty$.

\vspace{0.12cm}
\textit{Step III:} Let $\mathcal{G}=\overline{\mathcal{F}}$. We show that there exists a compact subspace $K$ of $Y$ that contains the set $\mathcal{G}(X) = \bigcup_{g\in\mathcal{G}} g(X) = \bigcup_{x\in X}\mathcal{G}(x) = \{g(x): g\in\mathcal{G},x\in X\}$.

Fix $\epsilon > 0$. For each $z\in X$, we can choose an open neighborhood $U_z$ of $z$ such that $d(g(x),g(z))<\epsilon$ for all $x\in U_z$ and all $g\in\mathcal{G}$ by equicontinuity of $\mathcal{G}$. Following compactness of $X$, we choose finitely many $z_1,\cdots,z_n\in X$ such that $\bigcup_{k=1}^n U_{z_k}= X$. By pointwise boundedness of $\mathcal{G}$, $\mathcal{G}(z_k)$ is a bounded subset of $Y$ for each $k$, and the union $\bigcup_{k=1}^n\mathcal{G}(z_k)$ is also bounded. Take any $y\in\bigcup_{k=1}^n\mathcal{G}(z_k)$, then $O_Y(y,D)\supset\bigcup_{k=1}^n\mathcal{G}(z_k)$, where $D$ is the diameter of $\bigcup_{k=1}^n\mathcal{G}(z_k)$. As a result, $\mathcal{G}(X)$ is contained in the open ball $O_Y(y,D+\epsilon)$, and  $K=\overline{O_Y(y,D+\epsilon)}$ is the desired compact subspace of $Y$.

\vspace{0.12cm}
\textit{Step IV:} We show that $\mathcal{G}$ is complete and totally bounded. Then \hyperref[thm:7.6]{Theorem 7.6 (ii)} implies that $\mathcal{F}$ is relatively compact. Since $\mathcal{G}$ is a closed subspace of the complete space $(C(X,Y),\rho)$, it is complete. 

To show total boundedness of $\mathcal{G}$ under $\rho$, apply \hyperref[lemma:7.21]{Lemma 7.21}: Step II states that $\mathcal{G}$ is equicontinuous under metric $d$, and Step III implies that $\mathcal{G}\subset C(X,K)$, where $(K,d)$ is a compact metric space.
\end{proof}

We also have the following criterion for compactness.

\paragraph{Corollary 7.24.\label{cor:7.24}} Let $X$ be a compact space. Let $(Y,d)$ be a metric space in which all closed bounded subspaces are compact, and give $C(X,Y)$ the corresponding uniform topology. A subset $\mathcal{F}$ of $C(X,Y)$ is compact if and only if $\mathcal{F}$ is closed, bounded under $\rho$ and equicontinuous under $d$.
\begin{proof}
If $\mathcal{F}$ is bounded under $\rho$, then it is pointwise bounded under $d$. If we further assume $\mathcal{F}$ is equicontinuous under $d$, then $\mathcal{F}$ is relatively compact by \hyperref[thm:7.23]{Theorem 7.23}. Since $\mathcal{F}$ is closed, it is compact.

Conversely, if $\mathcal{F}$ is compact, then it is closed and bounded. By \hyperref[thm:7.23]{Theorem 7.23}, it is equicontinuous.
\end{proof}

\paragraph{Corollary 7.25\label{cor:7.25}} (Arzelà's theorem). Let $X$ be a compact space. Let $(Y,d)$ be a metric space in which all closed bounded subspaces are compact, and give $C(X,Y)$ the corresponding uniform topology. If a sequence $(f_n)$ is equicontinuous and pointwise bounded, then it has a uniformly convergent subsequence.
\begin{proof}
By \hyperref[thm:7.23]{Theorem 7.23}, the collection $\{f_n\}$ is relatively compact. Then it is relatively sequentially compact, i.e., it has a subsequence $(f_{n_k})$ under the supremum metric $\rho$ that converges to some $f\in C(X,Y)$, which implies $f_{n_k}\rightrightarrows f$.
\end{proof} 

\subsection{The Topologies of Pointwise and Compact Convergence}
In addition to the uniform topology, there are other useful topologies on the function spaces $Y^X$ and $C(X,Y)$.

\paragraph{Definition 7.26\label{def:7.26}} (Topology of pointwise convergence/point-open topology). Let $X$ be a set, and let $Y$ be a topological space. Given a point $x$ of $X$ and an open set $U$ in $Y$, let
\begin{align*}
S(x,U)=\{f\in Y^X: f(x)\in U\}.
\end{align*}
The sets $\{S(x,U):x\in X,\ U\ \text{is open in}\ Y\}$ are a subbasis for a topology on $Y^X$, which is called the \textit{topology of pointwise convergence} (or the \textit{point-open topology}).

\paragraph{Remark.} The set of all finite intersections of subbasis elements $S(x,U)$ forms a basis for the topology of pointwise convergence. Then a typical basis element containing the function $f$ consists of all functions $g$ that are close to $f$ at finitely many points. Note that if we use the tuple notation $\mathbf{y}=(y_\alpha)_{\alpha\in X}$ instead of the functional notation $f:X\to Y$, we have
\begin{align*}
	S(\beta,U)=\{\mathbf{y}=(y_\alpha)_{\alpha\in X}\in Y^X:y_\beta\in U\}=\pi_\beta^{-1}(U).
\end{align*}
It is seen that the topology of pointwise convergence is the product topology we introduced in \hyperref[def:6.1]{Definition 6.1}. The terminology ``topology of pointwise convergence'' follows from \hyperref[lemma:6.4]{Lemma 6.4}. We reformulate it below.

\paragraph{Lemma 7.27.\label{lemma:7.27}} A sequence $f_n:X\to Y$ of functions converges to the function $f$ in the topology of pointwise convergence if and only if for each $x\in X$, the sequence $f_n(x)$ of points of $Y$ converges to the point $f(x)$.

\paragraph{Remark.} Let $X$ be a topological space and let $Y$ be a metric space. In \hyperref[thm:7.17]{Theorem 7.17}, we have shown that the function space $C(X,Y)$ is closed in the uniform topology. However, it is not necessarily closed in the topology of pointwise convergence.

To see a counterexample, consider the sequence $(f_n)$ in $C([0,1],\mathbb{R})$, where $f_n(x)=x^n$ for $x\in[0,1]$. This sequence converges to the function $f=\chi_{(0,1]}$, which is not continuous at point $x=0$.

\paragraph{Definition 7.28} (Topology of compact convergence). Let $X$ be a topological space and let $(Y,d)$ be a metric space. Given a point $f$ of $Y^X$, a compact subspace $C$ of $X$, and a number $\epsilon > 0$, define $B_C(f,\epsilon)$ by
\begin{align*}
	B_C(f,\epsilon) = \left\{g\in Y^X:\sup_{x\in C}d\left(f(x),g(x)\right)<\epsilon\right\}.
\end{align*}
Then the sets $\{B_C(f,\epsilon):f\in Y^X,\epsilon>0\ \text{and}\ C\ \text{is a compact subset of}\ X\}$ form a basis for a topology on $Y^X$, which is called the \textit{topology of compact convergence} (or the \textit{topology of uniform convergence on compact sets}).

\paragraph{Remark.} We check that the sets $B_C(f,\epsilon)$ satisfy the conditions for a basis. It suffices to show that the intersection of any two basis elements $B_{C_1}(f_1,\epsilon_1)$ and $B_{C_2}(f_2,\epsilon_2)$ is a union of basis elements. We prove that for every $g\in B_{C_1}(f_1,\epsilon_1)\cap B_{C_2}(f_2,\epsilon_2)$, there exists a basis element containing $g$ and contained in this intersection. Let
\begin{align*}
	\delta = \min\left\{\epsilon_1 - \sup_{x\in C_1}d\left(f_1(x),g(x)\right),\epsilon_2 - \sup_{x\in C_2}d\left(f_2(x),g(x)\right)\right\} > 0.
\end{align*}
Then $B_{C_1\cup C_2}(g,\delta)\subset B_{C_1}(f_1,\epsilon_1)\cap B_{C_2}(f_2,\epsilon_2)$, since for any $h\in B_{C_1\cup C_2}(g,\delta)$,
\begin{align*}
	\sup_{x\in C_1}d\left(f_1(x),h(x)\right)\leq\sup_{x\in C_1}\left\{d\left(f_1(x),g(x)\right)+d\left(g(x),h(x)\right)\right\} \leq \sup_{x\in C_1}d\left(f_1(x),g(x)\right) + \delta < \epsilon_1,
\end{align*}
and the case for $B_{C_2}(f_2,\epsilon_2)$ is similar. 

In contrast to the topology of pointwise convergence, a typical basis element of the topology of compact convergence containing $f$ requires a function $g$ to be close to $f$ at all points of some compact set rather than merely at finitely many points. The choice of terminology ``topology of compact convergence'' is justified by the following lemma.

\paragraph{Lemma 7.29.\label{lemma:7.29}} A sequence $f_n:X\to Y$ of functions converges to the function $f$ in the topology of compact convergence if and only if for each compact subspace $C$ of $X$, the sequence $f_n|_C$ converges uniformly to $f|_C$.
\begin{proof}
For any compact subspace $C$ of $X$, we have
\begin{align*}
	f_n|_C \rightrightarrows f|_C\ \Leftrightarrow\ \sup_{x\in C} d\left(f_n(x),f(x)\right)\to 0\ \Leftrightarrow\ \forall \epsilon > 0,\ \exists N_\epsilon\in\mathbb{N}\ \text{such that}\ f_n\in B_C(f,\epsilon)\ \forall n\geq N_\epsilon.
\end{align*}
Then the proof is immediate.
\end{proof}

\paragraph{Review.} Up to now, we have three topologies for the function space $Y^X$ when $Y$ is a metric space. We summarize the form of basis elements in these topologies below.
\begin{itemize}
\item The topology of pointwise convergence: given finitely many points $x_1,\cdots,x_n$ of $X$, a function $f\in Y^X$ and a number $\epsilon > 0$,
\begin{align*}
	B(f,\epsilon;x_{1:n})=\left\{g\in Y^X:\max_{1\leq j\leq n} d\left(f(x_j),g(x_j)\right)<\epsilon\right\}.
\end{align*}
\item The topology of compact convergence: given a compact subspace $C$ of $X$, a function $f\in Y^X$ and a number $\epsilon > 0$,
\begin{align*}
	B_C(f,\epsilon)=\left\{g\in Y^X:\sup_{x\in C} d\left(f(x),g(x)\right)<\epsilon\right\};
\end{align*}
\item The uniform topology: given a function $f\in Y^X$ and a number $\epsilon>0$,
\begin{align*}
	B_\rho(f,\epsilon)=\left\{g\in Y^X: \rho\left(f,g\right)<\epsilon\right\}=\left\{g\in Y^X:\sup_{x\in X} d\left(f(x),g(x)\right)<\epsilon\right\};
\end{align*}
\end{itemize}
The relation between them is stated in the following theorem.

\paragraph{Theorem 7.30.\label{thm:7.30}} Let $X$ be a topological space, and let $Y$ be a metric space. For the function space $Y^X$, the topology $\mathscr{T}_\mathrm{p}$ of pointwise convergence, the topology $\mathscr{T}_\mathrm{c}$ of compact convergence, and the uniform topology $\mathscr{T}_\mathrm{u}$ admits the following inclusions:
\begin{align*}
	\mathscr{T}_\mathrm{p}\subset \mathscr{T}_\mathrm{c} \subset \mathscr{T}_\mathrm{u}.
\end{align*}
If $X$ is discrete, the first two coincide, and if $X$ is compact, the second two coincide.
\begin{proof}
For the first inclusion, note that any basis element for $\mathscr{T}_\mathrm{p}$ is also a basis element for $\mathscr{T}_\mathrm{c}$, since the finite-point sets are automatically compact. If $X$ is discrete, then only the finite-point sets are compact because all one-point sets are open, hence the bases for $\mathscr{T}_\mathrm{c}$ and $\mathscr{T}_\mathrm{p}$ are the same.

For the second inclusion, it suffices to show every basis element $B_C(f,\epsilon)$ for $\mathscr{T}_\mathrm{c}$ is an open set in $\mathscr{T}_\mathrm{u}$. For each $g\in B_C(f,\epsilon)$, set $\delta_g=\epsilon - \sup_{x\in C}d(f(x),g(x))$. Then $B_\rho(g,\delta_g)\subset B_C(g,\delta_g)\subset B_C(f,\epsilon)$, and $B_C(f,\epsilon)=\bigcup_{g\in B_C(f,\epsilon)}B_\rho(g,\delta_g)$ is an open set in $\mathscr{T}_\mathrm{u}$. If $X$ is compact, then every basis element for $\mathscr{T}_\mathrm{u}$ is also a basis element for $\mathscr{T}_\mathrm{c}$.
\end{proof}

\paragraph{Remark.} By \hyperref[thm:7.30]{Theorem 7.30}, the collections of neighborhoods in the three topologies also admit the same inclusions. Then we have the following relation between convergences:
\begin{align*}
	\text{Uniform convergence}\ \Rightarrow\ \text{Compact convergence}\ \Rightarrow\ \text{Pointwise convergence}.
\end{align*}

Now we introduce a special class of topological spaces called compactly generated spaces.

\paragraph{Definition 7.31\label{def:7.31}} (Compactly generated spaces). A space $X$ is said to be \textit{compactly generated} if it satisfies the following condition: a set $A$ is open in $X$ if $A\cap C$ is open in $C$ for each compact subspace $C$ of $X$.

\paragraph{Remark.} Another equivalent condition for compactly generated spaces requires a set $B$ to be closed in $X$ if $B\cap C$ is closed in $C$ for each compact subspace $C$ of $X$.

\paragraph{Lemma 7.32.\label{lemma:7.32}} If $X$ is locally compact, or if $X$ is first-countable, then $X$ is compactly generated.
\begin{proof}
Let $X$ be locally compact and let $A$ be a set such that $A\cap C$ is open in $C$ for each compact subspace $C$ of $X$. Given $x\in A$, choose an open neighborhood $U$ of $x$ such that $U$ lies in a compact subspace $C$ of $X$. Then $A\cap C$ is open in $C$ by assumption, and $A\cap U$ is open in $U$, hence is open in $X$. Therefore $A$ contains a neighborhood of $x$ in $X$ for each $x\in A$.

Let $X$ be first-countable and let $B$ be a set such that $B\cap C$ is closed in $C$ for each compact subspace $C$ of $X$. Given $x\in B$, we show that $x\in B$. Since $X$ has a countable basis at $x$, we can find a sequence $(x_n)$ of points of $B$ such that $x_n\to x$. The subspace $C=\{x_n,n\in\mathbb{N}\}\cap \{x\}$ is compact, because for each open set $U\ni x$, $C\backslash U$ contains only finitely many points of $C$. By assumption, $B\cap C$ is closed in $C$, and $x\in B\cap C$. Therefore, $x\in B$, and $B$ is closed, as desired.
\end{proof}

\paragraph{Lemma 7.33.\label{lemma:7.33}} Let $X$ be a compact generated space and let $f:X\to Y$. Then $f$ is continuous if the restriction $f|_C$ is continuous for each compact subspace $C$ of $X$.
\begin{proof}
Let $V$ be an open set in $Y$. We show that $f^{-1}(V)$ is open in $X$. The restriction $f|_C:C\to Y$ is continuous for each compact subspace $C$ of $X$, the set
\begin{align*}
	f|_C^{-1}(V)=f^{-1}(V)\cap C
\end{align*}
is open in $C$. Since $X$ is compactly generated, $f^{-1}(V)$ is open in $X$.
\end{proof}

A crucial fact about continuous functions on compactly generated spaces is the following.

\paragraph{Theorem 7.34.\label{thm:7.34}} Let $X$ be a compact generated space and let $(Y,d)$ be a metric space. Then $C(X,Y)$ is a closed subspace of $Y^X$ in the topology of compact convergence.
\begin{proof}
Let $f$ be a limit point of $C(X,Y)$, we wish to show that $f$ is continuous. By \hyperref[lemma:7.33]{Lemma 7.33}, it suffices to show that $f|_C$ is continuous for each compact subspace $C$ of $X$.

For each $n\in\mathbb{N}$, consider the neighborhood $B_C(f,1/n)$ of $f$, which intersects $C(X,Y)$. So we choose $f_n\in B_C(f,1/n)\cap C(X,Y)$. Then $f_n|_C:C\to Y$ converges uniformly to $f|_C$. By the uniform limit theorem, $f|_C$ is continuous.
\end{proof}

An immediate corollary is stated below.

\paragraph{Corollary 7.35.\label{cor:7.35}} Let $X$ be a compact generated space and let $(Y,d)$ be a metric space. If a sequence of continuous functions $f_n:X\to Y$ converges to a function $f\in Y^X$ in the topology of compact convergence, then $f$ is continuous.

\paragraph{} Now we consider another topology on function spaces.

\paragraph{Definition 7.36\label{def:7.36}} (Compact-open topology). Let $X$ and $Y$ be topological spaces. Given a compact subspace $C$ of $X$ and an open set $U$ of $Y$, define
\begin{align*}
	S(C,U) = \left\{f\in C(X,Y):f(C)\subset U\right\}.
\end{align*}
The sets $\{S(C,U):C\ \text{is a compact subspace of}\ X,\ U\ \text{is an open set in}\ Y\}$ form a subbasis for a topology on $C(X,Y)$ that is called the compact-open topology.

\paragraph{} Clearly, the compact-open topology is finer than the topology of pointwise convergence. In fact, we can establish the equivalence between the topology of compact convergence and the compact-open topology on the subspace $C(X,Y)$.

\paragraph{Theorem 7.37.\label{thm:7.37}}  Let $X$ be a space and let $(Y,d)$ be a metric space. On the set $C(X,Y)$,
the compact-open topology and the topology of compact convergence coincide.
\begin{proof}
\textit{Step I:} Given a subset $A$ of $Y$ and $\epsilon>0$, let $O(A,\epsilon)=\bigcup_{a\in A}O(a,\epsilon)$. If $A$ is a compact subset of $Y$ and $V$ is an open set containing $A$, then $U(A,\epsilon)\subset V$ for $\epsilon=\min_{a\in A}d(a,X\backslash V) > 0$.

\paragraph{}
\textit{Step II:} We first prove that the topology of compact convergence is finer than the compact-open topology. It suffices to show that every $S(C,U)$ is open in the topology of compact convergence. We show that for each $f\in S(C,U)$, there exists a basis element $B_C(f,\epsilon)$ contained in $S(C,U)$. Since $C$ is compact and $f$ is continuous, $f(C)$ is compact. By Step I, there exists $\epsilon > 0$ such that $O(f(C),\epsilon)\subset U$. Then for any $g\in B_C(f,\epsilon)$, $\sup_{x\in C}d(f(x),g(x))<\epsilon$, and $g(C)\subset O(f(C),\epsilon)\subset U$. Hence $B_C(f,\epsilon)\subset S(C,U)$.

\vspace{0.12cm}
\textit{Step III:} We then prove that the compact-open topology is finer than the topology of compact convergence. Since every open set in the topology of compact convergence contains some $B_C(f,\epsilon)$ for each of its element $f$, it suffices to find a basis element $B$ for the compact-open topology such that $f\in B\subset B_C(f,\epsilon)$.

For each $x\in X$, we can choose a neighborhood $U_x$ such that $f(U_x)\subset O(f(x),\epsilon/4)$ by continuity of $f$. Then $f(\overline{U}_x)\subset O(f(x),\epsilon/3)$. Since $C$ is compact, we cover it by finitely many $U_{x_1},\cdots,U_{x_n}$, and set $C_i=\overline{U}_{x_i}\cap C$, which is compact for each $i$. Let
\begin{align*}
	B=\bigcap_{i=1}^n S\left(C_i,O\left(f(x_i),\frac{\epsilon}{3}\right)\right).
\end{align*}

Clearly, we have $f(C_i)\subset f(\overline{U}_{x_i})\subset O(f(x_i),\epsilon/3)$ for each $i$. This implies $f\in B$. For any $g\in B$, we also have $g(C_i)\subset O(f(x_i),\epsilon/3)$. Given $x\in C$, we have $x\in C_i$ for some $i$. Then
$$d(f(x),g(x)) < d(f(x),f(x_i)) + d(f(x_i),g(x)) < 2\epsilon/3.$$ 
Hence $\sup_{x\in C} d(f(x),g(x))\leq 2\epsilon/3 < \epsilon,$ which implies $B\subset B_C(f,\epsilon)$.
Thus $B$ is the desired basis element.
\end{proof}

The following corollary is immediately implied by \hyperref[thm:7.37]{Theorem 7.37}.
\paragraph{Corollary 7.38.\label{cor:7.38}} Let $Y$ be a metric space. The topology of compact convergence on $C(X,Y)$ does not depend on the metric of $Y$. As a result, if $X$ is compact, the uniform
topology on $C(X,Y)$ does not depend on the metric of $Y$.

Another important fact about the compact-open topology is that it satisfies the requirement of joint continuity: the expression $f(x)$ is continuous jointly in both the variables $x$ and $f$.

\paragraph{Theorem 7.39\label{thm:7.39}} (Continuous evaluation map). Give $C(X,Y)$ the compact-open topology, and define the \textit{evaluation map} $e:X\times C(X,Y)\to Y$ by the equation $e(x,f)=f(x)$. If $X$ is locally compact Hausdorff, then the evaluation map $e$ is continuous.
\begin{proof}
Given a point $(x,f)$ of $X\times C(X,Y)$ and an open set $V$ in $Y$ containing the image point $e(x,f)=f(x)$, we wish to find a neighborhood $W$ of $(x,f)$ such that $e(W)\subset V$.

The function $f\in C(X,Y)$, the inverse image $f^{-1}(V)$ is open. By \hyperref[thm:4.40]{Theorem 4.40}, we can find an open set $U\ni x$ of which the closure $\overline{U}$ is compact and contained in $f^{-1}(V)$, since $X$ is locally compact Hausdorff. Then $f$ belongs to the subbasis element $S(\overline{U},V)$ for the compact-open topology, and $W=U\times S(\overline{U},V)$ is the desired neighborhood of $(x,f)$.
\end{proof}

\paragraph{Remark.} For any $x\in X$, the evaluation functional $e(x,\cdot):C(X,Y)\to Y$ is in fact the restriction of projection map $\pi_x:Y^X\to Y$ on $C(X,Y)$.

\paragraph{Definition 7.40\label{def:7.40}} (Induced map). Given a continuous function $f:X\times Z\to Y$, there is a corresponding function $F:Z\to C(X,Y)$, defined by the equation
\begin{align*}
	(F(z))(x)= f(x,z).
\end{align*}
Conversely, given $F:Z\to C(X,Y)$, this equation defines a corresponding function $f:X\times Z\to Y$. We say that $F$ is the map of $Z$ into $C(X,Y)$ that is \textit{induced} by $f$.

\paragraph{Remark.} We need to check that $F(z):x\mapsto f(x,z)$ is a continuous function from $X$ to $Y$. Let $V$ be an open set in $Y$. We show that $(F(z))^{-1}(V) = \left\{x:(x,z)\in f^{-1}(V)\right\}$ is an open set in $X$: for any $x\in (F(z))^{-1}(V)$, there is a basis element $U\times W$ containing $(x,z)$ and contained in $f^{-1}(V)$. Then $x\in U\subset(F(z))^{-1}(V)$.

\paragraph{Theorem 7.41.\label{thm:7.41}} Let $X$, $Y$ and $Z$ be topological spaces, and give $C(X,Y)$ the compact-open topology. If $f:X\times Z\to Y$ is continuous, then so is the induced map $F:Z\to C(X,Y)$. The converse holds if $X$ is locally compact Hausdorff.
\begin{proof}
Suppose $f:X\times Z$ is continuous. To prove the continuity of $F$, we take $z_0\in Z$ and show that the inverse image of any basis element $S(C,V)$ containing $F(z_0)$ is a neighborhood of $z_0$ in $Z$.

Since $F(z_0)\in S(C,V)$, we have $f(C\times\{z_0\})\subset V$. The continuity of $f$ implies that $f^{-1}(V)$ is open in $C\times Z$. As a result, the intersection $f^{-1}(V)\cap C\times Z$ is an open set in the subspace $C\times Z$ containing the slice $C\times \{z_0\}$. By tube lemma (\hyperref[lemma:4.5]{Theorem 4.5}), there exists a neighborhood $U_{z_0}$ of $z_0$ in $Z$ such that the tube $C\times U_{z_0}\subset f^{-1}(V)$. Then we have $f(x,z)\in V$ for any $(x,z)\in C\times U_{z_0}$, which implies $F(U_{z_0})\subset S(C,V)$.

Conversely, suppose $X$ is a locally compact Hausdorff space and $F$ is continuous. Then the evaluation map $e$ is continuous. Note that $f:X\times Z\to Y$ satisfies $f(x,z)=e(x,F(z))$. Hence $f$ as the composition $e\circ(i_X\times F)$ is continuous.
\end{proof}

Finally, we introduce a more general version of Arzelà-Ascoli theorem.
\paragraph{Theorem 7.42\label{thm:7.42}} (Ascoli's theorem). Let $X$ be a topological space and let $(Y,d)$ be a metric space. Give the space $C(X,Y)$ the topology of compact convergence. Let $\mathcal{F}$ be a subset of $C(X,Y)$.
\begin{itemize}
	\item[(i)] If $\mathcal{F}$ is equicontinuous under $d$ and the set $\mathcal{F}(x)=\{f(x):f\in\mathcal{F}\}$ is relatively compact for each $x\in X$, then $\mathcal{F}$ is a relatively compact subset of $C(X,Y)$.
	\item[(ii)] The converse holds if $X$ is locally compact Hausdorff.
\end{itemize}
\begin{proof}
(i) \textit{Step I:} We first show that $\mathcal{F}$ is relatively compact in the topology of pointwise convergence on $Y^X$. By assumption, $\overline{\mathcal{F}(x)}$ is a compact subspace of $Y$ for each $x\in X$. By the Tychonoff theorem (\hyperref[thm:6.8]{Theorem 6.8}), the product space
\begin{align*}
	\prod_{x\in X}\overline{\mathcal{F}(x)}
\end{align*}
is a compact set in the topology of pointwise convergence on $Y^X$ that contains $\mathcal{F}$.

Now we show that $Y^X$ given the topology of pointwise convergence is Hausdorff, which implies the relative compactness of $\mathcal{F}$. Given distinct functions $f,g\in Y^X$, there exists $x\in X$ such that $f(x)\neq g(x)$. Choose disjoint open sets $U\ni f(x)$ and $V\ni g(x)$ in $Y$, then $S(x,U)$ and $S(x,V)$ are disjoint subbasis elements that containing $f$ and $g$, respectively.

\vspace{0.12cm}
\textit{Step II:} Let $\mathcal{G}$ be the closure of $\mathcal{F}$ in the pointwise convergence topology on $Y^X$. We prove $\mathcal{G}\subset C(X,Y)$.

Let $g$ be a limit point of $\mathcal{F}$, $x_0\in X$, and $\epsilon > 0$. By equicontinuity of $\mathcal{F}$, there exists a neighborhood $U$ of $x_0$ such that $d(f(x),f(x_0))<\epsilon/3$ for all $x\in U$ and $f\in\mathcal{F}$. Then for any $x\in U$, the basis element $S\bigl(x,O(g(x),\epsilon/3)\bigr)\cap S\bigl(x_0,O(g(x_0),\epsilon/3)\bigr)$ intersects $\mathcal{F}$. Choose some $f$ from this intersection, we have 
$$d(g(x),g(x_0))\leq d(g(x),f(x))+d(f(x),f(x_0))+d(f(x_0),g(x_0))<\epsilon.$$

Hence $g$ is continuous. Moreover, the closure $\mathcal{G}$ is also equicontinuous.

\vspace{0.12cm}
\textit{Step III:} We prove that the topology of pointwise convergence and the topology of compact convergence coincides on $\mathcal{G}$. By \hyperref[thm:7.30]{Theorem 7.30}, the latter is finer than the former. We need to prove the converse.

Given a basis element $B_C(g,\epsilon)\cap\mathcal{G}$ for the topology of compact convergence, we wish to find a basis element $B\cap\mathcal{G}$ for the topology of pointwise convergence, which contains $g$ and is contained by $B_C(g,\epsilon)\cap\mathcal{G}$. Using the compactness of $C$ and the equicontinuity of $\mathcal{G}$, we choose finitely many $x_1,\cdots,x_n\in C$ such that for every $x\in X$, there exists $j$ satisfying
\begin{align*}
	d\left(h(x),h(x_j)\right) < \frac{\epsilon}{4},\ \forall h\in\mathcal{G}.\tag{*}
\end{align*}
Then we let the desired basis element be
\begin{align*}
	B = \bigcap_{j=1}^n S\left(x_j,O\left(g(x_j),\frac{\epsilon}{4}\right)\right).
\end{align*}
For any $h\in B\cap\mathcal{G}$ and $x\in C$, we choose an index $j$ satisfying (*), then
\begin{align*}
	d\left(g(x),h(x)\right) \leq d\left(g(x),g(x_j)\right) + d\left(g(x_j),h(x_j)\right) + d\left(h(x_j),h(x)\right) < \frac{3\epsilon}{4}.
\end{align*}
Then $\sup_{x\in C}d(g(x),h(x))\leq 3\epsilon/4 < \epsilon$, as desired.

\vspace{0.12cm}
\textit{Step IV:} By Steps I and III, $\mathcal{G}$ is also compact in the topology of compact convergence. Following Step I and \hyperref[thm:7.30]{Theorem 7.30}, the space $C(X,Y)$ given the the topology of compact convergence is also Hausdorff. Hence $\mathcal{F}\subset\mathcal{G}$ is relatively compact in the topology of compact convergence.

\vspace{0.25cm}
(ii) Let $\mathcal{F}$ be a relatively compact subset of $C(X,Y)$. We show that the closure $\mathcal{H}$ of $\mathcal{F}$ is equicontinuous under $d$ and that $\mathcal{H}(x)=\{h(x):h\in\mathcal{H}\}$ is compact for each $x\in X$.

We first show the equicontinuity of $\mathcal{H}$ under $d$. Given $x_0\in X$, choose a compact subspace $K$ of $X$ containing a neighborhood $U$ of $x_0$. It suffices to show the equicontinuity of $\mathcal{H}|_K=\{f|_K:f\in\mathcal{H}\}$ at $x_0$.

We define the restriction map $r_K:C(X,Y)\to C(K,Y),f\mapsto f|_K$, and give both spaces the topology of compact convergence. Then the inverse image of every basis element $B_C(f|_K,\epsilon)$ under $r_K$ contains $B_C(f,\epsilon)$, which implies the continuity of $r_K$. Hence $\mathcal{H}|_K=r_K(\mathcal{H})$ is compact. Since $K$ is compact, the uniform and compact convergence topologies coincide on $C(K,Y)$. By \hyperref[thm:7.7]{Theorem 7.7} is totally bounded under the uniform metric. The equicontinuity of $\mathcal{H}|_K$ follows from \hyperref[lemma:7.20]{Lemma 7.20}.

Next we show the compactness of $\mathcal{H}(x)=\{h(x):h\in\mathcal{H}\}$ for each $x\in X$. We define $j_x:C(X,Y)\to X\times C(X,Y),f\mapsto (x,f)$, which is clearly continuous. By \hyperref[thm:7.39]{Theorem 7.39}, the evaluation map $e:X\times C(X,Y)\to Y$ is continuous. Since $\mathcal{H}(x)=e(x,\mathcal{H})$ is the image of $\mathcal{H}$ under the composition $e\circ j$, it is compact.
\end{proof}

\newpage
\section{Miscellaneous}
\subsection{Net}
The concept of net is introduced to generalize the definition of sequence, which is viewed as a mapping from the natural number set $\bbN$ into a target set $X$.
\begin{definition}[Directed set and net]
A \textit{directed set} is a set $I$ equipped with a binary relation $\preceq$ such that
\begin{itemize}
	\item $i\preceq i$ for all $i\in I$;
	\item If $i\preceq j$ and $j\preceq k$, then $i\preceq k$;
	\item For every $i,j\in I$ there exists $k\in I$ such that $i\preceq k$ and $j\preceq k$.
\end{itemize}
We also write $j\succeq i$ for $i\preceq j$. A net in a set $X$ is a mapping $i\mapsto x_i$ from a directed set $I$ into $X$. We denote such a mapping by $(x_i)_{i\in I}$, and we call it a net \textit{indexed by $I$}.
\end{definition}
Like the sequences, we can discuss the convergence property of nets.
\begin{definition}[Nets in topological spaces]
Let $X$ be a topological space, $A\subset X$, and $x\in X$. Let $I$ be a directed set, and let $(x_i)_{i\in I}$ be a net in $X$.
\begin{itemize}
	\item $(x_i)_{i\in I}$ is said to be \textit{eventually in $A$} if there exists $i_0\in I$ such that $x_i\in A$ for all $i\succeq i_0$;
	\item $(x_i)_{i\in I}$ is said to be \textit{frequently in $A$} if for every $i\in I$ there exists $j\succeq i$ such that $x_j\in A$;
	\item $(x_i)_{i\in I}$ is said to \textit{converge to $x$} if it is eventually in every neighborhood of $x$;
	\item $x$ is said to be a \textit{cluster point of $(x_i)_{i\in I}$} if it is frequently in every neighborhood of $x$.
\end{itemize}
\end{definition}

\paragraph{Example.} Here are some examples of directed sets:
\begin{itemize}
	\item The set of positive integers $\bbN$, with $i\preceq j$ if and only if $i\leq j$.
	\item The set $\bbR\backslash\{a\}$, with $x\preceq y$ if and only if $\vert x-a\vert\geq\vert y-a\vert$. 
	\item The set of all partitions $(x_j)_{j=0}^n$ of a compact interval $[a,b]$ (i.e. $a=x_0\leq x_i\leq\cdots\leq x_n=b$), with $(x_j)_{j=0}^n\preceq(y_k)_{k=0}^m$ if and only if $\max_{1\leq j\leq n}\vert x_j-x_{j-1}\vert\geq \max_{1\leq k\leq m}\vert y_k-y_{k-1}\vert$. This set is used in the definition of the Riemann integral.
	\item The set $\mathscr{N}$ of all neighborhoods of a point $x$ in a topological space $X$, with $\mathscr{N}$ directed by reverse inclusion, i.e. $U\preceq V$ if and only if $U\supset V$. 
	\item The Cartesian product $I\times J$ of two directed sets, with $(i,j)\leq (i^\prime,j^\prime)$ if and only if $i\preceq i^\prime$ and $j\preceq j^\prime$.
\end{itemize}
\begin{proposition}\label{netacc}
Let $X$ be a topological space, $A\subset X$, and $x\in X$. Then $x$ is a limit point of $A$ if and only if there is a net in $A\backslash\{x\}$ that converges to $x$, and $x\in\ol{A}$ if and only if there is a net in $A$ that converges to $x$.
\end{proposition}
\begin{proof}
Let $x$ be a limit point of $E$, and let $\mathscr{N}$ be the set of neighborhoods of $x$ directed by reverse inclusion. For each $U\in\mathscr{N}$, take $x_U\in U\backslash\{x\}$. Then the net $x_U\to x$. Conversely, if $(x_i)_{i\in I}$ is a net in $A\backslash\{x\}$ that converges to $x$, then every punctured neighborhood of $x$ contains some $x_i$, and $x$ is a limit point of $A$. The second result follows by noting that $\ol{A}$ contains the set $A$ itself and the limit points of $A$.
\end{proof}

We provide an analogue of Theorem \hyperref[thm:5.2]{5.2}. The property can be generalized to nets, which allows us to drop the requirement of first countability.
\begin{proposition}\label{netconvcont}
Let $X$ and $Y$ be two topological spaces. A function $f:X\to Y$ is continuous at $x\in X$ if and only if for every net $(x_i)_{i\in I}$ converging to $x$, the net $(f(x_i))_{i\in I}$ converges to $f(x)$ in $Y$.
\end{proposition}

\begin{proof}
If $f$ is continuous at $x$ and $V\subset Y$ is a neighborhood of $f(x)$, then $f^{-1}(V)$ is a neighborhood of $x$. For any net $(x_i)_{i\in I}$ with $x_i\to x$, it is eventually in $f^{-1}(V)$, and $(f(x_i))_{i\in I}$ is eventually in $V$. Hence $f(x_i)\to f(x)$.

On the other hand, if $f$ is not continuous at $x$, there is a neighborhood $V$ of $f(x)$ such that $f^{-1}(V)$ is not a neighborhood of $x$. Hence $x\notin (f^{-1}(V))\mathring{}$, and $x\in\ol{f^{-1}(V^c)}$. By Proposition \ref{netacc}, there exists a net $(x_i)_{i\in I}$ in $f^{-1}(V^c)$ that converges to $x$. But then $f(x_i)\notin V$, so $f(x_i)\not\to f(x)$.
\end{proof}

Similarly, the definition of subnet generalizes the idea of subsequence. 
\begin{definition}
A \textit{subnet} of a net $(x_i)_{i\in I}$ is a net $(y_j)_{j\in J}$ together with a map $j\mapsto i_j$ from $J$ to $I$ such that (i) for every $i_0\in I$ there exists $j_0\in J$ such that $i_j\succeq i_0$ for all $j\succeq j_0$, and
(ii) $y_j=x_{i_j}$. We also write $(x_{i_j})_{j\in J}$ for the subnet. Clearly, if the net $(x_i)_{i\in I}$ converges to $x$, so does any subnet $(x_{i_j})_{j\in J}$.
\end{definition} 

\paragraph{Remark.} Though we use the term ``subnet'' here, we point out that the set $J$ can have greater cardinality than $I$, and the mapping $j\mapsto i$ may not be injective.
\begin{proposition}
If $(x_i)_{i\in I}$ is a net in a topological space $X$, then $x\in X$ is a cluster point of $(x_i)_{i\in I}$ if and only if $(x_i)_{i\in I}$ has a subnet that converges to $x$.
\end{proposition}
\begin{proof}
Let $(y_j)_{j\in J}=(x_{i_j})_{j\in J}$ be a subnet of $(x_i)_{i\in I}$ converging to $x$. Given any neighborhood $U$ of $x$, we choose $j_1\in J$ such that $y_j\in U$ for $j\succeq j_1$. Also, for any $i_0\in I$, we take $j_2\in J$ such that $i_j\succeq i_0$ for all $j\succeq j_2$. We take $j\in J$ with $j\succeq j_1$ and $j\succeq j_2$. Then $i_j\succeq i_0$ and $x_{i_j}=y_j\in U$. Hence $(x_i)_{i\in I}$ is frequently in $U$, and $x$ is a cluster point of $(x_i)_{i\in I}$.

Conversely, if $x$ is a cluster point of $(x_i)_{i\in I}$, let $\mathscr{N}$ be the set of neighborhoods of $x$ directed by reverse inclusion. To proceed, we consider the directed set $\mathscr{N}\times I$, where $(U,i)\preceq (U^\prime,i^\prime)$ if and only if $U\supset U^\prime$ and $i\preceq i^\prime$. For each $(U,j)\in\mathscr{N}\times I$, take $i_{(U,j)}\in I$ such that $i_{(U,j)}\succeq j$ and $x_{i_{(U,j)}}\in U$. Then if $(U^\prime,j^\prime)\succeq(U,j)$,
\begin{align*}
	i_{(U^\prime,j^\prime)}\succeq j^\prime\succeq j,\quad \text{and}\quad x_{i_{(U^\prime,j^\prime)}}\in U^\prime\subset U.
\end{align*}
Hence $(x_{i_{(U,j)}})_{(U,j)\in\mathscr{N}\times I}$ is a subnet that converges to $x$.
\end{proof}

Recall that any sequence in a Hausdorff space converges to at most one point. A similar property also holds for nets, and we can even characterize a Hausdorff space by this property.
\begin{proposition}[Characterization of Hausdorff space]
A topological space $X$ is Hausdorff if and only if every net in $X$ converges to at most one point.
\end{proposition}
\begin{proof}
Assume that $X$ is Hausdorff and $(x_i)_{i\in I}$ is a net in $X$ that converges to $x$. If $y\neq x$, we can find two disjoint neighborhood $U_x$ and $U_y$. Since $(x_i)_{i\in I}$ is eventually in $U_x$, it is not eventually in $U_y$, and $x_i\not\to y$.

On the other hand, if $X$ is not Hausdorff, let $x$ and $y$ be distinct points with no disjoint neighborhoods. We consider the set $\mathscr{N}_x\times\mathscr{N}_y$ directed by reverse inclusion, where $\mathscr{N}_x$ and $\mathscr{N}_y$ are the sets of neighborhoods of $x$ and $y$, respectively. For each $(U,V)\in\mathscr{N}_x\times\mathscr{N}_y$, take $x_{(U,V)}\in U\cap V$. Then the net $\{x_{(U,V)}\}_{(U,V)\in\mathscr{N}_x\times\mathscr{N}_y}$ converges to both $x$ and $y$.
\end{proof}

\subsection{Topological Vector Space (TVS)}
The vector space is an algebraic structure that is closed under addition and scalar multiplication. In this section, we discuss how the algebraic properties interact with topological properties.
\begin{definition}[Topological vector space]
A topological vector space (TVS) is a vector space $X$ over the field $\mathbb{K}$ ($=\bbR$ or $\bbC$) which is endowed with a topology such that the mappings $(x,y)\mapsto x+y$ and $(\lambda,x)\mapsto \lambda x$ are continuous from $X\times X$ and $\mathbb{K}\times X$ to $X$.
\end{definition}

Endowed with the vector operations, it is possible to discuss the convexity.
\begin{definition}[Locally convex TVS]
A topological vector space is said to be locally convex if there is a basis for the topology consisting of convex sets.
\end{definition}
The most common way of defining locally convex topologies on vector spaces is in terms of seminorms.
\begin{theorem}[Characterization of locally convex TVS]\label{lctvschar}
Let $\{p_\alpha\}_{\alpha\in A}$ be a family of seminorms on the vector space $X$. For each $x\in X$, $\alpha\in A$ and $\epsilon>0$, define
\begin{align*}
	U_{x,\alpha}^\epsilon=\left\{y\in X:p_\alpha(y-x)<\epsilon\right\},
\end{align*} 
and define $\cal{T}$ to be the topology generated by the sets $U_{x,\alpha}^\epsilon$.
\begin{itemize}
\item[(i)] For each $x\in X$, the finite intersections of sets $U_{x,\alpha}^\epsilon$ ($\alpha\in A,\epsilon>0$) form a neighborhood basis at $x$.
\item[(ii)] If $(x_i)_{i\in I}$ is a net in $X$, then $x_i\to x$ if and only if $p_\alpha(x_i-x)\to 0$ for all $\alpha\in A$.
\item[(iii)] $(X,\cal{T})$ is a locally convex topological vector space.
\end{itemize}
\end{theorem}
\begin{proof}
(i) Every neighborhood of $x$ must contain a finite intersection $\bigcap_{i=1}^n U_{x_j,\alpha_j}^{\epsilon_j}$. Let $\delta_j=\epsilon_j-p_\alpha(x-x_j)$. By triangle inequality, we have $x\in\bigcap_{i=1}^n U_{x,\alpha_j}^{\delta_j}\subset\bigcap_{i=1}^n U_{x_j,\alpha_j}^{\epsilon_j}$.

(ii) Following (i), every neighborhood of $x$ contains some set $U_{x,\alpha}^\epsilon$. We fix $\alpha\in A$. It suffices to observe that $p_\alpha(x_i-x)\to 0$ if and only if $(x_i)_{i\in I}$ is eventually in $U_{x,\alpha}^\epsilon$ for every $\epsilon>0$.

(iii) The continuity of vector operations follows from Proposition \ref{netconvcont}. Indeed, if nets $x_i\to x$, $y_i\to y$ and $\lambda_i\to\lambda$, then $(\lambda_i)$ is bounded, and
\begin{align*}
	p_\alpha((x_i+y_i)-(x+y))\leq p_\alpha(x_i-x)+p_\alpha(y_i-y)&\to 0,\\
	p_\alpha(\lambda_i x_i-\lambda x)\leq \vert\lambda_i\vert p_\alpha(x_i-x)+\vert\lambda_i-\lambda\vert p_\alpha(x)&\to 0.
\end{align*}
Hence $x_i+y_i\to x+y$ and $\lambda_ix_i\to\lambda x$, and $(X,\cal{T})$ is a topological vector space by (ii). Furthermore, if $y,z\in U_{x,\alpha}^\epsilon$ and $0\leq\lambda\leq 1$, we have
\begin{align*}
	p_\alpha(\lambda y+(1-\lambda)z-x)\leq\lambda p_\alpha(y-x)+(1-\lambda)p_\alpha(z-x)<\epsilon,
\end{align*}
and $U_{x,\alpha}^\epsilon$ is a convex set. By (i), the basis for $\cal{T}$ consists of convex sets, and the local convexity follows.
\end{proof}

In topological vector spaces, the continuity of linear maps is also associated to boundedness.

\begin{proposition}
Let $X$ and $Y$ be topological vector spaces with topologies defined, respectively, by families $\{p_\alpha\}_{\alpha\in A}$ and $\{q_\beta\}_{\beta\in B}$ of seminorms. Let $T:X\to Y$ be a linear map. Then $T$ is continuous if and only if for each $\beta\in B$, there exists $\alpha_1,\cdots,\alpha_k\in A$ and $C>0$ such that $q_\beta(Tx)\leq C\sum_{j=1}^k p_{\alpha_j}(x)$ for all $x\in X$.
\end{proposition}
\textit{\hspace{-1.5em}Proof.} Suppose the latter condition holds, and take a net $(x_i)_{i\in I}$ converging to $x$. By Theorem \ref{lctvschar} (b), we have $p_\alpha(x-x_i)\to 0$ for all $\alpha\in A$, and $q_\beta(Tx_i-Tx)\to 0$ for all $\beta\in B$. Hence $Tx_i\to Tx$, and $T$ is continuous by Proposition \ref{netconvcont}. Conversely, if $T$ is continuous,then for every $\beta\in B$ there is a neighborhood $U$ of $0$ in $X$ such that $q_\beta(Tx)<1$ for all $x\in U$. By Theorem \ref{lctvschar} (a), we may assume $U=\bigcap_{j=1}^k U_{x,\alpha_j}^{\epsilon_j}$, and let $\epsilon=\min\{\epsilon_1,\cdots,\epsilon_k\}$. Then $q_\beta(Tx)<1$ whenever $p_{\alpha_j}(x)<\epsilon$ for all $j$. Given $x\in X$, one of the following holds:
\begin{itemize}
	\item If $p_{\alpha_j}(x)>0$ for some $j$, let $y=\epsilon x/\sum_{j=1}^k p_{\alpha_j}(x)$. Then $p_{\alpha_j}(y)<\epsilon$ for all $j$, and 
	\begin{align}
		q_\beta(Tx)\leq\frac{1}{\epsilon}\sum_{j=1}^k p_{\alpha_j}(x)q_\beta(Ty)\leq\frac{1}{\epsilon}\sum_{j=1}^k p_{\alpha_j}(x).\label{qbest}
	\end{align}
\item If $p_{\alpha_j}(x)=0$ for all $j$, then $p_{\alpha_j}(\lambda x)=0$ for all $j$ and all $\lambda>0$, and $$\lambda q_\beta(Tx)=q_\beta(T(\lambda x))<1,\quad\forall\lambda>0.$$ Hence $q_\beta(Tx)=0$, and the estimate (\ref{qbest}) also holds in this case.\qed
\end{itemize}
\newpage
\begin{theorem}\label{seminormlctvs}
Let $X$ be a topological vector space with topology defined by the family $\{p_\alpha\}_{\alpha\in A}$ of seminorms.
\begin{itemize}
\item[(i)] $X$ is a Hausdorff space if and only if for each $x\neq 0$, there exists $\alpha\in A$ such that $p_\alpha(x)\neq 0$.
\item[(ii)] If $X$ is Hausdorff and $A$ is countable, then $X$ is metrizable with a translation invariant metric $\rho$, i.e. $\rho(x,y)=\rho(x+z,y+z)$ for all $x,y,z\in X$.
\end{itemize}
\end{theorem}
\begin{proof}
(i) If $X$ is a Hausdorff space, we can separate $0$ and any nonzero $x\in X$ by disjoint neighborhoods $\bigcap_{i=1}^n U_{0,\alpha_i}^{\epsilon_i}$ and $\bigcap_{j=1}^nU_{x,\beta_j}^{\delta_j}$. Since $x\notin\bigcap_{i=1}^n U_{0,\alpha_i}^{\epsilon_i}$, we have $p_{\alpha_j}(x)\geq\epsilon_j>0$ for at least one $\alpha_j$. 

Conversely, if the latter condition holds, we separate $0$ and any nonzero $x\in X$ by disjoint $U_{0,\alpha}^\epsilon$ and $U_{x,\alpha}^{\epsilon}$, where $\epsilon=\frac{1}{3}p_\alpha(x)>0$.  For any distinct pair $x,y\in X$, we separate $0$ and $x-y$. Hence $X$ is Hausdorff.

(ii) We suppose $X$ is topologized by countably many seminorms $(p_n)_{n=1}^\infty$. We define $\rho:X\times X\to\bbR_+$ by
\begin{align*}
	\rho(x,y)=\sum_{n=1}^\infty 2^{-n}\frac{p_n(x-y)}{1+p_n(x-y)},\quad x,y\in X.
\end{align*}
Clearly this mapping is translation invariant, and it remains to verify that $\rho$ is a metric on $X$.
\begin{itemize}
	\item If $x\neq y$, since $X$ is Hausdorff, there exists $n\in\bbN$ such that $p_\alpha(x-y)\neq 0$, and $\rho(x,y)>0$.
	\item The symmetry $\rho(x,y)=\rho(y,x)$ follows from the homogeneity of seminorm.
	\item For all $x,y,z\in X$, we have $p_n(x-z)\leq p_n(x-y)+p_n(y-z)$ for all $n\in\bbN$, and
	\begin{align*}
		\rho(x,z)&=\sum_{n=1}^\infty 2^{-n}\frac{p_n(x-z)}{1+p_n(x-z)}\leq\sum_{n=1}^\infty 2^{-n}\frac{p_n(x-y)+p_n(y-z)}{1+p_n(x-y)+p_n(y-z)}\\
		&\leq\sum_{n=1}^\infty 2^{-n}\frac{p_n(x-y)}{1+p_n(x-y)+p_n(y-z)}+\sum_{n=1}^\infty 2^{-n}\frac{p_n(y-z)}{1+p_n(x-y)+p_n(y-z)}\\
		&\leq\sum_{n=1}^\infty 2^{-n}\frac{p_n(x-y)}{1+p_n(x-y)}+\sum_{n=1}^\infty 2^{-n}\frac{p_n(y-z)}{1+p_n(y-z)}=\rho(x,y)+\rho(y,z).
	\end{align*}
\end{itemize}
Now we show that the $\rho$ indeed induces the topology on $X$. We denote by $\cal{T}_\rho$ the topology induced by $\rho$. Since any base set $\bigcap_{j=1}^k U_{x,n_j}^{\epsilon_j}$ contains the open ball $B_\rho(x,\epsilon)$, where $\epsilon=\min\{2^{-n_1}\frac{\epsilon_1}{1+\epsilon_1},\cdots,2^{-n_k}\frac{\epsilon_k}{1+\epsilon_k}\}$, the metric topology $\cal{T}_\rho$ is finer than $\cal{T}$. On the other hand, for any open ball $B_\rho(x,\epsilon)$, we choose $2^{-N}<\epsilon/2$. Then all $y\in\bigcup_{n=1}^NU_{x,n}^{\delta}$, with $\delta=\epsilon/N$, satisfy
\begin{align*}
	\sum_{n=1}^\infty 2^{-n}\frac{p_n(x-y)}{1+p_n(x-y)}&\leq	\sum_{n=1}^N 2^{-n}\frac{p_n(x-y)}{1+p_n(x-y)}+2^{-N}\\
	&\leq\sum_{n=1}^N2^{-n}\frac{\delta}{1+\delta}+\frac{\epsilon}{2}<\epsilon.
\end{align*}
Hence $\bigcup_{n=1}^NU_{x,n}^{\delta}$ is contained in $B_\rho(x,\epsilon)$, and $\cal{T}$ is finer than $\cal{T}_\rho$. To summarize, $\cal{T}$ and $\cal{T}_\rho$ coincide.
\end{proof}

We can generalize the concept of Cauchy property of sequences to nets.
\begin{definition}[Cauchy net and completeness]
Let $X$ be a topological vector space. A net $(x_i)_{i\in I}$ in $X$ is said to be a \textit{Cauchy net} if the net $(x_i-x_j)_{(i,j)\in I\times I}$ converges to $0$, where $I\times I$ is directed in the usual way, i.e. $(i,j)\preceq(i^\prime,j^\prime)$ if and only if $i\preceq i^\prime$ and $j\preceq j^\prime$. Furthermore, the space $X$ is said to be \textit{complete} if every Cauchy sequence in $X$ converges.
\end{definition}

When determining the completeness of a first countable topological vector space, it suffices to consider Cauchy sequences.
\begin{theorem}\label{1ctcauchy}
	Let $X$ be a first countable topological vector space. Then $X$ is complete if and only if every Cauchy sequence in $X$ converges.
\end{theorem}
We require some technical result to prove the theorem.
\begin{lemma}\label{1ctcauchylemma}
	Let $X$ be a topological vector space. If $W\subset X$ is an neighborhood of $0$, then there exists a neighborhood $U$ of $X$ such that $U+U\subset W$.
\end{lemma}
\begin{proof}
Since the addition operation $p:X\times X\to X$ is continuous at $(0,0)$, the set $p^{-1}(W)$ is a neighborhood of $(0,0)$ in the product topology on $X\times X$, and there exist base sets $V_1,V_2\ni 0$ such that $V_1+V_2\subset W$. Putting $U=V_1\cap V_2$, we get the desired property $U+U\subset W$.
\end{proof}
\begin{proof}[Proof of Theorem \ref{1ctcauchy}]
The ``only if'' direction is easy since every Cauchy sequence is also a Cauchy net. Conversely, assume every Cauchy sequence in $X$ converges. If $(x_i)_{i\in I}$ is a Cauchy net, the net $(x_i-x_j)_{(i,j)\in I\times I}$ converges to $0$. 
Take a countable basis $(U_n)_{n=1}^\infty$ at $0$, and take an index sequence $i_1\preceq i_2\preceq\cdots$ in $I$ such that $x_i-x_j\in U_n$ for all $i,j\succeq i_n$. Then the net $(x_{i_k}-x_{i_m})_{(k,m)\in\bbN\times\bbN}$ is eventually in every $U_n$, hence converges to $0$. As a result, $(x_{i_k})_{k\in\bbN}$ is a Cauchy sequence, which converges to some $x\in X$.

Now we show that that $(x_i-x)_{i\in I}$ converges to $0$. By Lemma \ref{1ctcauchylemma}, if $W$ is a neighborhood of $0$, there is a neighborhood $U$ of $0$ such that $U+U\subset W$. Therefore the sets $V_n=U_n+U_n$ also form a neighborhood basis at $0$. We argue that $(x_i-x)_{i\in I}$ is eventually in each $V_n$. Since $x_{i_m}\to x$, we take $k\in\bbN$ such that $x_{i_m}-x\in U_n$ for all $m\geq k$. Then for all $i\succeq\max\{i_k,i_n\}$, we take $m\geq\max\{k,n\}$ to obtain
\begin{align*}
	x_i - x=(x_i-x_{i_m})+(x_{i_m}-x)\in U_n+U_n=V_n.
\end{align*}
Thus $x_i-x\to 0$, and the net $(x_i)_{i\in I}$ converges to $x$.
\end{proof}

Finally we introduce the definition of Fréchet space, which is a generalization of the Banach space.
\begin{definition}[Fréchet space]
Let $X$ be a topological vector space. $X$ is called a Fréchet space if it satisfies the following three properties:
\begin{itemize}
	\item[(i)] $X$ is a Hausdorff space,
	\item[(ii)] the topology of $X$ is induced by a countable family of seminorms $(p_n)_{n=1}^\infty$, and
	\item[(iii)] $X$ is complete (with respect to the family of seminorms).
\end{itemize}
\end{definition}
\paragraph{Remark.} By Theorem \ref{seminormlctvs}, the property (ii) implies that $X$ is first countable. Therefore, to determine completeness, it suffices to consider the Cauchy sequences in $X$.

\paragraph{Example.} Following are some examples of Fréchet spaces.
\begin{itemize}
\item Every Banach space is a Fréchet space, as the norm induces a translation-invariant metric and the space is complete with respect to this metric.
\item The space $L^1_\loc(\bbR^n)$ of locally integrable functions on $\bbR^n$ is a Fréchet space with the topology induced by the seminorms
\begin{align*}
	p_n(f)=\int_{\{\vert x\vert\leq n\}}\vert f(x)\vert\,dx,\quad n\in\bbN.
\end{align*}
\item The Schwartz space $\cal{S}(\bbR^n)$ is a class of fast decreasing $C^\infty$ functions. To be specific,
\begin{align*}
	\cal{S}(\bbR^n)=\left\{f\in C^\infty(\bbR^n):\sup_{x\in\bbR^n}(1+\vert x\vert)^N\vert\partial^\alpha f\vert<\infty,\ \mathrm{for all}\ N\in\bbN_0,\ \alpha\in\bbN_0^n\right\}
\end{align*}
This is a Fréchet space with the topology induced by the norms
\begin{align*}
	\Vert f\Vert_{(N,\alpha)}=\sup_{x\in\bbR^n}\left\vert(1+\vert x\vert)^N\partial^\alpha f\right\vert,\quad N\in\bbN_0,\alpha\in\bbN_0^n.
\end{align*}
\end{itemize}

\newpage
\subsection{The Krein-Milman Theorem}
The Krein-Milman theorem is a general result about compact
convex subsets of a locally convex Hausdorff topological vector space. We first give the definition of extreme point and face.
\begin{definition}[Extreme point and face]
Let $X$ be a topological vector space and let $K\subset X$ be a nonempty convex subset. 
\begin{itemize}
	\item[(i)] A point $x$ of $K$ is called an \textit{extreme point} of $K$ if there do not exist $y,z\in K$ and $0<\lambda<1$ such that $\lambda y+(1-\lambda)z=x$. We denote by $\mathrm{ext}(K)$ the set of extreme points of $K$.
	\item[(ii)] A nonempty convex subset $F\subset K$ is called a \textit{face} of $K$ if for all $x,y\in K$ and $0<\lambda<1$ such that $\lambda x+(1-\lambda)y\in F$, we have $x,y\in F$.
\end{itemize} 
\begin{remark}
A point $x\in K$ is an extreme point of $K$ if and only if the singleton $\{x\}$ is a face of $K$.
\end{remark}
\end{definition}

\begin{lemma}\label{kmlemma1}
Let $X$ be a vector space, and let $A,B,C$ be convex subsets of $K$. If $B$ is a face of $A$ and $C$ is a face of $B$, then $C$ is a face of $A$.
\end{lemma}
\begin{proof}
Let $x,y\in A$ and $0<\lambda<1$. If $\lambda x+(1-\lambda)y\in C$, since $C\subset B$ and $B$ is a face of $A$, we have $x,y\in B$. Again, since $C$ is a face of $B$, we have $x,y\in C$. Therefore $C$ is a face of $A$.
\end{proof}

\begin{lemma}\label{kmlemma3}
Let $X$ be a locally convex Hausdorff topological space. If $K\in\scr{K}$ is a compact convex set and $\ell:X\to\bbR$ is a continuous linear functional, the set
\begin{align*}
	F_\ell:=\left\{x\in K:\ell(x)=\sup_{y\in K}\ell(y)\right\}
\end{align*}
is a nonempty compact convex subset of $K$, and $F_\ell$ is a face of $K$.
\end{lemma}
\begin{proof}
We abbreviate $c=\sup_{y\in K} \ell(y)$. 
\begin{itemize}
	\item Since $K$ is compact and $\ell$ is continuous, there exists $x\in K$ such that $\ell(x)=c$, and $F$ is nonempty. 
	\item Since $X$ is Hausdorff and $\ell$ is continuous, both $K$ and $\ell^{-1}(\{c\})$ is closed. Hence $F_\ell$ is closed and compact.
	\item Since $K$ is convex and $f$ is linear, $\ell^{-1}(\{c\})$ is convex, and so is $F$.
\end{itemize}
To summarize, $F$ is nonempty, compact and convex. To prove that $F$ is a face of $K$, we fix $x,y\in K$ and $0<\lambda<1$ such that $\lambda x+(1-\lambda)y\in F$. Then $\lambda \ell(x)+(1-\lambda)\ell(y)=\ell(\lambda x+(1-\lambda)y)=c$. Since both $\ell(x)$ and $\ell(y)$ are no greater than $c$, we have $\ell(x)=\ell(y)=c$, and $x,y\in F$. Hence $F_\ell$ is a face of $K$.
\end{proof}


\begin{lemma}[Existence]\label{kmlemma2}
Let $X$ be a locally convex Hausdorff topological vector space, and let $K\subset X$ be a nonempty compact convex set. Then the set of extreme points of $K$ is nonempty.
\end{lemma}
\begin{proof}
The proof is divided to three steps.
\item \textbf{Step I.} Let $\scr{K}$ be the set of all nonempty compact convex subset of $X$, and define the relation $\preceq$ on $\scr{K}$ by $F\preceq K$ if and only if $F$ is a face of $K$. By Lemma \ref{kmlemma1}, $(K,\preceq)$ is a partially ordered set. Since $X$ is Hausdorff, every nonempty chain $\scr{C}\subset\scr{K}$ has a infimum $C_0=\bigcap_{C\in\scr{C}}C\in\mathscr{K}$.
\item \textbf{Step II.} We claim that every minimal element of $\scr{K}$ is a singleton. 

If $K\subset\scr{K}$ is not a singleton, we take $x,y\in K$ such that $x\neq y$ and take a convex open neighborhood $U$ of $x$ that does not contain $y$. Using the hyperplane separation theorem, there exists a continuous linear functional $\ell:X\to\bbR$ such that $\ell(y)<\ell(z)$ for all $z\in U$. By Lemma \ref{kmlemma3}, the set $F_\ell\in\scr{K}$ is a face of $K$ and $y\in K\backslash F$. Hence $K$ is not a minimal element of $\scr{K}$.

\item \textbf{Step IV.} By Step I and Zorn's lemma, there exists a minimal element $E\subset\scr{K}$. By Step III, the minimal element $E$ is a singleton $\{x\}$. Then $x\in\mathrm{ext}(K)$.
\end{proof}

Now we introduce the Krein-Milman theorem.
\begin{theorem}[Krein-Milman theorem]
	Let $X$ be a locally convex Hausdorff topological vector space, and let $K\subset X$ be a nonempty compact convex set. Then $K$ is the closed convex hull of its extreme points, i.e. $$K=\ol{\mathrm{conv}}(\mathrm{ext}(K)).$$
\end{theorem}
\begin{proof}
Following the proof of Lemma \ref{kmlemma2}, we have $K\in\scr{K}$. To prove the desired result, it suffices to show $K\subset\ol{\mathrm{conv}}(\mathrm{ext}(K))$. We argue by contradiction. If  $x\in K\backslash\ol{\mathrm{conv}}(\mathrm{ext}(K))$, there exists an open convex neighborhood $U\subset X$ of $x$ such that $U\cap \ol{\mathrm{conv}}(\mathrm{ext}(K))=\emptyset$. Since $\mathrm{ext}(K)$ is nonempty by Lemma \ref{kmlemma2}, there exists a continuous linear functional $\ell$ such that $\ell(x)>\sup_{y\in\ol{\mathrm{conv}}(\mathrm{ext}(K))} \ell(y)$.  By Lemma \ref{kmlemma3}, the set $F_\ell=\left\{x\in K:f(x)=\sup f(K)\right\}$ is a face of $K$ and $F_\ell\cap\mathrm{ext}(K)=\emptyset$. On the other hand, by Lemma \ref{kmlemma2}, the compact convex set $F_\ell$ has an extreme point $x$, which is also an extreme point of $K$ by Lemma \ref{kmlemma1}. This contradicts the fact that $F_\ell\cap\mathrm{ext}(K)=\emptyset$. Thus we complete the proof.
\end{proof}

\newpage
\subsection{The Stone-Weierstrass Theorem}
The classical Stone-Weierstrass theorem asserts that each continuous function on a closed interval $[a,b]$ can be approximated by a polynomial under the uniform metric. In this section, we discuss a generalized version of the Stone-Weierstrass theorem. Throughout this section, $X$ is a compact Hausdorff space, and $C(X,\bbR)$ (resp. $C(X)$) is the space of all continuous real-valued (resp. complex-valued) functions on $X$ equipped with the uniform metric. Before we proceed, here is some useful concepts:
\begin{itemize}
\item A subset $\cal{A}$ of $C(X,\bbR)$ or $C(X)$ is said to \textit{separate points of $X$}, if for every $x,y\in X$ with $x\neq y$ there exists $f\in\cal{A}$ such that $f(x)\neq f(y)$.
\item A set $\cal{A}$ of functions is called an \textit{algebra}, if it is a real or complex vector space that is closed under function multiplication, i.e. $fg\in\cal{A}$ for all $f,g\in\cal{A}$. Clearly, $C(X,\bbR)$ and $C(X)$ are algebras.
\item A subset $\cal{A}$ of $C(X,\bbR)$ is called a \textit{lattice}, if $\min\{f,g\}\in\cal{A}$ and $\max\{f,g\}\in\cal{A}$ for every pair $f,g\in\cal{A}$.
\item A subset $\cal{A}$ of $C(X)$ is said to be \textit{non-vanishing}, if for each $x\in X$, there exists $f\in\cal{A}$ with $f(x)\neq 0$.
\item Since the algebra and lattice operations are continuous, if $\cal{A}$ is an algebra or a lattice, so is its closure $\overline{\cal{A}}$ under the uniform metric. 
\end{itemize}
We first introduce the result in the real case.

\begin{theorem}[Stone-Weierstrass theorem]\label{stoneweierstrassreal}
Let $X$ be a compact Hausdorff space. If $\cal{A}$ is a sub-algebra of $C(X,\bbR)$ that separates points of $X$, then either $\cal{A}$ is dense in $C(X,\bbR)$ or $\ol{\cal{A}}=\{f\in C(X,\bbR):f(x_0)=0\}$ for some $x_0\in X$. The first alternative holds if and only if $\cal{A}$ is non-vanishing.
\end{theorem}

\begin{lemma}\label{swlemma1}
Consider $\bbR^2$ as an algebra under coordinate-wise addition and multiplication. Then the only sub-algebras of $\bbR^2$ are $\bbR^2$, $\{(0,0)\}$, $\mathrm{span}\{(1,0)\}$, $\mathrm{span}\{(0,1)\}$ and $\mathrm{span}\{(1,1)\}$.
\end{lemma}
\begin{proof}
Clearly, all the five subspaces of $\bbR^2$ listed above are algebras. If $\cal{A}\subset\bbR^2$ is a nonzero algebra and $(0,0)\neq(a,b)\in\cal{A}$, then $(a^2,b^2)\in\cal{A}$, and one of the following four cases holds:
\begin{itemize}
	\item If $a\neq 0$, $b\neq 0$ and $a\neq b$, then $(a,b)$ and $(a^2,b^2)$ are linearly independent, and $\cal{A}=\bbR^2$;
	\item If $a=b\neq 0$, then $\cal{A}=\mathrm{span}\{(1,1)\}$;
	\item If $a\neq 0$ and $b\neq 0$, then $\cal{A}=\mathrm{span}\{(1,0)\}$;
	\item If $a=0$ and $b\neq 0$, then $\cal{A}=\mathrm{span}\{(0,1)\}$.
\end{itemize}
Then we conclude the proof.
\end{proof}

\begin{lemma}\label{swlemma2}
For any $\epsilon>0$, there is a polynomial $P$ on $[-1,1]$ such that $P(0)=0$ and $$\sup_{x\in[-1,1]}\left\vert P(x)-\vert x\vert\right\vert <\epsilon.$$
\end{lemma}
\begin{proof}
Consider the Taylor series of $\sqrt{1-t}$:
\begin{align*}
	\sqrt{1-t}=1-\sum_{n=1}^\infty\frac{(2n-3)!!}{2^n}\frac{t^n}{n!}=1-\sum_{n=1}^\infty c_n t^n,
\end{align*}
where $c_n>0$. This series converges for $\vert t\vert<1$. By monotone convergence theorem, 
\begin{align*}
	\sum_{n=1}^\infty c_n=\lim_{t\uparrow 1}\sum_{n=1}^\infty c_nt^n=1-\lim_{t\uparrow 1}\sqrt{1-t}=1.
\end{align*}
Since $\sum_{n=1}^\infty c_n$ is finite, the series $1-\sum_{n=1}^\infty c_nt^n$ converges uniformly on $[-1,1]$. Therefore, given $\epsilon>0$, by taking a suitable partial sum of this series we obtain a polynomial $Q$ such that $\vert\sqrt{1-t}-Q(t)\vert<\frac{\epsilon}{2}$ for $t\in[-1,1]$. Setting $t=1-x^2$ and $R(x)=Q(1-x^2)$, we obtain a polynomial $R$ such that $\left\vert R(x)-\vert x\vert\right\vert<\frac{\epsilon}{2}$ on $[-1,1]$. Since $\vert R(0)\vert=\vert Q(1)\vert<\frac{\epsilon}{2}$, the desired polynomial is obtained by setting $P(x)=R(x)-R(0)$.
\end{proof}

\begin{lemma}\label{swlemma3}
Let $\cal{A}$ be a sub-algebra of $C(X,\bbR)$. Then $\vert f\vert\in\ol{\cal{A}}$ for all $f\in\cal{A}$, and $\ol{\cal{A}}$ is a lattice.
\end{lemma}
\begin{proof}
If $f\in\cal{A}$ and $f\neq 0$, let $h=f/\Vert f\Vert_\infty$. Then $h(X)\subset[-1,1]$. If $\epsilon>0$ and $P$ is as in Lemma \ref{swlemma2}, we have $\Vert P\circ h-\vert h\vert\Vert_\infty<\epsilon$. Since $P(0)=0$, the function $P\circ h$ has no constant term, hence is contained in the algebra $\cal{A}$. Letting $\epsilon\downarrow 0$, we have $\vert h\vert\in\ol{\cal{A}}$, and $\vert f\vert=\Vert f\Vert_\infty\vert h\vert\in\cal{A}$. Note that
\begin{align*}
	\max\{f,g\}=\frac{1}{2}\left(f+g+\vert f-g\vert\right),\quad \min\{f,g\}=\frac{1}{2}\left(f+g-\vert f-g\vert\right).
\end{align*}
Then the second result follows.
\end{proof}

\begin{lemma}\label{swlemma4}
Let $\cal{A}$ be a lattice in $C(X,\bbR)$, and let $f\in C(X,\bbR)$. If for every $x,y\in X$, there exists a function $g_{x,y}\in\cal{A}$ such that $g_{xy}(x)=f(x)$ and $g_{xy}(y)=f(y)$, then $f\in\ol{\cal{A}}$.
\end{lemma}
\begin{proof}
For every $\epsilon>0$ and $x,y\in X$, we take
\begin{align*}
	U_{x,y}^\epsilon=\left\{z\in X:f(z)<g_{xy}(z)+\epsilon\right\},\quad\text{and}\quad V_{x,y}^\epsilon=\left\{z\in X:f(z)> g_{xy}(z)-\epsilon\right\}
\end{align*}
These sets are open and contain $x$ and $y$. By the compactness of $X$, for each $y\in X$, there exists a finite cover $\{U_{x_j,y}^\epsilon\}_{j=1}^n$ of $X$. We define $g_y=\max\{g_{x_1,y},\cdots,g_{x_n,y}\}$. Then $f<g_y+\epsilon$ on $X$, and $f>g_y-\epsilon$ on $V_y^\epsilon=\bigcap_{j=1}^n V_{x_j,y}^\epsilon$, which is an open neighborhood of $y$. 

Again, by the compactness of $X$, we take a finite cover $\{V_{x,y_j}^\epsilon\}_{j=1}^m$ of $X$, and take $g=\min\{g_{y_1},\cdots,g_{y_m}\}$. Then $\Vert f-g\Vert_\infty<\epsilon$. Since $\cal{A}$ is a lattice, $g\in\cal{A}$, and $f\in\ol{\cal{A}}$.
\end{proof}

\begin{proof}[Proof of Theorem \ref{stoneweierstrassreal}]
Given $x,y\in X$ with $x\neq y$, define
\begin{align*}
	\cal{A}_{x,y}=\left\{(f(x),f(y)):f\in\cal{A}\right\}.
\end{align*}
Then $\cal{A}_{x,y}$ is a sub-algebra of $\bbR^2$, since $f\mapsto(f(x),f(y))$ is an algebra homomorphism. Since $\cal{A}$ separates points in $X$, it cannot be $\{(0,0)\}$ or the linear span of $(1,1)$. By Lemma \ref{swlemma1}, one of the following cases holds:
\begin{itemize}
	\item If $\cal{A}_{x,y}=\bbR^2$ for all $x,y\in X$, by Lemmata \ref{swlemma3} and \ref{swlemma4}, $\ol{\cal{A}}=C(X,\bbR)$.
	\item If $\cal{A}_{x,y}$ is $\mathrm{span}\{(0,1)\}$ or $\mathrm{span}\{(1,0)\}$ for some $x,y\in X$, there exists $x_0\in X$ such that $f(x_0)=0$ for all $f\in\cal{A}$. Since $\cal{A}$ separates points in $X$, there is only one such $x_0$. Furthermore, $\cal{A}_{x,y}=\bbR^2$ when neither $x$ nor $y$ is $x_0$. Again by Lemmata \ref{swlemma3} and \ref{swlemma4}, $\ol{\cal{A}}=\{f\in C(X,\bbR):f(x_0)=0\}$.
\end{itemize}
Finally, the first alternative holds if and only if $\cal{A}$ is non-vanishing. Thus we complete the proof.
\end{proof}

\begin{theorem}[Stone-Weierstrass theorem]\label{stoneweierstrasscomplex}
Let $X$ be a compact Hausdorff space. If $\cal{A}$ is a sub-algebra of $C(X)$ that separates points of $X$ and is closed under complex conjugation, then either $\cal{A}$ is dense in $C(X)$ or $\ol{\cal{A}}=\{f\in C(X):f(x_0)=0\}$ for some $x_0\in X$. The first alternative holds if and only if $\cal{A}$ is non-vanishing.
\end{theorem}
\begin{proof}
Since $\Re f=\frac{f+\ol{f}}{2}$ and $\Im f=\frac{f-\ol{f}}{2i}$, the set $\cal{A}_{\bbR}$ of real and imaginary parts of functions in $\cal{A}$ is a sub-algebra of $C(X,\bbR)$ to which Theorem [\ref{stoneweierstrassreal}] applies. Since $\cal{A}=\{f+ig:f,g\in\cal{A}_{\bbR}\}$, the result follows.
\end{proof}


\newpage
\begin{thebibliography}{99}
\bibitem{arms1983} Armstrong, M.A. (1983). \textit{Basic Topology}. Springer.
\bibitem{munk2014} Munkres, J.R. (2014). \textit{Topology}. Pearson. 
\bibitem{folland} Folland, G.B. (1999). \textit{Real Analysis: Modern Techniques and Their Applications.} 2nd ed. John Wiley \& Sons, Inc.
\end{thebibliography}

\end{document} 