\documentclass{article}
\usepackage[utf8]{inputenc}
\usepackage{algorithm}
\usepackage{algorithmic}
\usepackage{amsfonts}
\usepackage{amsmath}
\usepackage{amssymb}
\usepackage{amsthm}
\usepackage{bm}
\usepackage{bbm}
\usepackage{booktabs}
\usepackage{dsfont}
\usepackage{enumitem}
\usepackage{extarrows}
\usepackage{float} 
\usepackage{graphicx}
\usepackage{hyperref}
\usepackage{inconsolata}
\usepackage{listings}
\usepackage{makecell}
\usepackage{mathrsfs}
\usepackage{multicol}
\usepackage{multirow}
\usepackage{setspace}
\usepackage{subfigure} 
\usepackage{threeparttable}
\usepackage{ulem}
\usepackage[many]{tcolorbox}
\usepackage{tikz}
\usetikzlibrary{positioning, arrows.meta}
\setitemize[1]{itemsep=1pt,partopsep=0.8pt,parsep=\parskip,topsep=0.8pt}
\DeclareMathAlphabet{\mathpzc}{OT1}{pzc}{m}{it}
%% Number of equations.
\numberwithin{equation}{section}
%% New symbols.
\newcommand{\rmb}{\mathrm{b}}
\newcommand{\e}{\mathrm{e}}
\newcommand{\E}{\mathbb{E}}
\newcommand{\ind}{\perp\!\!\!\perp}
\newcommand{\bfw}{\mathbf{w}}
\newcommand{\bbC}{\mathbb{C}}
\newcommand{\bbN}{\mathbb{N}}
\newcommand{\bbP}{\mathbb{P}}
\newcommand{\bbQ}{\mathbb{Q}}
\newcommand{\bbR}{\mathbb{R}}
\newcommand{\bbT}{\mathbb{T}}
\newcommand{\bbZ}{\mathbb{Z}}
\newcommand{\scr}{\mathscr}
\renewcommand{\cal}{\mathcal}
\newcommand{\loc}{\mathrm{loc}}
\newcommand{\ol}{\overline}
\newcommand{\wh}{\widehat}
\newcommand{\wt}{\widetilde}
\DeclareFontFamily{U}{mathx}{}
\DeclareFontShape{U}{mathx}{m}{n}{<-> mathx10}{}
\DeclareSymbolFont{mathx}{U}{mathx}{m}{n}
\DeclareMathAccent{\widecheck}{0}{mathx}{"71}
\DeclareMathOperator{\id}{Id}
\DeclareMathOperator{\gr}{Gr}
\DeclareMathOperator{\tr}{tr}
\DeclareMathOperator{\Le}{Le}
\DeclareMathOperator{\cov}{Cov}
\DeclareMathOperator{\var}{Var}
\DeclareMathOperator{\conv}{Conv}
\DeclareMathOperator{\supp}{supp}
\DeclareMathOperator{\diam}{diam}
\DeclareMathOperator{\esssup}{ess\,sup}
\DeclareMathOperator{\argmax}{argmax}
\DeclareMathOperator{\argmin}{argmin}
\renewcommand{\d}{\mathrm{d}}
\renewcommand{\Re}{\mathrm{Re}}
\renewcommand{\Im}{\mathrm{Im}}
\renewcommand{\i}{\mathrm{i}}
\renewcommand{\proofname}{\textit{Proof}}
\renewcommand*{\thesubfigure}{(\arabic{subfigure})}
\renewcommand{\baselinestretch}{1.15}

\theoremstyle{plain}
\newtheorem{theorem}{Theorem}[section]
\newtheorem{lemma}[theorem]{Lemma}
\newtheorem{proposition}[theorem]{Proposition}
\newtheorem{corollary}[theorem]{Corollary}
\newtheorem{example}[theorem]{Example}
\theoremstyle{definition}
\newtheorem{definition}[theorem]{Definition}
\newtheorem*{remark}{Remark}

\title{\bf Further Topics in Measure Theory}
\usepackage{geometry}
\geometry{a4paper, scale=0.80}
\author{\textsc{Jyunyi Liao}}
\date{}
\begin{document}
\maketitle
\tableofcontents
	
\newpage

\section{Interpolation of $L^p$ spaces}
\subsection{The Riesz-Thorin Interpolation Theorem}
We begin from the interpolation of $L^p$ norms. Let $1\leq p<r<q<\infty$. If $f\in L^p(X,\scr{F},\mu)\cap L^q(X,\scr{F},\mu)$, the Hölder's inequality implies
\begin{align*}
	\int_X\vert f\vert^r\,d\mu&=\int_X\vert f\vert^\frac{p(q-r)}{q-p}\vert f\vert^\frac{q(r-p)}{q-p}\,d\mu\\
	&\leq\left(\int_X\vert f\vert^p\,d\mu\right)^{\frac{q-r}{q-p}}\left(\int_X\vert f\vert^q\,d\mu\right)^{\frac{r-p}{q-p}}=\Vert f\Vert_{L^p}^{1-t}\Vert f\Vert_{L^q}^{t},
\end{align*}
where $t=\frac{q(r-p)}{q-p}$ satisfies $\frac{1}{r}=\frac{1-t}{p}+\frac{t}{q}$. This estimate holds even when $q=\infty$, since
\begin{align*}
	\int_X\vert f\vert^r\,d\mu&=\int_X\vert f\vert^p\vert f\vert^{r-p}\,d\mu
	\leq\int_X\vert f\vert^p\Vert f\Vert^{r-p}_{L^\infty}\,d\mu=\Vert f\Vert^p_{L^p}\Vert f\Vert^{r-p}_{L^\infty}.
\end{align*}
Therefore $f\in L^r(X,\scr{F},\mu)$, and $\Vert f\Vert_{L^r}$ can be bounded by $L^p$ and $L^q$ norms. More generally, we have the following interpolation theorem for linear operators.

\begin{theorem}[Riesz-Thorin interpolation theorem]\label{rieszthorin}
	Let $p_0,p_1,q_0,q_1\in[1,\infty]$. Let $(X,\scr{F},\mu)$ and $(Y,\scr{G},\nu)$ be measure spaces. If $q_0=q_1=\infty$, we further assume that $\nu$ is semifinite. Let $\,T$ be a linear operator from $L^{p_0}(X,\scr{F},\mu)+L^{p_1}(X,\scr{F},\mu)$ into $L^{q_0}(Y,\scr{G},\nu)+L^{q_1}(Y,\scr{G},\nu)$ such that $\Vert Tf\Vert_{L^{q_0}}\leq M_0\Vert f\Vert_{L^{p_0}}$  for all $f\in L^{p_0}(X,\scr{F},\mu)$, and $\Vert Tg\Vert_{L^{q_1}}\leq M_1\Vert g\Vert_{L^{p_1}}$ for all $g\in L^{p_1}(X,\scr{F},\mu)$. For each $0<t<1$, define
	\begin{align*}
		\frac{1}{p}=\frac{1-t}{p_0}+\frac{t}{p_1},\qquad \frac{1}{q}=\frac{1-t}{q_0}+\frac{t}{q_1}.
	\end{align*}
	Then $\Vert Tf\Vert_{L^q}\leq M_0^tM_1^{1-t}\Vert f\Vert_{L^p}$ for $f\in L^p(X,\scr{F},\mu)$.
\end{theorem}
We begin by introducing an estimate of $L^p$-norms using a dual space argument.
\begin{lemma}\label{rieszthorinlemma2}
Let $(X,\scr{F},\mu)$ be a measure space, and $p,q\in[1,\infty]$ conjugate exponents. If $q=\infty$, we further assume that $\mu$ is semifinite. For each $f\in L^q(X,\scr{F},\mu)$,
\begin{align}
	\Vert f\Vert_{L^q}=\sup\left\{\left\vert\int_Xfg\,d\mu\right\vert:\Vert g\Vert_{L^p}\leq 1,\ g\ \text{is simple}\right\}\label{Lqest}
\end{align}
\end{lemma}
\begin{proof} Let $M$ be the right-hand side of (\ref{Lqest}). By Hölder's inequality, we have $\Vert f\Vert_{L^q}\Vert g\Vert_{L^p}\geq\left\vert\int_X fg\,d\mu\right\vert$, and $\Vert f\Vert_{L^q}\geq M$. Then it suffices to show the other direction $\Vert f\Vert_{L^q}\leq M$. We discuss two cases.
\item\textbf{Case I: $1<p,q<\infty$.} Given $f\in L^q(X,\scr{F},\mu)$, we take a sequence $(f_n)$ of simple functions such that $\vert f_n\vert\uparrow\vert f\vert$ and $f_n\to f$ a.e., and define
\begin{align*}
	g_n=\frac{\vert f_n\vert^{q-1}\cdot\ol{\mathrm{sgn}f_n}}{\Vert f_n\Vert_{L^q}^{q-1}}.
\end{align*}
Then $\Vert g_n\Vert_{L^p}^p=1$, and
\begin{align*}
	\vert(f_n-f)g\vert\leq 2\vert f\vert\cdot\frac{\vert f_n\vert^{q-1}}{\Vert f_1\Vert_{L^q}^{q-1}}\leq \frac{2\vert f\vert^q}{\Vert f_1\Vert_{L^q}^{q-1}}\in L^1(X,\scr{F},\mu).
\end{align*}
By dominated convergence theorem,
\begin{align*}
	\lim_{n\to\infty}\int_X (f_n-f)g_n\,d\mu=0.\label{rieszthorindct}
\end{align*}
We then use Fatou's lemma to bound the $L^q$ norm of $f$:
\begin{align}
	\Vert f\Vert_{L^q}=\int_X\frac{\vert f\vert^q}{\Vert f\Vert_{L^q}^{q-1}}\,d\mu\leq\liminf_{n\to\infty}\int_X \frac{\vert f_n\vert^q}{\Vert f_n\Vert_{L^q}^{q-1}}\,d\mu=\liminf_{n\to\infty}\left\vert\int_X f_ng_n\,d\mu\right\vert.
\end{align}
Passing to a suitable subsequence and applying (\ref{rieszthorindct}), we obtain
\begin{align*}
	\Vert f\Vert_{L^q}\leq\lim_{n\to\infty}\left\vert\int_X f_ng_n\,d\mu\right\vert=\lim_{n\to\infty}\left\vert\int_X fg_n\,d\mu\right\vert\leq M.
\end{align*}
\item\textbf{Case II: $p=1$ and $q=\infty$.} Argue by contradiction. If $\Vert f\Vert_{L^\infty}>M$, we choose $\epsilon>0$ such that the set $E=\{x\in X:\vert f(x)\vert>M+\epsilon\}$ has positive measure. Since $\mu$ is a semifinite measure, we can choose $F\subset E$ with $0<\mu(F)<\infty$. Let $g=\chi_F\cdot\ol{\mathrm{sgn}f}/\mu(F)$, and take a sequence of simple functions $g_n\to g$ and $\vert g_n\vert\uparrow g$. Then $\Vert g\Vert_{L^1}=1$, $\Vert g_n\Vert_{L^1}\leq 1$, and
\begin{align*}
	\vert fg_n\vert =\frac{1}{\mu(F)}\chi_F\vert f\vert\leq\frac{\Vert f\Vert_{L^\infty}}{\mu(F)}\chi_F,
\end{align*} 
which is an integrable function. By dominated convergence theorem and definition of $F$,
\begin{align*}
	M\geq \int_X fg\,d\mu=\lim_{n\to\infty}\int_X fg_n\,d\mu=\frac{1}{\mu(F)}\int_F\vert f\vert\,d\mu\geq M+\epsilon,
\end{align*}
a contradiction! Hence $\Vert f\Vert_{L^\infty}\leq M$.
\end{proof}

\begin{lemma}[The three lines lemma]\label{rieszthorinlemma1}
	Let $\phi$ be a bounded continuous function on the strip $0\leq\Re(z)\leq 1$ that is holomorphic in the interior of the strip. If $\vert\phi(z)\vert\leq M_0$ on $\Re(z)=0$ and $\vert\phi(z)\vert\leq M_1$ on $\Re(z)=1$, then $\vert\phi(z)\vert\leq M_0^{1-t}M_1^{t}$ on $\Re(z)=t$, where $0<t<1$.
\end{lemma}
\begin{proof}
	We define the function
	\begin{align*}
		\phi_\epsilon(z)=\phi(z)M_0^{-z}M_1^{z-1}e^{-\epsilon z(1-z)},
	\end{align*}
	which also satisfies the hypothesis of the lemma with $M_0$ and $M_1$ replaced by $1$, and 
	\begin{align*}
		\vert\phi_\epsilon(x+iy)\vert=\vert \phi(x+iy)\vert M_0^{-x}M_1^{x-1}e^{-\epsilon x(1-x)-\epsilon y^2}\leq M_0^{1-x}M_1^{x-1}e^{-\epsilon y^2}.
	\end{align*}
	Hence $\phi_\epsilon(z)\to 0$ as $\vert\Im(z)\vert\to\infty$. By our hypothesis, for sufficiently large $A>0$, we have $\vert\phi_\epsilon\vert\leq 1$ on the boundary of the region $D=\{z:0\leq\Re(z)\leq 1,-A\leq\Im(z)\leq A\}$. By the maximum modulus principle, $\max_{z\in\partial D}\vert\phi_\epsilon(z)\vert=\max_{z\in D}\vert\phi_\epsilon(z)\vert$. Hence $\vert\phi_\epsilon\vert\leq 1$ on $D$, and hence on the strip $0\leq\Re(z)\leq 1$. Letting $\epsilon\to 0$, we obtain $\vert\phi(z)\vert M_0^{-t}M_1^{t-1}\leq\lim_{\epsilon\to 0}\vert\phi_\epsilon(z)\vert\leq 1$, where $t=\Re(z)$. 
\end{proof}
\begin{proof}[Proof of Theorem \ref{rieszthorin}] The proof has three steps.
\item\textbf{Step I:} We begin with the case $p_0=p_1=p$. Since the case $q_0=q_1$ is clear, we may assume $q_0<q_1$. Then
\begin{align*}
	\Vert Tf\Vert_{L^q}\leq\Vert Tf\Vert_{L^{q_0}}^{\frac{q_0(q_1-q)}{q(q_1-q_0)}}\Vert Tf\Vert_{L^{q_1}}^{\frac{q_1(q-q_0)}{q(q_1-q_0)}}=\Vert Tf\Vert_{L^{q_0}}^{t}\Vert Tf\Vert_{L^{q_1}}^{1-t}\leq M_0^tM_1^{1-t}\Vert f\Vert_{L^p}.
\end{align*}

\item\textbf{Step II.} Now we assume $p_0<p_1\leq\infty$, and in particular $p<\infty$ for all $0<t<1$. We begin by taking a simple function $f=\sum_{j=1}^n a_j\chi_{E_j}=\sum_{j=1}^n\vert a_j\vert e^{i\theta_j}\chi_{E_j}$ and show that $\Vert Tf\Vert_{L^q}\leq M_0^{1-t}M_1^{t}\Vert f\Vert_{L^p}$. 

By homogeneity of $\Vert\cdot\Vert$ and linearity of $T$, it suffices to show the case $\Vert f\Vert_{L^p}=1$. We estimate $\Vert Tf\Vert_{L^q}$ by taking $g=\sum_{k=1}^m b_j\chi_{F_k}=\sum_{k=1}^m\vert b_j\vert e^{i\xi_k}\chi_{F_k}$ with $\Vert g\Vert_{L^{q^\prime}}=1$ in (\ref{Lqest}). Define functions $\alpha$ and $\beta$ as follows:
\begin{align*}
	\alpha(z)=\frac{1-z}{p_0}+\frac{z}{p_1},\quad \beta(z)=\frac{1-z}{q_0}+\frac{z}{q_1},\quad z\in\bbC,\ 0\leq\Re(z)\leq 1.
\end{align*}
Then $\alpha(t)=1/p$ and $\beta(t)=1/q$. We let
\begin{align*}
	f_z=\sum_{j=1}^n\vert a_j\vert^{\alpha(z)/\alpha(t)} e^{i\theta_j}\chi_{E_j},\qquad g_z=\begin{cases}\sum_{k=1}^m\vert b_k\vert^{(1-\beta(z))/(1-\beta(t))}e^{i\xi_k}\chi_{F_k}, &\beta(t)\neq 1,\\
	g, &\beta(t)=1.
	\end{cases}
\end{align*}
Finally, we define
\begin{align*}
	\Phi(z)=\int_Y (Tf_z)g_z\,d\nu=\begin{cases}
		\sum_{j=1}^n\sum_{k=1}^m \vert a_j\vert^{\frac{\alpha(z)}{\alpha(t)}}\vert b_k\vert^{\frac{1-\beta(z)}{1-\beta(t)}} e^{i(\theta_j+\xi_k)}\int_Y(T\chi_{E_j})\chi_{F_k}\,d\nu, &\beta(t)\neq 1\\
		\sum_{j=1}^n\sum_{k=1}^m \vert a_j\vert^{\frac{\alpha(z)}{\alpha(t)}}\vert b_k\vert e^{i(\theta_j+\xi_k)}\int_Y(T\chi_{E_j})\chi_{F_k}\,d\nu, &\beta(t)=1.
	\end{cases}
\end{align*}
Then $\Phi(z)$ is a bounded and continuous function on the strip $0\leq\Re(z)\leq 1$ that is holomorphic in the strip. We claim that $\Phi(z)\leq M_0$ on $\Re(z)=0$ and $\Phi(z)\leq M_1$ on $\Re(z)=1$. We let $z=i\omega$, where $\omega\in\bbR$. Since $E_1,\cdots,E_n$ are disjoint, at most one $\chi_{E_j}$ is nonzero, and
\begin{align*}
	\vert f_{i\omega}\vert=\sum_{j=1}^n\vert a_j\vert^{p/p_0}\chi_{E_j}=\vert f\vert^{p/p_0}.
\end{align*}
A similar calculation yields
\begin{align*}
	\vert g_{i\omega}\vert=\sum_{k=1}^m \vert b_k\vert^{q^\prime/q_0^\prime}\chi_{F_k}=\vert g\vert^{q^\prime/q_0^\prime},
\end{align*}
where $q^\prime$ and $q_0^\prime$ are the conjugate exponents of $q$ and $q_0$, respectively, and we set $\frac{\infty}{\infty}=1$. By Hölder's inequality,
\begin{align*}
	\vert\Phi(i\omega)\vert\leq\Vert Tf_{i\omega}\Vert_{L^{q_0}}\Vert g_{i\omega}\Vert_{L^{q_0^\prime}}\leq M_0\Vert f_{i\omega}\Vert_{L^{p_0}}\Vert g_{i\omega}\Vert_{L^{q_0^\prime}}=M_0\Vert f\Vert_{L^p}\Vert g\Vert_{L^{q^\prime}}=M_0.
\end{align*}
Similarly, we can show $\vert\Phi(1+i\omega)\vert\leq M_1$. By three lines lemma [Lemma \ref{rieszthorinlemma1}] and Lemma \ref{rieszthorinlemma2}, we have
\begin{align*}
	\Vert Tf\Vert_{L^q}\leq\vert\Phi(t)\vert\leq M_0^{1-t}M_1^t.
\end{align*} 
\item\textbf{Step III.} We have shown that $\Vert Tf\Vert_{L^q}\leq  M_0^{1-t}M_1^t\Vert f\Vert_{L^p}$ for all simple functions. For each $f\in L^p(X,\scr{F},\mu)$ with $f\geq 0$, we choose a sequence of simple functions such that $\vert f_n\vert\uparrow\vert f\vert$ and $f_n\to f$ pointwise. We let $E=\{x\in X:\vert f(x)\vert>1\}$, and define
\begin{align*}
	g=f\chi_E,\quad g_n=f_n\chi_E,\quad h=f-g,\quad h_n=f_n-g_n.
\end{align*}
Since $p_0<p<p_1$, we have $g\in L^{p_0}(X,\scr{F},\mu)$ and $h\in L^{p_1}(X,\scr{F},\mu)$. By dominated convergence theorem, $\Vert f_n-f\Vert_{L^p}\to 0$, $\Vert g_n-g\Vert_{L^{p_0}}\to 0$ and $\Vert h-h_p\Vert_{L^{p_1}}\to 0$. Hence $\Vert Tg_n-Tg\Vert_{L^{q_0}}\leq M\Vert g_n-g\Vert_{L^{p_0}}\to 0$ and $\Vert Th_n-Th\Vert_{L^{q_1}}\leq M\Vert h_n-h\Vert_{L^{p_1}}\to 0$. By passing to a suitable subsequence we may also assume $Tg_n\to Tg$ a.e. and $Th_n\to Th$ a.e., and then $Tf_n\to Tf$. By Fatou's lemma,
\begin{align*}
	\Vert Tf\Vert_{L^q}\leq\liminf_{n\to\infty}\Vert Tf_n\Vert_{L^q}\leq \liminf_{n\to\infty}M_0^{1-t}M_1^t\Vert f_n\Vert_{L^p}=M_0^{1-t}M_1^t\Vert f\Vert_{L^p}.
\end{align*}
Then we finish the proof.
\end{proof}

\newpage
\subsection{The Marcinkiewicz Interpolation Theorem}
\paragraph{Distribution function and weak $L^p$ spaces.} Let $(X,\scr{F},\mu)$ be a measure space. For a measurable function $f$ on $(X,\scr{F},\mu)$, define its \textit{distribution function} $\lambda_f:(0,\infty)\to[0,\infty]$ by
\begin{align*}
	\lambda_f(\alpha)=\mu\left(\left\{x\in X:\vert f(x)\vert>\alpha\right\}\right).
\end{align*}
Some properties of the distribution function is clear:
\begin{itemize}
	\item $\lambda_f$ is decreasing on $(0,\infty)$.
	\item $\lambda_f$ is right-continuous, since $\{\vert f\vert>\alpha\}=\bigcup_{n=1}^\infty\{\vert f\vert>\alpha+\epsilon_n\}$ for all $\epsilon_n>0$ with $\epsilon_n\downarrow 0$.
	\item $\lambda_{f+g}(2\alpha)\leq\lambda_f(\alpha)+\lambda_g(\alpha)$. In addition, if $\vert f\vert\leq\vert g\vert$, then $\lambda_f\leq\lambda_g$. 
	\item If the sequence $(f_n)$ satisfies $\vert f_n\vert\uparrow\vert f\vert$, then $\lambda_{f_n}\to\lambda_f$ pointwise, since $\{\vert f\vert>\alpha\}=\bigcup_{n=1}^\infty\{\vert f_n\vert>\alpha\}$.
\end{itemize}
Let $1\leq p<\infty$. For a measurable function $f$, define\vspace{-0.15cm}
\begin{align*}
	[f]_p=\left(\sup_{\alpha>0}\alpha^p\lambda_{f}(\alpha)\right)^{1/p}.
\end{align*}
The \textit{weak $L^p$ space} is then defined to be the set of all measurable functions on $(X.\scr{F},\mu)$ such that $[f]_p<\infty$. Note that $[\cdot]_p$ is not a norm, because it does not satisfy the triangle inequality. By Chebyshev's inequality, $[f]_p\leq\Vert f\Vert_{L^p}$ for all $f\in L^p(X,\scr{F},\mu)$. Therefore the classical $L^p$ is contained in the weak $L^p$ space. 

\begin{lemma}\label{mckwzlemma1}
	If $f$ is a measurable function on $(X,\scr{F},\mu)$ and $0<p<\infty$, then
	\begin{align*}
		\int_X\vert f\vert^p\,d\mu=p\int_0^\infty\alpha^{p-1}\lambda_f(\alpha)\,d\alpha.
	\end{align*}
\end{lemma}
\begin{proof}
	We may assume $\lambda_f(\alpha)<\infty$ for all $\alpha>0$, otherwise both integrals are infinite. We may also assume $f\geq 0$ by replacing $f$ with $\vert f\vert$ if necessary. If $f$ is simple, $\lambda_f$ is a step function with jump discontinuities $0<\alpha_1<\cdots<\alpha_n$. We let $\alpha_0=0$. Then
	\begin{align*}
		p\int_0^\infty\alpha^{p-1}\lambda_f(\alpha)\,d\alpha&=\sum_{j=1}^n\int_{\alpha_{j-1}}^{\alpha_j} p\alpha^{p-1}\lambda_f(\alpha)\,d\alpha=\sum_{j=1}^n\left(\vert\alpha_j\vert^p-\vert\alpha_{j-1}\vert^p\right)\lambda_f(\alpha_{j-1})\\
		&=\sum_{j=1}^n\vert\alpha_j\vert^p\left(\lambda_f(\alpha_{j-1})-\lambda_f(\alpha_j)\right)=\sum_{j=1}^n\vert\alpha_j\vert^p\mu(\{f=\alpha_j\})=\int_X\vert f\vert^p\,d\mu.
	\end{align*}
	Since $f_n\uparrow f$ implies the pointwise convergence $\lambda_{f_n}\to\lambda_f$, the general result follows from simple function approximation and monotone convergence theorem.
\end{proof}

\begin{definition}[Sublinear operators of strong and weak types]
	Let $T$ be an operator on the some vector space $V$ of measurable functions from $(X,\scr{F},\mu)$ to the space of all measurable functions on $(Y,\scr{G},\nu)$. 
	\begin{itemize}
		\item[(i)] $T$ is said to be \textit{sublinear}, if $\vert T(f+g)\vert\leq\vert Tf\vert+\vert Tg\vert$ and $\vert T(cf)\vert=c\vert Tf\vert$ for all $f,g\in V$ and $c>0$.
		\item[(ii)] Let $1\leq p,q\leq\infty$. The operator $T$ is said to be \textit{of strong type $(p,q)$}, if $L^p(X,\scr{F},\mu)\subset V$ and there exists a constant $C_{p,q}>0$ such that for all $f\in L^p(X,\scr{F},\mu)$, $$\Vert Tf\Vert_{L^q}\leq C_{p,q}\Vert f\Vert_{L^p}.$$
		\item[(iii)] Let $1\leq p\leq\infty$ and $1\leq q<\infty$. The operator $T$ is said to be \textit{of weak type $(p,q)$}, if $L^p(X,\scr{F},\mu)\subset V$ and there exists a constant $C_{p,q}>0$  such that  for all $f\in L^p(X,\scr{F},\mu)$, 
		$$[Tf]_q\leq C_{p,q}\Vert f\Vert_{L^p}.$$ 
		Clearly, a sublinear operator $T$ of strong type $(p,q)$ is also of weak type $(p,q)$. Also, we say $T$ is of weak type $(p,\infty)$ if and only if it is of strong type $(p,\infty)$.
	\end{itemize}
\end{definition}

\begin{theorem}[Marcinkiewicz interpolation theorem]\label{mckwz}
	Let $1\leq p_0\leq q_0\leq\infty$, $1\leq p_1\leq q_1\leq\infty$ and $q_0\neq q_1$. Let $\,T$ be a sublinear operator from $L^{p_0}(X,\scr{F},\mu)+L^{p_1}(X,\scr{F},\mu)$ into the space of measurable functions on $(Y,\scr{G},\nu)$. For each $0<\gamma<1$, define
	\begin{align*}
		\frac{1}{p}=\frac{1-\gamma}{p_0}+\frac{\gamma}{p_1},\qquad \frac{1}{q}=\frac{1-\gamma}{q_0}+\frac{\gamma}{q_1}.
	\end{align*}
	If $\,T$ is a sublinear operator of weak types $(p_0,q_0)$ and $(p_1,q_1)$, then $T$ is of strong type $(p,q)$.
\end{theorem}
The proof Marcinkiewicz interpolation theorem requires the following lemma.

\begin{lemma}[Minkowski's integral inequality]\label{mkwsk}
Let $(X,\scr{F},\mu)$ and $(Y,\scr{G},\nu)$ be two measure spaces and let $\Phi:X\times Y\to\bbC$ be a measurable function on the produce space. If $p\geq 1$, we have
\begin{align*}
	\left(\int_X\left\vert\int_Y\phi(x,y)\,d\nu(y)\right\vert^p d\mu(x)\right)^{\frac{1}{p}}\leq\int_X\left(\int_X\vert\phi(x,y)\vert^{p}\,d\mu(x)\right)^{\frac{1}{p}}\,d\mu(y).
\end{align*}
\end{lemma}
\begin{proof}
Let $\Phi(x)=\int_Y\phi(x,y)\,d\nu(y)$. Similar to the proof of Minkowski's inequality, we estimate $\Vert\Phi\Vert_{L^p}^p$ by
\begin{align*}
	\int_X\vert\Phi\vert^p\,d\mu&\leq\int_X\vert\Phi(x)\vert^{p-1}\int_Y\left\vert\phi(x,y)\right\vert d\nu(y)\,d\mu(x)\\
	&=\int_Y\int_X\vert\Phi(x)\vert^{p-1}\left\vert\phi(x,y)\right\vert d\mu(x)\,d\nu(y)\\
	&\leq\int_Y\left(\int_X\vert\Phi(x)\vert^{(p-1)\cdot\frac{p}{p-1}}\,d\mu(x)\right)^{\frac{p-1}{p}}\left(\int_X\vert\phi(x,y)\vert^{p}\,d\mu(x)\right)^{\frac{1}{p}}\,d\mu(y)\\
	&=\Vert\Phi\Vert_{L^p}^{p-1}\int_X\left(\int_X\vert\phi(x,y)\vert^{p}\,d\mu(x)\right)^{\frac{1}{p}}\,d\mu(y),
\end{align*}
where we interchange the integrals by Fubini's theorem and use Hölder's inequality to the inner integral.
\end{proof}

\begin{lemma}\label{mckwzlemma2}
	If $f$ is a measurable function and $\alpha>0$, define
	\begin{align*}
		h_\alpha=f\chi_{\{\vert f\vert\leq\alpha\}}+\alpha(\mathrm{sgn}\,f)\chi_{\{\vert f\vert>\alpha\}},\quad\text{and}\quad g_\alpha=f-h_\alpha=(\mathrm{sgn}\,f)(\vert f\vert-\alpha)\chi_{\{\vert f\vert>\alpha\}}.
	\end{align*}
	Then
	\begin{align*}
		\lambda_{g_\alpha}(t)=\lambda_f(t+\alpha),\quad\text{and}\quad\lambda_{h_\alpha}(t)=\begin{cases}
			\lambda_f(t), &t<\alpha,\\
			0, &t\geq\alpha.
		\end{cases}
	\end{align*}
\end{lemma}
\begin{proof}
	By definition, $h_\alpha$ is in fact the $\alpha$-truncation of $f$, i.e. $h_\alpha=f$ when $\vert f\vert\leq\alpha$, and $h_\alpha=\alpha(\mathrm{sgn}\,f)$ when $\vert f\vert>\alpha$. Hence $\{\vert h\vert>t\}=\{\vert f\vert>t\}$ when $t<\alpha$, and $\{\vert h\vert>t\}=\{\vert f\vert>t\}=\emptyset$ when $t\geq\alpha$. On the other hand, note that $g_\alpha=0$ on $\{\vert f\vert<\alpha\}$. For any $t>0$, we have $\{\vert g_\alpha\vert>t\}=\{\vert f\vert-\alpha>t\}=\{\vert f\vert>t+\alpha\}$.
\end{proof}

Now we prove the Marcinkiewicz interpolation theorem.
\begin{proof}[Proof of Theorem \ref{mckwz}]
For notation simplicity we also write $[\cdot]_\infty=\Vert\cdot\Vert_{L^\infty}$. Since $T$ is of weak types $(p_0,q_0)$ and $(p_1,q_1)$, there exist constants $C_0$ and $C_1$ such that
\begin{align*}
	\lambda_{f}(\alpha)\leq \left(\frac{C_0}{\alpha}\right)^{q_0}\Vert f\Vert_{L^{p_0}}^{q_0}\quad\text{and}\quad \lambda_{g}(\alpha)\leq \left(\frac{C_1}{\alpha}\right)^{q_1}\Vert g\Vert_{L^{p_1}}^{q_1}
\end{align*}
for all $f\in L^{p_0}(X,\scr{F},\mu)$, $g\in L^{p_1}(X,\scr{F},\mu)$ and all $\alpha>0$. There are several cases to consider. 
\begin{itemize}
	\item Case I: $p_0=p_1=p$. We may assume $q_0<q_1$ by switching subscripts $0$ and $1$ when necessary.
	\item Case II: $p_0\neq p_1$. We may assume $p_0<p_1$ by switching subscripts $0$ and $1$ when necessary.
\end{itemize}

Now we prove the theorem case by case.
\item\textbf{Case I\,(1): $p_0=p_1=p$ and $q_0<q_1<\infty$.} If $f\in L^p(X,\scr{F},\mu)$,
\begin{align*}
	\Vert Tf\Vert_q^q&=q\int_{0}^\infty\alpha^{q-1}\lambda_{Tf}(\alpha)\,d\alpha\leq q\int_0^{\Vert f\Vert_{L^p}}\alpha^{q-1}\lambda_{Tf}(\alpha)\,d\alpha+q\int_{\Vert f\Vert_{L^p}}^\infty\alpha^{q-1}\lambda_{Tf}(\alpha)\,d\alpha\\
	&\leq q\int_0^{\Vert f\Vert_{L^p}}\alpha^{q-q_0-1} C_0^{q_0}\Vert f\Vert^{q_0}_{L^p}\,d\alpha+q\int_{\Vert f\Vert_{L^p}}^\infty\alpha^{q-q_1-1}C_1^{q_1}\Vert f\Vert^{q_1}_{L^p}\,d\alpha\\
	&=\left(\frac{qC_0^{q_0}}{q-q_0}+\frac{qC_1^{q_1}}{q_1-q}\right)\Vert f\Vert^{q}_{L^p}
\end{align*}
\item\textbf{Case I\,(2): $p_0=p_1=p$ and $q_0<q_1=\infty$.} If $f\in L^p(X,\scr{F},\mu)$, we have $\Vert Tf\Vert_{L^{q_1}}\leq C_1\Vert f\Vert_{L^p}$. Then $\lambda_{Tf}(\alpha)=0$ when $\alpha> C_1\Vert f\Vert_{L^p}$, and
\begin{align*}
	\Vert Tf\Vert_q^q&=q\int_{0}^{C_1\Vert f\Vert_{L^p}}\alpha^{q-1}\lambda_{Tf}(\alpha)\,d\alpha\leq q\int_0^{C_1\Vert f\Vert_{L^p}}\alpha^{q-q_0-1}[Tf]_{q_0}^{q_0}\,d\alpha\\
	&\leq q\int_0^{C_1\Vert f\Vert_{L^p}}\alpha^{q-q_0-1} C_0^{q_0}\Vert f\Vert^{q_0}_{L^p}\,d\alpha=\frac{qC_0^{q_0}C_1^{q-q_0}}{q-q_0}\Vert f\Vert^{q}_{L^p}.
\end{align*}

\item\textbf{Case II\,(1): $p_0<p_1<\infty$ and $q_0,q_1<\infty$.} 
For $f\in L^p(X,\scr{F},\mu)$, we take $g_\alpha$ and $h_\alpha$ as in Lemma \ref{mckwzlemma2}, where $\alpha>0$ is to be determined. Then
	\begin{align}
		\lambda_{Tf}(2\beta)&\leq \lambda_{Tg_\alpha}(\beta)+\lambda_{Th_\alpha}(\beta)\leq\left(\frac{C_0}{\beta}\right)^{q_0}\left(\int_X\vert g_\alpha\vert^{p_0}\,d\mu\right)^{\frac{q_0}{p_0}}+\left(\frac{C_1}{\beta}\right)^{q_1}\left(\int_X\vert h_\alpha\vert^{p_1}\,d\mu\right)^{\frac{q_1}{p_1}}.\label{mckwzeq1}
	\end{align}
	Here we allow $\alpha$ to depend on $\beta$. By Lemma \ref{mckwzlemma1}, we have
	\begin{align}
		\int_X\vert g_\alpha\vert^{p_0}\,d\mu&=p\int_0^\infty t^{p_0-1}\lambda_{g_\alpha}(t)\,dt=p_0\int_0^\infty t^{p_0-1}\lambda_f(t+\alpha)\,dt\notag\\
		&=p_0\int_\alpha^\infty(t-\alpha)^{p_0-1}\lambda_f(t)\,dt\leq p_0\int_\alpha^\infty t^{p_0-1}\lambda_f(t)\,dt,\label{galphaest}
	\end{align}
	and similarly, \vspace{-0.1cm}
	\begin{align}
		\int_X\vert h_\alpha\vert^{p_1}\,d\mu&=p_1\int_0^\infty t^{p_1-1}\lambda_{h_\alpha}(t)\,dt=p_1\int_0^\alpha t^{p_1-1}\lambda_f(t)\,dt.\label{halphaest}
	\end{align}
	We combine Lemma \ref{mckwzlemma1}, the inequality (\ref{mckwzeq1}) and the estimates (\ref{galphaest})-(\ref{halphaest}):
	\begin{align}
	\begin{aligned}
	\Vert Tf\Vert_{L^q}^q&=q\int_0^\infty (2\beta)^{q-1}\lambda_{Tf}(2\beta)\,d(2\beta)=q2^q\int_0^\infty \beta^{q-1}\lambda_{Tf}(2\beta)\,d\beta\\
	&\leq q2^q\int_0^\infty\left(C_0^{p_0}\beta^{q-q_0-1}\left(\int_X\vert g_\alpha\vert^{p_0}\,d\mu\right)^{\frac{q_0}{p_0}}+C_1^{p_1} \beta^{q-q_1-1}\left(\int_X\vert h_\alpha\vert^{p_1}\,d\mu\right)^{\frac{q_1}{p_1}}\right)d\beta\\
	&\leq q2^qC_0^{p_0}p_0^{\frac{q_0}{p_0}}\int_0^\infty\beta^{q-q_0-1}\left(\int_0^\infty\chi_{\{t>\alpha\}} t^{p_0-1}\lambda_f(t)\,dt\right)^{\frac{q_0}{p_0}}d\beta\\
	&\quad+q2^qC_1^{p_1}p_1^{\frac{q_1}{p_1}}\int_0^\infty\beta^{q-q_1-1}\left(\int_0^\infty\chi_{\{t\leq\alpha\}} t^{p_1-1}\lambda_f(t)\,dt\right)^{\frac{q_1}{p_1}}d\beta.
	\end{aligned}\label{mckwzest3}
	\end{align}
Since this estimate holds for any $\alpha>0$, we choose $\alpha=\beta^\sigma$, where $$\sigma=\frac{p_0(q-q_0)}{q_0(p-p_0)}=\frac{(1-\gamma)\left(\frac{q}{q_0}-1\right)}{(1-\gamma)\left(\frac{p}{p_0}-1\right)}=\frac{\gamma\left(1-\frac{q}{q_1}\right)}{\gamma\left(1-\frac{p}{p_1}\right)}=\frac{p_1(q_1-q)}{q_1(p_1-p)}.$$ 
We also write $\chi_0=\chi_{\{t>\alpha\}}$, $\chi_1=\chi_{\{t\leq\alpha\}}$, and $$\phi_i(t,\beta)=\beta^{\frac{p_i}{q_i}(q-q_i-1)}\chi_i t^{p_i-1}\lambda_f(t),$$ where $i=0,1$. Then (\ref{mckwzest3}) becomes
\begin{align}
\Vert Tf\Vert_{L^q}^q\leq\sum_{i=0}^1q2^qC_i^{p_i}p_i^{\frac{q_i}{p_i}}\int_0^\infty\left(\int_0^\infty\phi_i(t,\beta)\,dt\right)^{\frac{q_i}{p_i}}d\beta.\label{mckwzest4}
\end{align}
We write $\Phi(\beta)=\int_0^\infty\phi_i(t,\beta)\,dt$. Since $\frac{q_i}{p_i}\geq 1$, by Minkowski's inequality, in either case $i=0,1$,
\begin{align}
\begin{aligned}
	\int_0^\infty\left(\int_0^\infty\phi_i(t,\beta)\,dt\right)^{\frac{q_i}{p_i}}d\beta&\leq\left(\int_0^\infty\left(\int_0^\infty\vert\phi_i(t,\beta)\vert^{\frac{q_i}{p_i}}\,d\beta\right)^{\frac{p_i}{q_i}}dt\right)^{\frac{q_i}{p_i}}\\
	&=\left(\int_0^\infty\left(\int_0^\infty\beta^{q-q_i-1}\chi_i\,d\beta\right)^{\frac{p_i}{q_i}}t^{p_i-1}\lambda_f(t)dt\right)^{\frac{q_i}{p_i}}.
\end{aligned}\label{mkwskest}
\end{align}
If $q_1>q_0$, the exponents $q-q_0$ and $\sigma$ are positive, and $\{t>\beta^\sigma\}=\{\beta<t^{1/\sigma}\}$. Then (\ref{mkwskest}) becomes
\begin{align*}
	\int_0^\infty\left(\int_0^\infty\phi_0(t,\beta)\,dt\right)^{\frac{q_0}{p_0}}d\beta&\leq\left(\int_0^\infty\left(\int_0^{t^{1/\sigma}}\beta^{q-q_0-1}\,d\beta\right)^{\frac{p_0}{q_0}}t^{p_0-1}\lambda_f(t)dt\right)^{\frac{q_0}{p_0}}\\ &=\frac{1}{q-q_0}\left(\int_0^\infty t^{p-1}\lambda_f(t)dt\right)^{\frac{q_0}{p_0}}=\frac{1}{q-q_0}\left(\frac{1}{p}\right)^{\frac{q_0}{p_0}}\Vert f\Vert_{L^p}^{\frac{pq_0}{p_0}}.
\end{align*}
On the other hand, $q-q_1<0$, and $\{t\leq\beta^\sigma\}=\{\beta\geq t^{1/\sigma}\}$. Then (\ref{mkwskest}) becomes
\begin{align*}
	\int_0^\infty\left(\int_0^\infty\phi_1(t,\beta)\,dt\right)^{\frac{q_1}{p_1}}d\beta &\leq\left(\int_0^\infty\left(\int_{t^{1/\sigma}}^\infty\beta^{q-q_1-1}\,d\beta\right)^{\frac{p_1}{q_1}}t^{p_1-1}\lambda_f(t)dt\right)^{\frac{q_1}{p_1}} \\
	&=\frac{1}{q_1-q}\left(\int_0^\infty t^{p-1}\lambda_f(t)dt\right)^{\frac{q_1}{p_1}}=\frac{1}{q_1-q}\left(\frac{1}{p}\right)^{\frac{q_1}{p_1}}\Vert f\Vert_{L^p}^{\frac{pq_1}{p_1}}.
\end{align*}
If $q_1<q_0$, the exponents $q-q_0$ and $\sigma$ are negative, and $\{t>\beta^\sigma\}=\{\beta>t^{1/\sigma}\}$. Then (\ref{mkwskest}) becomes
\begin{align*}
	\int_0^\infty\left(\int_0^\infty\phi_0(t,\beta)\,dt\right)^{\frac{q_0}{p_0}}d\beta&\leq\left(\int_0^\infty\left(\int_{t^{1/\sigma}}^\infty\beta^{q-q_0-1}\,d\beta\right)^{\frac{p_0}{q_0}}t^{p_0-1}\lambda_f(t)dt\right)^{\frac{q_0}{p_0}}\\ &=\frac{1}{q_0-q}\left(\int_0^\infty t^{p-1}\lambda_f(t)dt\right)^{\frac{q_0}{p_0}}=\frac{1}{q_0-q}\left(\frac{1}{p}\right)^{\frac{q_0}{p_0}}\Vert f\Vert_{L^p}^{\frac{pq_0}{p_0}}.
\end{align*}
A similar calculation gives
\begin{align*}
	\int_0^\infty\left(\int_0^\infty\phi_1(t,\beta)\,dt\right)^{\frac{q_1}{p_1}}d\beta &\leq\frac{1}{q-q_1}\left(\frac{1}{p}\right)^{\frac{q_1}{p_1}}\Vert f\Vert_{L^p}^{\frac{pq_1}{p_1}}.
\end{align*}
In either case, we plug in (\ref{mkwskest}) to (\ref{mckwzest4}) to get
\begin{align*}
\Vert Tf\Vert_{L_q}^q\leq\sum_{i=0}^1\frac{q2^qC_i^{p_i}}{\vert q-q_i\vert}\left(\frac{p_i}{p}\right)^{\frac{q_i}{p_i}}\Vert f\Vert_{L^p}^{\frac{pq_i}{p_i}}
\end{align*}
Therefore
\begin{align*}
	\sup\left\{\Vert Tf\Vert_{L^q}:\Vert f\Vert_{L^p}=1\right\}\leq B_{p,q}:= 2q^{1/q}\left[\sum_{i=0}^1\frac{C_i^{p_i}}{\vert q-q_i\vert}\left(\frac{p_i}{p}\right)^{\frac{q_i}{p_i}}\right]^{1/q}.
\end{align*}
By homogeneity of norms and sublinearity of $T$, we have $\Vert Tf\Vert_{L^q}\Vert\leq B_{p,q}\Vert f\Vert_{L^p}$ for all $f\in L^p(X,\scr{F},\mu)$. The remaining cases follow by modifying this procedure.

\item\textbf{Case II\,(2): $p_0<p_1=\infty$ and $q_0<q_1=\infty$.} We have $p_0=pt$ and $q_0=qt$. We take $\alpha=\beta/C_1$, so $\Vert Th_\alpha\Vert\leq C_1\Vert h_\alpha\Vert\leq\beta$, and $\lambda_{Th_\alpha}(\beta)=0$. Then the second term in the estimate (\ref{mckwzeq1}) vanishes, and $\phi_1=0$ in the estimate (\ref{mckwzest4}). We then apply an analogue of (\ref{mkwskest}) to get
\begin{align*}
	&\Vert Tf\Vert_{L^q}^q\lesssim\int_0^\infty\left(\int_0^\infty\phi_0(t,\beta)\,dt\right)^{\frac{q_0}{p_0}}\,d\beta\leq\left(\int_0^\infty\left(\int_0^\infty\vert\phi_0(t,\beta)\vert^{\frac{q_0}{p_0}}\,d\beta\right)^{\frac{p_0}{q_0}}dt\right)^{\frac{q_0}{p_0}}\\
	&\quad=\left(\int_0^\infty\left(\int_0^{C_1t}\beta^{q-q_0-1}\,d\beta\right)^{\frac{p_0}{q_0}}t^{p_0-1}\lambda_f(t)dt\right)^{\frac{q_0}{p_0}}=\frac{1}{\vert q-q_0\vert}\left(C_1^{q-q_0}\int_0^\infty t^{p-1}\lambda_f(t)dt\right)^{\frac{q_0}{p_0}}=\frac{C_1^{q-q_0}}{\vert q-q_0\vert}\Vert f\Vert_{L^p}^q.
\end{align*}

\item\textbf{Case II\,(3): $p_0<p_1<\infty$ and $q_0<q_1=\infty$.} Since $\Vert Th_\alpha\Vert_{L^\infty}\leq C_1\Vert h_\alpha\Vert_{L^{p_1}}$, 
\begin{align}
\Vert Th_\alpha\Vert_{L^\infty}^{p_1}&\leq C_1^{p_1}\Vert h_\alpha\Vert_{L^{p_1}}^{p_1}=C_1^{p_1}p_1\int_0^\alpha t^{p_1-1}\lambda_f(t)\,dt\notag\\
&\leq C_1^{p_1}p_1\alpha^{p_1-p}\int_0^\alpha t^{p-1}\lambda_f(t)\,dt\leq C_1^{p_1}\frac{p_1}{p}\alpha^{p_1-p}\Vert f\Vert_{L^p}^p.\label{mckwzest5}
\end{align}
We take $\alpha=(\frac{\beta}{\kappa})^\sigma$,
where $\kappa=C_1(\frac{p_1}{p}\Vert f\Vert_{L^p}^p)^{1/p_1}$ and $\sigma=\frac{p_1}{p_1-p}=\frac{p_0(q-q_0)}{q_0(p-p_0)}>0$. The the estimate (\ref{mckwzest5}) is $\beta^{p_1}$. Since $\Vert Th_\alpha\Vert_{L^\infty}\leq\beta$, the second term in the estimate (\ref{mckwzeq1}) vanishes, and $\phi_1=0$ in the estimate (\ref{mckwzest4}). Then $\chi_0=\chi_{\{t>(\beta/\kappa)^{\sigma}\}}=\chi_{\{\beta<\kappa t^{1/\sigma}\}}$, and we apply an analogue of (\ref{mkwskest}) to get
\begin{align*}
	&\Vert Tf\Vert_{L^q}^q\lesssim\int_0^\infty\left(\int_0^\infty\phi_0(t,\beta)\,dt\right)^{\frac{q_0}{p_0}}\,d\beta\leq\left(\int_0^\infty\left(\int_0^\infty\vert\phi_0(t,\beta)\vert^{\frac{q_0}{p_0}}\,d\beta\right)^{\frac{p_0}{q_0}}dt\right)^{\frac{q_0}{p_0}}\\
	&\quad=\left(\int_0^\infty\left(\int_0^{\kappa t^{1/\sigma}}\beta^{q-q_0-1}\,d\beta\right)^{\frac{p_0}{q_0}}t^{p_0-1}\lambda_f(t)dt\right)^{\frac{q_0}{p_0}}=\frac{1}{q-q_0}\left(\kappa^{q-q_0}\int_0^\infty t^{p-1}\lambda_f(t)dt\right)^{\frac{q_0}{p_0}}=B_{p,q}\Vert f\Vert^{\frac{pq_0}{p_0}}.
\end{align*}

\item\textbf{Case II\,(4): $p_0<p_1<\infty$ and $q_1<q_0=\infty$.} Since $\Vert Tg_\alpha\Vert_{L^\infty}\leq C_0\Vert g_\alpha\Vert_{L^{p_0}}$, 
\begin{align}
\Vert Tg_\alpha\Vert_{L^\infty}^{p_0}&\leq C_0^{p_0}\Vert g_\alpha\Vert_{L^{p_0}}^{p_0}\leq C_0^{p_0}p_0\alpha^{p_0-p}\int_0^\alpha t^{p-1}\lambda_f(t)\,dt\leq C_0^{p_0}\frac{p_0}{p}\alpha^{p_0-p}\Vert f\Vert_{L^p}^p.\label{mckwzest6}
\end{align}
We take $\alpha=(\frac{\beta}{\kappa})^\sigma$,
where $\kappa=C_0(\frac{p_0}{p}\Vert f\Vert_{L^p}^p)^{1/p_0}$ and $\sigma=\frac{p_0}{p_0-p}=\frac{p_1(q_1-q)}{q_1(p_1-p)}<0$, so the estimate (\ref{mckwzest6}) is $\beta^{p_0}$. Since $\Vert Tg_\alpha\Vert_{L^\infty}\leq\beta$, the first term in the estimate (\ref{mckwzeq1}) vanishes, and $\phi_0=0$ in the estimate (\ref{mckwzest4}). Then $\chi_1=\chi_{\{t\leq(\beta/\kappa)^{\sigma}\}}=\chi_{\{\beta\leq\kappa t^{1/\sigma}\}}$, and we apply an analogue of (\ref{mkwskest}) to get
\begin{align*}
	&\Vert Tf\Vert_{L^q}^q\lesssim\frac{1}{q-q_1}\left(\kappa^{q-q_1}\int_0^\infty t^{p-1}\lambda_f(t)dt\right)^{\frac{q_1}{p_1}}=B_{p,q}\Vert f\Vert^{\frac{pq_0}{p_0}}.
\end{align*}
Then we complete the whole proof.
\end{proof}

\begin{corollary}[Marcinkiewicz interpolation theorem]\label{mckwz2}
	Let $1\leq p_0<p_1\leq \infty$. If $\,T$ is a sublinear operator of weak types $(p_0,p_0)$ and $(p_1,p_1)$, then $T$ is of strong type $(p,p)$ for each $p\in(p_0,p_1)$.
\end{corollary}

\section{Radon Measures}
\subsection{Locally Compact Hausdorff (LCH) Spaces}
\paragraph{Topology review.} Throughout this section, we are mainly concerned with the \textit{Locally Compact Hausdorff (LCH)} space. To be specific, the topological space $X$ of our interest has the following topological properties:
\begin{itemize}
	\item $X$ is \textit{Hausdorff}, i.e. for each pair of distinct points $x$ and $y$ in $X$, there exists a neighborhood $U_x$ of $x$ and a neighborhood $U_y$ of $y$ such that $U$ and $V$ are disjoint.
	\item $X$ is \textit{locally compact}, i.e. every point in $X$ has a compact neighborhood.
\end{itemize}
The following proposition describes that, for any set $K$ compactly included in an open set $U$, we can always find a set $V$ between them in sense of compact inclusion.
\begin{proposition}\label{precpbetween}
	If $X$ is an LCH space and  $K\subset U\subset X$, where $K$ is compact and $U$ is open, there exists a precompact open set $V$ such that $K\subset V\subset\ol{V}\subset U$.
\end{proposition}
\begin{proof} Our proof are divided into three steps.
	
	\item\textbf{Step I.} We first show that, in a Hausdorff space $X$, we can separate a compact set $K$ and a single point $x\notin K$ outside the set with disjoint neighborhoods. Formally, we find two disjoint open sets $U\supset K$ and $V\ni x$. 
	
	For each $y\in K$, by Hausdorff property, we can find two disjoint neighborhoods $U_y$ of $y$ and $V_y$ of $x$. By compactness of $K$, it is possible to cover $K$ by finitely many such neighborhoods $U_{y_1},\cdots,U_{y_n}$. We then set $U=\bigcap_{j=1}^n U_{y_j}$ and $V=\bigcap_{j=1}^n V_{y_j}$, which has the desired properties.
	
	\item\textbf{Step II.} Next, we assume $X$ is LCH and show that \emph{any open neighborhood $U$ of a point $x$ contains a compact neighborhood of $x$}. We may assume that $\ol{U}$ is compact, otherwise we may replace $U$ by its intersection with the interior of a compact neighborhood of $x$. Then $\partial U$ is also a compact set, and we can separate $x$ and $\partial U$ by two disjoint open sets $V\ni x$ and $W\supset\partial U$ in $U$. Hence $V$ satisfies $\ol{V}\subset(W^c\cap\ol{U})\subset U$, and since $U$ is precompact, $\ol{V}$ is a compact subset of $U$. Therefore $\ol{V}$ is a compact neighborhood of $x$.
	
	\item\textbf{Step III.} Finally we come to the original proposition. By Step II, we find a precompact open neighborhood $V_x$ for each $x\in K$ such that $x\subset V_x\subset\ol{V}_x\subset U$. By compactness of $K$, we take finitely many such neighborhoods $V_{x_1},\cdots,V_{x_n}$ to cover $K$. Setting $V=\bigcup_{j=1}^n V_{x_j}$, we have $K\subset V\subset\ol{V}\subset U$, and $\ol{V}$ is compact.
\end{proof}

Now we discuss the generalized version of Urysohn's lemma and Tietze extension theorem in LCH spaces. Recall that \textit{every compact Hausdorff is normal}, to which the original version of these theorems applies.

\begin{theorem}[Urysohn's lemma in LCH spaces]\label{urysohnlch}
	Let $X$ be an LCH space and $K\subset U\subset X$, where $K$ is compact and $U$ is open. There exists $f\in C(X,[0,1])$ such that $f=1$ on $K$ and $f=0$ outside a compact subset of $U$.
\end{theorem}
\begin{proof}
	We take a precompact open set $V$ such that $K\subset V\subset\ol{V}\subset U$, as in Proposition \ref{precpbetween}, so $\ol{V}$ is normal. By Urysohn's lemma for normal spaces, there exists $f\in C(\ol{V},[0,1])$ such that $f=1$ on $K$ and $f=0$ on $\partial V$. We extend $f$ to $X$ by setting $f=0$ on $\ol{V}^c$. It remains to show that $f\in C(X)$.
	
	Let $E$ be a closed subset of $[0,1]$. If $0\notin E$, we have $f^{-1}(E)=(f|_{\ol{V}})^{-1}(E)$, and if $0\in E$, we have $f^{-1}(E)=(f|_{\ol{V}})^{-1}(E)\cup\ol{V}^c=(f|_{\ol{V}})^{-1}(E)\cup V^c$ since $(f|_{\ol{V}})^{-1}(E)\supset\partial V$. In either case, $f^{-1}(E)$ is closed. Therefore $f$ is continuous.
\end{proof}

The following theorem can be proved in a similar approach.
\begin{theorem}[Tietze extension theorem in LCH spaces]
	Let $X$ be an LCH space and $K\subset X$, where $K$ is compact. If $f\in C(K)$, there exists $F\in C(X)$ such that $F|_K=f$. Moreover, $F$ may be taken to vanish outside a compact set, i.e. $F\in C_c(X)$.
\end{theorem}
\begin{proof}
	We take a precompact set $V$ such that $K\subset V\subset\ol{V}\subset X$, so $\ol{V}$ is normal. By Tietze extension theorem for normal spaces, we can extend $f$ to a function $g\in C(\ol{V})$ with $g|_K=f$. We also take a function $\phi\in C(\ol{V},[0,1])$ such that $\phi=1$ on $K$ and $\phi=0$ on $\partial V$ by Urysohn's lemma. Then $g\phi\in C(\ol{V})$ agrees with $f$ on $K$. We take $F=g\phi$ on $\ol{V}$ and $F=0$ in $\ol{V}^c$. Then $F\in C_c(X)$ and $F|_K=f$. 
\end{proof}

\paragraph{Alexandroff compactification.} 
If $X$ is a noncompact LCH space, it is possible to make $X$ into a compact Hausdorff space by adding a single point at the ``infinity''. Let us take some object that is not a point of $X$, denoted by the symbol $\infty$ for convenience, and adjoin it to $X$, forming the set $X^*=X\cup\{\infty\}$. We topologize $X^*$ by defining the collection $\mathscr{T}^*$ of open sets of $X^*$ to consist of 
\begin{itemize}
	\item[(i)] all sets $U$ that are open in $X$, and
	\item[(ii)] all sets of the form $X^*\backslash K$, where $K$ is a compact subset of $X$.
\end{itemize}
We first check that such collection is indeed a topology on $X^*$. 
\begin{itemize}
	\item The empty set $\emptyset$ and $X^*$ are open sets of type (i) and (ii), respectively.
	\item Let $U_1$ and $U_2$ be open sets in $X$, and let $K_1$ and $K_2$ be compact sets in $X$. Then \vspace{-0.05cm}
	\begin{itemize}
		\item $U_1\cap U_2$ is of type (i), \vspace{-0.05cm}
		\item $(X^*\backslash K_1)\cap (X^*\backslash K_2) = X^*\backslash(K_1\cup K_2)$ is of type (ii), and \vspace{-0.05cm}
		\item $U\cap (X^*\backslash K) = U\cap (X\backslash K)$ is of type (i).\vspace{-0.05cm}
	\end{itemize}
	Hence $\mathscr{T}^*$ is closed under the finite intersection operation.
	\item Let $\{U_\alpha\}$ be a collection of open sets of $X$, and let $\{K_\beta\}$ be a collection of compact sets in $X$. Then \vspace{-0.05cm}
	\begin{itemize}
		\item $\bigcup_{\alpha}U_\alpha = U$ is of type (i),\vspace{-0.05cm}
		\item $\bigcup_{\beta}(X^*\backslash K_\beta) = X^*\backslash\bigcap_\beta K_\beta = X^*\backslash K$ is of type (ii), and \vspace{-0.05cm}
		\item $U \cup (X^*\backslash K) = X^*\backslash (K\backslash U)$ is of type (ii) since $K\backslash U$ is a compact subset of $X$.\vspace{-0.05cm}
	\end{itemize}
	Hence $\mathscr{T}^*$ is closed under the union operation.
\end{itemize} 
Then we need to verify that $X$ is a subspace of $X^*$:
\begin{itemize}
	\item Given any open set in $X^*$, its intersection with $X$ is open in $X$. If the open set is of type (i), it is clearly open in $X$. If it is of type (ii), then $(X^*\backslash K)\cap X = X\backslash K$ is open in Hausdorff space $X$.
	\item Conversely, given any open set in $X$, it is a type (i) open set in $X^*$.
\end{itemize}
Next we verify that $X^*$ is a compact topological space. 
\begin{itemize}
	\item If $\mathscr{A}$ is an open cover of $X^*$, it must contain at least one open set $X^*\backslash K$ of type (ii), to contain $\infty$. 
	\item Taking all members in $\mathscr{A}$ but $X^*\backslash K$ and intersect them with $X$, we obtain a cover of $X$. Since $K$ is a compact subset of $X$, finitely many of them cover $K$. Then the corresponding finite collection of elements of $\mathscr{A}$ along with $X^*\backslash K$ form a cover of $X^*$.
\end{itemize}
Finally we verify that $X^*$ is a Hausdorff space. Let $x$ and $y$ be two distinct points of $X^*$:
\begin{itemize}
	\item The case that both $x$ and $y$ lies in $X$ is clear since $X$ is Hausdorff.
	\item If $y=\infty$, we choose a compact set $K$ in $X$ that contains a neighborhood $U$ of $x$, then $U$ and $X^*\backslash K$ are disjoint neighborhoods of $x$ and $\infty$, respectively, in $X^*$.
\end{itemize}
The comapact Hausdorff space $X^*$ is called the \textit{one point compactification/Alexandroff compactification of $X$}.

\paragraph{Functions vanishing at infinity.} Let $X$ be a topological space. A continuous function $f\in C(X)$ is said to \textit{vanish at infinity} if the set $\{x\in X:\vert f(x)\vert\geq\epsilon\}$ is compact for every $\epsilon>0$. We define $C_0(X)$ to be the space of functions vanishing at infinity.
\begin{proposition}
	Let $X$ be an LCH space, and $f\in C(X)$. The function $f$ extends continuously to the Alexandroff compactification $X^*$ of $X$ if and only if there exists function $g\in C_0(X)$ and $z\in\bbC$ such that $f=g+c$, in which case the continuous extension is given by $f(\infty)=c$.
\end{proposition}
\begin{proof}
	Assume $f=g+c$, where $g\in C_0(X)$ and $c\in\bbC$. Replacing $f$ by $f-c$, we may further assume $c=0$. We extend $f$ to $X^*$ by setting $f(\infty)=0$, and show that $f$ is continuous. Let $U$ be an open subset of $\bbC$. 
	\begin{itemize}
		\item If $0\notin U$, then $f^{-1}(U)=(f|_X)^{-1}(U)$, which is open by continuity of $f|_X$. 
		\item If $0\in U$, there exists $\epsilon>0$ such that $\vert z\vert\geq\epsilon$ for all $z\in U^c$. Since $f|_X\in C_0(X)$, $(f|_X)^{-1}(U^c)$ is a closed subset of the compact set $\{x\in X:\vert f(x)\vert\geq\epsilon\}$ in $X$. Hence $f^{-1}(U)=X^*\backslash (f|_X)^{-1}(U^c)$ is open.
	\end{itemize}
	
	Conversely, if $f\in C(X)$ extends continuously to $X^*$, we let $c=f(\infty)$ and $g=f-c$. For each $\epsilon>0$, the set $g^{-1}(B(0,\epsilon))=\{x\in X^*:\vert g(x)\vert<\epsilon\}$ is open in $X^*$ and contains $\infty$. Consequently, the complement $\{x\in X^*:\vert g(x)\vert\geq\epsilon\}$ is a compact set in $X$. Therefore $g\in C_0(X)$.
\end{proof}
\paragraph{Topologies on $\bbC^X$.} Let $X$ be a topological space. There are various ways to topologize the space $\bbC^X$ of all complex-valued functions on $X$:
\begin{itemize}
	\item\textit{The topology of pointwise convergence/the product topology} is generated by the sets
	\begin{align*}
		U_{x_1,\cdots,x_m}^\epsilon(f)=\left\{g\in\bbC^X:\vert f(x_j)-g(x_j)\vert<\epsilon,\ j=1,2,\cdots,m\right\},
	\end{align*}
	where $f\in\bbC^X$, $\epsilon>0$ and $x_1,\cdots,x_m\in X$. In this topology, a sequence $(f_n)$ of functions converges to $f$ when $f_n\to f$ pointwise.
	\item\textit{The topology of compact convergence} is generated by the sets
	\begin{align*}
		U_K^\epsilon(f)=\left\{g\in\bbC^X:\sup_{x\in K}\vert f(x)-g(x)\vert<\epsilon\right\},
	\end{align*}
	where $f\in\bbC^X$, $\epsilon>0$ and $K$ is a compact subset of $X$. In this topology, a sequence $(f_n)$ of functions converges to $f$ when $f_n\to f$ uniformly on every compact subset $K$ of $X$.
	\item\textit{The topology of uniform convergence} is generated by the sets
	\begin{align*}
		U_\infty^\epsilon(f)=\left\{g\in\bbC^X:\sup_{x\in X}\vert f(x)-g(x)\vert<\epsilon\right\},
	\end{align*}
	where $f\in\bbC^X$ and $\epsilon>0$. In this topology, a sequence $(f_n)$ of functions converges to $f$ when $f_n\to f$ uniformly on $X$.
\end{itemize}

Basic analysis shows that the space $C(X)$ of continuous functions on $X$ is not a closed subspace of $\bbC^X$ in the topology of pointwise convergence, but when we switch to the uniform topology, it is. The following theorem asserts that $C(X)$ is also closed in the topology of compact convergence when $X$ is an LCH space.

\begin{proposition}
	If $X$ is an LCH space, $C(X)$ is closed in $\bbC^X$ in the topology of compact convergence. 
\end{proposition}
\begin{proof}
	We claim that, \textit{a subset $E$ of $X$ is closed if and only if $E\cap K$ is closed for each compact set $K\subset X$}. In fact, if $E$ is closed, $E\cap K$ must be closed since it is the intersection of two closed sets. On the other hand, if $E$ is not closed, we choose a point $x\in\ol{E}\backslash E$ and let $K$ be a compact neighborhood of $x$. Then $x$ is a limit point of $E\cap K$, however it is not in $E\cap K$.
	
	Now we prove the desired result. If $f$ is in the closure of $C(X)$, then for each compact subset $K$ of $X$, the restriction $f|_K$, being a uniform limit of continuous functions on $K$, is continuous. Then for any closed set $E\subset X$, the intersection $f^{-1}(E)\cap K=(f|_K)^{-1}(E)$ is closed for all compact subset $K$ of $X$, and hence $f^{-1}(E)$ is closed. Therefore $f$ is also in $C(X)$.
\end{proof}

\begin{proposition}\label{c0closecc}
	If $X$ is an LCH space, $C_0(X)=\ol{C_c(X)}$ in the uniform topology.
\end{proposition}
\begin{proof}
	If $f$ is in the closure of $C_c(X)$, for every $\epsilon>0$, we can take some $g\in C_c(X)$ such that $\Vert f-g\Vert_\infty<\epsilon$. Then $\{x\in X:\vert f(x)\vert\geq\epsilon\}\subset\supp g$, which are compact sets. 
	
	Conversely, if $f\in C_0(X)$, we show how to find a function $g\in C_c(X)$ with $\Vert f-g\Vert_\infty<\epsilon$ for any $\epsilon>0$. We take the compact set $K=\{x\in X:\vert f(x)\vert>\epsilon\}$, and take $\phi\in C_c(X,[0,1])$ such that $\phi=1$ on $K$ by Urysohn's lemma [Theorem \ref{urysohnlch}]. Setting $g=f\phi$ completes the proof.
\end{proof}

\begin{proposition}[Partition of unity]\label{partunity}
	Let $X$ be an LCH space, $K$ a compact subset of $X$, and $(U_j)_{j=1}^n$ an open cover of $K$. There exists a family of functions $\phi_j\in C_c(U_j,[0,1])$ such that $\sum_{j=1}^n \phi_j(x)=1$ for all $x\in K$. 
\end{proposition}
\begin{proof}
	By Proposition \ref{precpbetween}, for each $x\in X$, we take a precompact open neighborhood $V_x$ of $x$ contained in some $U_j$. Then by compactness of $K$, there exist finitely many $V_{x_1},\cdots,V_{x_m}$ that form a cover of $K$. We denote by $K_j$ the union of neighborhoods $V_{x_k}$ contained in $U_j$. By Urysohn's lemma, for each $j=1,2,\cdots,n$ we can find a function $g_j\in C_c(U_j,[0,1])$ such that $g_j=1$ on $K_j$. Furthermore, there also exists a function $f\in C_c(X,[0,1])$ such that $f=1$ on $K$ and $\supp(f)\subset\{x\in X:\sum_{j=1}^n g_k(x)>0\}$. Let $g_{n+1}=1-f$, so that $\sum_{j=1}^{n+1}g_j>0$ everywhere. Taking $\phi_j=g_j/\sum_{k=1}^{n+1}g_k$, we have $\phi_j\in C_c(U_j,[0,1])$ and $\sum_{j=1}^n\phi_j=1$ on $K$.
\end{proof}

\paragraph{$\sigma$-compactness.} A topological space is said to be \textit{$\sigma$-compact} if it is a countable union of compact sets. Formally, if $X$ is $\sigma$-compact, there exists compact subsets $K_n\subset X$ such that $X=\bigcup_{n=1}^\infty K_n$. Replacing $K_n$ by the union of itself and all preceding members, we may assume that $(K_n)$ is an increasing sequence.

A second countable LCH space is $\sigma$-compact. To see this, we take a precompact open neighborhood $U_x$ for each $x\in X$. Consequently, we can find a base set $B_x\in\mathscr{B}$ such that $x\in B_x\subset U_x$, and $\overline{B}_x$ is compact. We choose $\mathscr{B}_c\subset\mathscr{B}$ to be the collection of all precompact base sets. Then $B_x\in\mathscr{B}_c$ for all $x\in X$, and $X=\bigcup_{B\in\mathscr{B}_c}\overline{B}$ is a countable union of compact sets. Therefore, $X$ is a $\sigma$-compact topological space.

\begin{proposition}\label{sigmacompactinc}
	Let $X$ be a $\sigma$-compact LCH space. There exists a sequence $(U_n)_{n=1}^\infty$ of precompact open sets such that $U_1\subset\ol{U}_1\subset U_2\subset\ol{U}_2\subset U_3\subset\cdots U_n\subset\ol{U}_n\subset U_{n+1}\subset\cdots$ and $X=\bigcup_{n=1}^\infty U_n$. Furthermore, for all compact set $K\subset X$, there exists $n\in\bbN$ such that $U_n\supset K$.
\end{proposition}
\begin{proof}
	By $\sigma$-compactness of $X$, there exists a sequence $(K_n)_{n=1}^\infty$ of compact sets increasing to $X$. We start by taking a precompact open neighborhood $U_x$ for each $x\in X$ and setting $U_0=\emptyset$. With $U_{n-1}$ constructed, the union $\ol{U}_{n-1}\cup K_n$ is compact, and there exists finitely many $x_1,\cdots,x_k\in X$ such that $(\ol{U}_{n-1}\cup K_n)\subset\bigcup_{j=1}^k U_{x_j}$. We construct $U_n=\bigcup_{j=1}^k U_{x_j}$, which is also precompact open. Then we have $\ol{U}_{n-1}\subset U_n$. Moreover,
	\begin{align*}
		\bigcup_{n=1}^\infty U_n\supset\bigcup_{n=1}^\infty K_n=X.
	\end{align*} 
	Hence the sequence $(U_n)$ has the desired property. Moreover, for any compact subset $K$ of $X$, $\{U_n\}_{n=1}^\infty$ is an open cover of $K$, hence there exists $U_n$ such that $K\subset U_n$.
\end{proof}

\begin{proposition}
	Let $X$ be a $\sigma$-compact LCH space, and let $(U_n)_{n=1}^\infty$ be a sequence of precompact sets as in Proposition \ref{sigmacompactinc}. Then for each $f\in\bbC^X$, the sets
	\begin{align}
		\left\{g\in\bbC^X:\sup_{x\in\ol{U}_n}\vert g(x)-f(x)\vert< \frac{1}{m}\right\},\quad m,n\in\bbN\label{cmpcvgnb}
	\end{align}
	form a neighborhood base for $f$ in the topology of compact convergence. Hence this topology is first countable, and $f_k\to f$ uniformly on compact sets if and only if $f_n\to f$ uniformly on each $\ol{U}_n$.
\end{proposition}
\begin{proof}
	For $f\in\bbC^X$, any neighborhood of $f$ in the topology of compact convergence contains a set of the form
	\begin{align*}
		U_K^\epsilon(f)=\left\{g\in\bbC^X:\sup_{x\in K}\vert g(x)-f(x)\vert<\epsilon\right\},
	\end{align*}
	where $K$ is a compact subset of $X$ and $\epsilon>0$. We choose $n,m\in\bbN$ such that $K\subset U_n$ and $\frac{1}{m}<\epsilon$. Then
	\begin{align*}
		U_K^\epsilon(f)\supset\left\{g\in\bbC^X:\sup_{x\in\ol{U}_n}\vert g(x)-f(x)\vert< \frac{1}{m}\right\}.
	\end{align*}
	Therefore the sets of the form (\ref{cmpcvgnb}) form a neighborhood base for $f$.
\end{proof}

\newpage
\subsection{Positive Linear Functionals on $C_c(X)$ and Radon Measures}
Throughout this section, we assume that $X$ is an LCH space. One of the vector spaces we are interested in is the space $C_c(X)$ of continuous functions on $X$ with compact support.
\begin{definition}[Positive linear functionals]
	Let $X$ be an LCH space. A \textit{positive linear functional on $C_c(X)$} is a linear functional $T:C_c(X)\to\bbC$ such that $Tf\geq 0$ for all $f\in C_c(X)$ with $f\geq 0$.
\end{definition}

The positivity condition implies a continuity property of $T$.
\begin{proposition}\label{plfcont}
	If $\,T$ is a positive linear functional on $C_c(X)$, for each compact set $K\subset X$, there exists a constant $C_K>0$ such that $\vert Tf\vert\leq C_K\Vert f\Vert_\infty$ for all $f\in C_c(X)$ with $\supp(f)\subset K$.
\end{proposition}
\begin{proof}
	By dividing $f\in C_c(X)$ into real and imaginary parts, it suffices to consider real-valued functions $f$. By Urysohn's lemma, for any compact $K\subset X$, there is a function $\phi\in C_c(U,[0,1])$ such that $\phi=1$ on $K$. Then if $\supp(f)\subset K$, we have $\vert f\vert\leq\Vert f\Vert_\infty\phi$. Hence both $\Vert f\Vert_\infty\phi-f$ and $\Vert f\Vert_\infty\phi+f$ are nonnegative, and
	\begin{align*}
		T\phi\Vert f\Vert_\infty-Tf\geq 0,\quad T\phi\Vert f\Vert_\infty+Tf\geq 0
	\end{align*}
	Therefore $\vert Tf\vert\leq T\phi\Vert f\Vert_\infty$, which concludes the proof by setting $C_K=T\phi$.
\end{proof}
\begin{remark}
If we replace $C_c(X)$ by $C^\infty(X)$, this proposition still holds, because we can make $\phi\in C_c^\infty(U,[0,1])$ in our proof by $C^\infty$-Urysohn Lemma.
\end{remark}
The positive linear functionals on $C_c(X)$ is closely related to a family of Borel measures on $X$ with some regular properties. Intuitively, we let $\mu$ be a Borel measure on $X$ such that $\mu(K)<\infty$ for all compact $K\subset U$. Then the map $f\mapsto\int_X f\,d\mu$ is a positive linear functional on $C_c(X)$, since $f\in C_c(X)\subset L^1(\mu)$.

\begin{definition}[Radon measures]
	Let $X$ be a topological space, $\scr{B}$ the Borel $\sigma$-algebra on $X$, and $\mu$ a measure on $(X,\mathscr{B})$. Let $E$ be a Borel subset of $X$.
	\begin{itemize}
		\item[(i)] $\mu$ is said to be \textit{outer regular on $E$}, if
		\begin{align*}
			\mu(E)=\inf\left\{\mu(U):U\supset E,\ U\ \text{is open}\right\}.
		\end{align*}
		\item[(ii)] $\mu$ is said to be \textit{inner regular on $E$}, if
		\begin{align*}
			\mu(E)=\inf\left\{\mu(K):K\subset E,\ K\ \text{is compact}\right\}.
		\end{align*}
		\item[(iii)] $\mu$ is said to be \textit{regular}, if it is outer and inner regular on all Borel sets.
		\item[(iv)] $\mu$ is called a \textit{Radon measure}, if it is finite on all compact sets, outer regular on all Borel sets, and inner regular on all open sets.
	\end{itemize}
\end{definition}

The following theorem relates every positive linear functional on $C_c(X)$ with a Radon measure on $X$.
\begin{theorem}[Riesz representation theorem]\label{rieszcc}
	Let $X$ be a LCH space. If $\,T$ is a positive linear functional on $C_c(X)$, there exists a unique Radon measure $\mu$ on $X$ such that
	\begin{align*}
		Tf=\int_X f\,d\mu,\quad\forall f\in C_c(X).
	\end{align*}
	Furthermore, for all open sets $U\subset X$, $\mu$ satisfies
	\begin{align*}
		\mu(U)=\sup\left\{Tf:f\in C_c(U),\ 0\leq f\leq 1\right\},
	\end{align*}
	and for all compact sets $K\subset X$,
	\begin{align*}
		\mu(K)=\inf\left\{Tf:f\in C_c(X),\ f\geq\chi_K\right\}.
	\end{align*}
\end{theorem}

We begin by constructing a Radon measure from a positive linear functional on $C_c(X)$.
\begin{lemma}\label{radonrieszlemma1}
	Let $T$ be a positive linear functional on $C_c(X)$. For each open $U\subset X$, define
	\begin{align*}
		\mu(U)=\sup\left\{Tf:f\in C_c(U,[0,1])\right\},
	\end{align*}
	and for each subset $E\in 2^X$, define
	\begin{align}
		\mu^*(E)=\inf\left\{\mu(U):U\supset E,\ U\ \text{\rm is open}\right\}.\label{radonouterreg}
	\end{align}
	Then $\mu^*$ is an outer measure on $X$, and every open set $U\subset X$ is $\mu^*$-measurable, i.e.
	\begin{align}
		\mu^*(E)=\mu^*(E\cap U)+\mu^*(E\backslash U)\quad\text{\rm for all}\ E\in 2^X.\label{caratheodory}
	\end{align}
\end{lemma}
\begin{proof}
	By definition of $\mu$, we have $\mu(\emptyset)=0$, and $\mu(U)\leq\mu(V)$ for any open sets $U\subset V$. Hence $\mu^*(E)\leq\mu^*(F)$ for all $E\subset F\subset X$, and $\mu^*(U)=\mu(U)$ for all open $U$. We then show that for a sequence of open sets $(U_n)_{n=1}^\infty$ and $U=\bigcup_{n=1}^\infty U_n$, it holds $\mu(U)\leq\sum_{n=1}^\infty\mu(U_n)$. For any $f\in C_c(U,[0,1])$, let $K=\supp(f)$. By compactness of $K$, we have $K\subset\bigcup_{j=1}^n\mu(U_j)$ for some finite $n\in\bbN$. By Proposition \ref{partunity}, there exists a family of functions $g_j\in C_c(U_j,[0,1])$ such that $\sum_{j=1}^n g_n=1$ on $K$. Then $f=\sum_{j=1}^n fg_j$, and
	\begin{align*}
		Tf=\sum_{j=1}^n T(fg_j)\leq\sum_{j=1}^n\mu(U_j)\leq \sum_{n=1}^\infty \mu(U_n).
	\end{align*}
	By taking the supremum over $f\in C_c(U,[0,1])$, we have $\mu(U)\leq\sum_{n=1}^\infty \mu(U_n)$. More generally, if $(E_n)_{n=1}^\infty$ is a sequence of subsets of $X$ and $E=\bigcup_{n=1}^\infty E_n$, we take an open set $U_n\supset E_n$ for each $E_n$ and get
	\begin{align*}
		\sum_{n=1}^\infty\mu(U_n)\geq\mu\left(\bigcup_{n=1}^\infty U_n\right)\geq \mu(E).
	\end{align*}
	By taking the infimum over $(U_n)_{n=1}^\infty$, we have $\sum_{n=1}^\infty\mu(E_n)\geq \mu(E)$. Hence $\mu^*$ is an outer measure on $X$.
	
	Now we verify the condition \ref{caratheodory}. We first assume that $E$ is open, so that $E\cap U$ is open. For any $\epsilon>0$, we can find $f\in C_c(E\cap U,[0,1])$ such that $Tf>\mu(E\cap U)-\epsilon$. Similarly, we can find $g\in C_c(E\backslash\supp(f),[0,1])$ such that $Tg>\mu(E\backslash\supp(f))-\epsilon$. Then $f+g\in C_c(E,[0,1])$, and
	\begin{align*}
		\mu(E)\geq Tf+Tg\geq\mu(E\cap U)+\mu(E\backslash\supp(f))-2\epsilon \geq \mu^*(E\cap U)+\mu^*(E\backslash U)-2\epsilon.
	\end{align*}
	Letting $\epsilon\to 0$, we obtain the desired inequality. For the general case $E\in 2^X$, we may assume $\mu^*(E)<\infty$ and find an open $V\supset E$ such that $\mu^*(V)<\mu^*(E)+\epsilon$, and hence
	\begin{align*}
		\mu^*(E)+\epsilon>\mu^*(V)\geq\mu^*(V\cap U)+\mu^*(V\backslash U)\geq\mu^*(E\cap U)+\mu^*(E\backslash U).
	\end{align*}
	Letting $\epsilon\to 0$, we are done.
\end{proof}

\begin{remark} By Carathéodory's extension theorem, the family of $\mu^*$-measurable sets is a $\sigma$-algebra on $X$, which contains the Borel $\sigma$-algebra $\mathscr{B}$. By taking the restriction $\mu=\mu^*|_{\mathscr{B}}$, we obtain a Borel measure on $X$.
\begin{lemma}\label{radonrieszlemma2}
	The restriction $\mu=\mu^*|_{\mathscr{B}}$ of the outer measure $\mu^*$ in Lemma \ref{radonrieszlemma1} on the Borel algebra $\scr{B}$ defines a Radon measure on $X$. Furthermore, for each compact set $K\subset X$,
	\begin{align}
		\mu(K)=\inf\left\{Tf:f\in C_c(X),\ f\geq\chi_K\right\}.\label{radoncmpapprox}
	\end{align}
\end{lemma}
\end{remark}
\begin{proof}
	By (\ref{radonouterreg}), the Borel measure $\mu$ is outer regular on all Borel sets in $X$. If $K$ is compact, $f\in C_c(X)$ and $f\geq\chi_K$, we define $U_\epsilon=\left\{x\in X:f(x)\geq 1-\epsilon\right\}$, which is an open set. If $g\in C_c(U_\epsilon,[0,1])$, we have $f-(1-\epsilon)g\geq 0$, and $Tf\geq(1-\epsilon)Tg$. Hence
	\begin{align*}
		\mu(K)\leq\mu(U_\epsilon)=\inf\{Tg:g\in C_c(U_\epsilon,[0,1])\}\leq\frac{Tf}{1-\epsilon}.
	\end{align*}
	Letting $\epsilon\to 0$, we have $\mu(K)\leq Tf$, and hence $\mu(K)<\infty$. On the other hand, for any open $U\supset K$, by Urysohn's lemma, there exists $f\in C_c(U,[0,1])$ such that $f\geq\chi_K$, and we have $Tf\leq\mu(U)$ by definition of $\mu$ in Lemma \ref{radonrieszlemma1}. Since $\mu$ is outer regular, the result (\ref{radoncmpapprox}) follows.
	
	To verify that $\mu$ is a Radon measure, it remains to show that it is inner regular on all open sets. If $U$ is open and $\epsilon>0$, we choose $f\in C_c(U,[0,1])$ such that $Tf>\mu(U)-\epsilon$ and let $K=\supp(f)$. If $g\in C_c(X)$ and $g\geq\chi_K$, we have $g-f\geq 0$ and $Tg\geq Tf>\mu(U)-\epsilon$.  Then $\mu(K)>\mu(U)-\epsilon$, and $\mu$ is inner regular on $U$.
\end{proof}

\begin{proof}[Proof of Theorem \ref{rieszcc}]
	We start by establishing the uniqueness. Assume $\mu$ is a Radon measure such that $\int_X f\,d\mu=Tf$ for all $f\in C_c(X)$. If $U\subset X$ is open, we have $Tf=\int_X f\,d\mu\leq\mu(U)$ for all $f\in C_c(U,[0,1])$. On the other hand, if $K\subset U$ is a compact set, we take $f\in C_c(U,[0,1])$ such that $f=1$ on $K$ by Urysohn's lemma, so that $\mu(K)\leq \int_X f\,d\mu=Tf$. Since $\mu$ is inner regular on $U$, we have $$\mu(U)=\sup\left\{Tf:f\in C_c(U,[0,1])\right\}.$$ Thus $\mu$ is determined by $T$ on all open sets, hence on all Borel sets by outer regularity.
	
	To prove the existence, we take the Radon measure constructed in Lemmata \ref{radonrieszlemma1} and \ref{radonrieszlemma2}. It remains to show that $Tf=\int_X f\,d\mu$ for all $f\in C_c(X)$. We may assume $0\leq f\leq 1$, since $f$ is a linear combination of functions in $C_c(X,[0,1])$. Fix $N\in\bbN$. We define $K_j=\left\{x\in X:f(x)\geq\frac{j}{N}\right\}$ for each $j=1,2,\cdots,N$ and $K_0=\supp(f)$. Also, we divide $f$ by $f=\sum_{j=1}^Nf_j$, where $f_1,\cdots,f_N\in C_c(X)$ are defined as the truncation of $f$ on the interval $\left[\frac{j-1}{N},\frac{j}{N}\right]$:\vspace{-0.1cm}
	\begin{align*}
		f_j=\min\left\{\max\left\{f-\frac{j-1}{N},0\right\},\frac{1}{N}\right\}.
	\end{align*}
	Then $N^{-1}\chi_{K_j}\leq f_j\leq N^{-1}\chi_{j-1}$, and\vspace{-0.1cm}
	\begin{align*}
		\frac{\mu(K_j)}{N}\leq \int_X f_j\,d\mu\leq\frac{\mu(K_{j-1})}{N}.
	\end{align*}
	If $U\supset K_{j-1}$ is an open set, we have $Nf_j\in C_c(U,[0,1])$, and
	$Tf_j\leq\frac{\mu(U)}{N}$. Hence by (\ref{radoncmpapprox}) and outer regularity,\vspace{-0.1cm}
	\begin{align*}
		\frac{\mu(K_j)}{N}\overset{(\ref{radoncmpapprox})}{\leq} Tf_j\leq\frac{1}{N}\inf\left\{\mu(U):U\supset K_{j-1},\ U\ \text{is open}\right\}=\frac{\mu(K_{j-1})}{N}.
	\end{align*}
	Using $f=\sum_{j=1}^Nf_j$, we have\vspace{-0.1cm}
	\begin{align*}
		\frac{1}{N}\sum_{j=1}^N\mu(K_j)\leq \int_X f\,d\mu\leq \frac{1}{N}\sum_{j=0}^{N-1}\mu(K_j),\quad\text{and}\quad \frac{1}{N}\sum_{j=1}^N\mu(K_j)\leq Tf\leq \frac{1}{N}\sum_{j=0}^{N-1}\mu(K_j).
	\end{align*}
	Hence\vspace{-0.1cm}
	\begin{align*}
		\left\vert Tf-\int_X f\,d\mu\right\vert\leq\frac{\mu(K_0)-\mu(K_N)}{N}\leq\frac{\mu(\supp(f))}{N}.
	\end{align*}
	Since $\mu(\supp(f))<\infty$, we let $N\to\infty$ and conclude that $Tf=\int_X f\,d\mu$.
\end{proof}
\begin{remark} 
For any $f\in C_c(X)$ supported on $K$, we can take a mollification sequence $f^\epsilon=\phi_\epsilon*f\in C_c^\infty(X)$ that converges to $f$ uniformly.
Therefore, if $T$ is a positive linear functional on $C_c^\infty(X)$, by the remark under Proposition \ref{plfcont}, we can extend $T$ to a linear functional on $C_c(X)$. Through this procedure, we can also relate each positive linear functional on a subspace of $C_c(X)$ containing $C_c^\infty(X)$ to a unique Radon measure on $X$.
\end{remark}

\newpage
\subsection{Regularity and Approximation of Radon Measures}
In this section we discuss more properties of Radon measures. 
\begin{proposition}\label{radonreg}
	Every Radon measure is inner regular on all of its $\sigma$-finite sets.
\end{proposition}
\begin{proof}
	Let $\mu$ be a Radon measure on $X$ and $E\subset X$ a $\sigma$-finite set. If $\mu(E)<\infty$, for any $\epsilon>0$, we take an open set $U\supset E$ with $\mu(U)<\mu(E)+\epsilon$ and a compact set $F\subset U$ such that $\mu(F)>\mu(U)-\epsilon$. Since $\mu(U\backslash E)<\epsilon$, we can also take an open set $V\supset U\backslash E$ such that $\mu(V)<\epsilon$. Let $K=F\backslash V$, which is compact. Then $K\subset U\backslash V\subset E$, and
	\begin{align*}
		\mu(K)=\mu(F)-\mu(F\cap V)>\mu(U)-\epsilon-\mu(V)>\mu(E)-2\epsilon.
	\end{align*}
	Hence $\mu$ is inner regular on $E$. On the other hand, if $\mu(E)=\infty$, $E$ is the limit of an increasing sequence $(E_n)_{n=1}^\infty$ of $\mu$-finite sets such that $\mu(E_n)\to\infty$. Hence for any $N>0$ there exists $n\in\bbN$ such that $\mu(E_n)>N$. By the preceding argument, one can take a compact $K\subset E_n$ with $\mu(K)>N$. Hence the supremum of $\mu(K)$ over compact $K\subset E$ is $\infty$, and $\mu$ is inner regular on $E$.
\end{proof}

We have some immediate corollaries of this proposition.
\begin{corollary}
	Every $\sigma$-finite Radon measure is regular. Particularly, if $X$ is a $\sigma$-compact space, every Radon measure on $X$ is regular.
\end{corollary}
\begin{proposition}\label{sandwichapprox}
	Let $\mu$ be a $\sigma$-finite Radon measure on $X$ and $E$ a Borel set in $X$.
	\begin{itemize}
		\item[(i)] For every $\epsilon>0$, there exists an open $U$ and a closed $F$ with $F\subset E\subset U$ and $\mu(U\backslash F)<\epsilon$.
		\item[(ii)] There exists an $F_\sigma$ set $A$ and a $G_\delta$ set $B$ such that $A\subset E\subset B$ and $\mu(B\backslash A)=0$.
	\end{itemize}
\end{proposition}
\begin{proof}
	We write $E=\bigcup_{n=1}^\infty E_n$ where the $E_j$'s are disjoint and have finite measure. For each $E_n$, choose an open $U_n\supset E_n$ with $\mu(U_n)<\mu(E_n)+2^{-1-n}\epsilon$ and let $U=\bigcup_{n=1}^\infty U_n$. Then $U$ is an open set containing $E$ and $\mu(U\backslash E)\leq\sum_{n=1}^\infty\mu(U_n\backslash E_n)<\epsilon/2$. Applying the same approach to $E^c$, we get an open $V\supset E^c$ with $\mu(V\backslash E^c)<\epsilon/2$. Let $F=V^c$. Then $F$ is a closed set contained in $E$, and
	\begin{align*}
		\mu(U\backslash F)=\mu(U\backslash E)+\mu(E\backslash F)=\mu(U\backslash E)+\mu(V\backslash E^c)<\epsilon.
	\end{align*}
	
	Now for each $k\in\bbN$, by the preceding argument, we choose an open $U_k$ and a closed $F_k$ with $F_k\subset E\subset U_k$ and $\mu(U_k\backslash F_k)<1/k$. We may also assume $U_k\subset U_{k-1}$ by taking $U_k\cap U_{k-1}$ if necessary. Similarly we assume $F_k\supset F_{k-1}$. Let $B=\bigcap_{k=1}^\infty U_k$, which is a $G_\delta$ set, and $A=\bigcup_{k=1}^\infty F_k$, which is an $F_\sigma$ set. Then
	\begin{align*}
		\mu(B\backslash A)=\mu\left(\bigcap_{k=1}^\infty(U_k\backslash F_k)\right)=\lim_{k\to\infty}\mu(U_k\backslash F_k)=0,
	\end{align*}
	and $A\subset E\subset B$, which concludes the proof.
\end{proof}

The following theorem discusses the regularity of Borel measures in LCH spaces.
\begin{theorem}\label{borelisradon}
	Let $\mu$ be a Borel measure on an LCH space $X$ in which every open set is $\sigma$-compact (which is the case, for example, if $X$ is second countable). If $\mu$ is finite on compact sets, it is regular.
\end{theorem}
\begin{proof}
	Since $\mu$ is finite on compact sets, we have $\int_X f\,d\mu<\infty$ for all $f\in C_c(X)$, and $T_\mu f=\int_X f\,d\mu$ defines a positive linear functional $T_\mu$ on $C_c(X)$. Let $\nu$ be the associated Radon measure according to Theorem \ref{rieszcc}. If $U\subset X$ is open, let $(K_n)_{n=1}^\infty$ be a sequence of compact sets increasing to $U$. We take $f_1\in C_c(U,[0,1])$ such that $f=1$ on $K_1$, and inductively take $f_n\in C_c(U,[0,1])$ such that $f=1$ on $K_n\cup\supp(f_{n-1})$. Then $f_n\uparrow \chi_U$ pointwise, and by monotone convergence theorem,
	\begin{align*}
		\mu(U)=\lim_{n\to\infty}\int_X f_n\,d\mu=\lim_{n\to\infty}\int_X f_n\,d\nu=\nu(U).
	\end{align*}
	
	Next, if $E$ is any Borel set and $\epsilon>0$, by Proposition \ref{sandwichapprox}, there exists open $U\supset E$ and closed $V\subset E$ with $\nu(U\backslash F)<\epsilon$. Since $U\backslash F$ is open, $\mu(U\backslash F)=\nu(U\backslash F)<\epsilon$. In particular, $\mu(U)<\mu(E)+\epsilon$, and $\mu$ is outer regular. Also, we have $\mu(F)>\mu(E)-\epsilon$. Since $X$ is $\sigma$-compact, there exist compact sets $K_n\subset F$ with $\mu(K_n)\to\mu(F)$, and $\mu$ is inner regular. Therefore $\mu$ is regular on $X$.
\end{proof}
\begin{remark} By the uniqueness part of Theorem \ref{rieszcc}, since $\mu$ is Radon, we have $\mu=\nu$.
\end{remark}

\begin{proposition}\label{ccdenseinlp}
	If $\mu$ a Radon measure on an LCH space $X$, $C_c(X)$ is dense in $L^p(\mu)$ for $1\leq p<\infty$. 
\end{proposition}
\begin{proof}
	Since the simple functions are dense in $L^p(\mu)$, it suffices to approximate each simple function $\chi_E$ in $L^p$-norm, where $E\subset X$ is a Borel set with $\mu(E)<\infty$. For any $\epsilon>0$, we pick an open set $U$ and a compact set $K$ such that $K\subset E\subset U$ and $\mu(U\backslash K)<\epsilon$. By Urysohn's lemma, there exists $f\in C_c(X)$ such that $\chi_K\leq f\leq\chi_U$. Then $\Vert\chi_E-f\Vert_p^p\leq\mu(U\backslash K)\leq\epsilon$, and we are done.
\end{proof}

\begin{theorem}[Lusin]\label{lusin}
	Let $\mu$ be a Radon measure on an LCH space $X$, and $f:X\to\bbC$ a measurable function that vanishes outside a $\mu$-finite set. Then for any $\epsilon>0$, there exists $\phi\in C_c(X)$ such that $\mu(\{\phi\neq f\})<\epsilon$. Moreover, if $f$ is bounded, we may take $\Vert\phi\Vert_\infty\leq\Vert f\Vert_\infty$.
\end{theorem}
\begin{proof}
	We assume first that $f$ is bounded, so $f\in L^1(\mu)$. Let $E=\{x\in X:f(x)\neq 0\}$. By Proposition \ref{ccdenseinlp}, there exists a sequence $(g_n)$ in $C_c$ that converges to $f$ in $L^1$. We take a subsequence that converges to $f$ a.e. and still denote it by $(g_n)$ for simplicity. By Egoroff's theorem, there exists $A\subset E$ with $\mu(E\backslash A)<\epsilon/3$ and $g_n\to f$ uniformly on $A$, and there exists a compact $B\subset A$ and an open $U\supset E$ such that $\mu(A\backslash B)<\epsilon/3$ and $\mu(U\backslash E)<\epsilon/3$. Since $g_n\to f$ uniformly on $B$, $f|_B$ is continuous, and by Tietze extension theorem, there exists $\phi\in C_c(U)$ such that $\phi=f$ on $B$. Since $\{\phi\neq f\}\subset U\backslash B$ and $\mu(U\backslash B)<\epsilon$, we have $\mu(\{\phi\neq f\})<\epsilon$. Furthermore, if $\vert\phi(x)\vert>\Vert f\Vert_\infty$, we may truncate $\phi(x)$ to $\Vert f\Vert_\infty\frac{\phi(x)}{\vert\phi(x)\vert}$, which does not change $\phi|_B$ and does not impact the continuity of $\phi$. Therefore we mat take $\Vert\phi\Vert_\infty<\Vert f\Vert_\infty$.
	
	On the other hand, if $f$ is unbounded, we make $A_n=\{0\leq\vert f\vert\leq n\}$, which increases to $E=\{f\neq 0\}$ as $n\to\infty$. Then there exists sufficient large $n$ such that $\mu(E\backslash A_n)<\epsilon/2$. By the preceding argument, there exists $\phi\in C_c(X)$ such that $\phi=f\chi_{A_n}$ except on a set of measure less than $\epsilon/2$. Hence $\mu(\{\phi\neq f\})<\epsilon$. 
\end{proof}

Finally we discuss how to construct a Radon measure from another one.

\begin{proposition}\label{l1induceradon}
	Let $\mu$ be a Radon measure on a topological space $X$. If $\phi\in L^1(\mu)$ and $\phi\geq 0$, we define
	\begin{align*}
		\nu(E)=\int_E\phi\,d\mu,\quad E\in\mathscr{B}.
	\end{align*}
	Then $\nu$ is also a Radon measure on $X$.
\end{proposition}
\begin{proof}
	One can easily verify that $\nu$ is a Borel measure on $X$, and $\nu\ll\mu$. Then for each $\epsilon>0$, there exists $\delta>0$ such that $\nu(E)<\epsilon$ for all $\mu(E)<\delta$. Now we verify that $\nu$ is a Radon measure on $X$.
	\begin{itemize}
		\item If $K\subset X$ is a compact set, $\nu(K)=\int_K\phi\,d\mu\leq\int_X\phi\,d\mu<\infty$. 
		\item For any Borel set $E\subset X$ and any $\epsilon>0$, there exists an open $U\supset E$ such that $\mu(U\backslash E)<\delta$, and $\nu(U\backslash E)<\epsilon$. Hence $\nu$ is outer regular on $E$.
		\item For any open set $U\subset X$ and any $\epsilon>0$, there exists a compact $K\subset U$ such that $\mu(U\backslash K)<\delta$, and $\nu(U\backslash K)<\epsilon$. Hence $\nu$ is inner regular on $U$.
	\end{itemize}
	To summarize, $\nu$ is a Radon measure on $X$.
\end{proof}

\newpage
\subsection{Riesz-Markov-Kakutani Representation of $C_0(X)^*$}
\paragraph{Positive bounded linear functionals on $C_0(X)$.} Let $X$ be an LCH space. Proposition \ref{c0closecc} states that $C_0(X)$ is the uniform closure of $C_c(X)$. If $\mu$ is a Radon measure on $X$, the functional $T_\mu f=\int_X f\,d\mu$ extends continuously to $C_0(X)$ if and only if it is bounded with respect to the uniform norm $\Vert\cdot\Vert_\infty$, i.e. there exists a constant $\gamma>0$ such that $\vert T_\mu f\vert\leq\gamma\Vert f\Vert_\infty$ for all $f\in C_c(X)$. In view of the equality
\begin{align*}
	\mu(X)=\sup\left\{\int_X f\,d\mu:f\in C_c(X),\ 0\leq f\leq 1\right\}=\sup\left\{T_\mu f:f\in C_c(X),\ 0\leq f\leq 1\right\},
\end{align*}
we know that $T_\mu:C_c(X)\to\bbC$ is bounded with respect to $\Vert\cdot\Vert_\infty$ if and only if $\mu(X)<\infty$, in which case $\mu(X)$ is the operator norm of $T_\mu$. Therefore, we have identified the positive bounded linear functionals on $C_0(X)$, which are given by integration against finite Radon measures.

In this section, we identify the dual space of $C_0(X)$, denoted by $C_0(X)^*$, which consists of all bounded linear functionals on $C_0(X)$. 

\begin{definition}[Signed Radon measures and complex Radon measures] Let $X$ be a topological space.
	\begin{itemize}
		\item[(i)] A \textit{signed Radon measure on $X$} is a signed Borel measure on $X$ whose positive and negative variations are Radon measures.
		\item[(ii)] A \textit{complex Radon measure on $X$} is a complex Borel measure on $X$ whose real and imaginary parts are signed Radon measures. We denote the space of complex Radon measures on $X$ by $M(X)$, and define $\Vert\mu\Vert=\vert\mu\vert(X)$, where $\vert\mu\vert$ is the total variation of $\mu$.
	\end{itemize}
\end{definition}
\begin{remark}
Since a complex measure is always finite, every complex Radon measure is regular. Furthermore, every complex Borel measure is Radon in an LCH space in which every open set is $\sigma$-compact (for example, a second countable LCH space).
\end{remark} 

\begin{theorem}
	If $\mu$ is a complex Borel measure on $X$, then $\mu$ is Radon if and only in $\vert\mu\vert$ is Radon. Furthermore, $M(X)$ is a vector space and $\mu\mapsto\Vert\mu\Vert$ is a norm on it.
\end{theorem}
\begin{proof}
	By Proposition \ref{radonreg}, we note that a finite positive Borel measure $\mu$ is Radon if and only if for every Borel set $E$ and every $\epsilon>0$, there exists compact $K\subset E$ and open $U\supset E$ such that $\mu(U\backslash K)<\epsilon$. 
	
	If $\mu=(\mu_1-\mu_2)+i(\mu_3-\mu_4)$ and $\vert\mu\vert(U\backslash K)<\epsilon$, we have $\mu_j(U\backslash K)<\epsilon$ for $j=1,2,3,4$. Conversely, if $\mu_j(U_j\backslash K_j)<\epsilon/4$ for all $j$, we have $\vert\mu\vert(U\backslash K)<\epsilon$ for $U=\bigcap_{j=1}^4 U_j$ and $K_j=\bigcup_{j=1}^4 K_j$. Hence $\mu$ is Radon if and only if its total variation $\vert\mu\vert$ is Radon. 
	
	For the second assertion, a similar argument shows that $M(X)$ is closed under addition and scalar multiplication. Finally, to show $\mu\to\Vert\mu\Vert$ is a norm on $X$, let $\mu_1,\mu_2\in M(X)$ and $\nu=\vert\mu_1+\mu_2\vert$, and take the Radon Nikodym derivative $f_1=d\mu_1/d\nu$ and $f_2=\mu_2/d\nu$. Then $$\vert\mu+\nu\vert(X)\leq \int_X\vert f_1+f_2\vert d\nu\leq \int_X\vert f_1\vert d\nu+ \int_X\vert f_2\vert d\nu\leq\vert\mu_1\vert(X)+\vert\mu_2\vert(X).$$
	Hence the triangle inequality holds, and $\Vert\mu\Vert=\vert\mu\vert(X)$ is a norm.
\end{proof}

We now discuss how to identify each $T\in C(X)^*$ with a complex Radon measure on $X$.

\begin{theorem}[Riesz-Markov-Kakutani]\label{rmk}
	Let $X$ be an LCH space. For each $\mu\in M(X)$, define
	\begin{align*}
		T_\mu f=\int_X f\,d\mu,\quad f\in C_0(X).
	\end{align*}
	Then the map $\mu\mapsto T_\mu$ defines an isometric isomorphism of $M(X)$ onto the dual space $C_0(X)^*$.
\end{theorem}

We begin from the real case. While studying a possibly non-positive linear functional on $C_0(X,\bbR)$, the following decomposition is extremely useful.

\begin{theorem}[Jordan decomposition]
	If $\,T\in C_0(X,\bbR)^*$, there exists positive bounded linear functionals $T^\pm\in C_0(X,\bbR)^*$ such that $T=T^+-T^-$.
\end{theorem}

\begin{proof}
	For $f\in C_0(X,\bbR)$ with $f\geq 0$, we define
	\begin{align*}
		T^+f=\sup\left\{Tg:g\in C_0(X,\bbR),\ 0\leq g\leq f\right\}
	\end{align*}
	We claim that $T^+$ is the restriction to $C_0(X,[0,\infty))$ of a positive bounded linear functional on $C_0(X,\bbR)$. 
	\begin{itemize}
		\item For $\lambda\geq 0$, we have $$T(\lambda f)=\sup\left\{Th:h\in C_0(X,\bbR),\ 0\leq h\leq\lambda f\right\}=\sup\left\{\lambda Tg:g\in C_0(X,\bbR),\ 0\leq g\leq f\right\}=\lambda Tf.$$
		\item If $0\leq g_1\leq f_1$ and $0\leq g_2\leq f_2$, we have $0\leq g_1+g_2\leq f_1+f_2$, so that $T^+(f_1+f_2)\geq Tg_1+Tg_2$, and hence $T^+(f_1+f_2)\geq T^+f_1+T^+f_2$. On the other hand, if $0\leq g\leq f_1+f_2$, let $g_1=\min\{f_1,g\}$ and $g_2=g-g_1=\max\{0,g-f_1\}$, so that $0\leq g_1\leq f_1$ and $0\leq g_2\leq f_2$. Then
		\begin{align*}
			Tg=Tg_1+Tg_2\leq T^+f_1+T^+f_2,
		\end{align*}
		and $T^+(f_1+f_2)\leq T^+f_1+T^+f_2$. Therefore $T^+(f_1+f_2)=T^+f_1+T^+f_2$.
		\item Since $\vert Tg\vert\leq\Vert T\Vert\left\Vert g\right\Vert_\infty\leq\Vert T\Vert\left\Vert f\right\Vert_\infty$ for $0\leq g\leq f$ and $T0=0$, we have $0\leq T^+f\leq\Vert T\Vert\left\Vert f\right\Vert_\infty$
	\end{itemize}
	
	Now for any $f\in C_0(X,\bbR)$, both its positive $f^+=\max\{f,0\}$ and negative parts $f^-=\max\{-f,0\}$ are in $C_0(X,[0,\infty))$, and we define
	$T^+f=T^+f^+-T^+f^-$. If $f=g-h$, where $g,h\geq 0$, we have $f^++h=g+f^-$, and $Tf=Tf^+-Tf^-=Tg-Th$. It follows easily that $T^+$ is a linear functional in $C_0(X,\bbR)$, and 
	\begin{align*}
		\vert T^+f\vert\leq\max\left\{T^+f^+,T^+f^-\right\}\leq\Vert T\Vert\max\left\{\Vert f^+\Vert_\infty,\Vert f^-\Vert_\infty\right\}=\Vert T\Vert\left\Vert f\right\Vert_\infty.
	\end{align*}
	Hence $T^+$ is bounded, and $\Vert T^+\Vert\leq\Vert T\Vert$.
	
	Finally, we define $T^-=T^+-T\in C_0(X,\bbR)^*$. By definition of $T^+$, we have $T^+f\geq Tf$ for $f\in C_0(X,\bbR)$ with $f\geq 0$, hence $T^-$ is a positive linear functional. Thus we concludes the proof.
\end{proof}
\paragraph{Remark.} For any $T\in C_0(X)^*$, consider its restriction $T_R=U+iV$ to $C_0(X,\bbR)$, where $U,V\in C_0(X,\bbR)^*$. If $f=u+iv\in C_0(X)$, where $u,v\in C_0(X,\bbR)$, by $\bbC$-linearity,
\begin{align*}
	Tf &= Tu+iTv= T_R u+iT_R v=(U+iV)u+i(U+iV)v=(Uu-Vv)+i(Uv+Vu).
\end{align*}
It is seen $T$ is uniquely determined by $T_R$. We then decompose $U=U^+-U^-$ and $V=V^+-V^-$, where $U^\pm,V^\pm\in C_0(X,\bbR)^*$ are positive. By Riesz representation theorem, we can find finite positive Radon measures $\mu_R^{\pm}$ and $\mu_I^{\pm}$ associated with $U^\pm$ and $V^\pm$, respectively. We define the complex Radon measure
\begin{align*}
	\mu=\left(\mu_R^+-\mu_R^-\right)+i\left(\mu_I^+-\mu_I^-\right).
\end{align*}
Then
\begin{align*}
	&\int_X f\,d\mu=\left(\int_X f\,d\mu_R^+-\int_X f\,d\mu_R^-\right)+i\left(\int_X f\,d\mu_I^+-\int_X f\,d\mu_I^-\right)\\
	&\quad=\left(\int_X u\,d\mu_R^+-\int_X u\,d\mu_R^--\int_X v\,d\mu_I^++\int_X v\,d\mu_I^-\right)+i\left(\int_X v\,d\mu_R^+-\int_X v\,d\mu_R^-+\int_X u\,d\mu_I^+-\int_X u\,d\mu_I^-\right)\\
	&\quad=\left(U^+ u-U^-u-V^+v+V^-v\right)+i\left(U^+v-U^-v+V^+u-V^-u\right)\\
	&\quad =(Uu-Vv)+i(Uv+Vu)=Tf.
\end{align*}
Therefore, every $T\in C_0(X)^*$ is associated with a complex Radon measure $\mu\in M(X)$ such that $Tf=\int_X f\,d\mu$. Furthermore, since $\mu_R^+,\, \mu_R^-,\,\mu_I^+,\,\mu_I^-$ are unique determined by $T$, the complex Radon measure $\mu$ is unique.
\begin{proof}[Proof of Theorem \ref{rmk}]
	We have already shown that every $T\in C_0(X)^*$ is of the form $T_\mu$. On the other hand, if $\mu\in M(X)$, we have
	\begin{align*}
		\left\vert\int_X f\,d\mu\right\vert\leq\int_X\vert f\vert\,d\vert\mu\vert\leq\Vert f\Vert_\infty\Vert\mu\Vert,\quad f\in C_0(X).
	\end{align*}
	Hence $T_\mu\in C_0(X)^*$, and $\Vert T_\mu\Vert\leq\Vert \mu\Vert$. Furthermore, we take $h=d\mu/d\vert\mu\vert$, so that $\vert h\vert=1$ $\vert\mu\vert$-a.e.. By Lusin's theorem [Theorem \ref{lusin}], for each $\epsilon>0$, there exists $\phi\in C_c(X)$ such that $\Vert\phi\Vert_\infty=1$ and $\phi=\ol{h}$ except on a set $E$ with $\vert\mu\vert(E)<\epsilon/2$. Then
	\begin{align*}
		\Vert\mu\Vert&=\int_X\vert h\vert^2\,d\vert\mu\vert=\int_X\ol{h}\,d\mu\leq\left\vert\int_X \phi\,d\mu\right\vert+\left\vert\int_X(\phi-\ol{h})\,d\mu\right\vert\\
		&=\left\vert\int_X \phi\,d\mu\right\vert+\left\vert\int_E(\phi-\ol{h})\,d\mu\right\vert\leq\Vert T_\mu\Vert\left\Vert\phi\right\Vert_\infty+\Vert\phi-\ol{h}\Vert_\infty\vert\mu\vert(E)\leq\Vert T_\mu\Vert+\epsilon.
	\end{align*}
	Letting $\epsilon\to 0$, we have $\Vert\mu\Vert\leq\Vert T_\mu\Vert$. Hence $\Vert\mu\Vert=\Vert T_\mu\Vert$, and the proof is complete.
\end{proof}
\begin{remark}
If we consider the real case, the mapping $\mu\mapsto T_\mu$ is an isometric isomorphism from the space of finite signed Radon measures to $C_0(X,\bbR)^*$.
\end{remark} 

\begin{corollary}\label{chsradon}
	Let $X$ be a compact Hausdorff space, $C(X)^*$ is isometrically isomorphic to the space $M(X)$ of complex Radon measures on $X$.
\end{corollary}
\begin{remark}
	If in addition, $X$ is metrizable, then $X$ is second countable, and we know that every finite Borel measure on $X$ is Radon by Theorem \ref{borelisradon}. Since complex measures are always finite, $M(X)$ is indeed the space of complex Borel measures on $X$, and $C(X)^*\simeq M(X)$.
\end{remark}
\begin{corollary}
	Let $\mu$ be a Radon measure on an LCH space $X$. For each $f\in L^1(\mu)$, define
	\begin{align*}
		\nu_f(E)=\int_E f\,d\mu,\quad E\in\mathscr{B}.
	\end{align*} 
	The mapping $f\mapsto\nu_f$ is an isometric embedding of $L^1(\mu)$ into $M(X)$ whose range consists precisely of those $\nu\in M(X)$ such that $\nu\ll\mu$.
\end{corollary}
\begin{proof}
	By Proposition \ref{l1induceradon}, the complex measure $\nu_f$ on $X$ is Radon and satisfies $\nu_f\ll\mu$. Moreover, $$\Vert\nu_f\Vert=\vert\nu\vert(X)=\int_X\vert f\vert\,d\mu=\Vert f\Vert_{L^1}.$$
	Finally, if $\nu\in M(X)$ and $\nu\ll\mu$, taking $f$ to be the Radon-Nikodym derivative $d\nu/d\mu$ yields $\nu_f=\nu$.
\end{proof}

\newpage
\subsection{Lebesgue Decomposition for Radon Measures on $\bbR^n$}
In this section, we work in the Euclidean space $(\bbR^n,m)$, which is a locally compact, Hausdorff and second countable space. According to Theorem \ref{borelisradon}, if a Borel measure $\mu$ on $\bbR^n$ is finite on compact sets, it is a Radon measure. By Lebesgue decomposition theorem, $\mu$ has a unique decomposition
\begin{align*}
	\mu=\rho+\nu,
\end{align*}
where
\begin{itemize}
	\item $\rho$ is \textit{absolutely continuous} with respect to the Lebesgue measure $m$, written $\rho\ll m$, i.e. $\rho(E)=0$ for all Borel sets $E$ with $m(E)=0$.
	\item $\nu$ and the Lebesgue measure $m$ are \textit{mutually singular}, written $\nu\perp m$, i.e. there is a Borel set $A$ such that $m(\bbR^n\backslash A)=\nu(A)=0$.
\end{itemize}
Clearly, both $\rho$ and $\nu$ are Radon measures on $\bbR^n$. The following theorem gives a further decomposition of $\mu$.

\begin{theorem}[Lebesgue decomposition for Radon measure on $\bbR^n$]\label{lebradondecomp}
	If $\mu$ is a Radon measure on $\bbR^n$, there exists a locally integrable function $f\in L_\loc^1(\bbR^n)$ and a Radon measure $\nu\perp m$ such that
	\begin{align}
		\mu(E)=\int_E f\,dm+\nu(E),\quad E\in\mathscr{B}(\bbR^n).\label{lebdecomp}
	\end{align}
	Furthermore, for almost every $x\in\bbR^n$,
	\begin{align}
		\lim_{r\to 0^+}\frac{\mu(B(x,r))}{m(B(x,r))}=f(x).\label{lebradondiff}
	\end{align}
\end{theorem}

The proof of this theorem requires a finite version of Vitali covering theorem.

\begin{lemma}[Vitali covering lemma]\label{vitalicover}
	For any finite collection $\cal{F}$ of open balls $B_1,B_2,\cdots,B_N$ in an arbitrary metric space $X$, there exists a subcollection $\cal{G}\subset\cal{F}$ of disjoint balls such that
	\begin{align*}
		\bigcup_{j=1}^NB_j\subset \bigcup_{B\in\cal{G}}3B,
	\end{align*}
	where $3B$ denotes the ball with the same center as $B$ but with 3 times the radius.
\end{lemma}
\begin{proof}
	We choose balls in $\cal{G}$ by the greedy algorithm. First take $B_1^\prime$ to be the largest ball among $\cal{F}$. Having chosen $\{B_1^\prime,B_2^\prime,\cdots,B_k^\prime\}$, repeat the inductive step:
	\begin{itemize}
		\item if the remaining balls each have nonempty intersection with $\bigcup_{i=1}^k B_i^\prime$, stop;
		\item otherwise, take $B_{k+1}^\prime$ to be the largest among $\cal{F}\backslash\{B_1^\prime,B_2^\prime,\cdots,B_k^\prime\}$ that are disjoint from $\bigcup_{i=1}^k B_i^\prime$.
	\end{itemize}
	This algorithm must stop after less than $N$ rounds, with the chosen balls $B_1^\prime,B_2^\prime,\cdots,B_n^\prime$ disjoint. Then it remains to show that $B_i\subset E:=\bigcup_{j=1}^n 3B_j^\prime$ for every $i=1,\cdots,N$. We claim $B_i\cap E\neq\emptyset$, otherwise the algorithm would not have stopped at $B_1^\prime,B_2^\prime,\cdots,B_n^\prime$. We let $j_0$ be the minimal $j$ such that $B_j^\prime\cap B_i\neq\emptyset$. Then $B_i$ does not intersect $\bigcup_{i=1}^{j_0-1}B_j^\prime$, and the radius of $B_i$ is no greater than $B_{j_0}$, since $B_{j_0}$ is maximal at step $j_0$. Recalling that $B_{j_0}^\prime\cap B_i\neq\emptyset$, by triangle inequality, $3B_{j_0}^\prime \supset B_i$.
\end{proof}

\begin{lemma}\label{radonsingdifflemma}
If $\nu$ is a Radon measure on $\bbR^n$, and $\nu\perp m$, then for almost every $x\in\bbR^n$,
\begin{align}
	\lim_{r\to 0^+}\frac{\nu(B(x,r))}{m(B(x,r))}=0.\label{radonsingdiff}
\end{align}
\end{lemma}
\begin{proof}
We take the Borel set $A$ such that $m(\bbR^n\backslash A)=\nu(A)=0$, and define
\begin{align*}
E_k=\left\{x\in A:\limsup_{r\to 0^+}\frac{\nu(B(x,r))}{m(B(x,r))}>\frac{1}{k}\right\},\quad k=1,2,\cdots.
\end{align*}
By outer regularity of $\nu$, for any $\epsilon>0$, we can find an open set $U\supset A$ with $\nu(U)<\epsilon$. By definition of $E_k$, for each $x\in E_k$, we can take a ball $B(x,r_x)\subset$ such that
\begin{align}
	\frac{\nu(B(x,r_x))}{m(B(x,r_x))}>\frac{1}{k}.\label{ballvolratioest}
\end{align}
We take a compact subset $K\subset E_k$, then $K$ is covered by finitely many such balls. By Vitali covering lemma [Lemma \ref{vitalicover}], we can further take finitely many disjoint balls $B(x_1,r_1),\cdots,B(x_N,r_N)$ such that
\begin{align*}
	K\subset\bigcup_{j=1}^N B(x_j,3r_j).
\end{align*}
Applying the estimate (\ref{ballvolratioest}), we have
\begin{align*}
	m(K)\leq \sum_{j=1}^Nm(B(x_j,3r_j))=3^n\sum_{j=1}^Nm(B(x_j,r_j))< 3^nk\sum_{j=1}^N\nu(B(x_j,r_j))\leq 3^nk\nu(U)\leq 3^nk\epsilon.
\end{align*}
Since the compact $K\subset E_k$ is arbitrary, by inner regularity of the Lebesgue measure, $m(F_k)\leq 3^nk\epsilon$. Also, since $\epsilon>0$ is arbitrary, $m(F_k)=0$ for all $k\in\bbN$. Hence
\begin{align*}
\left\{x\in A:\limsup_{r\to 0^+}\frac{\nu(B(x,r))}{m(B(x,r))}>0\right\}=\bigcup_{k=1}^\infty E_k
\end{align*}
has Lebesgue measure zero. Since $m(\bbR^n\backslash A)=0$, the limit (\ref{radonsingdiff}) holds for $m$-a.e. $x\in\bbR^n$.
\end{proof}
\begin{proof}[Proof of Theorem \ref{lebradondecomp}]
We take the Lebesgue decomposition $\mu=\rho+\nu$, where $\rho\ll m$ and $\nu\perp m$. By Radon Nikodym theorem, there exists $f\in L_\loc^1(\bbR^n)$ such that
\begin{align*}
	\rho(E)=\int_E f\,dm,\quad\forall E\in\scr{B}(\bbR^n).
\end{align*}
Then $f$ satisfies the identity (\ref{lebdecomp}). The second result (\ref{lebradondiff}) is an immediate consequence of the Lebesgue differentiation theorem [Theorem \ref{lebdiffthm}] and Lemma \ref{radonsingdifflemma}.
\end{proof}
\begin{remark}
The locally integrable function $f$ satisfying (\ref{lebradondiff}) is also called the \text{derivative} or \textit{density} of the Radon measure $\mu$ with respect to the Lebesgue measure $m$.
\end{remark}

\newpage
\section{The Hardy-Littlewood Maximal Inequality and Differentiation}
\subsection{The Hardy-Littlewood Maximal Inequality}
In this section, we work in the Euclidean space $\bbR^n$ with the Lebesgue measure $m$. For a locally integrable function $f\in L^1_\loc(\bbR^n)$, we define the local average
\begin{align*}
	(A_rf)(x)=\frac{1}{m(B(x,r))}\int_{B(x,r)}f(y)\,dy,\quad x\in\bbR^n.
\end{align*}
To obtain a uniform estimate for $A_rf$, we define the \textit{Hardy-Littlewood maximal operator} by
\begin{align*}
	(Mf)(x)=\sup_{r>0}\,(A_r\vert f\vert)(x)=\sup_{r>0}\frac{1}{m(B(x,r))}\int_{B(x,r)}\vert f(y)\vert\,dy,\quad x\in\bbR^n.
\end{align*}
Clearly $M$ is sublinear. The function $Mf$ is also called the \textit{Hardy-Littlewood maximal function of $f$}.

\begin{theorem}[Hardy-Littlewood maximal inequality, weak type]\label{hdmiw}
The Hardy-Littlewood operator $M$ is of weak type $(1,1)$. In other words, there exists a constant $C_n>0$ such that for all $f\in L^1(\bbR^n)$ and all $\lambda>0$,
\begin{align}
	m\left(\{Mf\geq\lambda\}\right)\leq\frac{C_n}{\lambda}\Vert f\Vert_{L^1}.\label{hdmi1}
\end{align}
\end{theorem}
\begin{remark}
The inequality (\ref{hdmi1}) may look a bit stricter than the condition $[Mf]_1\leq C_n\Vert Mf\Vert_{L^1(\bbR^n)}$ of weak type $(1,1)$. But, as we will see, the two assertions are indeed equivalent.
\end{remark}


\textit{\hspace{-2em} Proof of Hardy-Littlewood maximal inequality [Theorem \ref{hdmiw}].}
We will show that for all $f\in L^1(\bbR^n)$, $$m(\{Mf>\lambda\})\leq\frac{3^n}{\lambda}\Vert f\Vert_{L^1},\quad\lambda>0.$$
Noticing that $m(\{Mf\geq\lambda\})\leq m(\{Mf>\lambda-\epsilon\})\leq 3^n(\lambda-\epsilon)^{-1}\Vert f\Vert_{L^1}$ for sufficiently small $\epsilon>0$, the desired inequality (\ref{hdmi1}) follows by perturbing $\epsilon\downarrow 0$. 

Using the inner regularity of the Lebesgue measure, it suffices to show that $m(K)\leq 3^n\lambda^{-1}\Vert f\Vert_{L^1}$ for each compact subset $K\subset\{Mf>\lambda\}$. For each $x\in K$, we take $r_x>0$ such that $$\frac{1}{m(B(x,r_x))}\int_{B(x,r_x)}\vert f\vert\,dm>\lambda.$$
The collection of balls $B(x,r_x)$ forms an open cover of $K$, and we may take by compactness of $K$ a finite subcollection that covers $K$. By Vitali covering lemma [Lemma \ref{vitalicover}], we take a further collection of disjoint balls $B_1,B_2,\cdots,B_k$ such that $K\subset\bigcup_{j=1}^k B_j$. Consequently,
\begin{align*}
	m(K)\leq 3^n\sum_{j=1}^k m(B_j)\leq\frac{3^n}{\lambda}\sum_{j=1}^k\int_{B_j}\vert f\vert\,dm=\frac{3^n}{\lambda}\int_{\bigcup_{j=1}^k B_j}\vert f\vert\,dm\leq\frac{3^n}{\lambda}\Vert f\Vert_{L^1}.\tag*{\qed}
\end{align*}

Using the Marcinkiewicz interpolation theorem, we immediately obtain the following result.
\begin{theorem}[Hardy-Littlewood maximal inequality, strong type]
Let $1<p\leq\infty$. The Hardy-Littlewood operator $M$ is of strong type $p$. That is, there exists a constant $C_{n,n}>0$ such that for all $f\in L^p(\bbR^n)$,
\begin{align*}
	\Vert Mf\Vert_{L^p}\leq C_{p,n}\Vert f\Vert_{L^p}.
\end{align*}
\end{theorem}
\begin{proof}
The Hardy-Littlewood operator $M$ is sublinear and of weak type $1$. By definition of $Mf$, we also have $\Vert Mf\Vert_{L^\infty}\leq\Vert f\Vert_{L^\infty}$ when $f$ is a.e. bounded. Hence $M$ is of strong type $\infty$, and is of strong type $(p,p)$ for each $1<p\leq\infty$ by Marcinkiewicz interpolation theorem [Corollary \ref{mckwz2}] 
\end{proof}

\newpage
\subsection{The Lebesgue Differentiation Theorem and a.e. Differentiability}
In this section we apply the Hardy-Littlewood maximal inequality to prove some differentiation theorems. 
\begin{theorem}[Lebesgue differentiation theorem]\label{lebdiffthm}
Let $f\in L^1_\loc(\bbR^n)$. For almost every $x\in\bbR^n$,
\begin{align}
	\lim_{r\to 0^+}\frac{1}{m(B(x,r))}\int_{B(x,r)}\vert f(y)-f(x)\vert\,dy=0.\label{lebdiff}
\end{align}
Consequently, the local average function $A_rf$ converges almost everywhere to $f$, i.e.
\begin{align}
	\lim_{r\to 0^+}(A_rf)(x)=\lim_{r\to 0^+}\frac{1}{m(B(x,r))}\int_{B(x,r)}f(y)\,dy=f(x)\label{lebdiffavg}
\end{align}
for almost every $x\in\bbR^n$.
\end{theorem}
\begin{remark}
Let $f$ be a measurable function on $\bbR^n$. A point $x\in\bbR^n$ is said to be a \textit{Lebesgue point of $f$} if the identity (\ref{lebdiff}) holds. The Lebesgue differentiation theorem implies that if $f\in L^1_\loc(\bbR^n)$, then almost every point in $\bbR^n$ is a Lebesgue point of $f$.
\end{remark}
\begin{proof}
We first prove the result for $g\in C_c(\bbR^n)$. If $x\in\bbR^n$ and $\epsilon>0$, by uniform continuity of $g$, there exists $\delta>0$ such that $\vert g(y)-g(x)\vert<\epsilon$ for all $y\in B(x,\delta)$. Then for all $r<\delta$,
\begin{align*}
	\frac{1}{m(B(x,r))}\int_{B(x,r)}\vert g(y)-g(x)\vert\,dy<\epsilon.
\end{align*}
Hence (\ref{lebdiff}) holds for all continuous functions with compact support.

Now we prove the general case. Since differentiation is a local property, we may assume that $f\in L^1(\bbR^n)$. For $\epsilon>0$, choose $g\in C_c(\bbR^n)$ such that $\Vert f-g\Vert_{L^1}\leq\epsilon$. We put $h=f-g$. By the triangle inequality,
\begin{align*}
	\vert(A_rf)(x)-f(x)\vert\leq \vert (A_rg)(x)-g(x)\vert+\vert (A_rh)(x)-h(x)\vert\leq \vert (A_rg)(x)-g(x)\vert+(A_r\vert h\vert)(x)+\vert h(x)\vert.
\end{align*}
Let $\lambda>0$. Then
\begin{align*}
&m\left(\left\{x\in\bbR^n:\limsup_{r\to 0^+}\vert A_rf-f\vert(x)\geq\lambda\right\}\right)\\
&\quad\leq m\left(\left\{x\in\bbR^n:\limsup_{r\to 0^+}\vert A_rg-g\vert(x)\geq\frac{\lambda}{3}\right\}\right)+ m\left(\left\{x\in\bbR^n:\limsup_{r\to 0^+}\,(A_r\vert h\vert)(x)\geq\frac{\lambda}{3}\right\}\right)+m\left(\vert h\vert\geq\frac{\lambda}{3}\right)\\
&\quad\leq m\left(\left\{x\in\bbR^n:\sup_{r>0}\,(A_r\vert h\vert)(x)\geq\frac{\lambda}{3}\right\}\right)+m\left(\vert h\vert\geq\frac{\lambda}{3}\right).
\end{align*}
By weak $L^1$ Hardy-Littlewood maximal inequality [Theorem \ref{hdmi1}] and Markov inequality,
\begin{align*}
	m\left(\left\{x\in\bbR^n:\limsup_{r\to 0^+}\vert A_rf-f\vert(x)\geq\lambda\right\}\right)\leq\frac{3C_n}{\lambda}\Vert h\Vert_{L^1}+\frac{3}{\lambda}\Vert h\Vert_{L^1}\leq\frac{3(C_n+1)\epsilon}{\lambda}.
\end{align*}
Since $\epsilon>0$ is arbitrary, the left-hand side of the last display is zero. The result then follows by taking the union on the sequence $\lambda_n=\frac{1}{n}\downarrow 0$.
\end{proof}

Following is a particular case of Lebesgue differentiation theorem.
\begin{theorem}[Lebesgue density theorem]
Let $E\subset\bbR^n$ be a Lebesgue measurable set. For almost every point $x\in \bbR^n$, the density
\begin{align}
	\lim_{r\to 0^+}\frac{m(E\cap B(x,r))}{m(B(x,r))}=\begin{cases}
		1, &\text{if $x\in E$},\\
		0, &\text{if $x\notin E$}.
	\end{cases}.\label{densitypoint}
\end{align}
\end{theorem}
\begin{remark}
Let $E\subset\bbR^n$. A point $x\in\bbR^n$ is said to be a \textit{density point of $E$} if the $$\lim_{r\to 0^+}\frac{m(E\cap B(x,r))}{m(B(x,r))}=1.$$ The Lebesgue density theorem implies that almost every point of a measurable set is a density point, and almost every point outside the measurable set is not a density point.
\end{remark}
\begin{proof}
The identity (\ref{densitypoint}) is a special case of (\ref{lebdiffavg}) when $f=\chi_E$.
\end{proof}

We can employ the Lebesgue differentiation theorem to prove the Fundamental theorem of calculus.

\begin{theorem}[Fundamental Theorem of Calculus]
Let $F:\bbR\to\bbC$ be an absolutely continuous function. Then $F$ is almost everywhere differentiable, and the derivative $f=F^\prime$ satisfies $f\in L^1_\loc(\bbR)$, and
\begin{align*}
	F(x)=F(a)+\int_a^x f(t)\,dt,\quad -\infty<a<x<\infty.
\end{align*}
\end{theorem}
\begin{proof}
Since the differentiability is a local property, it suffices to deal with the restriction of $F$ on a compact interval $[a,b]$. Let $\mu_F$ be the Lebesgue-Stieltjes measure generated by $f$ on $[a,b]$.

\item\textbf{Step I.} We claim that $\mu_F$ is absolutely continuous with respect to the Lebesgue measure $m$.

We fix $\epsilon>0$, and choose $\delta>0$ such that $\sum_{j=1}^N\vert F(b_j)-F(a_j)\vert<\epsilon$ for all disjoint intervals $\{(a_j,b_j)\}_{j=1}^N$ with total length less than $\delta$. If $E$ is a Borel set with $m(E)=0$, by outer regularity of $m$, we take an open $U\supset E$ with $m(U)<\delta$. Then $U$ is a disjoint union of at most countably many intervals $\{(a_j,b_j)\}_{j=1}^\infty$, and
\begin{align*}
	\sum_{j=1}^N\mu_F((a_j,b_j))\leq\sum_{j=1}^N\left(F(b_j)-F(a_j)\right)\leq\epsilon.
\end{align*}
Letting $N\to\infty$, we have $\mu_F(U)<\epsilon$, and $\mu_F(E)<\epsilon$. Since $\epsilon>0$ is arbitrary, $\mu_F(E)=0$.

\item\textbf{Step II.} By Radon-Nikodym theorem, we take $f\in L^1([a,b])$ such that $\mu_F(E)=\int_E f\,dm$. We may further globalize this result and assert that there exists a locally integrable function $f\in L^1_\loc(\bbR)$ such that
\begin{align*}
	\mu_F((x,y])=F(y)-F(x)=\int_x^y f(t)\,dt\quad\text{for all}\ -\infty<x<y<\infty.
\end{align*}

\item\textbf{Step III.} If $x\in\bbR$ is a Lebesgue point of $f$, by Lebesgue differentiation theorem,
\begin{align*}
	\lim_{r\to 0^+}\frac{1}{2r}\int_{x-r}^{x+r}\vert f(y)-f(x)\vert\,dy=0.
\end{align*}
We split the integral to $[x-r,x]$ and $[x,x+r]$ to get
\begin{align*}
	\lim_{r\to 0^+}\frac{1}{r}\int_{x}^{x+r}\vert f(y)-f(x)\vert\,dy=\lim_{r\to 0^+}\frac{1}{r}\int_{x-r}^{x}\vert f(y)-f(x)\vert\,dy=0.
\end{align*}
Hence the right derivative of $F$ at $x$ is
\begin{align*}
	\lim_{r\to 0^+}\frac{F(x+r)-F(x)}{r}=\lim_{r\to 0^+}\frac{1}{r}\int_{x}^{x+r}\vert f(y)-f(x)\vert\,dy=f(x),
\end{align*}
and the same for the left derivative. Therefore $F$ is differentiable almost everywhere, and $F^\prime=f$.
\end{proof}
\begin{remark}
A special case of this theorem is the one-dimensional Rademacher's theorem. If we further assume that $F:\bbR\to\bbC$ is Lipschitz continuous, then $F$ is almost everywhere differentiable and $F^\prime\in L^\infty(\bbR)$. Indeed, the essential supremum of $F^\prime$ is bounded by the Lipschitz constant. 
\end{remark}
\begin{theorem}[Rademacher's Theorem]\label{rademacher}
If $f:\bbR^n\to\bbC$ is a locally Lipschitz continuous function, then $f$ is almost everywhere differentiable.
\end{theorem}
\begin{proof}
Since the differentiability is a local property, we may assume that $f$ is Lipschitz continuous on $\bbR^n$. From the one-dimensional case, we know that for each unit vector $\vert v\vert=1$, the directional derivative $f_v=\frac{\partial f}{\partial v}$ exists almost everywhere. In particular, the partial derivatives $(\frac{\partial f}{\partial x_j})_{j=1}^n$ exist almost everywhere. We write $G=\bigl(\frac{\partial f}{\partial x_1},\frac{\partial f}{\partial x_2},\cdots,\frac{\partial f}{\partial x_n}\bigr)$. We show that for almost every $x\in\bbR^n$,
\begin{align*}
	\lim_{\vert h\vert\to 0}\frac{f(x+h)-f(x)-G\cdot h}{\vert h\vert}=0.
\end{align*}
This implies that $f$ is almost everywhere differentiable, and the gradient $\nabla f=G$.

\item\textbf{Step I.} For each $\vert v\vert=1$, we claim that $f_v=G\cdot v$ almost everywhere. We take a test function $\phi\in C_c^\infty(\bbR^n)$. Since $f$ is Lipschitz, by Lebesgue dominated convergence theorem,
\begin{align*}
	\lim_{t\to 0}\int_{\bbR^n}\frac{f(x+tv)-f(x)}{t}\phi(x)\,dx=\int_{\bbR^n}f_v(x)\phi(x)\,dx.
\end{align*}
Since $\phi$ is smooth, $\phi_v=\nabla\phi\cdot v$. Applying integration by parts, we have
\begin{align*}
	\int_{\bbR^n}f_v(x)\phi(x)\,dx&=\lim_{t\to 0}\int_{\bbR^n}\frac{f(x+tv)-f(x)}{t}\phi(x)\,dx=\lim_{t\to 0}\int_{\bbR^n}f(x)\frac{\phi(x-tv)-\phi(x)}{t}\,dx\\
	&=-\int_{\bbR^n}f(x)\phi_v(x)\,dx=-\int_{\bbR^n}\sum_{j=1}^nf(x)v_j\frac{\partial\phi}{\partial x_j}(x)\,dx=\int_{\bbR^n}\sum_{j=1}^nv_j\frac{\partial f}{\partial x_j}(x)\phi(x)\,dx.
\end{align*}
Therefore $\int_{\bbR^n}(f_v-G\cdot v)\phi\,dm=0$ for all $\phi\in C_c^\infty(\bbR^n)$, and $f_v=G\cdot v$ a.e.. For each $\vert v\vert=1$, we write
$$A_v=\left\{x\in\bbR^n:G(x)\ \text{and}\ f_v(x)\ \text{exists, and}\ f_v(x)=G(x)\cdot v\right\}.$$

\item\textbf{Step II.} We take a countable dense subset $(v_k)$ of the unit sphere $\{\vert v\vert=1\}$. By Step I, $\mu(A_{v_k}^c)=0$ for all $k\in\bbN$, and $\mu(A^c)=0$, where we take $A=\bigcap_{k=1}^\infty A_{v_k}$. We claim that $f$ is differentiable at all $x\in A$. Since $f$ is Lipschitz continuous, there exists a constant $K>0$ such that $\vert f(x)-f(y)\vert\leq K\vert x-y\vert$, and all partial derivatives are bounded by $K$. We take $h\neq 0$ in $\bbR^n$. Then for all $k\in\bbN$,
\begin{align*}
	&\frac{\vert f(x+h)-f(x)-G\cdot h\vert}{\vert h\vert}\\
	&\qquad=\frac{\vert f(x+\vert h\vert v)-f(x+\vert h\vert v_k)\vert}{\vert h\vert}+\frac{\vert f(x+\vert h\vert v_k)-f(x)-G\cdot \vert h\vert v_k\vert}{\vert h\vert}+\frac{\vert G\cdot\vert h\vert(v_k-v)\vert}{\vert h\vert}\\
	&\qquad=\frac{\vert f(x+\vert h\vert v)-f(x+\vert h\vert v_k)\vert}{\vert h\vert}+\left\vert\frac{f(x+\vert h\vert v_k)-f(x)}{\vert h\vert}-G\cdot v_k\right\vert+\left\vert G\cdot (v_k-v)\right\vert\\
	&\qquad\leq K\vert v-v_k\vert+\left\vert\frac{f(x+\vert h\vert v_k)-f(x)}{\vert h\vert}-G\cdot v_k\right\vert+ K\sqrt{n}\cdot\vert v_k-v\vert
\end{align*}
where $v=h/\vert h\vert$ is a unit vector. By density of $(v_k)$ in the unit sphere, for each $\epsilon>0$, we take $v_k$ such that $\vert v_k-v\vert<\epsilon$. Then for all $x\in A$, 
\begin{align*}
	\lim_{\vert h\vert\to 0}\frac{\vert f(x+h)-f(x)-G\cdot h\vert}{\vert h\vert}&\leq K\vert v-v_k\vert+\left\vert f_v-G\cdot v_k\right\vert+K\sqrt{n}\cdot\vert v_k-v\vert\leq K(1+\sqrt{n})\epsilon,
\end{align*}
where the last inequality follows from $x\in A_{v_k}$. Since $\epsilon>0$ is arbitrary, the above limit is zero. Therefore $f$ is differentiable at $x$, and the gradient $\nabla f=G$. 
\end{proof}
\begin{remark}
Since each component of $\nabla f$ is the limit of bounded measurable functions, $\vert\nabla f\vert\in L^1_\loc(\bbR^n)$.
\end{remark}
\newpage
\subsection{Second Differentiability of Convex Functions}
In this section, we discuss the differentiation of convex functions on Euclidean spaces. Recall that a function $f:\bbR^n\to\bbR$ is called \textit{convex} if for all $x,y\in\bbR^n$ and $0\leq\lambda\leq 1$,
\begin{align*}
	f(\lambda x+(1-\lambda)y)\leq \lambda f(x)+(1-\lambda)f(y).
\end{align*}
\begin{proposition}\label{convexloclip}
Every convex function on $\bbR^n$ is locally Lipschitz continuous.
\end{proposition}
\begin{proof}
Let $f:\bbR^n\to\bbR$ be a convex function.
\item\textbf{Step I.} We first prove that $f$ is locally bounded. We consider the compact hypercube $Q=[-N,N]^n$, with vertices $(x_k)_{k=1}^{2^n}$. Then every $x\in Q$ is a convex combination $x=\sum_{k=1}^{2^n}\lambda_kx_k$ of the vertices, and
\begin{align*}
	f(x)\leq\sum_{k=1}^{2^n}\lambda_kf(x_k)\leq M:=\max_{1\leq k\leq 2^n}f(x_k)<\infty.
\end{align*}
Then $\sup_{x\in Q}f(x)\leq M$. To derive a lower bound, note that for every $x\in Q$,
\begin{align*}
	f(0)\leq\frac{1}{2}f(x)+\frac{1}{2}f(-x)\leq\frac{1}{2}f(x)+\frac{1}{2}M.
\end{align*}
Hence $\inf_{x\in Q}f(x)\geq 2f(0)-M$.

\item\textbf{Step II.} Now we prove the local Lipschitz continuity of $f$. Fix $x,y\in\ol{B(0,N)}$ with $x\neq y$, where $N>0$ and $\ol{B(0,N)}$ is the closed ball of radius $N$ centered at $0$. We choose $\mu>0$ such that $z=x+\mu(y-x)$ satisfies $\vert z\vert=3N$. Then $\mu=\frac{\vert z-x\vert}{\vert y-x\vert}\geq 1$, and
\begin{align*}
	f(y)&=f\left(\frac{1}{\mu}z+\left(1-\frac{1}{\mu}\right)x\right)\leq f(x)+\frac{f(z)-f(x)}{\mu}\leq f(x)+\frac{2}{\mu}\sup_{\xi\in B(0,3N)}\vert f(\xi)\vert.
\end{align*}
Since $\vert z-x\vert\geq 2N$, we obtain
\begin{align*}
	f(y)-f(x)\leq\frac{2}{\mu}\sup_{\xi\in B(0,3N)}\vert f(\xi)\vert=\frac{\vert y-x\vert}{N}\sup_{\xi\in B(0,3N)}\vert f(\xi)\vert.
\end{align*}
Interchanging $x$ and $y$, the same estimate holds for $f(x)-f(y)$. Hence $f$ is locally Lipschitz continuous.
\end{proof}

\begin{remark}
According to Proposition \ref{convexloclip} and Rademacher's theorem [Theorem \ref{rademacher}], every convex function is almost everywhere differentiable. In this section, we step further and deal with the second differentiability. We begin by discussing some properties of derivatives of convex functions.
\end{remark}

\begin{lemma}
Let $f:\bbR^n\to\bbR$ be a convex function.
\begin{itemize}
	\item[(i)] If $f$ is differentiable at $x$, then
	\begin{align}
		f(y)\geq f(x)+\nabla f(x)\cdot (y-x).\label{convex1est}
	\end{align}
	\item[(ii)] If, in addition, $f\in C^2(\bbR^n)$, then $\nabla^2 f\succeq 0$ on $\bbR^n$.
\end{itemize}
\end{lemma}
\begin{proof}
(i) For each $y\in\bbR^n$ and $0<\lambda<1$, by convexity of $f$, we have
\begin{align*}
	\frac{f(x+\lambda(y-x))-f(x)}{\lambda}\leq f(y)-f(x).
\end{align*}
Letting $\lambda\to 0$, we have
\begin{align*}
	f(y)\geq f(x)+\nabla f(x)\cdot (y-x).
\end{align*}
(ii) By Taylor's theorem,
\begin{align*}
	f(y)=f(x)+\nabla f(x)\cdot (y-x)+(y-x)^\top\int_0^1(1-t)\nabla^2 f(x+t(y-x))\,dt\cdot (y-x)
\end{align*}
Then the estimate \ref{convex1est} implies
\begin{align*}
(y-x)^\top\int_0^1(1-t)\nabla^2 f(x+t(y-x))\,dt\cdot (y-x)\geq 0.
\end{align*}
Hence given any $\xi\in\bbR^n$, we set $y=x+s\xi$ with $s>0$. Then the above inequality becomes
\begin{align*}
	\xi^\top\int_0^1(1-t)\nabla^2 f(x+st\xi)\,dt\cdot\xi\geq 0.
\end{align*}
Letting $s\to 0$, we have
\begin{align*}
	\xi^\top\nabla^2 f(x)\cdot\xi\geq 0.
\end{align*}
This proves assertion (ii). 
\end{proof}

Indeed, for any convex function, we can find its second derivatives in the distributional sense.
\begin{theorem}[Second derivatives of convex functions as measures]
	Let $f:\bbR^n\to\bbR$ to a convex function. Then there exist signed Radon measures $\mu^{ij}=\mu^{ji}$ such that for all functions $\phi\in C_c^2(\bbR^n)$,
	\begin{align*}
		\int_{\bbR^n}f\phi_{x_ix_j}\,dx=\int_{\bbR^n}\phi\,d\mu^{ij},\quad i,j=1,2,\cdots,n.
	\end{align*}
\end{theorem}
\begin{proof}
	We define $f^\epsilon=\eta_\epsilon\cdot f\in C^\infty(\bbR^n)$, where $\eta_\epsilon$ is the standard mollifier. For any $v\in\bbR^n$ with $\vert v\vert=1$,
	\begin{align*}
		\sum_{i,j=1}^n\int_{\bbR^n}f^\epsilon\phi_{x_ix_j}v_iv_j\,dx=\int_{\bbR^n}\phi\sum_{i,j=1}^nf^\epsilon_{x_ix_j} v_iv_j\,dx\geq 0,\quad \phi\in C^2_c(\bbR^n)\ \text{and}\ \phi\geq 0.
	\end{align*}
	Letting $\epsilon\downarrow 0$, we have \vspace{-0.1cm}
	\begin{align*}
		T_v\phi:=\sum_{i,j=1}^n\int_{\bbR^n}f\phi_{x_ix_j}v_iv_j\,dx.
	\end{align*}
	Thus we define a positive linear functional $T_v$ on $C^2_c(\bbR^n)$. According to the Remark under the Theorem \ref{rieszcc}, there exists a Radon measure $\mu^v$ on $\bbR^n$ such that
	\begin{align*}
		T_v\phi=\int_{\bbR^n}\phi\,d\mu^v,\quad\text{for all}\ \phi\in C^2(\bbR^n).
	\end{align*}
	For each $i=1,2,\cdots,n$, we define $\mu^{ii}=\mu^{e_i}$. If $i\neq j$, we set $v=\frac{e_i+e_j}{\sqrt{2}}$. Then
	\begin{align*}
		\int_{\bbR^n}f\phi_{x_ix_j}\,dx&=\int_{\bbR^n}f\left(\frac{\phi_{ii}+\phi_{jj}}{2}+\phi_{ij}\right)dx-\frac{1}{2}\int_{\bbR^n}f\phi_{x_ix_i}\,dx-\frac{1}{2}\int_{\bbR^n}f\phi_{x_jx_j}\,dx\\
		&=\sum_{k,l=1}^n\int_{\bbR^n}f\phi_{kl}v_kv_l\,dx-\frac{1}{2}\int_{\bbR^n}f\phi_{x_ix_i}\,dx-\frac{1}{2}\int_{\bbR^n}f\phi_{x_jx_j}\,dx\\
		&=\int_{\bbR^n}\phi\,d\mu^v-\frac{1}{2}\int_{\bbR^n}\phi\,d\mu^{ii}-\frac{1}{2}\int_{\bbR^n}\phi\,d\mu^{jj}=\int_{\bbR^n}\phi\,d\mu^{ij},
	\end{align*}
	where we set $\mu^{ij}=\mu^v-\frac{1}{2}\mu^{ii}-\frac{1}{2}\mu^{ij}$. Then we complete the proof.
\end{proof}
\begin{remark} 
	According to the Lebesgue decomposition theorem, every signed Radon measure $\mu^{ij}$ has a unique decomposition $\mu^{ij}=\mu^{ij}_\mathrm{ac}+\mu^{ij}_\mathrm{s}$, where $\mu_\mathrm{ac}^{ij}\ll m$ and $\mu_{\rm s}^{ij}\perp m$. We write
	\begin{align*}
		M=(\mu^{ij})_{i,j=1}^n,\quad M_\mathrm{ac}=(\mu^{ij}_\mathrm{ac})_{i,j=1}^n,\quad M_{\rm s}=(\mu^{ij}_{\rm s})_{i,j=1}^n.
	\end{align*}
	By Radon-Nikodym theorem, we define $f^{ij}=d\mu_{\rm ac}^{ij}/dm$ to be the density of the absolute continuous part of $\mu^{ij}$ with respect to the Lebesgue measure $m$. Then
	\begin{align*}
		D^2f=\begin{pmatrix}
			f^{11} & f^{12} & \cdots & f^{1n}\\
			f^{21} & f^{22} & \cdots & f^{2n}\\
			\vdots & \vdots & \ddots & \vdots\\
			f^{n1} & f^{n2} & \cdots & f^{nn}
		\end{pmatrix}
	\end{align*}
	is a matrix valued function, and every element $f^{ij}$ is locally integrable. According to the Theorem \ref{lebradondecomp}, we have the decomposition
	\begin{align*}
		M(E)=\int_E D^2f\,dm+M_\mathrm{s}(E).
	\end{align*}
In fact, a convex function has not only distributional second derivatives as Radon measures, but also classical derivatives almost everywhere.  The main result of this section is presented below.
\end{remark}
\begin{theorem}[Alexandrov Theorem]\label{alexandrov}
	Let $f:\bbR^n\to\bbR$ to a convex function. Then $f$ is almost everywhere twice differentiable. More precisely, there exists $\nabla^2f:\bbR^n\to\bbR^{n\times n}$ such that for almost every $x\in\bbR^n$,
	\begin{align}
		\lim_{y\to x}\frac{1}{\vert y-x\vert^2}\left(f(y)-f(x)-\nabla f(x)\cdot (y-x)-\frac{1}{2}(y-x)^\top\nabla f(x)(y-x)\right)=0.\label{alexandrovapprox}
	\end{align}
\end{theorem}

To prove this theorem, we need some maximal inequalities concerning convex functions in a ball.
\begin{lemma}
If $f:\bbR^n\to\bbR$ is convex, there exists a constant $C_n>0$ such that for each ball $B(x,r)\subset\bbR^n$,
	\begin{align}
		\sup_{y\in B(x,\frac{r}{2})}\vert f(y)\vert\leq\frac{C_n}{m(B(x,r))}\int_{B(x,r)}\vert f(y)\vert\,dy,\label{convexlebdiff1}
	\end{align}
	and 
	\begin{align}
		\underset{y\in B(x,\frac{r}{2})}{\esssup}\vert \nabla f(y)\vert\leq\frac{C_n}{r\cdot m(B(x,r))}\int_{B(x,r)}\vert f(y)\vert\,dy.\label{convexlebdiff2}
	\end{align}
\end{lemma}
\begin{proof}
\textbf{Step I.} We first prove (\ref{convexlebdiff1}) for $f\in C^2(\bbR^n)$. Given $B(x,r)\subset\bbR^n$, we fix $z\in B(x,\frac{r}{2})$. Then
\begin{align*}
	f(y)\geq f(z)+\nabla f(z)\cdot (y-z).
\end{align*}
We integrate this inequality with respect to $y$ over $B(z,\frac{r}{2})$ to obtain
\begin{align}
f(z)\leq\frac{1}{m(B(z,\frac{r}{2}))}\int_{B(z,\frac{r}{2})}f(y)\,dy\leq\frac{2^n}{m(B(x,r))}\int_{B(x,r)}\vert f(y)\vert\,dy.\label{convexhlest1}
\end{align}
Next, we choose $\phi\in C_c^\infty(\bbR^n)$ such that $0\leq\phi\leq 1$ on $\bbR^n$, $\phi=1$ on $B(0,\frac{1}{2})$ and $\phi=0$ outside $B(0,1)$, and write $M_1=\sup_{\vert y\vert\leq1}\vert\nabla\phi(y)\vert$. Then the function $\phi_{x,r}(y)=\phi(\frac{y-x}{r})$ satisfies
\begin{align*}
\begin{cases}
0\leq\phi_{x,r}\leq 1,\ \ \vert\nabla\phi_{x,r}\vert\leq\frac{M_1}{r},\\
\phi=1\ \text{on}\ B(x,\frac{r}{2}),\ \ \phi=0\ \text{on}\ \bbR^n\backslash B(x,r).
\end{cases}
\end{align*}
We multiply by $\phi_{x,r}(y)$ the estimate $f(z)\geq f(y)+\nabla f(y)\cdot (z-y)$ and integrate with respect to $y$ on $B(x,r)$:
\begin{align*}
	f(z)\int_{B(x,r)}\phi_{x,r}(y)\,dy&\geq \int_{B(x,r)}f(y)\phi_{x,r}(y)\,dy+\int_{B(x,r)}\phi_{x,r}(y)\nabla f(y)\cdot(z-y)\,dy\\
	&=\int_{B(x,r)}f(y)\left(\phi_{x,r}(y)-\nabla\cdot\phi_{x,r}(y)(z-y)\right)\,dy\geq -M_1\int_{B(x,r)}\vert f(y)\vert\,dy.
\end{align*}
This inequality implies
\begin{align}
	f(z)\geq -\frac{M_1}{m(B(x,r))}\int_{B(x,r)}\vert f(y)\vert\,dy.\label{convexhlest2}
\end{align}
Since $z\in B(x,\frac{r}{2})$ is arbitrary, we combine (\ref{convexhlest1}) and (\ref{convexhlest2}) to obtain the estimate (\ref{convexlebdiff1}).

\item\textbf{Step II.} More generally, for a convex function $f:\bbR^n\to\bbR$, we define $f^\epsilon=\eta_\epsilon*f$, where $\epsilon>0$ and $\eta_\epsilon(y)=\epsilon^{-n}\eta(\epsilon^{-1}y)$ is the standard mollifier. Clearly $f^\epsilon\in C^\infty(\bbR^n)$, and $f^\epsilon$ is convex, since
\begin{align*}
	f^\epsilon(\lambda x+(1-\lambda)y)&\leq\int_{\bbR^n}f(\lambda x+(1-\lambda)y-z)\eta_\epsilon(z)\,dz\\
	&\leq\lambda\int_{\bbR^n}f(x-z)\eta_\epsilon(z)\,dz+(1-\lambda)\int_{\bbR^n}f(y-z)\eta_\epsilon(z)\,dz\leq\lambda f^\epsilon(x)+(1-\lambda)f^\epsilon(y).
\end{align*}
Applying the assertion (i) for $C^2$ functions, we have
\begin{align*}
	\sup_{z\in B(x,\frac{r}{2})}\vert f^\epsilon(z)\vert\leq\frac{C_n}{m(B(x,r))}\int_{B(x,r)}\vert f^\epsilon(y)\vert\,dy
\end{align*}
for each ball $B(x,r)\subset\bbR^n$. Since $f$ is locally Lipschitz continuous, $f^\epsilon\to f$ uniformly on $B(x,r)$ as $\epsilon\to 0$, which gives the same estimate (\ref{convexlebdiff1}) for $f$.

\item\textbf{Step III.} For each $z\in B(x,\frac{r}{2})$ such that $\nabla f(z)$ exists, define
\begin{align*}
	S_z=\left\{y\in\bbR^n:\frac{r}{4}\leq\vert y-z\vert\leq\frac{r}{2},\ \nabla f(z)\cdot(y-z)\geq\frac{1}{2}\vert\nabla f(z)\vert\left\vert y-z\right\vert\right\}.
\end{align*}
Then $m(S_z)\geq Ar^n$, where $A>0$ is a constant only depending on $n$. Using the estimate (\ref{convex1est}), we have
\begin{align*}
	f(y)\geq f(z)+\frac{r}{8}\vert\nabla f(z)\vert.
\end{align*}
Integrating with respect to $y$ over $S(z)$ gives
\begin{align*}
	\vert\nabla f(z)\vert\leq\frac{8}{r}\cdot\frac{1}{m(S_z)}\int_{S_z}\vert f(y)-f(z)\vert\,dy\leq\frac{8}{Ar^{n+1}}\int_{B(x,r)}\vert f(y)-f(z)\vert\,dy.
\end{align*}
This estimate and (\ref{convexlebdiff1}) complete the proof of assertion (i) for convex functions $f$.
\end{proof}
\begin{proof}[Proof of Theorem \ref{alexandrov}]
The proof has four steps.
\item\textbf{Step I.} According to the Lebesgue differentiation theorem [Theorem \ref{lebdiffthm}] and Rademacher's theorem [Theorem \ref{rademacher}], for almost every $x\in\bbR^n$, the gradient $\nabla f(x)$ exists and satisfies
\begin{align}
&\lim_{r\to 0^+}\frac{1}{m(B(x,r))}\int_{B(x,r)}\vert\nabla f(y)-\nabla f(x)\vert\,dy=0,\label{alexandrovgradest}\\
&\hspace{-4.30cm}\text{and}\notag\\
&\lim_{r\to 0^+}\frac{1}{m(B(x,r))}\int_{B(x,r)}\Vert D^2f(y)-D^2f(x)\Vert_\mathrm{F}\,dy=0.\label{alexandrovhessest}
\end{align}
By singularity of $\mu_\mathrm{s}^{ij}$ and Lemma \ref{radonsingdifflemma}, for almost every $x\in\bbR^n$, the measures $(\mu_s^{ij})$ satisfy
\begin{align}
\lim_{r\to 0^+}\frac{\vert\mu_\mathrm{s}^{ij}\vert(B(x,r))}{m(B(x,r))}=0,\quad i,j=1,2,\cdots,n.\label{alexandrovsingest}
\end{align}
We fix such a point $x$, and we also assume $x=0$ for simplicity, since our proof is adapted to the convex function $(\tau_xf)(y)=f(y-x)$. We choose $r>0$ and let $f^\epsilon=\eta_\epsilon*f$. For $y\in B(0,r)$, by Taylor's theorem,
\begin{align*}
	f^\epsilon(y)&=f^\epsilon(0)+\nabla f^\epsilon(0)\cdot y+\int_0^1(1-t)y^\top \nabla^2f^\epsilon(ty)y\,dt\\
	&=f^\epsilon(0)+\nabla f^\epsilon(0)\cdot y+\frac{1}{2}y^\top D^2f(0)y+\int_0^1(1-t)y^\top[\nabla^2f^\epsilon(ty)-D^2f(0)]\,y\,dt.
\end{align*}
\textbf{Step II.} For any function $\phi\in C_c^2(B(0,r))$ with $\vert\phi\vert\leq 1$, we multiply the equation above by $\phi$ and take the average over $B(0,r)$. Then
\begin{align}\label{convextaylor2est}
	&\frac{1}{m(B(0,r))}\int_{B(0,r)}\phi(y)\left(f^\epsilon(y)-f^\epsilon(0)-\nabla f^\epsilon(0)\cdot y-\frac{1}{2}y^\top D^2f(0)y\right)dy\notag\\
	&\quad=\frac{1}{m(B(0,r))}\int_{B(0,r)}\phi(y)\left(\int_0^1(1-t)y^\top[\nabla^2f^\epsilon(ty)-D^2f(0)]\,y\,dt\right)dy\\
	&\quad=\int_0^1(1-t)\left(\frac{1}{m(B(0,r))}\int_{B(0,r)}\phi(y)y^\top[\nabla^2f^\epsilon(ty)-D^2f(0)]\,y\,dy\right)dt\tag{By Fubini's theorem}\\
	&\quad=\int_0^1\frac{1-t}{t^2}\left(\frac{1}{m(B(0,tr))}\int_{B(0,tr)}\phi\left(\frac{z}{t}\right)z^\top[\nabla^2f^\epsilon(z)-D^2f(0)]\,z\,dz\right)dt.\tag{Change the vairable $z=ty$}
\end{align}
To estimate the inner integral, we use integration by parts:
\begin{align*}
	g_\epsilon(t)&=\int_{B(0,tr)}\phi\left(\frac{z}{t}\right)z^\top \nabla^2f^\epsilon(z)z\,dz=\int_{B(0,tr)}f^\epsilon(z)\sum_{i,j=1}^n\frac{\partial^2}{\partial z_i\partial z_j}\left(\phi\left(\frac{z}{t}\right)z_iz_j\right)dz.
\end{align*}
Letting $\epsilon\to 0^+$, we obtain
\begin{align}
	\begin{aligned}
	\lim_{\epsilon\to 0^+}g_\epsilon(t)&=\int_{B(0,tr)}f(z)\sum_{i,j=1}^n\frac{\partial^2}{\partial z_i\partial z_j}\left(\phi\left(\frac{z}{t}\right)z_iz_j\right)dz\\
	&=\sum_{i,j=1}^n\int_{B(0,tr)}\phi\left(\frac{z}{t}\right)z_iz_j\,d\mu^{ij}\\
	&=\sum_{i,j=1}^n\int_{B(0,tr)}f^{ij}\phi\left(\frac{z}{t}\right)z_iz_j\,dz+\sum_{i,j=1}^n\int_{B(0,tr)}\phi\left(\frac{z}{t}\right)z_iz_j\,d\mu^{ij}_{\rm s}.
\end{aligned}\label{convexgibpest}
\end{align}
Furthermore, we have the following estimate:
\begin{align*}
\frac{g_\epsilon(t)}{t^2}&\leq r^2\int_{B(0,tr)}\Vert\nabla^2f(z)\Vert_\mathrm{F}\,dz=r^2\int_{B(0,tr)}\left\Vert\int_{\bbR^n}\nabla^2\eta_\epsilon(z-y)f(y)\,dy\right\Vert_\mathrm{F}\,dz\\
&=r^2\int_{B(0,tr)}\left\Vert\int_{B(z,\epsilon)}\eta_\epsilon(z-y)\,dM(y)\right\Vert_\mathrm{F}\,dz\\
&\leq r^2\int_{B(0,tr)}\sum_{i,j=1}^n\int_{B(z,\epsilon)}\vert\eta_\epsilon(z-y)\vert\, d\vert\mu^{ij}\vert(y)\,dz.
\end{align*}
By Fubini's theorem, we have
\begin{align*}
\frac{g_\epsilon(t)}{t^2}&\leq r^2\sum_{i,j=1}^n\int_{B(0,tr+\epsilon)}\left(\int_{B(y,\epsilon)\cap B(0,tr)}\vert\eta_\epsilon(z-y)\vert\,dz\right)d\vert\mu^{ij}\vert(y)\\
&\leq\frac{r^2}{\epsilon^n}\sum_{i,j=1}^n\int_{B(0,tr+\epsilon)}\left(\int_{B(y,\epsilon)\cap B(0,tr)}dz\right)d\vert\mu^{ij}\vert(y).
\end{align*}
Since $(\mu^{ij})$ are Radon measures, we have $\sum_{i,j=1}^n\vert\mu^{ij}\vert(B(0,r+1))<\infty$
\begin{align*}
	\frac{g_\epsilon(t)}{t^2}&\leq\frac{r^2\min\{\epsilon^n,t^nr^n\}}{\epsilon^n}\sum_{i,j=1}^n\vert\mu^{ij}\vert(B(0,tr+\epsilon))\leq Cr^2\min\left\{1,\frac{t^nr^n}{\epsilon^n}\right\}\leq Cr^2,
\end{align*}
where $C$ is a constant only depending on $f$, $r$ and $n$. Hence we applying dominated convergence theorem to let $\epsilon\to 0^+$ in (\ref{convextaylor2est}), and plug-in (\ref{convexgibpest}):
\begin{align*}
	&\frac{1}{m(B(0,r))}\int_{B(0,r)}\phi(y)\left(f(y)-f(0)-\nabla f(0)\cdot y-\frac{1}{2}y^\top D^2f(0)y\right)dy\notag\\
	&\quad=\int_0^1\frac{1-t}{t^2}\left(\frac{1}{m(B(0,tr))}\int_{B(0,tr)}\phi\left(\frac{z}{t}\right)z^\top[D^2f(z)-D^2f(0)]\,z\,dz\right)dt\\
	&\quad\quad+\sum_{i,j=1}^n\int_0^1\frac{1-t}{t^2}\left(\frac{1}{m(B(0,tr))}\int_{B(0,tr)}\phi\left(\frac{z}{t}\right)z_iz_j\,d\mu^{ij}_{\rm s}\right)\,dt\\
	&\quad\leq Cr^2\int_0^1\left(\frac{1}{m(B(0,tr))}\int_{B(0,tr)}\Vert D^2f(z)-D^2f(0)\Vert_\mathrm{F}\,dz\right)\,dt+Cr^2\max_{1\leq i,j\leq n}\int_0^1\frac{\vert M_\mathrm{s}\vert(B(0,tr))}{(tr)^n}\,dt,
\end{align*}
where $C$ is a constant depending on $f,r$ and $n$ only. According to the properties (\ref{alexandrovhessest}) and (\ref{alexandrovsingest}), and taking the supremum over all $\phi\in C_c^2(B(0,r))$ with $\vert\phi\vert\leq 1$, we have
\begin{align}
	\frac{1}{m(B(0,r))}\int_{B(0,r)}\vert h(y)\vert dy=o(r^2),\label{hlebdifforsq}
\end{align}
where
\begin{align}
	h(y)=f(y)-f(0)-\nabla f(0)\cdot y-\frac{1}{2}y^\top D^2f(0)y.\label{convextaylor2remainder}
\end{align}
\item\textbf{Step III.} We claim that there exists a constant $C$ depending on $f$ and $n$ only, such that
\begin{align}
	\underset{x\in B(0,\frac{r}{2})}{\esssup}\vert\nabla h(x)\vert\leq\frac{C}{r\cdot m(B(0,r))}\int_{B(0,r)}\vert h(y)\vert\,dy+Cr\label{alexandrovgradest2}
\end{align}
for all $r>0$. This estimate follows by applying (\ref{convexlebdiff2}) on the convex function $g(y)=h(y)+\frac{1}{2}\Vert D^2f(0)\Vert_2\vert y\vert^2$.

\item\textbf{Step IV.} We fix $0<\epsilon,\eta<1$. By (\ref{hlebdifforsq}), for some $r_0>0$ depending on $\eta$ and $\epsilon$, we have
\begin{align*}
	m(\{z\in B(0,r):\vert h(z)\vert\geq\epsilon r^2\})\leq\frac{1}{\epsilon r^2}\int_{B(0,r)}\vert h(z)\vert\,dz\leq\eta\cdot m(B(0,r))
\end{align*}
for all $0<r<r_0$. Thus for each point $y\in B(0,\frac{r}{2})$, there exists $z\in B(0,r)$ with $\vert y-z\vert\leq\eta^{1/n}r$ such that $\vert h(z)\vert<\epsilon r^2$. If not, a contradiction arises from
\begin{align*}
	m(\{z\in B(0,r):\vert h(z)\vert\geq\epsilon r^2\})\geq m(B(y,\eta^{1/n}r))=\eta\cdot m(B(0,r)).
\end{align*}
By the estimate (\ref{alexandrovgradest2}), for all $0<r<r_0$ and all $y\in B(0,\frac{r}{2})$,
\begin{align*}
	\frac{\vert h(y)\vert}{r^2}&\leq\frac{\vert h(z)\vert+\vert h(y)-h(z)\vert}{r^2}\leq \epsilon+\frac{\eta^{1/n}}{r}\underset{B(0,r)}{\esssup}\,\vert\nabla h\vert\\
	&\leq\epsilon+C\eta^{1/n}\left(\frac{1}{r^2\cdot m(B(0,r))}\int_{B(0,r)}\vert h(\xi)\vert\,d\xi+1\right)\leq\epsilon+C\eta^{1/n}\left(\eta\epsilon+1\right)
\end{align*}
Since $0<\epsilon,\eta<1$ are arbitrary, we have
\begin{align*}
	\lim_{r\to 0^+}\frac{1}{r^2}\sup_{y\in B(0,\frac{r}{2})}\vert h(y)\vert=0.
\end{align*}
Recalling the definition (\ref{convextaylor2remainder}) of $h$, we have
\begin{align*}
\lim_{r\to 0^+}\frac{1}{r^2}\sup_{y\in B(0,\frac{r}{2})}\left\vert f(y)-f(0)-\nabla f(0)\cdot y-\frac{1}{2}y^\top D^2f(0)y\right\vert=0,
\end{align*}
which implies (\ref{alexandrovapprox}) for $x=0$. The same estimate holds for every $x\in\bbR^n$ satisfying (\ref{alexandrovgradest}), (\ref{alexandrovhessest}) and (\ref{alexandrovsingest}), which concludes the proof.
\end{proof}


\newpage
\section{Ergodic Theory}
\paragraph{Setting.} Let $(X,\scr{F},\mu)$ be a $\sigma$-finite measure space. We consider a mapping $T:X\to X$ such that
\begin{itemize}
	\item[(i)] $T$ is \textit{measurable}, i.e. $T^{-1}(E)\in\scr{F}$ for each $E\in\scr{F}$;
	\item[(ii)] $T$ is \textit{measure-preserving}, i.e. $\mu(T^{-1}(E))=\mu(E)$ for each $E\in\scr{F}$.
	\item[(iii)] We call the quadruple $(X,\scr{F},\mu,T)$ a \textit{measure-preserving system}.
\end{itemize}
If in addition for such a transformation $T$ we have that $T$ is a bijection and $T^{-1}$ is also a measure-preserving transformation, then $T$ is called a \textit{measure-preserving isomorphism}.

If $f:X\to\bbC$ is a measurable function and $T$ is a measure-preserving transformation, the composition $f\circ T$ is measurable. Furthermore, if $f$ is integrable, so is $f\circ T$, and
\begin{align*}
	\int_X f\,d\mu=\int_X f\circ T\,d\mu.
\end{align*}

The setting described above is of interest, in part, because it
abstracts the idea of a dynamical system, one whose totality of states is
represented by the space $X$, with each point $x\in X$ giving a particular
state of the system. The mapping $T:X\to X$ describes the transformation of the system after a unit of time has elapsed. The iterates, $T^n=T\circ T\circ\cdots\circ T$ ($n$ times) describe the evolution of the system after $n$ units of time. In many scenarios, we are interested in the average behavior of the system as the time $n\to\infty$. To be specific, given a measurable function $f$ on $(X,\scr{F},\mu)$, we aim to study the \textit{ergodic averages}
\begin{align*}
	(A_nf)(x)=\frac{1}{n}\sum_{k=0}^{n-1} f(T^kx)
\end{align*}
and their limit as $n\to\infty$.

\subsection{The Mean Ergodic Theorem}
We first discuss a general ergodic result for Banach spaces. 
\begin{theorem}[Mean ergodic theorem]\label{meanerg}
Let $T:X\to X$ be a bounded linear operator on a Banach space $X$, and assume that $\sup_{n\in\bbN}\Vert T^n\Vert<\infty$. For $n\in\bbN$, define the ergodic average
\begin{align*}
	A_n=\frac{1}{n}\sum_{k=0}^{n-1}T^k.
\end{align*}
\begin{itemize}
\item[(i)] If $x\in X$, the sequence $(A_n x)_{n=1}^\infty$ converges if and only if it has a weakly convergent subsequence;
\item[(ii)] The set 
\begin{align*}
	L=\left\{x\in X:\textit{the sequence $(A_nx)_{n=1}^\infty$ converges}\right\}
\end{align*}
is a closed $T$-invariant subspace of $X$, and $L=\ker(\id-T)\oplus\ol{\mathfrak{R}(\id-T)}$. 
\item[(iii)] If $X$ reflexive, then $L=X$.
\item[(iv)] Define the operator $A:L\to L$ by $A(x_0+x_1)=x_0$ for $x_0\in\ker(\id-T)$ and $x_1\in\ol{\mathfrak{R}(\id-T)}$. Then
\begin{align*}
	\lim_{n\to\infty} A_n x=Ax
\end{align*}
for all $x\in L$, and $A$ satisfies
\begin{align*}
	AT=TA=A^2=A,\quad and\quad \Vert A\Vert\leq\sup_{n\in\bbN}\Vert T^n\Vert.
\end{align*}
\end{itemize}
\end{theorem}
The proof of this theorem requires some lemmata.
\begin{lemma}\label{meanerglemma1}
Assume $c=\sup_{n\in\bbN}\Vert T^n\Vert<\infty$.
\begin{itemize}
	\item[(i)] For each $n\in\bbN$, $\Vert A_n\Vert\leq c$ and $\Vert A_n(\id-T)\Vert\leq\frac{1+c}{n}$.
	\item[(ii)] If $x\in\ker(\id-T)$, then for each $n\in\bbN$, we have $A_nx=x$ and $\Vert x\Vert\leq c\Vert x+(\id-T)\xi\Vert$ for all $\xi\in X$.
	\item[(iii)] If $x\in\ker(\id - T)$ and $y\in\ol{\mathfrak{R}(\id-T)}$, then $\Vert x\Vert\leq c\Vert x+y\Vert$.
	\item[(iv)] $\ker(\id-T)\cap\ol{\mathfrak{R}(\id-T)}=0$, and $L:=\ker(\id-T)\oplus\ol{\mathfrak{R}(\id-T)}$ is a closed subspace of $X$.
	\item[(v)] $T(L)\subset L$.
	\item[(vi)] If $y\in\ol{\mathfrak{R}(\id-T)}$, then $\lim_{n\to\infty} A_ny=0$.
\end{itemize}
\end{lemma}
\begin{proof}
(i) Since $A_n=\frac{1}{n}(\id+T+T^2+\cdots+T^n)$, we have
\begin{align*}
	\Vert A_n\Vert\leq\frac{1}{n}\sum_{k=0}^{n-1}\Vert T^k\Vert\leq\sup_{n\in\bbN}\Vert T^n\Vert=c,\quad\text{and}\quad
	\Vert A_n(\id-T)\Vert=\frac{1}{n}\left\Vert\id-T^n\right\Vert\leq\frac{1+\Vert T^n\Vert}{n}\leq\frac{1+c}{n}.
\end{align*}
\item (ii) If $x\in\ker(\id-T)$, we have $Tx=x$ and by induction $T^nx=x$ for all $n\in\bbN$ and hence $A_nx=x$. Moreover, by (i) we have $A_n(\id-T)\xi\to 0$ as $n\to\infty$ for all $\xi\in X$, and
\begin{align*}
	\Vert x\Vert=\lim_{n\to\infty}\Vert x+A_n(\id-T)\xi\Vert=\lim_{n\to\infty}\Vert A_n(x+(\id-T)\xi)\Vert\leq c\Vert x+(\id-T)\xi\Vert.
\end{align*}
\item (iii) If $y\in\ol{\mathfrak{R}(\id-T)}$, there exists a sequence $\xi_n\in X$ such that $(\id-T)\xi_n\to y$. We take $\xi=\xi_n$ in (ii) and take the limit $n\to\infty$ to obtain $\Vert x\Vert\leq c\Vert x+y\Vert$.

\item (iv) We let $x\in\ker(\id-T)\cap\ol{\mathfrak{R}(\id-T)}$. Then $-x\in\ol{\mathfrak{R}(\id-T)}$, and by (iii) we have $\Vert x\Vert\leq c\Vert x+(-x)\Vert=0$. Next we show that $\ker(\id-T)\oplus\ol{\mathfrak{R}(\id-T)}$ is closed. Let $x_n\in\ker(\id-T)$ and $y_n\in\ol{\mathfrak{R}(\id-T)}$ be sequences whose sum $z_n=y_n+z_n$ converges to some element $z\in X$. Then $(z_n)$ is a Cauchy sequence in $X$, and by (iii) the sequence $(x_n)$ is also Cauchy, and hence $y_n=z_n-x_n$ is also Cauchy. Since $\ker(\id-T)$ and $\ol{\frak{R}(\id-T)}$ are closed subspaces of $X$, the Cauchy sequences $(x_n)$ and $(y_n)$ converge to $x\in\ker(\id-T)$ and $y\in\ol{\frak{R}(\id-T)}$, respectively, and $z=x+y\in\ker(\id-T)\oplus\ol{\mathfrak{R}(\id-T)}$.

\item (v) We take $x\in\ker(\id-T)$ and $y\in\ol{\mathfrak{R}(\id-T)}$, and take a sequence $\xi_n\in X$ such that $(\id-T)\xi_n\to y$. Then $$T(x+y)=x+Ty=x+\lim_{n\to\infty}T(\id-T)\xi_n=x+\lim_{n\to\infty}(\id-T)(T\xi_n)\in\ker(\id-T)\oplus\ol{\mathfrak{R}(\id-T)}.$$
\item (vi) For any $\epsilon>0$, we take $\xi\in X$ such that $c\Vert y-(\id-T)\xi\Vert<\frac{\epsilon}{3}$. By (i), we have $\Vert A_n(\id-T)\xi\Vert\leq\frac{1+c}{n}\Vert\xi\Vert$, which tends to $0$ as $n\to\infty$. Then there exists $N$ such that $\Vert (A_n-A_m)(\id-T)\xi\Vert\leq\frac{\epsilon}{3}$ for all $n,m\geq N$, and
\begin{align*}
	\Vert A_n y-A_m y\Vert&\leq\Vert A_n y-A_n(\id-T)\xi\Vert+\Vert (A_n-A_m)(\id-T)\xi\Vert+\Vert A_m(\id-T)\xi-A_m y\Vert\\
	&\leq\Vert A_n\Vert\,\Vert y-(\id-T)\xi\Vert+\Vert (A_n-A_m)(\id-T)\xi\Vert+\Vert A_m\Vert\,\Vert(\id-T)\xi-y\Vert\\
	&\leq 2c\Vert y-(\id-T)\xi\Vert+\frac{\epsilon}{3}\leq\epsilon.
\end{align*}
Hence $(A_ny)$ is a Cauchy sequence, and
\begin{align*}
	\lim_{n\to\infty}\Vert A_ny\Vert = \lim_{n\to\infty}\Vert A_n(y-(\id-T)\xi)\Vert+\lim_{n\to\infty}\Vert A_n(\id-T)\xi\Vert<\frac{\epsilon}{3},
\end{align*}
which implies $A_ny\to 0$ as $n\to\infty$.
\end{proof}
\begin{lemma}\label{meanerglemma2}
Let $x,x_0\in X$. The following are equivalent:
\begin{itemize}
\item[(a)] $x_0\in\ker(\id-T)$ and $x-x_0\in\ol{\frak{R}(\id-T)}$.
\item[(b)] $\lim_{n\to\infty}\Vert A_nx-x_0\Vert =0$.
\item[(c)] There exists a subsequence $n_k$ such that for all $f\in X^*$,
\begin{align*}
	\lim_{k\to\infty} f(A_{n_k}x)=f(x_0).
\end{align*}
\end{itemize}
\end{lemma}
\begin{proof}
The Lemma \ref{meanerglemma1} (vi) implies (a) $\Rightarrow$ (b), and obviously (a) $\Rightarrow$ (c). Then it remains to prove (c) $\Rightarrow$ (a). If (c) holds, we take $f\in X^*$. Then $T^*f=f\circ T:X\to\bbC$ is a bounded linear functional, and
\begin{align*}
	f(x_0-Tx_0)=(f-T^*f)(x_0)=\lim_{k\to\infty}(f-T^*f)(A_{n_k}x)=\lim_{k\to\infty}f((\id-T)A_{n_k}x)=0,
\end{align*}
where the last equality follows from Lemma \ref{meanerglemma1} (i), and we have $Tx_0=x_0$ by Hahn-Banach theorem. 

Now we assume that $x-x_0\in\ol{\frak{R}(\id-T)}$. By Hahn-Banach theorem, there exists $f\in X^*$ such that $f(x-x_0)=1$ and $f(\xi-T\xi)=0$ for all $\xi\in X$. This implies that $f(T^{k+1}\xi-T^k\xi)=0$ for all $\xi\in X$ and all $k\in\bbN_0$. By induction, we have $f(T^k\xi)=f(\xi)$. Hence
\begin{align*}
	f(A_nx)=\frac{1}{n}\sum_{k=0}^{n-1}f(T^kx_0)=f(x_0)
\end{align*}
for all $n\in\bbN_0$. According to (c), we have $f(x)=f(x_0)$, and $f(x-x_0)=0$, which is a contradiction. Therefore $x-x_0\in\ol{\frak{R}(\id-T)}$, and we complete the proof. 
\end{proof}

Now we prove the main theorem.
\begin{proof}[Proof of Theorem \ref{meanerg}] 
By Lemma \ref{meanerglemma2}, the sequence $(A_n x)_{n=1}^\infty$ converges in norm if and only if it has a weakly convergent subsequence, if and only if $x\in L=\ker(\id-T)\oplus\ol{\frak{R}(\id-T)}$. By Lemma \ref{meanerglemma1} (iv) and (v), the subspace $L$ is closed and $T$-invariant. Furthermore, since $\Vert A_n\Vert\leq c$ for all $n\in\bbN$, for every $x\in X$, the sequence $(A_nx)$ is bounded. If $X$ is reflexive, by Banach-Alaoglu theorem, every $(A_n x)$ has a weakly convergent subsequence $(A_{n_k}x)$, which implies $x\in L$, and hence $L=X$.

Finally we consider the operator $A$ defined in (iv). Then $A^2=A$ by definition. By Lemma \ref{meanerglemma1} (iii), we have $\Vert A\Vert\leq c$, and by Lemma \ref{meanerglemma1} (vi), $\lim_{n\to\infty} A(x_0+x_1)=Ax_0$. Since $A$ commutes with $T|_L$, and $A$ vanishes on the range of operator $\id-T$, we have $TA=AT=A$.
\end{proof}

Since Hilbert spaces are reflexive, we have the following mean ergodic theorem for Hilbert spaces.
\begin{corollary}[Mean ergodic theorem]
Let $T$ be a bounded linear operator on the Hilbert space $H$ such that $\sup_{n\in\bbN}\Vert T^n\Vert<\infty$, and let $P_T$ be the projection operator onto the subspace
\begin{align*}
	\ker(\id-T)=\{x\in H:Tx=x\}.
\end{align*}
Then for every $x\in H$, the ergodic average
\begin{align*}
	A_nx:=\frac{1}{n}\sum_{k=0}^{n-1}T^kx\to P_Tx\quad\text{in norm as $n\to\infty$}.
\end{align*}
\end{corollary}

In particular, we take the Hilbert space to be $L^2(X,\scr{F},\mu)$. If $T$ is a measure-preserving operator on $X$, we regard $T$ as a linear operator on $L^2(X,\scr{F},\mu)$ by writing $Tf=f\circ T$. Then $T$ is an isometry on $L^2(X,\scr{F},\mu)$, i.e. $\Vert Tf\Vert_{L^2}=\Vert f\Vert_{L^2}$ for all $f\in L^2(X,\scr{F},\mu)$, and $\Vert T\Vert=1$. Consequently, we have $\Vert T^n\Vert\leq 1$ for all $n\in\bbN$, and we can apply the mean ergodic theorem on this system.
\begin{corollary}[Mean ergodic theorem]\label{meanergl2}
Let $(X,\scr{F},\mu,T)$ be a measure-preserving system, and let $P_T$ be the projection operator onto the subspace
	\begin{align*}
		G=\left\{g\in L^2(X,\scr{F},\mu):g\circ T=g\right\}
	\end{align*}
	Then for every $f\in L^2(X,\scr{F},\mu)$, the ergodic average
	\begin{align*}
		A_nf:=\frac{1}{n}\sum_{k=0}^{n-1}f\circ T^k\to P_Tf\quad\text{in $L^2(X,\scr{F},\mu)$ as $n\to\infty$}.
	\end{align*}
\end{corollary}

In finite measure spaces, the ergodic average $A_nf$ also converges in $L^1$. This conclusion follows from the convergence result in $L^2$ and the density of $L^2$ in $L^1$.
\begin{corollary}\label{meanergl1}
Let $(X,\scr{F},\mu,T)$ be a measure-preserving system such that $\mu$ is finite. For each $f\in L^1(X,\scr{F},\mu)$, the ergodic average $A_nf=\sum_{k=0}^{n-1}f\circ T^k$ converges in $L^1$ to a $T$-invariant function $\bar{f}\in L^1(X,\scr{F},\mu)$.
\end{corollary}
\begin{proof}
Since $\mu$ is finite, we know by Cauchy-Schwartz inequality that $L^2(X,\scr{F},\mu)\subset L^1(X,\scr{F},\mu)$. For any $g\in L^2(X,\scr{F},\mu)$, by Cauchy's inequality and Corollary \ref{meanergl2}, 
\begin{align*}
	\Vert A_n g-P_Tg\Vert_{L^1}\leq\sqrt{\Vert A_n g-P_Tg\Vert_{L^2}\left\Vert\mathbf{1}\right\Vert_{L^2}}=\sqrt{\mu(X)\Vert A_n g-P_Tg\Vert_{L^2}}\to 0.
\end{align*}
Hence $(A_ng)$ is a Cauchy sequence in $L^1$. If $f\in L^1(X,\scr{F},\mu)$ and $\epsilon>0$, we choose $g\in L^2(X,\scr{F},\mu)$ such that $\Vert f-g\Vert_{L^1}<\epsilon/3$. Since $\Vert T(f-g)\Vert_{L^1}=\Vert f-g\Vert_{L^1}$, we have $\Vert A_n(f-g)\Vert\leq\Vert f-g\Vert_{L^1}<\epsilon/3$ for all $n\in\bbN$. Furthermore, there exists $N$ such that $\Vert A_ng-A_mg\Vert_{L^1}<\epsilon/3$ for all $n,m>N$, and
\begin{align*}
	\Vert A_n f-A_m f\Vert_{L^1}\leq\Vert A_n f-A_n g\Vert_{L^1}+\Vert A_n g-A_m g\Vert_{L^1}+\Vert A_m g-A_m f\Vert_{L^1}<\epsilon.
\end{align*}
Hence $(A_nf)$ is also a Cauchy sequence in $L^1$, which converges to a function $\bar{f}\in L^1(X,\scr{F},\mu)$ by $L^1$-completeness. To show that $\bar{f}$ is $T$-invariant, note that
\begin{align*}
	\Vert A_nf\circ T-A_nf\Vert_{L^1}=\left\Vert\frac{1}{n}\left(f\circ T^n-f\right)\right\Vert_{L^1}\leq\frac{2}{n}\Vert f\Vert_{L^1},
\end{align*}
which converges to $0$ as $n\to\infty$. Hence $\bar{f}\circ T=\bar{f}$ a.e., and $\bar{f}$ is $T$-invariant.
\end{proof}
\subsection{The Maximal Ergodic Theorem}
We now turn to the question of almost everywhere convergence of the
ergodic averages. As in the case of the averages that occur in the Lebesgue differentiation theorem, the key to dealing with such pointwise
limits lies in estimate for the corresponding maximal function:
\begin{align*}
	f^*=\sup_{1\leq n<\infty}A_nf=\sup_{1\leq n<\infty}\frac{1}{n}\sum_{k=0}^{n-1}f\circ T^k.
\end{align*}
We first state our main result below.
\begin{theorem}[Maximal ergodic theorem]\label{maxerg}
Let $(X,\scr{F},\mu,T)$ be a measure-preserving system, and fix $\alpha\in\bbR$. For each $f\in L^1(X,\scr{F},\mu)$, define
\begin{align*}
	E_\alpha^f=\left\{x\in X:\sup_{1\leq n<\infty}\frac{1}{n}\sum_{k=0}^{n-1}f(T^k x)>\alpha\right\}.
\end{align*}
Then
\begin{align*}
	\alpha\mu(E_\alpha^f)\leq\int_{E_\alpha} f\,d\mu\leq\Vert f\Vert_{L^1}.
\end{align*}
\end{theorem}
\paragraph{Remark.} If $\alpha>0$, the result can be written as
\begin{align}
	\mu(E_\alpha^f)\leq\frac{1}{\alpha}\int_{E_\alpha} f\,d\mu\leq\frac{1}{\alpha}\Vert f\Vert_{L^1}.\label{maxergineq}
\end{align}
This result is a corollary of the following maximal inequality.
\begin{proposition}[Maximal inequality]\label{maxerglemma}
Let $U:L^1(X,\scr{F},\mu)\to L^1(X,\scr{F},\mu)$ be a positive linear operator such that $\Vert U\Vert\leq 1$. For $g\in L^1(X,\scr{F},\mu)$, define the functions
\begin{align*}
	g_n=g+Ug+U^2g+\cdots U^{n-1}g
\end{align*}
for $n\in\bbN$, with $g_0=0$. Let $G_N(x)=\max_{0\leq n\leq N} g_n(x)$ for all $x\in X$. Then for every $N\geq 1$,
\begin{align*}
	\int_{\{G_N>0\}}g\,d\mu\geq 0.
\end{align*}
\end{proposition}
\begin{proof}
Since $U$ is a positive linear operator, for $0\leq n\leq N$, we have $UG_N+g\geq Ug_n+g=g_{n+1}$. Hence
\begin{align*}
	UG_N+g\geq\max_{1\leq n\leq N+1}g_n\geq \max_{1\leq n\leq N}g_n.
\end{align*}
Since $g_0=0$, on the set $E=\{G_N>0\}$, we have $$UG_N+g\geq \max_{1\leq n\leq N}g_n=\max_{0\leq n\leq N}g_n=G_N.$$
Therefore $g\geq G_N-UG_N$ on $E$. Since $G_N\geq 0$, we have $UG_N\geq 0$, and
\begin{align*}
	\int_E g\,d\mu&\geq\int_E G_N\,d\mu-\int_E UG_N\,d\mu=\int_X G_N\,d\mu-\int_E UG_N\,d\mu\\
	&\geq\int_X G_N\,d\mu-\int_X UG_N\,d\mu=\Vert G_N\Vert_{L^1}-\Vert UG_N\Vert_{L^1}\geq 0,
\end{align*}
where the last inequality follows from $\Vert U\Vert\leq 1$. Then we complete the proof.
\end{proof}

Now we prove the main theorem.
\begin{proof}[Proof of Theorem \ref{maxerg}]
Define $g=f-\alpha$ and $Ug=g\circ T$ in Proposition \ref{maxerglemma}. Then
\begin{align*}
	E_\alpha^f=\left\{x\in X:\sup_{n\in\bbN}\frac{1}{n}\sum_{k=0}^{n-1}f(T^k x)>\alpha\right\}=\bigcup_{N=0}^\infty\left\{x\in X:G_N(x)>0\right\}.
\end{align*}
By Proposition \ref{maxerglemma} and Lebesgue dominated convergence theorem,
\begin{align*}
	\int_{E_\alpha^f} f\,d\mu-\alpha\mu(E_\alpha^f)=\int_{E_\alpha^f} g\,d\mu\geq 0.
\end{align*}
Thus we complete the proof.
\end{proof}
\begin{remark}
When $\alpha>0$, we apply the same result on the negation $-f\in L^1(X,\scr{F},\mu)$, we have
\begin{align*}
	\mu\left(\inf_{1\leq n<\infty}\frac{1}{n}\sum_{k=0}^{n-1}f\circ T^k<-\alpha\right)\leq\frac{1}{\alpha}\Vert f\Vert_{L^1}.
\end{align*}
Combining this with (\ref{maxergineq}), we get the two-sided bound:
\begin{align*}
	\mu\left(\sup_{1\leq n<\infty}\vert A_nf \vert>\alpha\right)=\mu\left(\sup_{1\leq n<\infty}\left\vert\frac{1}{n}\sum_{k=0}^{n-1}f\circ T^k\right\vert>\alpha\right)\leq\frac{2}{\alpha}\Vert f\Vert_{L^1}.
\end{align*}
We later use this conclusion in the proof of pointwise convergence result.
\end{remark}

\subsection{The Birkhoff Ergodic Theorem}
In this section, we focus on the pointwise convergence theorem of ergodic averages. Our result is established on finite measure spaces, and it is convenient to assume that the measure-preserving system $(X,\scr{F},\mu,T)$ is on a probability space. Before we proceed, we first introduce the definition of ergodicity.
\begin{definition}[Ergodic transformation]
A measure-preserving transformation $T:X\to X$ on a measure space $(X,\scr{F},\mu)$ is said to be \textit{$\mu$-ergodic} if every $T$-invariant subset of $X$ is trivial, i.e. for any $E\in\scr{F}$, $$T^{-1}(E)=E\quad\Rightarrow\quad\mu(E)=0\ \ \text{or}\ \ \mu(X\backslash E)=0.$$
\end{definition}
Following are some alternate characterizations of ergodicity.
\begin{proposition}\label{ergodicfunc}
Let $(X,\scr{F},\mu,T)$ be a measure-preserving system. The following are equivalent:
\begin{itemize}
	\item[(i)] $T$ is $\mu$-ergodic;
	\item[(ii)] For any $E\in\scr{F}$, if $T^{-1}(E)$ and $E$ only differ by a $\mu$-null set, i.e. $\mu(T^{-1}(E)\backslash E)+\mu(E\backslash T^{-1}(E))=0$, then $\mu(E)=0$ or $\mu(X\backslash E)=0$;
	\item[(iii)] For any measurable function $f:X\to\bbC$, if $f\circ T=f\ a.e.$, then $f$ is constant a.e..
\end{itemize}
\begin{proof}
(i) $\Rightarrow$ (ii). Let $E$ be a set such that $\mu(T^{-1}(E)\backslash E)+\mu(E\backslash T^{-1}(E))=0$. Then
\begin{align*}
	T^{-n}(E)\backslash E\subset\bigcup_{k=0}^{n-1} T^{-k-1}(E)\backslash T^{-k}(E)=\bigcup_{k=0}^{n-1} T^{-k}(T^{-1}(E)\backslash E),\\ 
	E\backslash T^{-n}(E)\subset\bigcup_{k=0}^{n-1} T^{-k}(E)\backslash T^{-k-1}(E)=\bigcup_{k=0}^{n-1} T^{-k}(E\backslash T^{-1}(E)).
\end{align*}
Since $T$ is measure-preserving, we have $\mu(T^{-n}(E)\backslash E)+\mu(E\backslash T^{-n}(E))=0$. We define
\begin{align*}
	F=\bigcap_{N=1}^\infty\bigcup_{n=N}^\infty T^{-n}(E).
\end{align*}
Then $T^{-1}(F)=F$, and we have either $\mu(F)=0$ or $\mu(X\backslash F)=0$ by ergodicity of $\mu$. Moreover,
\begin{align*}
	F\backslash E&=\bigcap_{N=1}^\infty\left(\bigcup_{n=N}^\infty T^{-n}(E)\right)\backslash E=\bigcap_{N=1}^\infty\bigcup_{n=N}^\infty T^{-n}(E)\backslash E,\\
	E\backslash F&=\bigcup_{N=1}^\infty E\backslash\left(\bigcup_{n=N}^\infty T^{-n}(E)\right)=\bigcup_{N=1}^\infty\bigcap_{n=N}^\infty E\backslash T^{-n}(E).
\end{align*}
Hence $\mu(E\backslash F)=\mu(F\backslash E)=0$, and we have either $\mu(E)=0$ or $\mu(X\backslash E)=0$.

\item (ii) $\Rightarrow$ (iii). For $f$ given in (iii), by considering $\Re f$ and $\Im f$ separately, we may assume $f:X\to\bbR$ and $f\circ T=f$ a.e.. For any $t\in\bbR$, the sets $E_t=\{f\leq t\}$ and $T^{-1}(E_t)=\{f\circ T\leq t\}$ only differ by a $\mu$-null set. By (ii), we have $\mu(\{f\leq t\})=0$ or $\mu(\{f>t\})=0$. We take $$c=\sup\{t\in\bbR:\mu(\{f\leq t\})=0\}=\inf\{t\in\bbR:\mu(\{f>t\})=0\}.$$
Since $\{f>c\}=\bigcup_{n\in\bbN}\{f>c+n^{-1}\}$ and $\{f<c\}=\bigcup_{n\in\bbN}\{f<c+n^{-1}\}$, we have $\mu(\{f>c\})=\mu(\{f<c\})=0$. Hence $f=c$ a.e..

\item (iii) $\Rightarrow$ (i). If $E$ is a $T$-invariant set, i.e. $E=T^{-1}(E)$, we take $f=\chi_E$ in (iii). Then $\chi_E\circ T=\chi_{T^{-1}(E)}$, which equals $\chi_E$ a.e.. By (iii), $\chi_E=c$ a.e., where $c\in\{0,1\}$. Hence either $\mu(E)=0$ or $\mu(X\backslash E)=0$.
\end{proof}
\end{proposition}
Now we are ready to introduce the main result.
\begin{theorem}[Birkhoff's theorem]
Let $(X,\scr{F},\mu,T)$ be a measure-preserving system on a probability space. \begin{itemize}
	\item[(i)] For each $f\in L^1(X,\scr{F},\mu)$, the ergodic average
	\begin{align*}
		A_nf=\frac{1}{n}\sum_{k=0}^{n-1}f\circ T^k
	\end{align*}
	converges almost everywhere to a $T$-invariant function $\bar{f}\in L^1(X,\scr{F},\mu)$, where $\int_X f\,d\mu=\int_X\bar{f}\,d\mu$. 
	\item[(ii)] In addition, if $T$ is $\mu$-ergodic, $\bar{f}=\int_X f\,d\mu$.
\end{itemize}
\end{theorem}
\begin{proof}
We first assume that $g\in L^\infty(X,\scr{F},\mu)$. By Corollary \ref{meanergl1}, there is a $T$-invariant function $\bar{g}\in L^1(X,\scr{F},\mu)$ such that $A_ng\to\bar{g}$ in $L^1$. For any $\epsilon>0$, we choose $n$ sufficiently large so that $\Vert\bar{g}-A_ng\Vert_{L^1}<\epsilon^2$. Applying maximal ergodic theorem [Theorem \ref{maxerg}] to the function $h=\bar{g}-A_ng$, we have
\begin{align*}
	\mu\left(\left\{x\in X:\sup_{m\in\bbN}\left\vert A_m(\bar{g}-A_ng)\right\vert>\epsilon\right\}\right)\leq\frac{2}{\epsilon}\Vert\bar{g}-A_ng\Vert_{L^1}<2\epsilon.
\end{align*}
Since $\bar{g}$ is $T$-invariant, $A_m\bar{g}=\bar{g}$. Also,
\begin{align*}
	A_m(A_ng)&=\frac{1}{mn}\sum_{j=0}^{m-1}\sum_{k=0}^{n-1}g\circ T^{j+k}=\frac{1}{mn}\sum_{j=0}^{m-1}\sum_{k=0}^{n-1}\left(g\circ T^j+g\circ T^{j+k}-g\circ T^j\right)\\
	&=A_mg+\frac{1}{mn}\sum_{k=0}^{n-1}\sum_{j=0}^{m-1}\left(g\circ T^{j+k}-g\circ T^j\right)= A_mg+\frac{1}{mn}\sum_{k=1}^{n-1}\sum_{j=0}^{k-1}\left(g\circ T^{m+j}-g\circ T^j\right),
\end{align*}
and
\begin{align*}
	\vert A_m(A_ng)-A_mg\vert&\leq\frac{1}{mn}\sum_{k=1}^{n-1}\sum_{j=0}^{k-1}\left\vert g\circ T^{m+j}-g\circ T^j\right\vert\leq\frac{1}{mn}\frac{n(n-1)}{2} 2\Vert g\Vert_\infty=\frac{n-1}{m}\Vert g\Vert_\infty,
\end{align*}
which converges to $0$ as $m\to\infty$. Then $\limsup_{m\to\infty}\vert\bar{g}-A_mg\vert=\limsup_{m\to\infty}\vert\bar{g}-A_m(A_ng)\vert$, and
\begin{align*}
\mu\left(\limsup_{m\to\infty}\vert\bar{g}-A_mg\vert>\epsilon\right)&=\mu\left(\limsup_{m\to\infty}\vert\bar{g}-A_m(A_ng)\vert>\epsilon\right)=\mu\left(\limsup_{m\to\infty}\vert A_m(\bar{g}-A_ng)\vert>\epsilon\right)<2\epsilon.
\end{align*}
Thus $\limsup_{m\to\infty}\vert\bar{g}-A_mg\vert=0$, and $A_mg\to\bar{g}$ a.e. as $m\to\infty$. To generalize the result to $f\in L^1(X,\scr{F},\mu)$, we take $g\in L^\infty(X,\scr{F},\mu)$ such that $\Vert f-g\Vert_{L^1}<\epsilon^2$. By Corollary \ref{meanergl1}, we take a $T$-invariant $\bar{f}\in L^1(X,\scr{F},\mu)$ such that $\Vert A_nf-\bar{f}\Vert_{L^1}\to 0$. Since $\Vert A_nf-A_ng\Vert_{L^1}\leq\Vert f-g\Vert_{L^1}<\epsilon^2$, we have $\Vert\bar{f}-\bar{g}\Vert_{L^1}<\epsilon^2$. Thus
\begin{align*}
	\mu\left(\limsup_{n\to\infty}\vert\bar{f}-A_nf\vert>2\epsilon\right)&\leq\mu\left(\vert\bar{f}-\bar{g}\vert+\limsup_{n\to\infty}\vert\bar{g}-A_ng\vert+\sup_{n\in\bbN}\vert A_nf-A_ng\vert>2\epsilon\right)\\
	&\leq\mu\left(\vert\bar{f}-\bar{g}\vert>\epsilon\right)+\mu\left(\sup_{n\in\bbN}\vert A_n(f-g)\vert>\epsilon\right)\\
	&\leq\frac{\Vert\bar{f}-\bar{g}\Vert_{L^1}}{\epsilon}+\frac{2\Vert f-g\Vert_{L^1}}{\epsilon}<3\epsilon,
\end{align*}
where the third inequality follows from Markov's inequality and the maximal ergodic theorem. Hence $A_nf\to\bar{f}$ a.e. and in $L^1$. Since $T$ is measure preserving,
\begin{align*}
	\int_X\bar{f}\,d\mu=\int_XA_nf\,d\mu=\int_X f\,d\mu.
\end{align*}
In addition, if $T$ is a $\mu$-ergodic transformation, by Proposition \ref{ergodicfunc} the $T$-invariant function $\bar{f}$ is constant a.e.. Since $\mu(X)=1$, we have $\bar{f}=\int_X f\,d\mu$ a.e..
\end{proof}

\begin{remark}
For a measure-preserving system $(X,\scr{F},\mu,T)$ on a probability space and a function $f\in L^1(X,\scr{F},\mu)$, we define the \textit{time average at $x\in X$} to be $(A_nf)(x)=\frac{1}{n}\sum_{k=0}^{n-1} f(T^k x)$ and the \textit{space average} $\int_X f\,d\mu$. 

A brief interpretation for Birkhoff theorem is that, if $T$ is $\mu$-ergodic, then for almost every $x\in X$, the time average converges to the space average as the time $n$ goes to infinity.
\end{remark}

We can obtain a stronger mean ergodic theorem as a consequence of Birkhoff's theorem.
\begin{theorem}[Mean ergodic theorem]\label{meanerglp}
Let $(X,\scr{F},\mu,T)$ be a measure-preserving system on a probability space, and let $1\leq p<\infty$. If $f\in L^p(X,\scr{F},\mu)$, the ergodic average $A_nf=\sum_{k=0}^{n-1}f\circ T^k$ converges in $L^p$ to a $T$-invariant function $\bar{f}\in L^p(X,\scr{F},\mu)$, i.e.
\begin{align*}
	\lim_{n\to\infty}\left\Vert\,\sum_{k=0}^{n-1}f\circ T^k-\bar{f}\,\right\Vert_{L^p} = 0.
\end{align*}
\end{theorem}
\begin{proof}
We first take a bounded function $g\in L^\infty(X,\scr{F},\mu)$, so that $g\in L^1(X,\scr{F},\mu)$. By Birkhoff's theorem, there exists a $T$-invariant $\bar{g}\in L^1(X,\scr{F},\mu)$ such that $A_ng\to\bar{g}$ a.e.. Then for almost all $x\in X$,
\begin{align*}
	\lim_{n\to\infty}\left\vert (A_ng)(x)-\bar{g}(x)\right\vert^p=0.
\end{align*}
Since $g$ is bounded, so is $A_ng$ and $\bar{g}$, and $\bar{g}\in L^\infty(X,\scr{F},\mu)$. By Lebesgue dominated convergence theorem, 
\begin{align*}
	\lim_{n\to\infty}\left\Vert A_ng-\bar{g}\right\Vert_{L^p}^p=0.
\end{align*}
We now consider the general case $f\in L^p(X,\scr{F},\mu)$. Since $T$ is measure-preserving, we have $\Vert f\circ T^n\Vert_{L^p}=\Vert f\Vert_{L^p}$ for all $n\in\bbN$, and $\Vert A_nf\Vert_{L^p}\leq\Vert f\Vert_{L^p}$ by the triangle inequality. Given any $\epsilon>0$, we take a bounded $g\in L^\infty(X,\scr{F},\mu)$ such that $\Vert f-g\Vert_{L^p}<\epsilon/3$. Since $(A_ng)$ is a Cauchy sequence in $L^p$, we take $N>0$ such that $\Vert A_ng-A_mg\Vert_{L^p}<\epsilon/3$ for all $m,n\geq N$, and hence
\begin{align*}
	\Vert A_nf-A_mf\Vert_{L^p}&\leq\Vert A_nf-A_ng\Vert_{L^p}+\Vert A_ng-A_mg\Vert_{L^p} + \Vert A_mg-A_mf\Vert_{L^p}\\
	&\leq 2\Vert f-g\Vert_{L^p}+\Vert A_ng-A_mg\Vert_{L^p}<\epsilon.
\end{align*}
Thus $(A_nf)$ is Cauchy in $L^p$, and there exists $\bar{f}\in L^p(X,\scr{F},\mu)$ such that $\Vert A_nf-\bar{f}\Vert_{L^p}\to 0$. By Birkhoff's theorem, $(A_nf)$ also admits a pointwise limit, which coincides the $L^p$ limit $\bar{f}$. Hence $\bar{f}$ is $T$-invariant.
\end{proof}
\begin{remark}
To summarize, if $1\leq p<\infty$ and $f\in L^p(X,\scr{F},\mu)$, the ergodic average sequence $(A_nf)$ admits a limit $\bar{f}\in L^p(X,\scr{F},\mu)$ such that
\begin{align*}
	\lim_{n\to\infty}\Vert A_nf-\bar{f}\Vert_{L^p}=0,\quad\text{and}\quad \lim_{n\to\infty}(A_nf)(x)=\bar{f}(x)\ \ \text{for a.e.}\ x\in X.
\end{align*} In a nutshell, $A_nf\to\bar{f}$ both a.e. and in $L^p$.
\end{remark}

\subsection{The Krein-Milman Theorem}
In this section we introduce a general result about compact
convex subsets of a locally convex Hausdorff topological vector space, which is used in the proof of unique ergodicity. 
\begin{definition}[Extreme point and face]
	Let $X$ be a vector space and $K\subset X$ a nonempty convex subset. 
	\begin{itemize}
		\item[(i)] A point $x$ of $K$ is called an \textit{extreme point} of $K$ if there do not exist $y,z\in K$ and $0<\lambda<1$ such that $\lambda y+(1-\lambda)z=x$. We denote by $\mathrm{ext}(K)$ the set of extreme points of $K$.
		\item[(ii)] A nonempty convex subset $F\subset K$ is called a \textit{face} of $K$ if for all $x,y\in K$ and $0<\lambda<1$ such that $\lambda x+(1-\lambda)y\in F$, we have $x,y\in F$.
	\end{itemize} 
	\begin{remark}
		A point $x\in K$ is an extreme point of $K$ if and only if the singleton $\{x\}$ is a face of $K$.
	\end{remark}
\end{definition}

\begin{lemma}\label{kmlemma1}
	Let $X$ be a vector space, and let $A,B,C$ be convex subsets of $K$. If $B$ is a face of $A$ and $C$ is a face of $B$, then $C$ is a face of $A$.
\end{lemma}
\begin{proof}
	Let $x,y\in A$ and $0<\lambda<1$. If $\lambda x+(1-\lambda)y\in C$, since $C\subset B$ and $B$ is a face of $A$, we have $x,y\in B$. Again, since $C$ is a face of $B$, we have $x,y\in C$. Therefore $C$ is a face of $A$.
\end{proof}

\begin{lemma}\label{kmlemma3}
	Let $X$ be a locally convex Hausdorff topological space. If $K\in\scr{K}$ is a compact convex set and $\ell:X\to\bbR$ is a continuous linear functional, the set
	\begin{align*}
		F_\ell:=\left\{x\in K:\ell(x)=\sup_{y\in K}\ell(y)\right\}
	\end{align*}
	is a nonempty compact convex subset of $K$, and $F_\ell$ is a face of $K$.
\end{lemma}
\begin{proof}
	We abbreviate $c=\sup_{y\in K} \ell(y)$. 
	\begin{itemize}
		\item Since $K$ is compact and $\ell$ is continuous, there exists $x\in K$ such that $\ell(x)=c$, and $F$ is nonempty. 
		\item Since $X$ is Hausdorff and $\ell$ is continuous, both $K$ and $\ell^{-1}(\{c\})$ is closed. Hence $F_\ell$ is closed and compact.
		\item Since $K$ is convex and $f$ is linear, $\ell^{-1}(\{c\})$ is convex, and so is $F$.
	\end{itemize}
	To summarize, $F$ is nonempty, compact and convex. To prove that $F$ is a face of $K$, we fix $x,y\in K$ and $0<\lambda<1$ such that $\lambda x+(1-\lambda)y\in F$. Then $\lambda \ell(x)+(1-\lambda)\ell(y)=\ell(\lambda x+(1-\lambda)y)=c$. Since both $\ell(x)$ and $\ell(y)$ are no greater than $c$, we have $\ell(x)=\ell(y)=c$, and $x,y\in F$. Hence $F_\ell$ is a face of $K$.
\end{proof}


\begin{lemma}[Existence]\label{kmlemma2}
	Let $X$ be a locally convex Hausdorff topological vector space, and let $K\subset X$ be a nonempty compact convex set. Then the set of extreme points of $K$ is nonempty.
\end{lemma}
\begin{proof}
	The proof is divided to three steps.
	\item \textbf{Step I.} Let $\scr{K}$ be the set of all nonempty compact convex subset of $X$, and define the relation $\preceq$ on $\scr{K}$ by $F\preceq K$ if and only if $F$ is a face of $K$. By Lemma \ref{kmlemma1}, $(K,\preceq)$ is a partially ordered set. Since $X$ is Hausdorff, every nonempty chain $\scr{C}\subset\scr{K}$ has a infimum $C_0=\bigcap_{C\in\scr{C}}C\in\mathscr{K}$.
	\item \textbf{Step II.} We claim that every minimal element of $\scr{K}$ is a singleton. 
	
	If $K\subset\scr{K}$ is not a singleton, we take $x,y\in K$ such that $x\neq y$ and take a convex open neighborhood $U$ of $x$ that does not contain $y$. Using the hyperplane separation theorem, there exists a continuous linear functional $\ell:X\to\bbR$ such that $\ell(y)<\ell(z)$ for all $z\in U$. By Lemma \ref{kmlemma3}, the set $F_\ell\in\scr{K}$ is a face of $K$ and $y\in K\backslash F$. Hence $K$ is not a minimal element of $\scr{K}$.
	
	\item \textbf{Step IV.} By Step I and Zorn's lemma, there exists a minimal element $E\subset\scr{K}$. By Step III, the minimal element $E$ is a singleton $\{x\}$. Then $x\in\mathrm{ext}(K)$.
\end{proof}

Now we introduce the Krein-Milman theorem.
\begin{theorem}[Krein-Milman theorem]
	Let $X$ be a locally convex Hausdorff topological vector space, and let $K\subset X$ be a nonempty compact convex set. Then $K$ is the closed convex hull of its extreme points, i.e. $$K=\ol{\mathrm{conv}}(\mathrm{ext}(K)).$$
\end{theorem}
\begin{proof}
	Following the proof of Lemma \ref{kmlemma2}, we have $K\in\scr{K}$. To prove the desired result, it suffices to show $K\subset\ol{\mathrm{conv}}(\mathrm{ext}(K))$. We argue by contradiction. If  $x\in K\backslash\ol{\mathrm{conv}}(\mathrm{ext}(K))$, there exists an open convex neighborhood $U\subset X$ of $x$ such that $U\cap \ol{\mathrm{conv}}(\mathrm{ext}(K))=\emptyset$. Since $\mathrm{ext}(K)$ is nonempty by Lemma \ref{kmlemma2}, there exists a continuous linear functional $\ell$ such that $\ell(x)>\sup_{y\in\ol{\mathrm{conv}}(\mathrm{ext}(K))} \ell(y)$.  By Lemma \ref{kmlemma3}, the set $F_\ell=\left\{x\in K:f(x)=\sup f(K)\right\}$ is a face of $K$ and $F_\ell\cap\mathrm{ext}(K)=\emptyset$. On the other hand, by Lemma \ref{kmlemma2}, the compact convex set $F_\ell$ has an extreme point $x$, which is also an extreme point of $K$ by Lemma \ref{kmlemma1}. This contradicts the fact that $F_\ell\cap\mathrm{ext}(K)=\emptyset$. Thus we complete the proof.
\end{proof}

\subsection{Ergodic Measures and Unique Ergodicity}
\paragraph{Invariant measures.} For convenience, we focus on a compact metrizable space $X$ equipped with the Borel $\sigma$-algebra $\scr{B}$. Then $X$ is a second countable space, and the space $C(X)$ of all continuous functions $f:X\to\bbC$ with the supremum norm $\Vert\cdot\Vert_\infty$ is a separable Banach space. Furthermore, by Corollary \ref{chsradon}, the dual space of $C(X)$ is isomorphic to the space $M(X)$ of complex Borel measures on $X$. Let $T:X\to X$ is a homeomorphism on $X$. A Borel probability measure $\mu$ on $X$ is said to be \textit{$T$-invariant} if 
\begin{align*}
	\int_X f\circ T\,d\mu=\int_X f\,d\mu,\quad\text{for all}\ f\in C(X).
\end{align*}
We denote by $M_T(X)$ the set of all $T$-invariant Borel probability measures on $X$.

\begin{lemma}\label{invariantspace}
Let $X$ be a compact metrizable space, and $T:X\to X$ a homeomorphism.
\begin{itemize}
	\item[(i)] $M_T(X)$ is a weak* compact convex subset of the unit sphere in $M(X)$.
	\item[(ii)] $M_T(X)$ is nonempty.
	\item[(iii)] If $\mu\in M_T(X)$, then $(X,\scr{B},\mu, T)$ is a measure-preserving system, i.e. $\mu(E)=\mu(T^{-1}(E))$ for all $E\in\scr{B}$.
\end{itemize}
\end{lemma}
\begin{proof}
(i) By definition $M_T(X)$ is a convex subset of the unit sphere in $M(X)$. By Banach-Alaoglu theorem, the closed unit ball is compact in the weak* topology on $M(X)$. Then it suffices to show that $M_T(X)$ is weak* closed. We note that a sequence of complex Borel measures $\mu_n\to\mu$ in the weak* topology on $M(X)$ if and only if $\int_X f\,d\mu_n\to\int_X f\,d\mu$ for all $f\in C(X)$. If $\mu_n\in M_T(X)$, by setting $f=1$ we know that $\mu(X)=\lim_{n\to\infty}\mu_n(X)=1$. Furthermore,
\begin{align*}
	\int_X f\circ T\,d\mu=\lim_{n\to\infty} f\circ T\,d\mu_n=\lim_{n\to\infty} f\,d\mu_n=\int_X f\,d\mu,\quad\forall f\in C(X).
\end{align*}
Therefore $\mu\in M_T(X)$, and $M_T(X)$ is closed and hence compact in the weak* topology on $M(X)$.

\item (ii) Fix $x_0\in X$. For each $n\in\bbN$, define the Borel probability measure $\mu_n:\scr{B}\to[0,1]$ by
\begin{align*}
	\int_X f\,d\mu=\frac{1}{n}\sum_{k=0}^{n-1} f(T^kx_0),\quad f\in C(X).
\end{align*}
By Banach-Alaoglu theorem, the sequence has a weak* convergent subsequence $(\mu_{n_j})$. We denote by $\mu$ its weak* limit in $M(X)$. Then $\mu(X)=\int_X 1\,d\mu=\lim_{n\to\infty}\int_X1\,d\mu_n=1$, and for all $f\in C(X)$,
\begin{align*}
	\int_X f\circ T\,d\mu=\lim_{j\to\infty}\frac{1}{n_j}\sum_{k=1}^{n_j}f(T^kx_0)=\lim_{j\to\infty}\frac{1}{n_j}\sum_{k=0}^{n_j-1}f(T^kx_0)=\int_X f\,d\mu.
\end{align*}
Therefore $\mu\in M_T(X)$, and $M_T(X)$ is nonempty.

\item (iii) We defined by $\nu(E)=(T_*\mu)(E)=\mu(T^{-1}(E))$ the pushforward of $\mu$, which is also a measure on $\scr{B}$ by continuity of $T$. By the change-of-variable formula, it suffices to show that $\nu=\mu$ on $\scr{B}$.

For a closed subset $F\subset X$, define $f_n(x)=\max\{1-nd(x,F),0\}$. Then $f_n\in C(X)$ and $f_n\downarrow\chi_F$ as $n\to\infty$. By monotone convergence theorem,
\begin{align*}
	\nu(F)=\lim_{n\to\infty} \int_Xf_n\,d\nu=\lim_{n\to\infty} \int_X f_n\circ T\,d\mu=\lim_{n\to\infty}\int_Xf_n\,d\mu=\mu(F).
\end{align*}
Thus $\nu(F)=\mu(F)$ for all closed subset $F\subset X$, and $\mu(U)=\nu(U)$ for all open subset $U\subset X$. By outer-regularity of $\mu$, we have $\mu=\nu$ everywhere on $\scr{B}$.
\end{proof}
\paragraph{Remark.} Since $T:X\to X$ is an homeomorphism, both $T$ and $T^{-1}$ are measurable. For all $E\in\scr{B}$, we have $\mu(E)=\mu(T^{-1}(T(E)))=\mu(T(E))$. Hence the inverse $T^{-1}$ is also a measure-preserving transformation.

\paragraph{Ergodic measures.} A $T$-invariant probability measure $\mu$ is said to be \textit{$T$-ergodic} if $T$ is $\mu$-ergodic, i.e.
\begin{align*}
	T^{-1}(E)=E\quad\Rightarrow\quad\mu(E)\in\{0,1\}.
\end{align*}
We have the following characterization of $T$-ergodic measures.

\begin{theorem}[Ergodicity and Extremity]\label{ergext}
Let $X$ be a compact metrizable space, and let $T:X\to X$ be a homeomorphism. If $\mu\in M_T(X)$, the following are equivalent:
\begin{itemize}
	\item[(i)] $\mu$ is $T$-ergodic;
	\item[(ii)] $\mu$ is an extreme point of $M_T(X)$.
\end{itemize}
\end{theorem}
\begin{proof}
The proof has three steps.
\item\textbf{Step I.} Let $\mu_1,\mu_2$ be $T$-ergodic measures such that $\mu_1(E)=\mu_2(E)$ for every $T$-invariant Borel set $E\subset X$. We claim that $\int_X f\,d\mu_1=\int_X f\,d\mu_2$ for each $f\in C(X)$, hence $\mu_1=\mu_2$ by Riesz representation theorem.

By Corollary \ref{meanergl1}, the sequence $A_n f$ converges to $\int_X f\,d\mu_j$ in $L^1$, and hence a subsequence $A_{n_i}f$ converges a.e. to $\int_X f\,d\mu_j$, where $j=1,2$. Hence there exists $A_j\subset X$ such that $\mu(A_j)=1$ and
\begin{align*}
	\int_X f\,d\mu_j=\lim_{i\to\infty}\frac{1}{n_i}\sum_{k=0}^{n_i-1}f(T^kx)\quad\text{for all $x\in A_j$}.
\end{align*}
For $j=1,2$, define $E_j=\bigcap_{n\in\bbZ}T^n(A_j)$, so that $E_j$ is a $T$-invariant set with $\mu_j(E_j)=1$. By assumption, $\mu_1(E_1)=\mu_2(E_1)=\mu_1(E_2)=\mu_2(E_2)=1$. Then the $T$-invariant set $E:=E_1\cap E_2$ is nonempty, because $\mu(E)=\mu(E_1)+\mu(E_2)-\mu(E_1\cup E_2)=1$. Since $E\subset A_1\cap A_2$, we fix $x\in E$ and obtain
\begin{align*}
	\int_X f\,d\mu_1=\lim_{i\to\infty}\frac{1}{n_i}\sum_{k=0}^{n_i-1}f(T^kx)=\int_X f\,d\mu_2.
\end{align*}
\item\textbf{Step II.} If $\mu\in M_T(X)$ is ergodic, we claim that $\mu$ is an extreme point of $M_T(X)$. Take $\mu_1,\mu_2\in M_T(X)$ and $0<\lambda<1$ such that $\mu=(1-\lambda)\mu_1+\lambda\mu_2$. If $E\in\scr{B}$ is a $T$-invariant set, we have $\mu(E)\in\{0,1\}$. Then
\begin{itemize}
	\item If $\mu(E)=0$, we have $(1-\lambda)\mu_1(E)+\lambda\mu_2(E)=0$ and $\mu_1(E)=\mu_2(E)=0$. \item Similarly, if $\mu(E)=1$, we have $\mu_1(F)=\mu_2(F)=1$.
\end{itemize}
In either case, we have $\mu_1(E)=\mu_2(E)=\mu(E)\in\{0,1\}$. Hence $\mu_1$ and $\mu_2$ are $T$-ergodic measures that agree on all $T$-invariant Borel sets. By Step I, we have $\mu_1=\mu_2=\mu$, and hence $\mu$ is an extreme point of $M_T(X)$.

\item\textbf{Step III.} Conversely, if $\mu\in M_T(X)$ is not ergodic, we can find two probability measures $\mu_1,\mu_2\in M_T(X)$ with $\mu_1\neq\mu_2$ and $0<\lambda<1$ such that $(1-\lambda)\mu_1+\lambda\mu_2=\mu$, and hence $\mu$ is not an extreme point of $M_Y(X)$.

By non-ergodicity of $(\mu,T)$, there exists a Borel set $B\subset X$ such that $T^{-1}(B)=B$ and $0<\mu(B)<1$. We then define Borel probability measures
\begin{align*}
	\mu_1(E):=\frac{\mu(E\backslash B)}{\mu(X\backslash B)}\quad\text{and}\quad\mu_2(E):=\frac{\mu(E\cap B)}{\mu(B)},\quad E\in\scr{B},
\end{align*}
and take $\lambda=\mu(B)$. For each $E\in\scr{B}$,
\begin{align*}
\mu_2(T^{-1}(E))=\frac{\mu(T^{-1}(E)\cap B)}{\mu(B)}=\frac{\mu(T^{-1}(E\cap B))}{\mu(B)}=\frac{\mu(E\cap B)}{\mu(B)}=\mu_2(E).
\end{align*}
Hence $\mu_2$ is $T$-invariant, and similarly $\mu_1$ is $T$-invariant. Furthermore, $(1-\lambda)\mu_1+\lambda\mu_2=\mu$, as desired.
\end{proof}
\begin{corollary}
Every homeomorphism of a compact metrizable space admits an ergodic measure.
\end{corollary}
\begin{proof}
Since $M_T(X)$ is a nonempty compact convex subset of $M(X)$ by Lemma \ref{invariantspace}, it has an extreme point $\mu$ by Krein-Milman theorem. According to Theorem \ref{ergext}, $\mu$ is a $T$-ergodic measure.
\end{proof}

Aside from existence, we also wonder whether the ergodic measure of a homeomorphism $T$ is unique.
\begin{definition}[Unique ergodicity]
A homeomorphism $T$ of a compact metrizable space $X$ is said to be \textit{uniquely ergodic}, if there is only one Borel probability measure $\mu$ that is $T$-invariant, i.e. $\vert M_T(X)\vert=1$.
\end{definition}
\begin{remark}
Since $M_T(X)$ is the closed convex hull of the $T$-ergodic measures (extreme points), $T$ is uniquely ergodic if and only if there is only one Borel probability measure $\mu$ that is $T$-ergodic.
\end{remark}
\begin{theorem}[Birkhoff's theorem]
Let $T:X\to X$ be a homeomorphism of a compact metrizable space $X$. The following are equivalent.
\begin{itemize}
	\item[(i)] $T$ is uniquely ergodic.
	\item[(ii)] There exists $\mu\in M_T(X)$ such that for all $f\in C(X)$,
	\begin{align}
		\lim_{n\to\infty}\frac{1}{n}\sum_{k=0}^{n-1}f(T^kx)=\int_X f\,d\mu\quad\text{for all $x\in X$.}\label{ergcontpwlimit}
	\end{align}
\item[(iii)] For all $f\in C(X)$, the sequence of functions $A_nf=\frac{1}{n}\sum_{k=0}^{n-1}f\circ T^k$ converges pointwise to a constant.
\item[(iv)] For all $f\in C(X)$, the sequence of functions $A_nf=\frac{1}{n}\sum_{k=0}^{n-1}f\circ T^k$ converges uniformly to a constant.
\end{itemize}
\end{theorem}
\begin{proof}
(i) $\Rightarrow$ (iv): If $T$ is uniquely ergodic, we take for each $x\in X$ the sequence $$\mu_n=\frac{1}{n}\sum_{k=0}^{n-1}\delta_{T^kx},\quad n=1,2,\cdots.$$
By Banach-Alaoglu theorem, and since $\vert M_T(X)\vert=1$, every subsequence of $(\mu_n)$ has a further subsequence converging in the weak* topology to the unique element $\mu\in M_T(X)$, which is ergodic. We claim that $(\mu_n)$ converges to $\mu$ in the weak* topology. If there exists a neighborhood $U$ of $\mu$ in the weak* topology such that for each $k\in\bbN$, there exists $n_k>k$ with $\mu_{n_k}\notin U$, which gives a subsequence $(\mu_{n_k})$ outside $U$. Therefore,
\begin{align*}
\lim_{n\to\infty}\frac{1}{n}\sum_{k=0}^{n-1}\delta_{T^kx}\overset{w^*}{=}\mu.
\end{align*}
Integrating both sides with $f\in C(X)$ gives (\ref{ergcontpwlimit}). Argue (iv) by contradiction. If $(A_nf)$ does not converge uniformly to $\int_X f\,d\mu$, there exists $\epsilon>0$ such that for each $m\geq 1$, there exists $n_m\geq m$ and $x_m\in X$ such that
\begin{align}
	\left\vert\frac{1}{n_m}\sum_{k=0}^{n_m-1}f(T^kx_m)-\int f\,d\mu\right\vert\geq\epsilon.\label{birkcontra}
\end{align}
We consider the sequence $\nu_m=\frac{1}{n_m}\sum_{k=0}^{n_m-1}\delta_{T^k x_m}$, which also converges to $\mu\in M_T(X)$ in weak* topology, by passing to a subsequence if necessary. Then the left-hand side of (\ref{birkcontra}) goes to $0$ as $m\to\infty$, a contradiction.
\item (iv) $\Rightarrow$ (iii) is clear.
\item (iii) $\Rightarrow$ (ii): Define the positive linear functional $Af=\lim_{n\to\infty}\frac{1}{n}\sum_{k=0}^{n-1}f\circ T^k$. Then $\vert Af\vert\leq\Vert f\Vert_\infty$, and $A:C(X)\to\bbC$ is continuous. By Riesz representation theorem, there is a Borel measure $\mu\in M(X)$ such that $Af=\int_X f\,d\mu$. Since $\mu(X)=A1=1$ and $A(f\circ T)=Af$, the measure $\mu\in M_T(X)$.
\item (ii) $\Rightarrow$ (i): Let $\mu,\nu\in M_T(X)$, where $\mu$ is the measure such that the hypothesis holds. For any $f\in C(X)$, by $T$-invariance of $\nu$ and dominated convergence theorem, 
\begin{align}
	\int_X f\,d\nu=\lim_{n\to\infty}\int_X A_nf\,d\nu=\int_X\lim_{n\to\infty} A_nf\,d\nu=\int_X\left(\int_X f\,d\mu\right)d\nu=\int_X f\,d\mu.\label{birkchange}
\end{align}
By Riesz representation theorem, we have $\mu=\nu$, and $\vert M_T(X)\vert=1$.
\end{proof}


\subsection{The Recurrence Theorems}
In many scenarios, we are also interested in the recurrence property of a dynamical system $(X,\scr{F},\mu,T)$. Beginning from a state $x_0\in X$, we wonder if the system will return to a state arbitrarily closed to, or exactly the same as, the initial state $x_0$.
\begin{definition}[Recurrence]
Let $(X,\scr{F},\mu,T)$ be a measure-preserving system. For a subset $A\subset X$, the \textit{first return time} of $A$ is the map defined for almost every $x\in A$ by
\begin{align*}
	n_A(x)=\inf\left\{n\geq 1:T^nx\in A\right\}.
\end{align*}
We write $n_A=n_A^1$. For each integer $k\geq 2$, we define the \textit{$k^\text{th}$ return time} by
\begin{align*}
	n_A^k(x)=\inf\{n>n_A^{k-1}(x):T^nx\in A\}.
\end{align*}
We say that a point $x\in A$ is \textit{infinitely recurrent to $A$}, or \textit{returns infinitely to $A$}, if $(T^nx)_{n=1}^\infty$ contains a subsequence $(T^{n_k}x)_{k=1}^\infty\subset A$, or equivalently, $n_A^k(x)<\infty$ for every $k\in\bbN$.
\end{definition}

\begin{theorem}[Poincaré recurrence theorem]
Let $(X,\scr{F},\mu,T)$ be a measure-preserving system where $\mu$ is a probability measure. For each set $A\subset\scr{F}$, almost every $x\in A$ is infinitely recurrent to $A$. That is,
\begin{align*}
	\mu\left(\left\{x\in A:T^nx\in A\ \text{for infinitely many $n\in\bbN$}\right\}\right)=\mu(A).
\end{align*}
\end{theorem}
\begin{proof}
We let $B=\left\{x\in A:T^nx\in A\ \text{for infinitely many}\  n\in\bbN\right\}$. Then
\begin{align*}
	B&=\left\{x\in A:T^nx\in A\ \text{for infinitely many}\  n\in\bbN\right\}\\
	&=\left\{x\in A:\text{for every $n\in\bbN$, there exists $k\geq n$ such that}\ T^kx\in A\right\}\\
	&=\bigcap_{n=1}^\infty\bigcup_{k=n}^\infty A\cap T^{-k}(A)=A\cap \bigcap_{n=1}^\infty\bigcup_{k=n}^\infty T^{-k}(A).
\end{align*}
For every $n\in\bbN_0$, let $A_n=\bigcup_{k=n}^\infty T^{-k}(A)$. Then $T^{-n}(A_0)=A_n\subset A_0$. Since $A\backslash A_n\subset A_0\backslash A_n=A_0\backslash T^{-n}(A_0)$,
\begin{align*}
	0\leq\mu(A\backslash A_n)\leq\mu(A_0\backslash T^{-n}(A_0))=\mu(A_0)-\mu(T^{-n}(A_0))=0,
\end{align*}
where the last inequality follows from the facts that $T$ is measure-preserving and $\mu$ is finite. Then
\begin{align*}
	\mu(B)=\mu\left(A\cap\bigcap_{n=1}^\infty A_n\right)=\mu\left(A\backslash \bigcup_{n=1}^\infty(A\backslash A_n)\right)=\mu(A)-\mu\left(\bigcup_{n=1}^\infty(A\backslash A_n)\right)=\mu(A).
\end{align*}
Then we complete the proof.
\end{proof}

\paragraph{Asymptotic relative frequency.}  The Poincaré recurrence theorem implies that, for almost every $x\in A$, the trajectory $(T^nx)_{n=0}^\infty$ hits $A$ infinitely many times. However, it does
not predict the frequency of the visits that $x$ makes to the set $A$. The relative number of elements of $\{x,Tx,T^2x,\cdots,T^{n-1}x\}$ in $A$ is
\begin{align*}
	\frac{1}{n}\left\vert\left\{T^kx\in A:k=0,1,\cdots,n-1\right\}\right\vert=\frac{1}{n}\sum_{k=0}^{n-1}\chi_A(T^k x).
\end{align*}
By Birkhoff's theorem, if $T$ is $\mu$-ergodic, for almost all $x\in X$, the asymptotic relative frequency is
\begin{align*}
	\lim_{n\to\infty}\sum_{k=0}^{n-1}\chi_A(T^k x)=\int_X\chi_A\,d\mu=\mu(A).
\end{align*}

The Poincaré recurrence theorem asserts that almost every point in a positive measure set returns to the set after a sufficiently long but finite time, but does not give an estimate of the return time. The Kac's lemma states that, in an ergodic system, the points in a positive measure set return to the set within an average time inversely proportional to the measure of the set.
\begin{theorem}[Kac's lemma]
Let $(X,\scr{F},\mu,T)$ be an ergodic system on a probability space. For each set $A\in\scr{F}$ with $\mu(A)>0$, the first return time $n_A$ satisfies
\begin{align*}
	\int_A n_A\,d\mu=1.
\end{align*}
\end{theorem}
\begin{proof}
Let $A_n=\{x\in A:n_A(x)=n\}$ be the set of points in $A$ that return to $A$ after exactly $n$ times. Then
\begin{align*}
	A_n&=\left\{x\in A:T^nx\in A\ \text{and}\ Tx\notin A,\ T^2x\notin A,\cdots,T^{n-1}x\notin A\right\}=A\cap T^{-n}(A)\cap \bigcap_{k=1}^{n-1}T^{-k}(X\backslash A).
\end{align*}
Similarly, we define $$B_n=\{x\notin A: x\ \text{enters $A$ at time $n$}\}=T^{-n}(A)\cap \bigcap_{k=0}^{n-1}T^{-k}(X\backslash A).$$
Since $T$ is $\mu$-ergodic, and $\mu(A)>0$, almost every $x\in X$ enters $A$ after a sufficiently long time, and the set $\bigcap_{n=0}^\infty T^{-n}(X\backslash A)$ has measure zero. Hence both $\mu(A_n)$ and $\mu(B_n)$ goes to zero as $n\to\infty$. Furthermore, $(A_n,B_n)_{n=1}^\infty$ are disjoint sets that almost cover $X$. Also note that
\begin{align*}
	T^{-1}(B_n)=T^{-n-1}(A)\cap \bigcap_{k=1}^{n}T^{-k}(X\backslash A)=A_{n+1}\cup B_{n+1}.
\end{align*}
Since $T$ is measure preserving, $\mu(B_n)=\mu(T^{-1}(B_n))=\mu(A_{n+1})+\mu(B_{n+1})$, and by induction we have
\begin{align*}
	\mu(B_n)=\sum_{k=n+1}^\infty\mu(A_k)+\lim_{k\to\infty}\mu(B_k)=\sum_{k=n+1}^\infty\mu(A_k),\quad n\in\bbN.
\end{align*}
By Poincaré recurrence theorem, $A=\bigcup_{n=1}^\infty A_n$. Therefore
\begin{align*}
	1&=\mu(X)=\sum_{n=1}^\infty\left[\mu(A_n)+\mu(B_n)\right]=\sum_{n=1}^\infty\sum_{k=n}^{\infty}\mu(A_k)=\sum_{k=1}^\infty\sum_{n=1}^{k}\mu(A_k)\\
	&=\sum_{k=1}^\infty k\mu(A_k)=\sum_{k=1}^\infty\int_{A_k}n_A\,d\mu=\int_An_A\,d\mu.
\end{align*}
Thus we complete the proof.
\end{proof}
\begin{remark}
The Kac's lemma can also be stated as
\begin{align*}
	\frac{1}{\mu(A)}\int_A n_A\,d\mu=\frac{1}{\mu(A)},
\end{align*}
where the left-hand side of the equation is the \textit{mean return time} to $A$.
\end{remark}


\end{document}